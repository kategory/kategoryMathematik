%******************************************************** -*-LaTeX-*- ******************************
%                                                                                                  *
% v5.1.1.1.4.5 Quotienten-Moduln.tex                                                               *
%                                                                                                  *
% Copyright (C) 2024 Kategory GmbH \& Co. KG (joerg.kunze@kategory.de)                             *
%                                                                                                  *
% v5.1.1.1.4.5 Quotienten-Moduln is part of kategoryMathematik.                                    *
%                                                                                                  *
% kategoryMathematik is free software: you can redistribute it and/or modify                       *
% it under the terms of the GNU General Public License as published by                             *
% the Free Software Foundation, either version 3 of the License, or                                *
% (at your option) any later version.                                                              *
%                                                                                                  *
% kategoryMathematik is distributed in the hope that it will be useful,                            *
% but WITHOUT ANY WARRANTY; without even the implied warranty of                                   *
% MERCHANTABILITY or FITNESS FOR A PARTICULAR PURPOSE.  See the                                    *
% GNU General Public License for more details.                                                     *
%                                                                                                  *
% You should have received a copy of the GNU General Public License                                *
% along with this program.  If not, see <http://www.gnu.org/licenses/>.                            *
%                                                                                                  *
%***************************************************************************************************

\documentclass[a4paper]{amsart}
% \documentclass[a4paper]{book}

%-----------------------------------------------------------------------------------------------------*
% package:                                                                                            *
%-----------------------------------------------------------------------------------------------------*
\usepackage{amssymb}
\usepackage{amsfonts}
\usepackage{amsmath}
\usepackage{amsthm}

\usepackage{mathabx}

\usepackage{a4wide} % a little bit smaller margins

\usepackage{graphicx}
\usepackage{hyperref}
\usepackage{algorithmic}
\usepackage{listings}
\usepackage{color}
\usepackage{colortbl}
\usepackage{sidecap}
\usepackage{comment}
\usepackage{tcolorbox}
\usepackage{collect}

\usepackage{upgreek}

% \usepackage{diagrams}

\usepackage[german]{babel}
\usepackage[none]{hyphenat}
\emergencystretch=4em

\usepackage[utf8]{inputenc} % to be able to use äöü as characters in text
\usepackage[T1]{fontenc} % to be able to use äöü in lables
\usepackage{lmodern}     % to avoid pixelation introduced by fontenc

\usepackage{hyperref}

\usepackage{tikz}
\usepackage{tikz-cd}
\usetikzlibrary{babel}

\usepackage[only,llbracket,rrbracket]{stmaryrd}


%-----------------------------------------------------------------------------------------------------*
% theorem:                                                                                            *
%-----------------------------------------------------------------------------------------------------*
\theoremstyle{definition}
\newtheorem{theorem}{Theorem}[subsection]

\newcommand{\myTheorem}[1]{%
  \newtheorem{jk#1}[theorem]{#1}
  \newenvironment{#1}[1]{%
    \expandafter\begin{jk#1} \expandafter\label{#1:##1}\textbf{(##1):}
  }{%
    \expandafter\end{jk#1}
  }
}

\myTheorem{Definition}
\myTheorem{Proposition}
\myTheorem{Satz}
\myTheorem{Theorem}
\myTheorem{Axiom}
\myTheorem{Beispiel}
\myTheorem{Anmerkung}

\definecollection{jkjkFrage}
\newtheorem{jkFrage}[theorem]{Frage}
\newenvironment{Frage}[1]{%
  \expandafter\begin{jkFrage} \expandafter\label{Frage:#1}\textbf{(#1):}
  \begin{collect}{jkjkFrage}{}{}
    \item \ref{Frage:#1} #1
  \end{collect}
}{%
  \expandafter\end{jkFrage}
}

\newcommand{\myRef}[2]{[#1 \ref{#1:#2}, ``#2'']}

\renewcommand{\proofname}{Beweis}

%-----------------------------------------------------------------------------------------------------*
% operator:                                                                                           *
%-----------------------------------------------------------------------------------------------------*
\DeclareMathOperator{\End}{End}
\DeclareMathOperator{\Ker}{Ker}
\DeclareMathOperator{\Mat}{Mat}
\DeclareMathOperator{\rank}{rank}
\DeclareMathOperator{\ggT}{ggT}
\DeclareMathOperator{\len}{len}
\DeclareMathOperator{\ord}{ord}
\DeclareMathOperator{\kgV}{kgV}
\DeclareMathOperator{\id}{id}
\DeclareMathOperator{\red}{red}
\DeclareMathOperator{\supp}{supp}
\DeclareMathOperator{\Bild}{Bild}
\DeclareMathOperator{\Rang}{Rang}
\DeclareMathOperator{\Det}{Det}
\DeclareMathOperator{\Hom}{Hom}
\DeclareMathOperator{\GL}{GL}

\DeclareMathOperator{\sub}{sub}
\DeclareMathOperator{\blk}{blk}
\DeclareMathOperator{\minimal}{minimal}
\DeclareMathOperator{\maximal}{maximal}

\definecolor{mygreen}{rgb}{0,0.6,0}
\definecolor{mygray}{rgb}{0.5,0.5,0.5}
\definecolor{mymauve}{rgb}{0.58,0,0.82}

\lstset{ %
  backgroundcolor=\color{white},   % choose the background color
  basicstyle=\ttfamily\footnotesize,        % size of fonts used for the code
  breaklines=true,                 % automatic line breaking only at whitespace
  captionpos=b,                    % sets the caption-position to bottom
  commentstyle=\color{mygreen},    % comment style
  escapeinside={\%*}{*)},          % if you want to add LaTeX within your code
  keywordstyle=\color{blue},       % keyword style
  stringstyle=\color{mymauve},     % string literal style
  frame=single
}

\setcounter{MaxMatrixCols}{20}

%******************************************************************************************************
%                                                                                                     *
% definition:                                                                                         *
%                                                                                                     *
%******************************************************************************************************
\newcommand{\R}{\ensuremath{\mathbb{ R }}}
\newcommand{\Q}{\ensuremath{\mathbb{ Q }}}
\newcommand{\Z}{\ensuremath{\mathbb{ Z }}}
\newcommand{\N}{\ensuremath{\mathbb{ N }}}
\newcommand{\C}{\ensuremath{\mathbb{ C }}}
\newcommand{\A}{\ensuremath{\mathbb{ A }}}
\newcommand{\F}{\ensuremath{\mathbb{ F }}}
\newcommand{\K}{\ensuremath{\mathbb{ K }}}
\newcommand{\Pb}{\ensuremath{\mathbb{ P }}}

\newcommand{\M}{\ensuremath{\mathcal{ M }}}
\newcommand{\V}{\ensuremath{\mathcal{ V }}}

\newcommand{\AAA}{\ensuremath{\mathcal{ A }}}
\newcommand{\BB}{\ensuremath{\mathcal{ B }}}
\newcommand{\CC}{\ensuremath{\mathcal{ C }}}
\newcommand{\DD}{\ensuremath{\mathcal{ D }}}
\newcommand{\EE}{\ensuremath{\mathcal{ E }}}
\newcommand{\FF}{\ensuremath{\mathcal{ F }}}
\newcommand{\KK}{\ensuremath{\mathcal{ K }}}
\newcommand{\MM}{\ensuremath{\mathcal{ M }}}
\newcommand{\PP}{\ensuremath{\mathcal{ P }}}
\newcommand{\ZZ}{\ensuremath{\mathcal{ Z }}}

\newcommand{\Set}{\text{\textbf{Set}} }


\newcommand{\imporant}[1]{ \textcolor{red}{\textbf{#1}} }

\newcommand{\bb}[1]{\mathbf{#1}}
\newcommand{\balpha}{\boldsymbol{\upalpha}}
\newcommand{\bbeta}{\boldsymbol{\upbeta}}
\newcommand{\bgamma}{\boldsymbol{\upgamma}}
\newcommand{\bdelta}{\boldsymbol{\delta}}
\newcommand{\bmu}{\boldsymbol{\upmu}}

\newcommand{\z}[1]{\Z_{#1}}
\newcommand{\e}[1]{\z{#1}^*}
\newcommand{\q}[1]{(\e{#1})^2}
\newcommand{\m}{\mathcal}

\newcommand{\zz}[1]{\ensuremath{\Z /#1\Z}}

\excludecomment{book}
\excludecomment{example}
\excludecomment{backup}

\newcommand{\zb}{z.~B. }

\begin{document}

%******************************************************************************************************
%                                                                                                     *
\begin{titlepage}
%                                                                                                     *
%******************************************************************************************************
% \vspace*{\fill}
\centering
{\huge
(Höhere Grundlagen) Homologische Algebra\\[1cm]
\textbf{v5.1.1.1.4.5 Quotienten-Moduln}
}\\[1cm]

\textbf{Kategory GmbH \& Co. KG}\\
Präsentiert von Jörg Kunze\\
Copyright (C) 2024 Kategory GmbH \& Co. KG

\end{titlepage}

%\clearpage
%\setcounter{page}{2}
%
%\tableofcontents

\newpage

%******************************************************************************************************
%                                                                                                     *
\section*{Beschreibung}
%                                                                                                     *
%******************************************************************************************************

\subsection*{Inhalt}
Ist eine Äquivalenz-Relation kompatibel zu den Operationen eines Moduls, so bilden deren Äquivalenz-Klassen selber einen Modul. Diesen nennen wir den Quotienten-Modul. Die Äquivalenz-Klasse der $0$ ist ein Untermodul.

Umgekehrt definiert jeder Unter-Modul eine kompatible Äquivalenz-Relation.

Die Projektion auf den Quotienten-Modul ist surjektiv. Umgekehrt bilden die Fasern jedes surjektiven Modul-Homomorphismus eine kompatible Äquivalenz-Relation. Der zugehörige Quotienten-Modul ist isomorph zu dem Ziel-Modul.

Damit haben wir eine (bis auf Isomorphie) 1:1 Zuordnung von kompatible Äquivalenz-Relationen, Unter-Moduln und surjektiven Modul-Homomorphismen.

Bei der Quotienten-Bildung kann es zu Aufwickelungen kommen. Da sowohl der Ausgangs-Modul und sogar der Ring selber über Wickelungen verfügen kann, ist die Situation im Allgemeinen unübersichtlich. Die Untersuchung dieser Phänomene ist ein Ziel der homologischen Algebra.

\subsection*{Präsentiert}
Von Jörg Kunze

\subsection*{Voraussetzungen}
Ringe, Gruppen, Moduln, Kategorien, Vektorräume, Unter-Gruppen, Äquivalenz-Relationen, Quotienten in Gruppen und Vektorräumen

\subsection*{Text}
Der Begleittext als PDF und als LaTeX findet sich unter
{\tiny
   \url{https://github.com/kategory/kategoryMathematik/tree/main/v5%20H%C3%B6here%20Grundlagen/v5.1%20Homologische%20Algebra/v5.1.1.1.4.5%20Quotienten-Moduln}
}

\subsection*{Meine Videos}
Siehe auch in den folgenden Videos:\\
\\
v5.1.1.1 (Höher) Homologische Algebra - Moduln\\
\url{https://youtu.be/JY43_07kNmA}\\
\\
v5.1.1.1.4 (Höher) Homologische Algebra - Unter-Moduln\\
\url{https://youtu.be/4g2TgQx7JkI}\\
\\
v5.0.1.0.5 (Höher) Kategorien - Mono Epi Null\\
\url{https://youtu.be/n4-qZJK_sH0}

\subsection*{Quellen}
Siehe auch in den folgenden Seiten:\\
\url{https://de.wikipedia.org/wiki/Quotientenmodul}\\
\url{https://ncatlab.org/nlab/show/quotient+module}

\subsection*{Buch}
Grundlage ist folgendes Buch:\\
"`An Introduction to Homological Algebra"'\\
Joseph J. Rotman\\
2009\\
Springer-Verlag New York Inc.\\
978-0-387-24527-0 (ISBN)\\
{\tiny
   \url{https://www.lehmanns.de/shop/mathematik-informatik/6439666-9780387245270-an-introduction-to-homological-algebra}}\\
\\
Oft zitiert:\\
"`An Introduction to Homological Algebra"'\\
Charles A. Weibel\\
1995\\
Cambridge University Press\\
978-0-521-55987-4 (ISBN)\\
{\tiny
   \url{https://www.lehmanns.de/shop/mathematik-informatik/510327-9780521559874-an-introduction-to-homological-algebra}}\\
\\
Ohne Kategorien-Theorie:\\
"`Algébre 10. Algèbre homologique"'\\
Nicolas Bourbaki\\
1980\\
Springer-Verlag \\
978-3-540-34492-6 (ISBN)\\
{\tiny
   \url{https://www.lehmanns.de/shop/mathematik-informatik/7416782-9783540344926-algebre}}

\subsection*{Lizenz}
Dieser Text und das Video sind freie Software. Sie können es unter den Bedingungen der
GNU General Public License, wie von der Free Software Foundation veröffentlicht, weitergeben
und/oder modifizieren, entweder gemäß Version 3 der Lizenz oder (nach Ihrer Option) jeder späteren Version.

Die Veröffentlichung von Text und Video erfolgt in der Hoffnung, dass es Ihnen von Nutzen sein wird,
aber OHNE IRGENDEINE GARANTIE, sogar ohne die implizite Garantie der MARKTREIFE oder der
VERWENDBARKEIT FÜR EINEN BESTIMMTEN ZWECK. Details finden Sie in der GNU General Public License.

Sie sollten ein Exemplar der GNU General Public License zusammen mit diesem Text erhalten haben
(zu finden im selben Git-Projekt).
Falls nicht, siehe \url{http://www.gnu.org/licenses/}.

%******************************************************************************************************
\subsection*{Das Video}
%******************************************************************************************************
Das Video hierzu ist zu finden unter
{\tiny
   \url{uups}
}

%******************************************************************************************************
%                                                                                                     *
\section{v5.1.1.1.4.5 Quotienten-Moduln}
%                                                                                                     *
%******************************************************************************************************

%******************************************************************************************************
\subsection{Modul-kompatible Äquivalenz-Relationen}
%******************************************************************************************************
Sei $R$ ein Ring mit $1$ und $M$ ein $R$-Modul. Sei $\sim$ eine Äquivalenz-Relation auf $M$.
\begin{Definition}{Modul-kompatibel}
   Die Äquivalenz-Relation $\sim$ auf dem $R$-Modul $M$ ist \textbf{Modul-kompatibel}, falls sie mit der Modulstruktur in folgendem Sinne verträglich ist. Für alle $x_1, x_2, y \in M$ und $r \in R$
   \begin{alignat}{6}
      &x_1 \sim x_2 \Rightarrow &&y &&+      &&x_1 \sim y &&+       &&x_2 \\
      &x_1 \sim x_2 \Rightarrow &&       &&-      &&x_1 \sim        &&-       &&x_2 \\
      &x_1 \sim x_2 \Rightarrow &&r      &&\cdot  &&x_1 \sim r      && \cdot  &&x_2 
   \end{alignat}
\end{Definition}
Beachte: die Null-ären Operatoren $0$ und $1$ müssen hier nicht berücksichtigt werden, weil in ihnen keine variablen Elemente vorkommen, die zu irgendetwas äquivalent sein müssten.

%******************************************************************************************************
\subsection{Äquivalenz-Relationen vs Unter-Moduln}
%******************************************************************************************************
\begin{Satz}{Modul-kompatible Äquivalenz-Relation ist Unter-Modul}
   Sei $\sim$ eine modul-kompatible Äquivalenz-Relation. Dann ist
   \begin{equation}\label{UnterModulZuRelation}
      N := \{ x \mid x \sim 0\}
   \end{equation}
   Unter-Modul.
\end{Satz}
\begin{proof}
   Wir müssen die Abgeschlossenheit von $N$ gegenüber den Operatoren zeigen. Seinen dazu $x, y \in N$ also $x \sim 0$ und $y \sim 0$. Damit haben wir 
   \begin{equation}\label{abgeschlossenPlus}
      x + y \sim x \ + 0 = x \sim 0.
   \end{equation}
   Und damit $x + y \in N$.
   $0 \in N$ da $0 \sim 0$, da $\sim$ als Äquivalenz-Relation reflexiv ist. 
   \begin{equation}\label{abgeschlossenMinus}
      - x \sim - 0 = 0.
   \end{equation}
   Und schließlich
   \begin{equation}\label{abgeschlossenMal}
      r \cdot x \sim r \cdot 0 = 0.
   \end{equation}
\end{proof}

\begin{Satz}{Unter-Modul ist modul-kompatible Äquivalenz-Relation}
   Sei $N$ ein Unter-Modul. Dann ist
   \begin{equation}\label{RelationZuUnterModul}
      x \sim y :\Leftrightarrow x-y \in N
   \end{equation}
   eine modul-kompatible Äquivalenz-Relation.
\end{Satz}
\begin{proof}
   Für die Eigenschaften einer Äquivalenz-Relation benötigen wir nur die Unter-Gruppen-Eigenschaften. Die Modul-Eingenschaften brauchen wir für die Kompatibilität mit der Skalar-Multiplikation.
   
   \emph{Reflexiv:} $x - x = 0 \in N$.
   
   \emph{Symmetrisch:} $x - y = -(y - x) \in N$.
   
   \emph{Transitiv:} $x - z = (x - y) + (y - z) \in N$.

   $+$: $(y + x_1) - (y + x_2) = x_1 - x_2 \in N$.
   
   $-$: $(-x_1) - (-x_2) = -(x_1 - x_2) \in N$.
   
   $\cdot$: $rx_1 - rx_2 = r(x_1 - x_2) \in N$.
\end{proof}

VORSICHT: Nicht alle Unter-Gruppen von $M$ sind auch Unter-Moduln. Die Multiplikation mit Elementen aus $R$ können die Gruppen-Elemente auch drehen. In den Beweisen oben sehen wir, dass die Kompatibilität der Skalarmultiplikation dem Distributiv-Gesetz und der Abgeschlossenheit entspricht.

{\color{red}Damit haben wir eine 1:1-Beziehung zwischen Unter-Moduln und modul-kompatiblen Äquivalenz-Relationen.}

%******************************************************************************************************
\subsection{Ausschließlich Unter-Moduln produzieren Quotienten-Moduln}
%******************************************************************************************************
\begin{Satz}{Ausschließlich Unter-Moduln produzieren Quotienten-Moduln}
Sei $M$ ein $R$-Modul und $N$ eine Unter-\textbf{Gruppe}. Sei $M/N$ ein $R$-Modul. Dann ist $N$ sogar Unter-$R$-Modul.
\end{Satz}
\begin{proof}
   Wir müssen die Abgeschlossenheit gegenüber der Skalar-Multiplikation mit Elementen aus $R$ zeigen. Seien dazu $n \in N$ und $r \in R$ beliebig. Da $M/N$ ein $R$-Modul ist, ist (in $M/N$)
   \begin{equation}\label{UnterModul}
      0 = r \cdot 0 = r \cdot \llbracket 0 \rrbracket = r \cdot \llbracket n \rrbracket =  \llbracket r \cdot n \rrbracket.
   \end{equation}
   Daraus folgt $r \cdot n \in N$.
\end{proof}

%******************************************************************************************************
\subsection{Projektion}
%******************************************************************************************************
Die Projektion
\begin{alignat}{6}
   &p \colon &&M &&\to &&M/N\\
   &         &&m &&\mapsto &&\llbracket m \rrbracket
\end{alignat}
ist ein $R$-Modul-Homomorphismus, was gezeigt werden muss. Dieser ist surjektiv, da jedes Element von $M/N$ von der Form $\llbracket m \rrbracket$ und damit Bild von $m$ ist.

%******************************************************************************************************
\subsection{Surjektiv}
%******************************************************************************************************
Sei $f \colon M \to L$ ein surjektiver $R$-Modul-Homomorphismus. Die Fasern von $f$, also die Urbilder  der $\{l\}$ für $l \in L$ sind Äquivalenz-Klassen. Zwei Elemente in $M$ sind äquivalent, wenn sie das selbe Bild in $L$ haben.
\begin{Satz}{Fasern einer Surjektion ergeben modul-verträgliche Äquivalenz-Relation}
   Mit den Bezeichnungen von oben erzeugen die Fasern eines surjektiven $R$-Modul-Homomorphismus eine modul-verträgliche Äquivalenz-Relation.
\end{Satz}
\begin{proof}
   Sei $x_1 \sim x_2$ also $f(x_1) = f(x_2)$ und $y \in M$ beliebig.
   \begin{alignat}{6}
      &f(y + x_1) &&= f(y) + f(x_1) &&= f(y) + f(x_2) ) &&= f(y + x_2)\\
      &f(-x_1)    &&= -f(x_1)       &&= -f(x_2)         &&= f(-x_2)\\
      &f(rx_1)    &&= rf(x_1)       &&= rf(x_2)         &&= f(rx_2)
   \end{alignat}
\end{proof}

Ähnlich wie bei Gruppen können wir zeigen
\begin{Satz}{Ziel eines surjektiven Morphismus ist isomorph zu Quotient}
   Der Quotient der durch die Fasern eines surjektiven $R$-Modul-Homomorphismus $f$ erzeugten Äquivalenz-Relation ist isomorph zum Ziel (Codomain) von $f$.
\end{Satz}

{\color{red}Damit haben wir (bis auf Isomorphie) eine 1:1-Beziehung zwischen Unter-Moduln, modul-kompatiblen Äquivalenz-Relationen und surjektiven $R$-Modul-Homomorphismen.}

%******************************************************************************************************
\subsection{Aufwickeln von Moduln}
%******************************************************************************************************
Sei $I$ ein echtes Ideal von $R$, also $0 \subsetneq I \subsetneq R$. Sei $M$ ein $R$-Modul. Wir betrachten den Unter-Modul $IM = \{im \mid i \in I \land m \in M\}$. Damit bilden wir den Quotienten $M/IM$.

...

Wenn $I$ Elemente aus $\Z$ enthält (genauer Elemente der Form $n\cdot 1$, wo $n \in \Z$ und $1 \in R$), sagen wir $z$, haben wir sogar $l + l + \cdots + l = 0$, mit $z$ mal das $l$ in der Summe. Und das auch dann, wenn dies in $M$ nicht der Fall war. Das empfinde ich als Aufwickeln eines Moduls.  




\begin{backup}
%******************************************************************************************************
%                                                                                                     *
\section{TODO}
%                                                                                                     *
%******************************************************************************************************
\begin{itemize}
     \item Überprüfe Symbolverzeichnis
\end{itemize}


\end{backup}

\begin{backup}
    (Zur Zeit nicht benötigter Inhalt)
\end{backup}

%******************************************************************************************************
%                                                                                                     *
\begin{thebibliography}{9}
%                                                                                                     *
%******************************************************************************************************
  \bibitem[Rotman2009]{Rotman}
   	Joseph J. Rotman, \emph{An Introduction to Homological Algebra},
   	2009 Springer-Verlag New York Inc., 978-0-387-24527-0 (ISBN)

   \bibitem[Bourbaki1970]{A1-3}
      Nicolas Bourbaki, \emph{Algébre 1-3},
      2006 Springer-Verlag, 978-3-540-33849-9 (ISBN)

   \bibitem[Bourbaki1980]{A10}
      Nicolas Bourbaki, \emph{Algébre 10. Algèbre homologique},
      2006 Springer-Verlag, 978-3-540-34492-6 (ISBN)

   \bibitem[MacLane1978]{MacLane}
      Saunders Mac Lane, \emph{Categories for the Working Mathematician},
      Springer-Verlag New York Inc., 978-0-387-98403-2 (ISBN)

\end{thebibliography}

%******************************************************************************************************
%                                                                                                     *
\begin{large}
    \centerline{\textsc{Symbolverzeichnis}}
\end{large}
%                                                                                                     *
%******************************************************************************************************
\bigskip

\renewcommand*{\arraystretch}{1}

\begin{tabular}{ll}
    $R$                                 & Ein kommutativer Ring mit Eins\\
    $G$                                 & Ein Generierendensystem\\
    $\asterisk$                         & Verknüpfung der Gruppe $G$\\
    $n\Z$                               & Das Ideal aller Vielfachen von $n$ in $\Z$\\
    $\zz{n}$                            & Der Restklassenring modulo $n$\\
    $\K$                                & Ein Körper\\
    $x, y$                    & Elemente eines Moduls, wenn wir sie von den Skalaren abheben wollen. \\
                                        & Entspricht in etwa Vektoren\\
    $r$                            & Element von $R^n$\\
    $\phi$                              & Gruppen-Homomorphismus\\
    $Z(R)$                              & Zentrum des Rings $R$\\
    $\Hom(X,Y)$                         & Menge/Gruppe der $R$-Homomorphismen von $X$ nach $Y$\\
    $IM$                                & $= \{ im \mid i \in I \land m\in M \}$\\
    $\langle G \rangle$                 & Der von den Elementen aus $G$ generierte Modul
    
\end{tabular}

\end{document}
