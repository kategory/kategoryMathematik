%******************************************************** -*-LaTeX-*- ******************************
%                                                                                                  *
% v5.1.1.1.5 R-Mod ist additiv.tex                                                                 *
%                                                                                                  *
% Copyright (C) 2025 Kategory GmbH \& Co. KG (joerg.kunze@kategory.de)                             *
%                                                                                                  *
% v5.1.1.1.5 R-Mod ist additiv is part of kategoryMathematik.                                    *
%                                                                                                  *
% kategoryMathematik is free software: you can redistribute it and/or modify                       *
% it under the terms of the GNU General Public License as published by                             *
% the Free Software Foundation, either version 3 of the License, or                                *
% (at your option) any later version.                                                              *
%                                                                                                  *
% kategoryMathematik is distributed in the hope that it will be useful,                            *
% but WITHOUT ANY WARRANTY; without even the implied warranty of                                   *
% MERCHANTABILITY or FITNESS FOR A PARTICULAR PURPOSE.  See the                                    *
% GNU General Public License for more details.                                                     *
%                                                                                                  *
% You should have received a copy of the GNU General Public License                                *
% along with this program.  If not, see <http://www.gnu.org/licenses/>.                            *
%                                                                                                  *
%***************************************************************************************************

\documentclass[a4paper]{amsart}
% \documentclass[a4paper]{book}

%-----------------------------------------------------------------------------------------------------*
% package:                                                                                            *
%-----------------------------------------------------------------------------------------------------*
\usepackage{amssymb}
\usepackage{amsfonts}
\usepackage{amsmath}
\usepackage{amsthm}

\usepackage{mathabx}

\usepackage{a4wide} % a little bit smaller margins

\usepackage{graphicx}
\usepackage{hyperref}
\usepackage{algorithmic}
\usepackage{listings}
\usepackage{color}
\usepackage{colortbl}
\usepackage{sidecap}
\usepackage{comment}
\usepackage{tcolorbox}
\usepackage{collect}

\usepackage{upgreek}

% \usepackage{diagrams}

\usepackage[german]{babel}
\usepackage[none]{hyphenat}
\emergencystretch=4em

\usepackage[utf8]{inputenc} % to be able to use äöü as characters in text
\usepackage[T1]{fontenc} % to be able to use äöü in lables
\usepackage{lmodern}     % to avoid pixelation introduced by fontenc

\usepackage{hyperref}

\usepackage{tikz}
\usepackage{tikz-cd}
\usetikzlibrary{babel}

\usepackage[only,llbracket,rrbracket]{stmaryrd}


%-----------------------------------------------------------------------------------------------------*
% theorem:                                                                                            *
%-----------------------------------------------------------------------------------------------------*
\theoremstyle{definition}
\newtheorem{theorem}{Theorem}[subsection]

\newcommand{\myTheorem}[1]{%
  \newtheorem{jk#1}[theorem]{#1}
  \newenvironment{#1}[1]{%
    \expandafter\begin{jk#1} \expandafter\label{#1:##1}\textbf{(##1):}
  }{%
    \expandafter\end{jk#1}
  }
}

\myTheorem{Definition}
\myTheorem{Proposition}
\myTheorem{Satz}
\myTheorem{Theorem}
\myTheorem{Axiom}
\myTheorem{Beispiel}
\myTheorem{Anmerkung}

\definecollection{jkjkFrage}
\newtheorem{jkFrage}[theorem]{Frage}
\newenvironment{Frage}[1]{%
  \expandafter\begin{jkFrage} \expandafter\label{Frage:#1}\textbf{(#1):}
  \begin{collect}{jkjkFrage}{}{}
    \item \ref{Frage:#1} #1
  \end{collect}
}{%
  \expandafter\end{jkFrage}
}

\newcommand{\myRef}[2]{[#1 \ref{#1:#2}, ``#2'']}

\renewcommand{\proofname}{Beweis}

%-----------------------------------------------------------------------------------------------------*
% operator:                                                                                           *
%-----------------------------------------------------------------------------------------------------*
\DeclareMathOperator{\End}{End}
\DeclareMathOperator{\Ker}{Ker}
\DeclareMathOperator{\Mat}{Mat}
\DeclareMathOperator{\rank}{rank}
\DeclareMathOperator{\ggT}{ggT}
\DeclareMathOperator{\len}{len}
\DeclareMathOperator{\ord}{ord}
\DeclareMathOperator{\kgV}{kgV}
\DeclareMathOperator{\id}{id}
\DeclareMathOperator{\red}{red}
\DeclareMathOperator{\supp}{supp}
\DeclareMathOperator{\Bild}{Bild}
\DeclareMathOperator{\Rang}{Rang}
\DeclareMathOperator{\Det}{Det}
\DeclareMathOperator{\Hom}{Hom}
\DeclareMathOperator{\GL}{GL}

\DeclareMathOperator{\sub}{sub}
\DeclareMathOperator{\blk}{blk}
\DeclareMathOperator{\minimal}{minimal}
\DeclareMathOperator{\maximal}{maximal}

\definecolor{mygreen}{rgb}{0,0.6,0}
\definecolor{mygray}{rgb}{0.5,0.5,0.5}
\definecolor{mymauve}{rgb}{0.58,0,0.82}

\lstset{ %
  backgroundcolor=\color{white},   % choose the background color
  basicstyle=\ttfamily\footnotesize,        % size of fonts used for the code
  breaklines=true,                 % automatic line breaking only at whitespace
  captionpos=b,                    % sets the caption-position to bottom
  commentstyle=\color{mygreen},    % comment style
  escapeinside={\%*}{*)},          % if you want to add LaTeX within your code
  keywordstyle=\color{blue},       % keyword style
  stringstyle=\color{mymauve},     % string literal style
  frame=single
}

\setcounter{MaxMatrixCols}{20}

%******************************************************************************************************
%                                                                                                     *
% definition:                                                                                         *
%                                                                                                     *
%******************************************************************************************************
\newcommand{\R}{\ensuremath{\mathbb{ R }}}
\newcommand{\Q}{\ensuremath{\mathbb{ Q }}}
\newcommand{\Z}{\ensuremath{\mathbb{ Z }}}
\newcommand{\N}{\ensuremath{\mathbb{ N }}}
\newcommand{\C}{\ensuremath{\mathbb{ C }}}
\newcommand{\A}{\ensuremath{\mathbb{ A }}}
\newcommand{\F}{\ensuremath{\mathbb{ F }}}
\newcommand{\K}{\ensuremath{\mathbb{ K }}}
\newcommand{\Pb}{\ensuremath{\mathbb{ P }}}

\newcommand{\M}{\ensuremath{\mathcal{ M }}}
\newcommand{\V}{\ensuremath{\mathcal{ V }}}

\newcommand{\AAA}{\ensuremath{\mathcal{ A }}}
\newcommand{\BB}{\ensuremath{\mathcal{ B }}}
\newcommand{\CC}{\ensuremath{\mathcal{ C }}}
\newcommand{\DD}{\ensuremath{\mathcal{ D }}}
\newcommand{\EE}{\ensuremath{\mathcal{ E }}}
\newcommand{\FF}{\ensuremath{\mathcal{ F }}}
\newcommand{\KK}{\ensuremath{\mathcal{ K }}}
\newcommand{\MM}{\ensuremath{\mathcal{ M }}}
\newcommand{\PP}{\ensuremath{\mathcal{ P }}}
\newcommand{\ZZ}{\ensuremath{\mathcal{ Z }}}

\newcommand{\Set}{\text{\textbf{Set}} }


\newcommand{\imporant}[1]{ \textcolor{red}{\textbf{#1}} }

\newcommand{\bb}[1]{\mathbf{#1}}
\newcommand{\balpha}{\boldsymbol{\upalpha}}
\newcommand{\bbeta}{\boldsymbol{\upbeta}}
\newcommand{\bgamma}{\boldsymbol{\upgamma}}
\newcommand{\bdelta}{\boldsymbol{\delta}}
\newcommand{\bmu}{\boldsymbol{\upmu}}

\newcommand{\z}[1]{\Z_{#1}}
\newcommand{\e}[1]{\z{#1}^*}
\newcommand{\q}[1]{(\e{#1})^2}
\newcommand{\m}{\mathcal}

\newcommand{\zz}[1]{\ensuremath{\Z /#1\Z}}

\excludecomment{book}
\excludecomment{example}
\excludecomment{backup}

\newcommand{\zb}{z.~B. }

\begin{document}

%******************************************************************************************************
%                                                                                                     *
\begin{titlepage}
%                                                                                                     *
%******************************************************************************************************
% \vspace*{\fill}
\centering
{\huge
(Höhere Grundlagen) Homologische Algebra\\[1cm]
\textbf{v5.1.1.1.5 R-Mod ist additiv}
}\\[1cm]

\textbf{Kategory GmbH \& Co. KG}\\
Präsentiert von Jörg Kunze\\
Copyright (C) 2024 Kategory GmbH \& Co. KG

\end{titlepage}

%\clearpage
%\setcounter{page}{2}
%
%\tableofcontents

\newpage

%******************************************************************************************************
%                                                                                                     *
\section*{Beschreibung}
%                                                                                                     *
%******************************************************************************************************

\subsection*{Inhalt}
Additive Kategorien sind definiert als \textbf{Ab}-Kategorien, die ein Null-Objekt und endliche Produkte und Koprodukte besitzen.

Das $R$-Mod eine \textbf{Ab}-Kategorie ist, haben wir schon gezeigt.

Hier zeigen wir, dass der Null-Modul $\{0\}$ ein Null-Objekt ist, und dass die direkte Summe, also das kartesische Produkt mit komponentenweiser Modul-Struktur, Produkt und Koprodukt ist.

In additiven Kategorien ist ein Produkt von zwei Objekten immer auch Koprodukt und umgekehrt. In diesen Kategorien ist dies darüber hinaus äquivalent zu Biprodukten.

\subsection*{Präsentiert}
Von Jörg Kunze

\subsection*{Voraussetzungen}
Ringe, Moduln, Kategorien, \textbf{Ab}-Kategorie, Produkt, Koprodukt, Null-Objekt.

\subsection*{Text}
Der Begleittext als PDF und als LaTeX findet sich unter
{\tiny
   \url{https://github.com/kategory/kategoryMathematik/tree/main/v5%20H%C3%B6here%20Grundlagen/v5.1%20Homologische%20Algebra/v5.1.1.1.5%20R-Mod%20ist%20additiv}
}

\subsection*{Meine Videos}
Siehe auch in den folgenden Videos:\\
\\
v5.1.1.1.3 (Höher) Homologische Algebra - Additiver Funktor Hom\\
\url{https://youtu.be/Xog_6hrbmx0}\\
\\
v5.0.1.0.5 (Höher) Kategorien - Mono Epi Null\\
\url{https://youtu.be/n4-qZJK_sH0}
\\
v5.1.1.1 (Höher) Homologische Algebra - Moduln\\
\url{https://youtu.be/JY43_07kNmA}\\

\subsection*{Quellen}
Siehe auch in den folgenden Seiten:\\
\url{https://ncatlab.org/nlab/show/additive+category}\\
\url{https://en.wikipedia.org/wiki/Additive_category}\\
\url{https://de.wikipedia.org/wiki/Produkt_und_Koprodukt}\\
\url{https://en.wikipedia.org/wiki/Biproduct}\\
\url{https://ncatlab.org/nlab/show/biproduct}\\
\url{https://de.wikipedia.org/wiki/Anfangsobjekt,_Endobjekt_und_Nullobjekt}

\subsection*{Buch}
Grundlage ist folgendes Buch:\\
"`An Introduction to Homological Algebra"'\\
Joseph J. Rotman\\
2009\\
Springer-Verlag New York Inc.\\
978-0-387-24527-0 (ISBN)\\
{\tiny
   \url{https://www.lehmanns.de/shop/mathematik-informatik/6439666-9780387245270-an-introduction-to-homological-algebra}}\\
\\
Oft zitiert:\\
"`An Introduction to Homological Algebra"'\\
Charles A. Weibel\\
1995\\
Cambridge University Press\\
978-0-521-55987-4 (ISBN)\\
{\tiny
   \url{https://www.lehmanns.de/shop/mathematik-informatik/510327-9780521559874-an-introduction-to-homological-algebra}}\\
\\
Ohne Kategorien-Theorie:\\
"`Algébre 10. Algèbre homologique"'\\
Nicolas Bourbaki\\
1980\\
Springer-Verlag \\
978-3-540-34492-6 (ISBN)\\
{\tiny
   \url{https://www.lehmanns.de/shop/mathematik-informatik/7416782-9783540344926-algebre}}

\subsection*{Lizenz}
Dieser Text und das Video sind freie Software. Sie können es unter den Bedingungen der
GNU General Public License, wie von der Free Software Foundation veröffentlicht, weitergeben
und/oder modifizieren, entweder gemäß Version 3 der Lizenz oder (nach Ihrer Option) jeder späteren Version.

Die Veröffentlichung von Text und Video erfolgt in der Hoffnung, dass es Ihnen von Nutzen sein wird,
aber OHNE IRGENDEINE GARANTIE, sogar ohne die implizite Garantie der MARKTREIFE oder der
VERWENDBARKEIT FÜR EINEN BESTIMMTEN ZWECK. Details finden Sie in der GNU General Public License.

Sie sollten ein Exemplar der GNU General Public License zusammen mit diesem Text erhalten haben
(zu finden im selben Git-Projekt).
Falls nicht, siehe \url{http://www.gnu.org/licenses/}.

%******************************************************************************************************
\subsection*{Das Video}
%******************************************************************************************************
Das Video hierzu ist zu finden unter
{\tiny
   \url{uups}
}

%******************************************************************************************************
%                                                                                                     *
\section{v5.1.1.1.5 R-Mod ist additiv}
%                                                                                                     *
%******************************************************************************************************

%******************************************************************************************************
\subsection{Axiome}
%******************************************************************************************************
Sei $\CC$ eine Kategorie. Dann benennen wir folgende Liste von 6 Axiomen (AK steht für Ab, additive, oder abelsche Kategorie):
\begin{itemize}
   \item \textbf{AK1:} $\Hom(X,Y)$ ist ablesche Gruppe 
   \item \textbf{AK2:} $(f + f')g = fg + f'g$ und $g(f + f') = gf + gf'$
   \item \textbf{AK3:} $\exists$ Null-Objekt
   \item \textbf{AK4:} $\exists$ endliche Ko/Produkte
   \item \textbf{AK5:} Jeder Hom hat Ko/Kern
   \item \textbf{AK6:} $\CC$ ist ko/normal (mono=Kern,epi=Kokern)
\end{itemize}

Eine Kategorie nennen wir \textbf{Ab}-Kategorie, wenn sie AK1 und AK2 erfüllt. Wir nennen sie \textbf{additiv}, wenn sie zusätzlich AK3 und AK4 erfüllt. Schließlich nennen wir sie \textbf{abelsch}, wenn sie alle 6 Axiome erfüllt.

Wir wissen bereits, dass $R$-Mod eine \textbf{Ab}-Kategorie ist (siehe v5.1.1.1.3 (Höher) Homologische Algebra - Additiver Funktor Hom).

%******************************************************************************************************
\subsection{$R$-Mod ist additiv}
%******************************************************************************************************
AK3: Sei $\{0\}$ eine abelsche Gruppe mit genau einem Element. Dieses Element ist das neutrale Element. Alle Gruppen mit genau einem Element sind isomorph. Wir statten diese Gruppe nun mit einer Skalarmultiplikation mit Elementen aus $R$ aus mit $r \cdot 0 := 0$. Zu dieser Festlegung werden wir durch die Axiome für $R$-Moduln gezwungen. Es ist leicht zu zeigen, dass $\{0\}$ dadurch zu einem $R$-Modul wird.

Ein Null-Objekt ist gleichzeitig Anfangs- wie End-Objekt (initial und terminal). D.~h. 
\begin{alignat}{7}
   &\forall X &&\in \CC &&\quad \exists! p &&\colon X &&\to 0\\
   &\forall Y &&\in \CC &&\quad \exists! i &&\colon 0 &&\to Y
\end{alignat}
Wenn wir die Null-Morphismen so bezeichnen,
\begin{alignat}{7}
   &0_{X,Y} \colon &&X &&\to     &&\, Y\\
   &               &&x &&\mapsto &&\, 0,
\end{alignat}
dann sind $p := 0_{X,0}$ und $i := 0_{0,Y}$ die gesuchten einzigen Homomorphismen.

AK4: Wir definieren
\begin{equation}\label{DirekteSumme}
   X \oplus Y := \{(x,y) \mid x \in X \land y \in Y\}.
\end{equation}
Die $R$-Modul-Struktur ergibt sich durch komponentenweise Festlegung:
\begin{alignat}{7}
   &(x_1,y_1) + (x_2,y_2) &&:= (x_1 + x_2, y_1 + y_2)\\
   &r(x,y) &&:= (rx,ry)\\
   &0 &&:= (0, 0)
\end{alignat}
Die Modul-Axiome sind leicht zu sehen.

%******************************************************************************************************
\subsection{Produkt}
%******************************************************************************************************
Zu $X \oplus Y$ haben wir die \textbf{Projektionen}
\begin{alignat}{3}
   &p_X(x,y) &&:= x\\
   &p_Y(x,y) &&:= y.
\end{alignat}
$X \oplus Y, p_X, p_Y$ ist ein Produkt.
\begin{proof}
   Seien $f_X \colon W \to X$ und $f_Y \colon W \to Y$, zwei $R$-Modul-Homomorphismen. 
   Ein $h \colon W \to X \oplus  Y$ besteht aus zwei Morphismen $h(w) = (h_X(w), h_Y(w))$. Damit
   $f_X = p_X \circ h$ und $f_Y = p_Y \circ h$ gilt, muss, wegen $p_X \circ h = h_X$ und $p_Y \circ h = h_Y$, gelten, dass $h_X = f_X$ und $h_Y = f_Y$. Damit ist Existenz und Eindeutigkeit des gesuchten $h$ gezeigt.
   Also
   \begin{alignat}{3}
      &h \colon &&W \to &&X \oplus Y\\
      &         &&w \mapsto && (f_X(w), f_Y(w))
   \end{alignat}
\end{proof}

%******************************************************************************************************
\subsection{Koprodukt}
%******************************************************************************************************
Zu $X \oplus Y$ haben wir die \textbf{Injektionen}
\begin{alignat}{3}
   &i_X(x) &&:= (x, 0)\\
   &i_Y(y) &&:= (0, y).
\end{alignat}
$X \oplus Y, i_X, i_Y$ ist ein Koprodukt.
\begin{proof}
   Hier müssen wir zu zwei $g_X \colon X \to Z$ und $g_Y \colon Y \to Z$ genau ein $h \colon X \oplus  Y \to Z$ finden mit $g_X = h \circ i_X$ und $g_Y = h \circ i_Y$. Dies wird durch
   \begin{alignat}{3}
      &h \colon &&X \oplus Y &&\to Z\\
      &         &&(x,y) \mapsto && g_X(x) + g_Y(y)
   \end{alignat}
   bewerkstelligt. Die Nachweise, dass es sich um einen $R$-Mod-Hom handelt, dass die Gleichungen erfüllt werden und dass es eindeutig ist, müssen erbracht werden.
\end{proof}

{\color{red} Hier wird die additive Struktur der $R$-Moduln benötigt. Dies ist bei dem Produkt nicht nötig.}

%******************************************************************************************************
\subsection{Additiv}
%******************************************************************************************************
Damit haben wir insgesamt gezeigt, dass die Kategorie der $R$-Moduln die Axiome AK1, AK2, AK3 und AK4 erfüllt. \textbf{$R$-Mod ist additiv}.

%******************************************************************************************************
\subsection{Biprodukt}
%******************************************************************************************************
Dass $X \oplus Y$ sowohl im Produkt als auch im Koprodukt vorkommt ist kein Zufall, denn in allen additiven Kategorien gilt, dass jedes Produktobjekt auch Koproduktobjekt und umgekehrt ist. Produkt und Koprodukt zusammen erfüllen in allen additiven Kategorien darüber hinaus die Biprodukt-Gleichungen:
\begin{alignat}{3}
   &p_X \circ i_X &&= \id_X\\
   &p_Y \circ i_Y &&= \id_Y\\
   &i_X \circ p_X + i_Y \circ p_Y &&= \id_{X \oplus Y}.
\end{alignat}
Den Beweis finden wir \zb in \cite{MacLane} im Kapitel "`Abelian Categories"' und dort im Abschnitt "`2. Additive Categories"'.

Wir können das hier aber auch für den Fall von $R$-Mod direkt zeigen. Z.~B. für die 3. Gleichung:
\begin{alignat}{3}
   &(i_X \circ p_X &&+ i_Y \circ p_Y)(x,y) &&=\\ 
   &(i_X \circ p_X)(x,y) &&+ (i_Y \circ p_Y)(x,y) &&=\\ 
   &i_X( p_X(x,y) ) &&+ i_Y( p_Y(x,y)) &&=\\ 
   &i_X(x)  &&+ i_Y(y) &&=\\ 
   &( x,0)  &&+ ( 0,y) &&=\\ 
   &(x,y)
\end{alignat}

\begin{backup}
%******************************************************************************************************
%                                                                                                     *
\section{TODO}
%                                                                                                     *
%******************************************************************************************************
\begin{itemize}
     \item Überprüfe Symbolverzeichnis
\end{itemize}


\end{backup}

\begin{backup}
    (Zur Zeit nicht benötigter Inhalt)
\end{backup}

%******************************************************************************************************
%                                                                                                     *
\begin{thebibliography}{9}
%                                                                                                     *
%******************************************************************************************************
  \bibitem[Rotman2009]{Rotman}
   	Joseph J. Rotman, \emph{An Introduction to Homological Algebra},
   	2009 Springer-Verlag New York Inc., 978-0-387-24527-0 (ISBN)

   \bibitem[Bourbaki1970]{A1-3}
      Nicolas Bourbaki, \emph{Algébre 1-3},
      2006 Springer-Verlag, 978-3-540-33849-9 (ISBN)

   \bibitem[Bourbaki1980]{A10}
      Nicolas Bourbaki, \emph{Algébre 10. Algèbre homologique},
      2006 Springer-Verlag, 978-3-540-34492-6 (ISBN)

   \bibitem[MacLane1978]{MacLane}
      Saunders Mac Lane, \emph{Categories for the Working Mathematician},
      Springer-Verlag New York Inc., 978-0-387-98403-2 (ISBN)

\end{thebibliography}

%******************************************************************************************************
%                                                                                                     *
\begin{large}
    \centerline{\textsc{Symbolverzeichnis}}
\end{large}
%                                                                                                     *
%******************************************************************************************************
\bigskip

\renewcommand*{\arraystretch}{1}

\begin{tabular}{ll}
    $R$                                 & Ein kommutativer Ring mit Eins\\
    $G$                                 & Ein Generierendensystem\\
    $\asterisk$                         & Verknüpfung der Gruppe $G$\\
    $n\Z$                               & Das Ideal aller Vielfachen von $n$ in $\Z$\\
    $\zz{n}$                            & Der Restklassenring modulo $n$\\
    $\K$                                & Ein Körper\\
    $x, y$                    & Elemente eines Moduls, wenn wir sie von den Skalaren abheben wollen. \\
                                        & Entspricht in etwa Vektoren\\
    $r$                            & Element von $R^n$\\
    $\phi$                              & Gruppen-Homomorphismus\\
    $Z(R)$                              & Zentrum des Rings $R$\\
    $\Hom(X,Y)$                         & Menge/Gruppe der $R$-Homomorphismen von $X$ nach $Y$\\
    $IM$                                & $= \{ im \mid i \in I \land m\in M \}$\\
    $\langle G \rangle$                 & Der von den Elementen aus $G$ generierte Modul
    
\end{tabular}

\end{document}
