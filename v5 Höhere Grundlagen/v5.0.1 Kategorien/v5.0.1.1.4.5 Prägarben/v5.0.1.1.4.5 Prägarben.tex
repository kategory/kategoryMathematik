%******************************************************** -*-LaTeX-*- ******************************
%                                                                                                  *
% v5.0.1.1.4.5 Prägarben.tex                                                                       *
%                                                                                                  *
% Copyright (C) 2024 Kategory GmbH \& Co. KG (joerg.kunze@kategory.de)                             *
%                                                                                                  *
% v5.0.1.1.4.5 Prägarben is part of kategoryMathematik.                                            *
%                                                                                                  *
% kategoryMathematik is free software: you can redistribute it and/or modify                       *
% it under the terms of the GNU General Public License as published by                             *
% the Free Software Foundation, either version 3 of the License, or                                *
% (at your option) any later version.                                                              *
%                                                                                                  *
% kategoryMathematik is distributed in the hope that it will be useful,                            *
% but WITHOUT ANY WARRANTY; without even the implied warranty of                                   *
% MERCHANTABILITY or FITNESS FOR A PARTICULAR PURPOSE.  See the                                    *
% GNU General Public License for more details.                                                     *
%                                                                                                  *
% You should have received a copy of the GNU General Public License                                *
% along with this program.  If not, see <http://www.gnu.org/licenses/>.                            *
%                                                                                                  *
%***************************************************************************************************

\documentclass[a4paper]{amsart}
% \documentclass[a4paper]{book}

%-----------------------------------------------------------------------------------------------------*
% package:                                                                                            *
%-----------------------------------------------------------------------------------------------------*
\usepackage{amssymb}
\usepackage{amsfonts}
\usepackage{amsmath}
\usepackage{amsthm}

\usepackage{mathabx}

\usepackage{a4wide} % a little bit smaller margins

\usepackage{graphicx}
\usepackage{hyperref}
\usepackage{algorithmic}
\usepackage{listings}
\usepackage{color}
\usepackage{colortbl}
\usepackage{sidecap}
\usepackage{comment}
\usepackage{tcolorbox}
\usepackage{collect}

\usepackage{upgreek}

% \usepackage{diagrams}

\usepackage[german]{babel}
\usepackage[none]{hyphenat}
\emergencystretch=4em

\usepackage[utf8]{inputenc} % to be able to use äöü as characters in text
\usepackage[T1]{fontenc} % to be able to use äöü in lables
\usepackage{lmodern}     % to avoid pixelation introduced by fontenc

\usepackage{hyperref}

\usepackage{tikz}
\usepackage{tikz-cd}
\usetikzlibrary{babel}

%-----------------------------------------------------------------------------------------------------*
% theorem:                                                                                            *
%-----------------------------------------------------------------------------------------------------*
\theoremstyle{definition}
\newtheorem{theorem}{Theorem}[subsection]

\newcommand{\myTheorem}[1]{%
  \newtheorem{jk#1}[theorem]{#1}
  \newenvironment{#1}[1]{%
    \expandafter\begin{jk#1} \expandafter\label{#1:##1}\textbf{(##1):}
  }{%
    \expandafter\end{jk#1}
  }
}

\myTheorem{Definition}
\myTheorem{Proposition}
\myTheorem{Satz}
\myTheorem{Theorem}
\myTheorem{Axiom}
\myTheorem{Beispiel}
\myTheorem{Anmerkung}

\definecollection{jkjkFrage}
\newtheorem{jkFrage}[theorem]{Frage}
\newenvironment{Frage}[1]{%
  \expandafter\begin{jkFrage} \expandafter\label{Frage:#1}\textbf{(#1):}
  \begin{collect}{jkjkFrage}{}{}
    \item \ref{Frage:#1} #1
  \end{collect}
}{%
  \expandafter\end{jkFrage}
}

\newcommand{\myRef}[2]{[#1 \ref{#1:#2}, ``#2'']}

\renewcommand{\proofname}{Beweis}

%-----------------------------------------------------------------------------------------------------*
% operator:                                                                                           *
%-----------------------------------------------------------------------------------------------------*
\DeclareMathOperator{\End}{End}
\DeclareMathOperator{\Ker}{Ker}
\DeclareMathOperator{\Mat}{Mat}
\DeclareMathOperator{\rank}{rank}
\DeclareMathOperator{\ggT}{ggT}
\DeclareMathOperator{\len}{len}
\DeclareMathOperator{\ord}{ord}
\DeclareMathOperator{\kgV}{kgV}
\DeclareMathOperator{\id}{id}
\DeclareMathOperator{\red}{red}
\DeclareMathOperator{\supp}{supp}
\DeclareMathOperator{\Bild}{Bild}
\DeclareMathOperator{\Rang}{Rang}
\DeclareMathOperator{\Det}{Det}
\DeclareMathOperator{\Hom}{Hom}
\DeclareMathOperator{\GL}{GL}

\DeclareMathOperator{\sub}{sub}
\DeclareMathOperator{\blk}{blk}
\DeclareMathOperator{\minimal}{minimal}
\DeclareMathOperator{\maximal}{maximal}

\DeclareMathOperator{\Dom}{Dom}
\DeclareMathOperator{\Cod}{Cod}
\DeclareMathOperator{\Obj}{Obj}

\definecolor{mygreen}{rgb}{0,0.6,0}
\definecolor{mygray}{rgb}{0.5,0.5,0.5}
\definecolor{mymauve}{rgb}{0.58,0,0.82}

\lstset{ %
  backgroundcolor=\color{white},   % choose the background color
  basicstyle=\ttfamily\footnotesize,        % size of fonts used for the code
  breaklines=true,                 % automatic line breaking only at whitespace
  captionpos=b,                    % sets the caption-position to bottom
  commentstyle=\color{mygreen},    % comment style
  escapeinside={\%*}{*)},          % if you want to add LaTeX within your code
  keywordstyle=\color{blue},       % keyword style
  stringstyle=\color{mymauve},     % string literal style
  frame=single
}

\setcounter{MaxMatrixCols}{20}

%******************************************************************************************************
%                                                                                                     *
% definition:                                                                                         *
%                                                                                                     *
%******************************************************************************************************
\newcommand{\R}{\ensuremath{\mathbb{ R }}}
\newcommand{\Q}{\ensuremath{\mathbb{ Q }}}
\newcommand{\Z}{\ensuremath{\mathbb{ Z }}}
\newcommand{\N}{\ensuremath{\mathbb{ N }}}
\newcommand{\C}{\ensuremath{\mathbb{ C }}}
\newcommand{\A}{\ensuremath{\mathbb{ A }}}
\newcommand{\F}{\ensuremath{\mathbb{ F }}}
\newcommand{\K}{\ensuremath{\mathbb{ K }}}
\newcommand{\Pb}{\ensuremath{\mathbb{ P }}}

\newcommand{\M}{\ensuremath{\mathcal{ M }}}
\newcommand{\V}{\ensuremath{\mathcal{ V }}}

\newcommand{\AAA}{\ensuremath{\mathcal{ A }}}
\newcommand{\BB}{\ensuremath{\mathcal{ B }}}
\newcommand{\CC}{\ensuremath{\mathcal{ C }}}
\newcommand{\DD}{\ensuremath{\mathcal{ D }}}
\newcommand{\EE}{\ensuremath{\mathcal{ E }}}
\newcommand{\FF}{\ensuremath{\mathcal{ F }}}
\newcommand{\KK}{\ensuremath{\mathcal{ K }}}
\newcommand{\MM}{\ensuremath{\mathcal{ M }}}
\newcommand{\PP}{\ensuremath{\mathcal{ P }}}
\newcommand{\ZZ}{\ensuremath{\mathcal{ Z }}}

\newcommand{\Set}{\text{\textbf{Set}}}
\newcommand{\Ab}{\text{\textbf{Ab}}}

\newcommand{\imporant}[1]{ \textcolor{red}{\textbf{#1}} }

\newcommand{\bb}[1]{\mathbf{#1}}
\newcommand{\balpha}{\boldsymbol{\upalpha}}
\newcommand{\bbeta}{\boldsymbol{\upbeta}}
\newcommand{\bgamma}{\boldsymbol{\upgamma}}
\newcommand{\bdelta}{\boldsymbol{\delta}}
\newcommand{\bmu}{\boldsymbol{\upmu}}

\newcommand{\z}[1]{\Z_{#1}}
\newcommand{\e}[1]{\z{#1}^*}
\newcommand{\q}[1]{(\e{#1})^2}
\newcommand{\m}{\mathcal}

\excludecomment{book}
\excludecomment{example}
\excludecomment{backup}

\newcommand{\zb}{z.~B. }

\begin{document}

%******************************************************************************************************
%                                                                                                     *
\begin{titlepage}
%                                                                                                     *
%******************************************************************************************************
% \vspace*{\fill}
\centering
{\huge
(Höhere Grundlagen) Kategorien\\[1cm]
\textbf{v5.0.1.1.4.5 Prägarben}
}\\[1cm]

\textbf{Kategory GmbH \& Co. KG}\\
Präsentiert von Jörg Kunze\\
Copyright (C) 2024 Kategory GmbH \& Co. KG

\end{titlepage}

%\clearpage
%\setcounter{page}{2}
%
%\tableofcontents

\newpage

%******************************************************************************************************
%                                                                                                     *
\section*{Beschreibung}
%                                                                                                     *
%******************************************************************************************************

%******************************************************************************************************
\subsection*{Inhalt}
%******************************************************************************************************
Eine Prägarbe ist eine kontravariante Funktor-Kategorie nach \textbf{Set}, also ein $\Set^{\CC^*}$ für eine kleine Kategorie $\CC$, die Kategorie der kontravarianten Funktoren von $\CC$ nach $\Set$, oder der Funktoren von der dualen Kategorie $\CC^*$ nach $\Set$.
\\\\
Tauchen ganz natürlich immer und überall auf: \zb Die Prägarbe der stetigen Funktionen.
\\\\
Viele Kategorien entpuppen sich als Prägarbe, \zb die Kategorie der Graphen.
\\\\
Prägarben sind Verallgemeinerungen von $\Set$. $\Set$ ist die Prägarbe $\Set^\bullet$, wobei $\bullet$ die Kategorie mit nur einem Objekt und nur einem Morphismus ist.

%******************************************************************************************************
\subsection*{Präsentiert}
%******************************************************************************************************
Von Jörg Kunze

%******************************************************************************************************
\subsection*{Voraussetzungen}
%******************************************************************************************************
Kategorie, Funktor, Natürliche Transformation, Funktor-Kategorie

%******************************************************************************************************
\subsection*{Text}
%******************************************************************************************************
Der Begleittext als PDF und als LaTeX findet sich unter
{\tiny
   \url{}
}

%******************************************************************************************************
\subsection*{Meine Videos}
%******************************************************************************************************
Siehe auch in den folgenden Videos:\\
\\
v5.0.1.0.3 (Höher) Kategorien - Funktoren\\
\url{https://youtu.be/Ojf5LQGeyOU}\\
\\
v5.0.1.0.3.5 (Höher) Kategorien - Kategorien von Homomorphismen\\
\url{https://youtu.be/v1F5BFH8nbo}\\
\\
v5.0.1.0.4 (Höher) Kategorien - Natürliche Transformationen\\
\url{https://youtu.be/IN7Qa-SwlD0}\\
\\
vv5.0.1.1.4 (Höher) Kategorien - Funktorkategorien\\
\url{https://youtu.be/byvbGxRz1hs}\\
\\

%******************************************************************************************************
\subsection*{Quellen}
%******************************************************************************************************
Siehe auch in den folgenden Seiten:\\
\url{https://en.wikipedia.org/wiki/Presheaf_(category_theory)}\\
\url{https://ncatlab.org/nlab/show/presheaf}\\
\url{https://mathoverflow.net/questions/302934/why-sheaves-are-important-and-why-do-we-care-about-them}\\
\url{https://de.wikipedia.org/wiki/Garbe_(Mathematik)}\\
\url{https://de.wikipedia.org/wiki/Garbe_(Landwirtschaft)}

%******************************************************************************************************
\subsection*{Buch}
%******************************************************************************************************
Grundlage ist folgendes Buch:\\
"`Categories for the Working Mathematician"'\\
Saunders Mac Lane\\
1998 | 2nd ed. 1978\\
Springer-Verlag New York Inc.\\
978-0-387-98403-2 (ISBN)\\
{\tiny
   \url{https://www.amazon.de/Categories-Working-Mathematician-Graduate-Mathematics/dp/0387984038}}\\

Gut für die kategorische Sichtweise ist:\\
"`Topology, A Categorical Approach"'\\
Tai-Danae Bradley\\
2020 MIT Press\\
978-0-262-53935-7 (ISBN)\\
{\tiny
\url{https://www.lehmanns.de/shop/mathematik-informatik/52489766-9780262539357-topology}}\\

Einige gut Erklärungen finden sich auch in den Einführenden Kapitel von:\\
"`An Introduction to Homological Algebra"'\\
Joseph J. Rotman\\
2009 Springer-Verlag New York Inc.\\
978-0-387-24527-0 (ISBN)\\
{\tiny \url{https://www.lehmanns.de/shop/mathematik-informatik/6439666-9780387245270-an-introduction-to-homological-algebra}}\\

Etwas weniger umfangreich und weniger tiefgehend aber gut motivierend ist:
"`Category Theory"'\\
Steve Awodey\\
2010 Oxford University Press\\
978-0-19-923718-0 (ISBN)\\
{\tiny\url{https://www.lehmanns.de/shop/mathematik-informatik/9478288-9780199237180-category-theory}}\\

Mit noch weniger Mathematik und die Konzepte motivierend ist:
"`Conceptual Mathematics: a First Introduction to Categories"'\\
F. William Lawvere, Stephen H. Schanuel\\
2009 Cambridge University Press\\
978-0-521-71916-2 (ISBN)\\
{\tiny\url{https://www.lehmanns.de/shop/mathematik-informatik/8643555-9780521719162-conceptual-mathematics}}

%******************************************************************************************************
\subsection*{Lizenz}
%******************************************************************************************************
Dieser Text und das Video sind freie Software. Sie können es unter den Bedingungen der
GNU General Public License, wie von der Free Software Foundation veröffentlicht, weitergeben
und/oder modifizieren, entweder gemäß Version 3 der Lizenz oder (nach Ihrer Option) jeder späteren Version.

Die Veröffentlichung von Text und Video erfolgt in der Hoffnung, dass es Ihnen von Nutzen sein wird,
aber OHNE IRGENDEINE GARANTIE, sogar ohne die implizite Garantie der MARKTREIFE oder der
VERWENDBARKEIT FÜR EINEN BESTIMMTEN ZWECK. Details finden Sie in der GNU General Public License.

Sie sollten ein Exemplar der GNU General Public License zusammen mit diesem Text erhalten haben
(zu finden im selben Git-Projekt).
Falls nicht, siehe \url{http://www.gnu.org/licenses/}.

\subsection*{Das Video}
%******************************************************************************************************
Das Video hierzu ist zu finden unter
{\tiny
   \url{XXX}
}

%******************************************************************************************************
%                                                                                                     *
\section{v5.0.1.1.4.5 Prägarben}
%                                                                                                     *
%******************************************************************************************************

%******************************************************************************************************
\subsection{Definition Prägarbe}
%******************************************************************************************************
Eine Prägarbe ist ein kontravarianter Funktor von einer kleinen Kategorie $\CC$ nach $\Set$. Diese können wir auch als Diagramm der Form $\CC$ in $\Set$ mit umgedrehten Pfeilen auffassen.


Eine Kategorie von Prägarben ist eine kontravariante Funktor-Kategorie nach \textbf{Set}, also ein $\Set^{\CC^*}$ für eine kleine Kategorie $\CC$, die Kategorie der Prägarben über $\CC$, der kontravarianten Funktoren von $\CC$ nach $\Set$, oder der Funktoren von der dualen Kategorie $\CC^*$ nach $\Set$.

Wenn wir $\CC$ als Graph oder Form ansehen, dann ist $\Set^{\CC^*}$ die Kategorie aller Diagramme in $\Set$ von der Form $\CC^*$, also von der Form $\CC$ aber mit umgedrehten Pfeilen.

Sei \zb $\CC$:

\begin{equation}
   \begin{tikzcd}
      & \bullet\\
      \bullet \arrow[ur]  \arrow[dr] \\
      & \bullet
   \end{tikzcd}
\end{equation}
Dann sind die Objekte von $\Set^{\CC^*}$ alle Figuren über $\Set$
\begin{equation}
   \begin{tikzcd}
      A \arrow[dr]\\
      &B    \\
      C \arrow[ur]
   \end{tikzcd}
\end{equation}
mit Mengen $A,B,C$.

%******************************************************************************************************
\subsection{Kontravariante Funktoren}
%******************************************************************************************************
Wir wollen eine sehr einfache Konzeptstruktur rund um das Konzept Auto aufbauen. Konzepte sind hierbei Objekte einer Kategorie, der Kategorie der verwendeten Konzepte. Ein Morphismus $f \colon A \to B$ von einem Konzept $A$ zu einem Konzept $B$ bedeutet, dass das Konzept $A$ in der Rolle $f$ in $B$ vorkommt. Wir betrachten stark vereinfachend nur die Konzepte Auto $A$, Rad $R$, Felge $F$ und Halter $H$. Die Modellierung kann dann so aussehen:  
\begin{equation}
   \begin{tikzcd}
      F \arrow[r] & R \arrow[dr,"vl{,}vr{,}hl{,}hr"] \\
      && A\\
      & H \arrow[ur]
   \end{tikzcd}
\end{equation}
Dabei sind $vl{,}vr{,}hl{,}hr$ 4 Morphismen (ich weiß nur nicht, wie man 4 parallele Pfeile malt), welche die Rollen "`Rad vorne links"', "`Rad vorne rechts"', "`Rad hinten links"' und "`Rad hinten rechts"' modellieren. Diese Kategorie nennen wir $\CC$.

Ein Fuhrpark ist nun eine Prägarbe über $\CC$, also ein kontravarianter Funktor $\FF \colon \CC^* \to \Set$. Zu jedem Konzept gibt es eine Menge von Dingen zu diesem Konzept: die Menge der Felgen $\FF(F)$, die Menge der Räder $\FF(R)$, die Menge der Autos $\FF(A)$ und die Menge der Halter $\FF(H)$.
Die Abbildung $\FF(A) \to \FF(H)$ ordnet jedem Auto seinem Halter zu, die Abbildung $\FF(vl) \colon \FF(A) \to \FF(R)$ ordnet jedem Auto sein vorderes linkes Rad zu. Beachte, dass diese Pfeile in umgekehrter Richtung verlaufen.

Die Funktion $\FF(vl) \colon \FF(A) \to \FF(R)$ ordnet jedem Auto sein vorderes linkes Rad zu. So entsteht das Umdrehen der Pfeile.

%******************************************************************************************************
\subsection{Prägarbe der stetigen Funktionen}
%******************************************************************************************************
Wenn wir die stetigen Funktionen $\R \to \R$ betrachten, können wir,
statt mit partiellen Funktionen zu arbeiten, deren Definitionsmenge bestimmt werden muss, auch die Struktur aller offenen Teilmengen von $\R$ und jeweils den auf ihnen vollständig definierten stetigen Funktionen betrachten.

Die Menge der offenen Mengen bildet eine Kategorie: die Objekte sind die offenen Mengen selber, die Morphismen sind die Identitäten sowie für jedes $U \subseteq V$ die Einbettung $i_{UV} \colon U \to V$.
Wir haben also zwischen zwei Objekten maximal einen Morphismus. Hier müssen wir eigentlich noch zeigen, dass die Axiome einer Kategorie erfüllt sind. Und wir haben noch ein oft genutztes Rudel an Beispielen für Kategorien. Diese Kategorie nennen wir $\mathcal{T}$ für Topologie.

Nun sei $\CC \colon \mathcal{T}^* \to \Set$ der Funktor, der jeder offenen Mengen $U$ die Menge der stetigen Funktionen $U \to \R$ zuordnet. Dem Morphismus $i_{UV} \colon U \to V$ wird die Einschränkung der Funktionen auf $V$ auf $U$ zugeordnet. Wenn also $f \colon V \to U$ stetig ist, dann ist
\begin{equation}
   \CC(i_{UV})(f) = f_{|U}.
\end{equation}
Es ist leicht zu zeigen, dass $\CC$ eine Prägarbe ist. 

In ähnlicher Weise gibt es die Prägarbe der differenzierbaren Funktionen, die der allgemeinen Funktionen und viele mehr. Diese befinden sich alle in der Kategorie der Prägarben über $\mathcal{T}$.

%******************************************************************************************************
\subsection{Kategorie der Graphen als Prägarbe}
%******************************************************************************************************
Sei $\mathcal{G}$ folgende Kategorie.
\[
\begin{tikzcd}
   V \arrow[shift right, swap]{r}{s} \arrow[shift left]{r}{t} & A
\end{tikzcd}
\]
Dann besteht eine Prägarbe $G$ über $\mathcal{G}$ aus einer Menge $G(A)$, die wir die Menge der Pfeile (arrows) nennen, und einer Menge $G(V)$, die wir die Menge der Knoten (vertices) nennen. Sei $a \in G(A)$ ein Pfeil, dann nennen wir $G(s)(a)$ den Startpunkt und $G(t)(a)$ den Endpunkt (target) von $a$.

Gerichtete Graphen sind Prägarben über $\mathcal{G}$.

Die Kategorie der gerichteten Graphen ist eine Prägarben-Kategorie!

%******************************************************************************************************
\subsection{Prägarben sind Verallgemeinerungen von Set}
%******************************************************************************************************
Prägarben sind Verallgemeinerungen von $\Set$. $\Set$ ist die Prägarbe $\Set^{\bullet^*}$, wobei $\bullet$ die Kategorie mit nur einem Objekt und nur einem Morphismus ist. Anders gesagt ist $\Set$ selber eine Prägarben-Kategorie.

\begin{backup}
%******************************************************************************************************
%                                                                                                     *
\section{TODO}

%                                                                                                     *
%******************************************************************************************************
\begin{itemize}
     \item Überprüfe Symbolverzeichnis
\end{itemize}

\end{backup}

\begin{backup}
    (Zur Zeit nicht benötigter Inhalt)
\end{backup}

%******************************************************************************************************
%                                                                                                     *
\begin{thebibliography}{9}
%                                                                                                     *
%******************************************************************************************************
   \bibitem[Awodey2010]{Awodey}
      Steve Awode, \emph{Category Theory},
      2010 Oxford University Press, 978-0-19-923718-0 (ISBN)

   \bibitem[Bradley2020]{Bradley}
      Tai-Danae Bradley, \emph{Topology, A Categorical Approach},
      2020 MIT Press, 978-0-262-53935-7 (ISBN)

   \bibitem[LawvereSchanuel2009]{Lawvere}
      F. William Lawvere, Stephen H. Schanuel, \emph{Conceptual Mathematics: a First Introduction to Categories},
      2009 Cambridge University Press, 978-0-521-71916-2 (ISBN)

   \bibitem[MacLane1978]{MacLane}
      Saunders Mac Lane, \emph{Categories for the Working Mathematician},
      Springer-Verlag New York Inc., 978-0-387-98403-2 (ISBN)

   \bibitem[Rotman2009]{Rotman}
   	Joseph J. Rotman, \emph{An Introduction to Homological Algebra},
   	2009 Springer-Verlag New York Inc., 978-0-387-24527-0 (ISBN)

\end{thebibliography}

%******************************************************************************************************
%                                                                                                     *
\begin{large}
    \centerline{\textsc{Symbolverzeichnis}}
\end{large}
%                                                                                                     *
%******************************************************************************************************
\bigskip

\renewcommand*{\arraystretch}{1}

\begin{tabular}{ll}
    $\bullet$                           & Kategorie mit genau einem Objekt und einem Morphismus\\
    $A, B, C, \cdots, X, Y, Z$          & Objekte\\
    $f, g, h, r, s, \cdots$             & Homomorphismen\\
    $\mathcal C, \mathcal D, \mathcal E, \cdots$ & Kategorien\\
    $\PP$                               & Potenzmengen-Funktor\\
    \Set                                & Die Kategorie der kleinen Mengen\\
    \Ab                                 & Kategorie der kleinen abelschen Gruppen\\
    $\Hom( X, Y)$                       & Die Klasse der Homomorphismen von $X$ nach $Y$\\
    $\alpha, \beta, \cdots$             & natürliche Transformationen oder Ordinalzahlen\\
    $\mathcal C ^{\text{op}}$ oder $\mathcal C ^{*}$         & Duale Kategorie\\
    $\DD^\CC$                           & Funktorkategorie\\
    $U, U', U''$                        & Universen\\
    $V_\alpha$                          & eine Menge der Von-Neumann-Hierarchie zur Ordinalzahl
                                          $\alpha$

\end{tabular}

\end{document}
