%******************************************************** -*-LaTeX-*- ******************************
%                                                                                                  *
% v5.0.1.0.4.4.8 Äquivalenz von Kategorien.tex                                                     *
%                                                                                                  *
% Copyright (C) 2023 Kategory GmbH \& Co. KG (joerg.kunze@kategory.de)                             *
%                                                                                                  *
% v5.0.1.0.4.4.8 Äquivalenz von Kategorien is part of kategoryMathematik.                          *
%                                                                                                  *
% kategoryMathematik is free software: you can redistribute it and/or modify                       *
% it under the terms of the GNU General Public License as published by                             *
% the Free Software Foundation, either version 3 of the License, or                                *
% (at your option) any later version.                                                              *
%                                                                                                  *
% kategoryMathematik is distributed in the hope that it will be useful,                            *
% but WITHOUT ANY WARRANTY; without even the implied warranty of                                   *
% MERCHANTABILITY or FITNESS FOR A PARTICULAR PURPOSE.  See the                                    *
% GNU General Public License for more details.                                                     *
%                                                                                                  *
% You should have received a copy of the GNU General Public License                                *
% along with this program.  If not, see <http://www.gnu.org/licenses/>.                            *
%                                                                                                  *
%***************************************************************************************************

\documentclass[a4paper]{amsart}
% \documentclass[a4paper]{book}

%-----------------------------------------------------------------------------------------------------*
% package:                                                                                            *
%-----------------------------------------------------------------------------------------------------*
\usepackage{amssymb}
\usepackage{amsfonts}
\usepackage{amsmath}
\usepackage{amsthm}

\usepackage{mathabx}

\usepackage{a4wide} % a little bit smaller margins

\usepackage{graphicx}
\usepackage{hyperref}
\usepackage{algorithmic}
\usepackage{listings}
\usepackage{color}
\usepackage{colortbl}
\usepackage{sidecap}
\usepackage{comment}
\usepackage{tcolorbox}
\usepackage{collect}

\usepackage{upgreek}

% \usepackage{diagrams}

\usepackage[german]{babel}
\usepackage[none]{hyphenat}
\emergencystretch=4em

\usepackage[utf8]{inputenc} % to be able to use äöü as characters in text
\usepackage[T1]{fontenc} % to be able to use äöü in lables
\usepackage{lmodern}     % to avoid pixelation introduced by fontenc

\usepackage{hyperref}

\usepackage{tikz}
\usepackage{tikz-cd}
\usetikzlibrary{babel}

%-----------------------------------------------------------------------------------------------------*
% theorem:                                                                                            *
%-----------------------------------------------------------------------------------------------------*
\theoremstyle{definition}
\newtheorem{theorem}{Theorem}[subsection]

\newcommand{\myTheorem}[1]{%
  \newtheorem{jk#1}[theorem]{#1}
  \newenvironment{#1}[1]{%
    \expandafter\begin{jk#1} \expandafter\label{#1:##1}\textbf{(##1):}
  }{%
    \expandafter\end{jk#1}
  }
}

\myTheorem{Definition}
\myTheorem{Proposition}
\myTheorem{Satz}
\myTheorem{Theorem}
\myTheorem{Beispiel}
\myTheorem{Anmerkung}

\definecollection{jkjkFrage}
\newtheorem{jkFrage}[theorem]{Frage}
\newenvironment{Frage}[1]{%
  \expandafter\begin{jkFrage} \expandafter\label{Frage:#1}\textbf{(#1):}
  \begin{collect}{jkjkFrage}{}{}
    \item \ref{Frage:#1} #1
  \end{collect}
}{%
  \expandafter\end{jkFrage}
}

\newcommand{\myRef}[2]{[#1 \ref{#1:#2}, ``#2'']}

\renewcommand{\proofname}{Beweis}

%-----------------------------------------------------------------------------------------------------*
% operator:                                                                                           *
%-----------------------------------------------------------------------------------------------------*
\DeclareMathOperator{\End}{End}
\DeclareMathOperator{\Ker}{Ker}
\DeclareMathOperator{\Mat}{Mat}
\DeclareMathOperator{\rank}{rank}
\DeclareMathOperator{\ggT}{ggT}
\DeclareMathOperator{\len}{len}
\DeclareMathOperator{\ord}{ord}
\DeclareMathOperator{\kgV}{kgV}
\DeclareMathOperator{\id}{id}
\DeclareMathOperator{\red}{red}
\DeclareMathOperator{\supp}{supp}
\DeclareMathOperator{\Bild}{Bild}
\DeclareMathOperator{\Rang}{Rang}
\DeclareMathOperator{\Det}{Det}
\DeclareMathOperator{\Hom}{Hom}
\DeclareMathOperator{\GL}{GL}

\DeclareMathOperator{\sub}{sub}
\DeclareMathOperator{\blk}{blk}
\DeclareMathOperator{\minimal}{minimal}
\DeclareMathOperator{\maximal}{maximal}

\definecolor{mygreen}{rgb}{0,0.6,0}
\definecolor{mygray}{rgb}{0.5,0.5,0.5}
\definecolor{mymauve}{rgb}{0.58,0,0.82}

\lstset{ %
  backgroundcolor=\color{white},   % choose the background color
  basicstyle=\ttfamily\footnotesize,        % size of fonts used for the code
  breaklines=true,                 % automatic line breaking only at whitespace
  captionpos=b,                    % sets the caption-position to bottom
  commentstyle=\color{mygreen},    % comment style
  escapeinside={\%*}{*)},          % if you want to add LaTeX within your code
  keywordstyle=\color{blue},       % keyword style
  stringstyle=\color{mymauve},     % string literal style
  frame=single
}

\setcounter{MaxMatrixCols}{20}

%******************************************************************************************************
%                                                                                                     *
% definition:                                                                                         *
%                                                                                                     *
%******************************************************************************************************
\newcommand{\R}{\ensuremath{\mathbb{ R }}}
\newcommand{\Q}{\ensuremath{\mathbb{ Q }}}
\newcommand{\Z}{\ensuremath{\mathbb{ Z }}}
\newcommand{\N}{\ensuremath{\mathbb{ N }}}
\newcommand{\C}{\ensuremath{\mathbb{ C }}}
\newcommand{\A}{\ensuremath{\mathbb{ A }}}
\newcommand{\F}{\ensuremath{\mathbb{ F }}}
\newcommand{\K}{\ensuremath{\mathbb{ K }}}
\newcommand{\Pb}{\ensuremath{\mathbb{ P }}}

\newcommand{\M}{\ensuremath{\mathcal{ M }}}
\newcommand{\V}{\ensuremath{\mathcal{ V }}}

\newcommand{\AAA}{\ensuremath{\mathcal{ A }}}
\newcommand{\BB}{\ensuremath{\mathcal{ B }}}
\newcommand{\CC}{\ensuremath{\mathcal{ C }}}
\newcommand{\DD}{\ensuremath{\mathcal{ D }}}
\newcommand{\EE}{\ensuremath{\mathcal{ E }}}
\newcommand{\FF}{\ensuremath{\mathcal{ F }}}
\newcommand{\KK}{\ensuremath{\mathcal{ K }}}
\newcommand{\MM}{\ensuremath{\mathcal{ M }}}
\newcommand{\PP}{\ensuremath{\mathcal{ P }}}
\newcommand{\ZZ}{\ensuremath{\mathcal{ Z }}}

\newcommand{\imporant}[1]{ \textcolor{red}{\textbf{#1}} }

\newcommand{\bb}[1]{\mathbf{#1}}
\newcommand{\balpha}{\boldsymbol{\upalpha}}
\newcommand{\bbeta}{\boldsymbol{\upbeta}}
\newcommand{\bgamma}{\boldsymbol{\upgamma}}
\newcommand{\bdelta}{\boldsymbol{\delta}}
\newcommand{\bmu}{\boldsymbol{\upmu}}

\newcommand{\z}[1]{\Z_{#1}}
\newcommand{\e}[1]{\z{#1}^*}
\newcommand{\q}[1]{(\e{#1})^2}
\newcommand{\m}{\mathcal}

\excludecomment{book}
\excludecomment{example}
\excludecomment{backup}

\begin{document}

%******************************************************************************************************
%                                                                                                     *
\begin{titlepage}
%                                                                                                     *
%******************************************************************************************************
% \vspace*{\fill}
\centering
{\huge
(Höhere Grundlagen) Kategorien\\[1cm]
\textbf{v5.0.1.0.4.4.8 Äquivalenz von Kategorien}
}\\[1cm]

\textbf{Kategory GmbH \& Co. KG}\\
Präsentiert von Jörg Kunze\\
Copyright (C) 2023 Kategory GmbH \& Co. KG

\end{titlepage}

%\clearpage
%\setcounter{page}{2}
%
%\tableofcontents

\newpage

%******************************************************************************************************
%                                                                                                     *
\section*{Beschreibung}
%                                                                                                     *
%******************************************************************************************************

%******************************************************************************************************
\subsection*{Inhalt}
%******************************************************************************************************
Äquivalente Kategorien können mit den Mitteln der Kategorien-Theorie kaum unterschieden werden. Viele Tatsachen übe eine Kategorie gelten automatisch immer auch für alle zu ihr äquivalenten.

Intuitiv unterscheiden sie sich ausschließlich in den Mächtigkeiten ihrer Äquivalenzklassen isomorpher Objekte. Isomorphe Objekte, also solche zwischen denen es einen Isomorphismus gibt, können mit den Mitteln der Kategorien-Theorie gar nicht unterschieden werden.

Im Extremfall gibt es in einer Kategorie zu jedem Objekt kein weiteres isomorphes, jedes Objekt ist also das einzige seiner Art. Wählen wir aus einer Kategorie zu jeder Klasse isomorpher Objekte jeweils genau eins aus, erhalten wir das Skelett der Kategorie. Das Skelett ist äquivalent zur Ausgangskategorie. Zwei Kategorien sind äquivalent genau dann, wenn deren Skelette isomorph sind.

Äquivalente Kategorien können gegenseitig ineinander funktoriel abgebildet werden und zwar so, dass diese beiden Einbettungen hintereinandergeschaltet eine Einbettung in sich selbst ergeben, die  natürlich isomorph zur Identität ist.

Äquivalenzen sind nicht immer mit bloßem Auge zu erkennen, was das folgende Beispiel auch zeigt.

Als Beispiel nehmen wir einen Satz von Gelfand-Neumark, der sagt, dass die Kategorie der kompakten Hausdorffräume zu der der kommutativen $C^*$-Algebren mit Einselement äquivalent ist. Also zwei völlig verschiedene Gebilde sind kategorisch fast gleich. Diese Korrespondenz kann dazu genutzt werden topologische Eigenschaften algebraisch zu kodieren und dann auf den nicht-kommutativen Fall zu erweitern: nicht-kommutative Topologie.  

%******************************************************************************************************
\subsection*{Präsentiert}
%******************************************************************************************************
Von Jörg Kunze

%******************************************************************************************************
\subsection*{Voraussetzungen}
%******************************************************************************************************
Kategorie, Funktor, natürliche Transformation, Isomorphismus

%******************************************************************************************************
\subsection*{Text}
%******************************************************************************************************
Der Begleittext als PDF und als LaTeX findet sich unter
{\tiny
   \url{https://github.com/kategory/kategoryMathematik/tree/main/v5%20H%C3%B6here%20Grundlagen/v5.0.1%20Kategorien/v5.0.1.0.4.4.8%20%C3%84quivalenz%20von%20Kategorien}
}

%******************************************************************************************************
\subsection*{Meine Videos}
%******************************************************************************************************
Siehe auch in den folgenden Videos:\\ 
\\
v5.0.1.6.1 (Höher) Kategorien - Abelsche - Nullobjekt\\
\url{https://youtu.be/XbOf-nVZ1t0}\\
\\
v5.0.1.6.1.4 (Höher) Kategorien - Abelsche - Additiver Funktor\\
\url{https://youtu.be/zSP_a2RvoYE}

%******************************************************************************************************
\subsection*{Quellen}
%******************************************************************************************************
Siehe auch in den folgenden Seiten:\\
\url{https://de.wikipedia.org/wiki/%C3%84quivalenz_(Kategorientheorie)}\\
\url{https://ncatlab.org/nlab/show/equivalence+of+categories}\\
\url{https://de.wikipedia.org/wiki/Nat%C3%BCrliche_Transformation}\\
\url{https://ncatlab.org/nlab/show/natural+isomorphism}\\
\url{https://de.wikipedia.org/wiki/Skelett_(Kategorientheorie)}\\
\url{https://de.wikipedia.org/wiki/Satz_von_Gelfand-Neumark}

%******************************************************************************************************
\subsection*{Buch}
%******************************************************************************************************
Grundlage ist folgendes Buch:\\
"`Categories for the Working Mathematician"'\\
Saunders Mac Lane\\
1998 | 2nd ed. 1978\\
Springer-Verlag New York Inc.\\
978-0-387-98403-2 (ISBN)\\
{\tiny
   \url{https://www.amazon.de/Categories-Working-Mathematician-Graduate-Mathematics/dp/0387984038}}\\

Gut für die kategorische Sichtweise ist:\\
"`Topology, A Categorical Approach"'\\
Tai-Danae Bradley\\
2020 MIT Press\\
978-0-262-53935-7 (ISBN)\\ 
{\tiny
\url{https://www.lehmanns.de/shop/mathematik-informatik/52489766-9780262539357-topology}}\\

Einige gut Erklärungen finden sich auch in den Einführenden Kapitel von:\\
"`An Introduction to Homological Algebra"'\\
Joseph J. Rotman\\
2009 Springer-Verlag New York Inc.\\
978-0-387-24527-0 (ISBN)\\ 
{\tiny \url{https://www.lehmanns.de/shop/mathematik-informatik/6439666-9780387245270-an-introduction-to-homological-algebra}}\\

Etwas weniger umfangreich und weniger tiefgehend aber gut motivierend ist:
"`Category Theory"'\\
Steve Awodey\\
2010 Oxford University Press\\
978-0-19-923718-0 (ISBN)\\
{\tiny\url{https://www.lehmanns.de/shop/mathematik-informatik/9478288-9780199237180-category-theory}}\\

Mit noch weniger Mathematik und die Konzepte motivierend ist:
"`Conceptual Mathematics: a First Introduction to Categories"'\\
F. William Lawvere, Stephen H. Schanuel\\
2009 Cambridge University Press\\
978-0-521-71916-2 (ISBN)\\
{\tiny\url{https://www.lehmanns.de/shop/mathematik-informatik/8643555-9780521719162-conceptual-mathematics}}

%******************************************************************************************************
\subsection*{Lizenz}
%******************************************************************************************************
Dieser Text und das Video sind freie Software. Sie können es unter den Bedingungen der 
GNU General Public License, wie von der Free Software Foundation veröffentlicht, weitergeben 
und/oder modifizieren, entweder gemäß Version 3 der Lizenz oder (nach Ihrer Option) jeder späteren Version.

Die Veröffentlichung von Text und Video erfolgt in der Hoffnung, dass es Ihnen von Nutzen sein wird, 
aber OHNE IRGENDEINE GARANTIE, sogar ohne die implizite Garantie der MARKTREIFE oder der 
VERWENDBARKEIT FÜR EINEN BESTIMMTEN ZWECK. Details finden Sie in der GNU General Public License.

Sie sollten ein Exemplar der GNU General Public License zusammen mit diesem Text erhalten haben 
(zu finden im selben Git-Projekt). 
Falls nicht, siehe \url{http://www.gnu.org/licenses/}.

\subsection*{Das Video}
%******************************************************************************************************
Das Video hierzu ist zu finden unter 
{\tiny
   \url{huch!}
}

%******************************************************************************************************
%                                                                                                     *
\section{Äquivalente Kategorien}
%                                                                                                     *
%******************************************************************************************************

%******************************************************************************************************
\subsection{Ein endliches Beispiel}
%******************************************************************************************************
\begin{equation}
	\begin{tikzcd}
		\bullet \arrow[leftrightarrow,d] \arrow[dr] \arrow[Rightarrow,rrr, bend left, green, "F"]&&& 
		   \color{green}\bullet \arrow[leftrightarrow,green,d] \arrow[green,r] \arrow[dr] & \color{green}\bullet \arrow[leftrightarrow,d]\\
		\color{red}\bullet \arrow[leftrightarrow,red,d] \arrow[red,r] & \color{red}\bullet && 
		   \color{green}\bullet                          \arrow[r] \arrow[green,ur] \arrow[Rightarrow,red,llld, bend left, "G"] & \bullet \\
		\color{red}\bullet           \arrow[red,ur]
	\end{tikzcd}
\end{equation}

%******************************************************************************************************
\subsection{Definition der Äquivalenz von Kategorien }
%******************************************************************************************************
\begin{Definition}{Äquivalenz von Kategorien}
	Zwei Kategorien $\CC, \DD$ heißen \textbf{äquivalent}, wenn es zwei Funktoren
	\begin{alignat}{3}
		&F &&\colon \CC && \to \DD \\
		&G &&\colon \DD && \to \CC
	\end{alignat}
	und zwei natürliche Isomorphismen
	\begin{alignat}{3}
		&\alpha &&\colon FG && \to 1_\CC \\
		&\beta  &&\colon GF && \to 1_\DD
	\end{alignat}
	gibt.
\end{Definition}

%******************************************************************************************************
\subsection{Endo-Funktoren}
%******************************************************************************************************
\usetikzlibrary{shapes.geometric,fit}
\tikzset{
	dashellipse/.style={ellipse,draw,dashed,inner sep=0pt,blue,fit={#1}}
}
\begin{equation}
	\begin{tikzcd}[
		execute at end picture={
			\node[dashellipse=(tikz@f@1-1-1)(tikz@f@1-1-1)]{};
		}
	]
		X \arrow[dd] \arrow[dr] \arrow[rrr, blue, dashed, bend left, "\mathcal F"{name=0}] \arrow[rrrrrd, red, dashed, bend right] &&& A \arrow[dd, blue] \arrow[dr, blue] \arrow[dddrr, bend right] \arrow[drr, green] \arrow[ddrrr, bend left]\\ 
		& Y \arrow[dl] \arrow[rrr, blue, dashed, bend left] \arrow[rrrrrd, red, dashed, bend right] &&&B \arrow[dl, blue] \arrow[ddr] \arrow[drr, green] &D \arrow[dd, red] \arrow[dr, red]\\
		Z \arrow[rrr, blue, dashed, bend left] \arrow[rrrrrd, red, dashed, bend right, "\mathcal G"{name=1}] &&&C  \arrow[drr, green] &&& E \arrow[dl, red]\\
		&&&&&F
		\arrow[Rightarrow, green, from=0, to=1,shorten >= 6pt, yshift = -3pt, "\alpha"{yshift=3pt}]
	\end{tikzcd}
\end{equation}



TODO: drei Bilder: Nat zwischen 1. Funktoren 2. Endo-Funktoren 3. 1 und Endo-Funktor


%******************************************************************************************************
\subsection{Das Skelett einer Kategorie}
%******************************************************************************************************
\begin{Definition}{Skelett einer Kategorie}
   Ein \textbf{Skelett} einer Kategorie $\CC$ ist eine Unter-Kategorie bestehend aus genau einem Objekt von $\CC$ pro Isomorphie-Klassen von Objekten und zwischen zwei Objekten im Skelett bestehen alle Homomorphismen wie in $\CC$.
\end{Definition}

\begin{itemize}
   \item Zu jedem Objekt in $\CC$ gibt es genau ein zum ihm isomorphes Objekt im Skelett.
   \item Je zwei Objekte im Skelett sind nicht isomorph.
   \item Die Ausgangskategorie ist äquivalent im Sinne von \myRef{Definition}{Äquivalenz von Kategorien}.
   \item Zwei Kategorien sind genau dann äquivalent, wenn sie isomorphe Skelette besitzen.
\end{itemize}

TODO: Male in Skelett

Grundlegend ist, dass jede Kategorie ein Skelett besitzt. (Diese Aussage ist zum Auswahlaxiom für Klassen äquivalent, wie es etwa die Neumann-Bernays-Gödel-Mengenlehre bereitstellt.) Wenn auch eine Kategorie mehrere verschiedene Skelette besitzen kann, sind sie jedoch als Kategorien isomorph. Also besitzt jede Kategorie bis auf Isomorphie ein eindeutiges Skelett.

Die Bedeutung von Skeletten rührt daher, dass sie (bis auf Isomorphie) kanonische Vertreter der Äquivalenzklassen bezüglich der Äquivalenz von Kategorien sind. Das ergibt sich daraus, dass jede Kategorie zu einem Skelett äquivalent ist, und dass zwei Kategorien genau dann äquivalent sind, wenn sie isomorphe Skelette besitzen. (Die letzten beiden Absätze sind dem Wikipedia-Eintrag über Skelette entnommen.)

In der folgenden Skizze ist jeweils die Kategorie
\begin{equation}
   \begin{tikzcd}
      \color{blue}\bullet \arrow[blue,r] & \color{blue}\bullet
   \end{tikzcd}
\end{equation}
als Skelett blau hervorgehoben:
\begin{equation}
   \begin{tikzcd}
      \bullet \arrow[leftrightarrow,d] \arrow[dr] \arrow[Rightarrow,rrr, bend left, "F"]&&& 
      \bullet \arrow[leftrightarrow,d] \arrow[r] \arrow[dr] & \bullet \arrow[leftrightarrow,d]\\
      \color{blue}\bullet \arrow[leftrightarrow,d] \arrow[blue,r] & \color{blue}\bullet && 
      \color{blue}\bullet                          \arrow[r,blue] \arrow[ur] \arrow[Rightarrow,llld, bend left, "G"] & \color{blue}\bullet \\
      \bullet           \arrow[ur]
   \end{tikzcd}
\end{equation}

\begin{backup}
Noch zu erledigen sind
%******************************************************************************************************
%                                                                                                     *
\section{TODO}
%                                                                                                     *
%******************************************************************************************************
\begin{itemize}
   \item leer
\end{itemize}
\end{backup}

\begin{backup}
    (Zur Zeit nicht benötigter Inhalt)
\end{backup}

%******************************************************************************************************
%                                                                                                     *
\begin{thebibliography}{9}
%                                                                                                     *
%******************************************************************************************************
   \bibitem[Awodey2010]{Awodey}
      Steve Awode, \emph{Category Theory},
      2010 Oxford University Press, 978-0-19-923718-0 (ISBN)

   \bibitem[Bradley2020]{Bradley}
      Tai-Danae Bradley, \emph{Topology, A Categorical Approach},
      2020 MIT Press, 978-0-262-53935-7 (ISBN)

   \bibitem[LawvereSchanuel2009]{Lawvere}
      F. William Lawvere, Stephen H. Schanuel, \emph{Conceptual Mathematics: a First Introduction to Categories},
      2009 Cambridge University Press, 978-0-521-71916-2 (ISBN)

   \bibitem[MacLane1978]{MacLane}
      Saunders Mac Lane, \emph{Categories for the Working Mathematician},
      Springer-Verlag New York Inc., 978-0-387-98403-2 (ISBN)

   \bibitem[Rotman2009]{Rotman}
   	Joseph J. Rotman, \emph{An Introduction to Homological Algebra},
   	2009 Springer-Verlag New York Inc., 978-0-387-24527-0 (ISBN)
      
\end{thebibliography}

%******************************************************************************************************
%                                                                                                     *
\begin{large}
    \centerline{\textsc{Symbolverzeichnis}}
\end{large}
%                                                                                                     *
%******************************************************************************************************
\bigskip

\renewcommand*{\arraystretch}{1}

\begin{tabular}{ll}
    $A, B, C, \cdots, X, Y, Z$          & Objekte\\
    $F,G$                               & Funktoren\\
    $f, g, h, r, s, \cdots$             & Homomorphismen\\
    $\mathcal C, \mathcal D, \mathcal E, \cdots$ & Kategorien\\
    \textbf{Set}                        & Die Kategorie der Mengen\\
    $\Hom( X, Y)$                       & Die Menge der Homomorphismen von $X$ nach $Y$\\
    $\alpha, \beta, \cdots$             & natürliche Transformationen\\
    $\mathcal C ^{\text{op}}$           & Duale Kategorie\\
    \textbf{Ring} nach \textbf{Gruppe}  & Kategorie der Ringe und der Gruppen\\
    $\GL_n(R)$                          & Allgemeine lineare Gruppe über dem Ring $R$\\
    $R^*$                               & Einheitengruppe des Rings $R$\\
    $\Det_n^R$                          & $n$-dimensionale Determinante für Matrizen mit Koeffizienten in $R$. 
    
\end{tabular}

\end{document}
