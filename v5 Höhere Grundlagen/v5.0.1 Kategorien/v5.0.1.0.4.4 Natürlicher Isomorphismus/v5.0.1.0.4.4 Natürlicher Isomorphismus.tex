%******************************************************** -*-LaTeX-*- ******************************
%                                                                                                  *
% v5.0.1.0.4.4 Natürlicher Isomorphismus.tex                                                       *
%                                                                                                  *
% Copyright (C) 2023 Kategory GmbH \& Co. KG (joerg.kunze@kategory.de)                             *
%                                                                                                  *
% v5.0.1.0.4.4 Natürlicher Isomorphismus is part of kategoryMathematik.                            *
%                                                                                                  *
% kategoryMathematik is free software: you can redistribute it and/or modify                       *
% it under the terms of the GNU General Public License as published by                             *
% the Free Software Foundation, either version 3 of the License, or                                *
% (at your option) any later version.                                                              *
%                                                                                                  *
% kategoryMathematik is distributed in the hope that it will be useful,                            *
% but WITHOUT ANY WARRANTY; without even the implied warranty of                                   *
% MERCHANTABILITY or FITNESS FOR A PARTICULAR PURPOSE.  See the                                    *
% GNU General Public License for more details.                                                     *
%                                                                                                  *
% You should have received a copy of the GNU General Public License                                *
% along with this program.  If not, see <http://www.gnu.org/licenses/>.                            *
%                                                                                                  *
%***************************************************************************************************

\documentclass[a4paper]{amsart}
% \documentclass[a4paper]{book}

%-----------------------------------------------------------------------------------------------------*
% package:                                                                                            *
%-----------------------------------------------------------------------------------------------------*
\usepackage{amssymb}
\usepackage{amsfonts}
\usepackage{amsmath}
\usepackage{amsthm}

\usepackage{mathabx}

\usepackage{a4wide} % a little bit smaller margins

\usepackage{graphicx}
\usepackage{hyperref}
\usepackage{algorithmic}
\usepackage{listings}
\usepackage{color}
\usepackage{colortbl}
\usepackage{sidecap}
\usepackage{comment}
\usepackage{tcolorbox}
\usepackage{collect}

\usepackage{upgreek}

% \usepackage{diagrams}

\usepackage[german]{babel}
\usepackage[none]{hyphenat}
\emergencystretch=4em

\usepackage[utf8]{inputenc} % to be able to use äöü as characters in text
\usepackage[T1]{fontenc} % to be able to use äöü in lables
\usepackage{lmodern}     % to avoid pixelation introduced by fontenc

\usepackage{hyperref}

\usepackage{tikz}
\usepackage{tikz-cd}
\usetikzlibrary{babel}

%-----------------------------------------------------------------------------------------------------*
% theorem:                                                                                            *
%-----------------------------------------------------------------------------------------------------*
\theoremstyle{definition}
\newtheorem{theorem}{Theorem}[subsection]

\newcommand{\myTheorem}[1]{%
  \newtheorem{jk#1}[theorem]{#1}
  \newenvironment{#1}[1]{%
    \expandafter\begin{jk#1} \expandafter\label{#1:##1}\textbf{(##1):}
  }{%
    \expandafter\end{jk#1}
  }
}

\myTheorem{Definition}
\myTheorem{Proposition}
\myTheorem{Theorem}
\myTheorem{Example}
\myTheorem{Remark}

\definecollection{jkjkFrage}
\newtheorem{jkFrage}[theorem]{Frage}
\newenvironment{Frage}[1]{%
  \expandafter\begin{jkFrage} \expandafter\label{Frage:#1}\textbf{(#1):}
  \begin{collect}{jkjkFrage}{}{}
    \item \ref{Frage:#1} #1
  \end{collect}
}{%
  \expandafter\end{jkFrage}
}

\newcommand{\myRef}[2]{[#1 \ref{#1:#2}, ``#2'']}

\renewcommand{\proofname}{Beweis}

%-----------------------------------------------------------------------------------------------------*
% operator:                                                                                           *
%-----------------------------------------------------------------------------------------------------*
\DeclareMathOperator{\End}{End}
\DeclareMathOperator{\Ker}{Ker}
\DeclareMathOperator{\Mat}{Mat}
\DeclareMathOperator{\rank}{rank}
\DeclareMathOperator{\ggT}{ggT}
\DeclareMathOperator{\len}{len}
\DeclareMathOperator{\ord}{ord}
\DeclareMathOperator{\kgV}{kgV}
\DeclareMathOperator{\id}{id}
\DeclareMathOperator{\red}{red}
\DeclareMathOperator{\supp}{supp}
\DeclareMathOperator{\Bild}{Bild}
\DeclareMathOperator{\Rang}{Rang}
\DeclareMathOperator{\Det}{Det}
\DeclareMathOperator{\Hom}{Hom}

\DeclareMathOperator{\sub}{sub}
\DeclareMathOperator{\blk}{blk}
\DeclareMathOperator{\minimal}{minimal}
\DeclareMathOperator{\maximal}{maximal}

\definecolor{mygreen}{rgb}{0,0.6,0}
\definecolor{mygray}{rgb}{0.5,0.5,0.5}
\definecolor{mymauve}{rgb}{0.58,0,0.82}

\lstset{ %
  backgroundcolor=\color{white},   % choose the background color
  basicstyle=\ttfamily\footnotesize,        % size of fonts used for the code
  breaklines=true,                 % automatic line breaking only at whitespace
  captionpos=b,                    % sets the caption-position to bottom
  commentstyle=\color{mygreen},    % comment style
  escapeinside={\%*}{*)},          % if you want to add LaTeX within your code
  keywordstyle=\color{blue},       % keyword style
  stringstyle=\color{mymauve},     % string literal style
  frame=single
}

\setcounter{MaxMatrixCols}{20}

%******************************************************************************************************
%                                                                                                     *
% definition:                                                                                         *
%                                                                                                     *
%******************************************************************************************************
\newcommand{\R}{\ensuremath{\mathbb{ R }}}
\newcommand{\Q}{\ensuremath{\mathbb{ Q }}}
\newcommand{\Z}{\ensuremath{\mathbb{ Z }}}
\newcommand{\N}{\ensuremath{\mathbb{ N }}}
\newcommand{\C}{\ensuremath{\mathbb{ C }}}
\newcommand{\A}{\ensuremath{\mathbb{ A }}}
\newcommand{\F}{\ensuremath{\mathbb{ F }}}
\newcommand{\K}{\ensuremath{\mathbb{ K }}}
\newcommand{\Pb}{\ensuremath{\mathbb{ P }}}

\newcommand{\M}{\ensuremath{\mathcal{ M }}}
\newcommand{\V}{\ensuremath{\mathcal{ V }}}

\newcommand{\AAA}{\ensuremath{\mathcal{ A }}}
\newcommand{\BB}{\ensuremath{\mathcal{ B }}}
\newcommand{\CC}{\ensuremath{\mathcal{ C }}}
\newcommand{\EE}{\ensuremath{\mathcal{ E }}}
\newcommand{\KK}{\ensuremath{\mathcal{ K }}}
\newcommand{\MM}{\ensuremath{\mathcal{ M }}}
\newcommand{\PP}{\ensuremath{\mathcal{ P }}}
\newcommand{\ZZ}{\ensuremath{\mathcal{ Z }}}

\newcommand{\imporant}[1]{ \textcolor{red}{\textbf{#1}} }

\newcommand{\bb}[1]{\mathbf{#1}}
\newcommand{\balpha}{\boldsymbol{\upalpha}}
\newcommand{\bbeta}{\boldsymbol{\upbeta}}
\newcommand{\bgamma}{\boldsymbol{\upgamma}}
\newcommand{\bdelta}{\boldsymbol{\delta}}
\newcommand{\bmu}{\boldsymbol{\upmu}}

\newcommand{\z}[1]{\Z_{#1}}
\newcommand{\e}[1]{\z{#1}^*}
\newcommand{\q}[1]{(\e{#1})^2}
\newcommand{\m}{\mathcal}

\excludecomment{book}
\excludecomment{example}
\excludecomment{backup}

\begin{document}

%******************************************************************************************************
%                                                                                                     *
\begin{titlepage}
%                                                                                                     *
%******************************************************************************************************
% \vspace*{\fill}
\centering
{\huge
(Höhere Grundlagen) Kategorien\\[1cm]
\textbf{v5.0.1.0.4.4 Natürlicher Isomorphismus}
}\\[1cm]

\textbf{Kategory GmbH \& Co. KG}\\
Präsentiert von Jörg Kunze\\
Copyright (C) 2023 Kategory GmbH \& Co. KG

\end{titlepage}

%\clearpage
%\setcounter{page}{2}
%
%\tableofcontents

\newpage

%******************************************************************************************************
%                                                                                                     *
\section*{Beschreibung}
%                                                                                                     *
%******************************************************************************************************

%******************************************************************************************************
\subsection*{Inhalt}
%******************************************************************************************************
Natürliche Isomorphismen werden oft beiläufig eingeführt. So sagen wir z.B. in der Kategorie der Mengen, dass es eine Bijektion zwischen den Morphismen des Produktes von $X$ und $Y$ nach $Z$ und den Morphismen von $X$ in die Morphismen von $Y$ nach $Z$ gibt. Wir sagen weiter salopp, dass dieser natürlich in $X$, $Y$ und $Z$ sei.

Was heißt denn hier "`natürlich in"'? Es bedeutet, dass wir einen natürlichen Isomorphismus (weil Bijektion) zwischen $\Hom( X \times Y, Z )$ und $\Hom( X, \Hom( Y, Z ))$ haben. D.h. die oben genannte Bijektion ist verträglich mit Morphismen (hier Funktionen) zwischen den Objekten.

Die Bijektion von oben ist nichts anderes als das Currying. Dessen Natürlichkeit ist ein oft verschwiegener Vorteil.

Wenn wir genauer hinsehen haben wir eine ganze Schar von Isomorphismen, nämlich für jede Wahl von $X$, $Y$ und $Z$ einen. Diese Schar sind die Komponenten der natürlichen Transformation.

Eine natürlicher Isomorphismus findet zwischen zwei Funktoren statt. Sie ist eine Schar von Isomorphismen, die mit den Bildern der Morphismen verträgliche sind (kommutieren).

Was wir also mit "`ist natürlich"' sagen ist, dass es zwei Kategorien gibt mit zwei Funktoren zwischen ihnen und dass die Schar der Isomorphismen Komponenten eines natürlichen Isomorphismuses sind, die somit verträglich mit den Bildern von Morphismen unter den beiden Funktoren sind.

Aber was sind denn im obigen Beispiel die Funktoren? Das ist die Funktion des "`in"' von "`ist natürlich in"': Natürlich in $X$ bedeutet, dass, wenn wir $X$ als Variable ansehen, dann erhalten wir zwei Funktoren, so dass ...

Im obigen Beispiel haben wir drei mal zwei Funktoren und zwar für jede Wahl der beiden anderen Variablen.

Wir sagen also in Wirklichkeit:
1. Wir haben zwei Kategorien
2. In jeder der drei Variablen erhalten wir bei fester Wahl der anderen beiden zwei Funktoren.
3. Diese Funktoren bilden auch Morphismen ab, und das so, dass Identitäten und Verknüpfungen erhalten bleiben.
4. Die Isomorphismen sind Isomorphismen in der Zielkategorie.
5. Die Verträglichkeitsbedingung für natürliche Transformationen ist für diese Isomorphismen und deren Inverse erfüllt.

%******************************************************************************************************
\subsection*{Präsentiert}
%******************************************************************************************************
Von Jörg Kunze

%******************************************************************************************************
\subsection*{Voraussetzungen}
%******************************************************************************************************
Axiome der Kategorien, Funktor, natürliche Transformation.

%******************************************************************************************************
\subsection*{Text}
%******************************************************************************************************
Der Begleittext als PDF und als LaTeX findet sich unter
{\tiny
   \url{https://github.com/kategory/kategoryMathematik/tree/main/v5%20H%C3%B6here%20Grundlagen/v5.0.1%20Kategorien/v5.0.1.0.4.4%20Nat%C3%BCrlicher%20Isomorphismus}
}

%******************************************************************************************************
\subsection*{Meine Videos}
%******************************************************************************************************
Siehe auch in den folgenden Videos:\\
v5.0.1.0.3 (Höher) Kategorien - Funktoren\\
\url{https://youtu.be/Ojf5LQGeyOU}\\
v5.0.1.0.4 (Höher) Kategorien - Natürliche Transformationen\\
\url{https://youtu.be/IN7Qa-SwlD0}\\


%******************************************************************************************************
\subsection*{Quellen}
%******************************************************************************************************
Siehe auch in den folgenden Seiten:\\
\url{https://de.wikipedia.org/wiki/Nat%C3%BCrliche_Transformation}\\
\url{https://ncatlab.org/nlab/show/natural+transformation}\\
\url{https://ncatlab.org/nlab/show/natural+isomorphism}\\
\url{https://de.wikipedia.org/wiki/Currying}\\
\url{https://en.wikipedia.org/wiki/Currying#Category_theory}\\
\url{}\\
\url{}\\
\url{}\\
\url{}\\

%******************************************************************************************************
\subsection*{Buch}
%******************************************************************************************************
Grundlage ist folgendes Buch:\\
"`Categories for the Working Mathematician"'\\
Saunders Mac Lane\\
1998 | 2nd ed. 1978\\
Springer-Verlag New York Inc.\\
978-0-387-98403-2 (ISBN)\\
{\tiny
   \url{https://www.amazon.de/Categories-Working-Mathematician-Graduate-Mathematics/dp/0387984038}}\\

Gut für die kategorische Sichtweise ist:\\
"`Topology, A Categorical Approach"'\\
Tai-Danae Bradley\\
2020 MIT Press\\
978-0-262-53935-7 (ISBN)\\ 
{\tiny
\url{https://www.lehmanns.de/shop/mathematik-informatik/52489766-9780262539357-topology}}\\

Einige gut Erklärungen finden sich auch in den Einführenden Kapitel von:\\
"`An Introduction to Homological Algebra"'\\
Joseph J. Rotman\\
2009 Springer-Verlag New York Inc.\\
978-0-387-24527-0 (ISBN)\\ 
{\tiny \url{https://www.lehmanns.de/shop/mathematik-informatik/6439666-9780387245270-an-introduction-to-homological-algebra}}\\

Etwas weniger umfangreich und weniger tiefgehend aber gut motivierend ist:
"`Category Theory"'\\
Steve Awodey\\
2010 Oxford University Press\\
978-0-19-923718-0 (ISBN)\\
{\tiny\url{https://www.lehmanns.de/shop/mathematik-informatik/9478288-9780199237180-category-theory}}\\

Mit noch weniger Mathematik und die Konzepte motivierend ist:
"`Conceptual Mathematics: a First Introduction to Categories"'\\
F. William Lawvere, Stephen H. Schanuel\\
2009 Cambridge University Press\\
978-0-521-71916-2 (ISBN)\\
{\tiny\url{https://www.lehmanns.de/shop/mathematik-informatik/8643555-9780521719162-conceptual-mathematics}}

%******************************************************************************************************
\subsection*{Lizenz}
%******************************************************************************************************
Dieser Text und das Video sind freie Software. Sie können es unter den Bedingungen der 
GNU General Public License, wie von der Free Software Foundation veröffentlicht, weitergeben 
und/oder modifizieren, entweder gemäß Version 3 der Lizenz oder (nach Ihrer Option) jeder späteren Version.

Die Veröffentlichung von Text und Video erfolgt in der Hoffnung, dass es Ihnen von Nutzen sein wird, 
aber OHNE IRGENDEINE GARANTIE, sogar ohne die implizite Garantie der MARKTREIFE oder der 
VERWENDBARKEIT FÜR EINEN BESTIMMTEN ZWECK. Details finden Sie in der GNU General Public License.

Sie sollten ein Exemplar der GNU General Public License zusammen mit diesem Text erhalten haben 
(zu finden im selben Git-Projekt). 
Falls nicht, siehe \url{http://www.gnu.org/licenses/}.

\subsection*{Das Video}
%******************************************************************************************************
Das Video hierzu ist zu finden unter 
{\tiny
   \url{huhu}
}

%******************************************************************************************************
%                                                                                                     *
\section{Natürlicher Isomorphismus}
%                                                                                                     *
%******************************************************************************************************
Als Beispiel nehmen wir das Currying in der Kategorie \textbf{Set}. Es besagt, dass es (pro $A, B, C$) einen Mengen-Isomorphismus (also eine Bijektion) folgender Art gibt.
\begin{equation}
   \Hom( A \times B, C ) \cong \Hom( A, \Hom( B, C)).
\end{equation}
Dieser Isomorphismus ist natürlich in $A, B, C$.

%******************************************************************************************************
\subsection{Definition einer natürlichen Transformation}
%******************************************************************************************************
Seien im Folgenden $\mathcal{C,D}$ zwei Kategorien und $\mathcal F,\mathcal G \colon \mathcal C \to \mathcal D$ zwei Funktoren mit $\mathcal C$ als Quell- und $\mathcal D$ als Ziel-Kategorie.

\begin{Definition}{Natürliche Transformation}
   Eine \textbf{natürliche Transformation} 
   \begin{equation}
      \alpha \colon \mathcal F \Rightarrow \mathcal G
   \end{equation}
   ist eine Schar von Homomorphismen in der Ziel-Kategorie $\mathcal{D}$ indiziert durch die Objekte der Quell-Kategorie $\mathcal C$:
   \begin{equation}
      \left (\alpha_X \colon \mathcal F(X) \to \mathcal G(X) \right )_{X \in \mathcal C}.
   \end{equation}
   mit denen jeweils die Bilder der Objekte der Quell-Kategorie verbunden werden. Diese müssen mit den Bildern der Morphismen in dem Sinne verträglich sein, dass alle Diagramme der folgenden Art, d.h. für alle $f \colon X \to Y$, mit $f \in \mathcal C$, kommutativ sind:
   \begin{equation}\label{kommutativ}
      \begin{tikzcd}
         \mathcal F(X) \arrow[r, "\alpha_X"] \arrow[d, "\mathcal F(f)"]
         & \mathcal  G(X) \arrow[d, "\mathcal G(f)"]\\
         \mathcal F(Y) \arrow[r, "\alpha_Y"]  
         & \mathcal  G(Y)\\
      \end{tikzcd}
   \end{equation}
\end{Definition}

\begin{Definition}{Natürlicher Isomorphismus}
   Ein \textbf{natürlicher Isomorphismus} ist eins der folgenden äquivalenten Dinge:
   \begin{itemize}
      \item Eine natürliche Transformation mit einem beidseitigen Inversen.
      \item Eine natürliche Transformation, deren Komponenten sämtlich Isomorphismen sind
      \item Ein Isomorphismus in der Funktorkategorie.
   \end{itemize}
\end{Definition}

%******************************************************************************************************
\subsection{Natürlich in $A, B, C$ ergibt 6 Funktoren}
%******************************************************************************************************
Natürlich in $A, B, C$ bedeutet, dass wir jede der drei Variablen als Argument eines Funktors auffassen und die jeweils anderen beiden fest lassen. In Wirklichkeit erhalten wir also eine Schar von Funktoren und das für die linke und die rechte Seite. Das macht folgende 6 Scharen von Funktoren die insgesamt 3 natürliche Isomorphismen ergeben. 
\begin{align}
   \alpha \colon \Hom( \_ \times B, C ) &\cong \Hom( \_, \Hom( B, C))\\
   \beta \colon \Hom( A \times \_, C ) &\cong \Hom( A, \Hom( \_, C))\\
   \gamma \colon Hom( A \times B, \_ ) &\cong \Hom( A, \Hom( B, \_)).
\end{align}
Hier stehen die $\_$ für die Variable des Funktors, während die $A, B, C$ für die Parameter der Schar stehen.

%******************************************************************************************************
\subsection{Zwei Kategorien}
%******************************************************************************************************
Was sind in unserem Beispiel die beiden Kategorien $\mathcal{C,D}$? Das Quell-Objekt ist, salopp gesagt, $A, B$ oder $C$. Damit ist Quell-Kategorie \textbf{Set}. Die Ziel-Kategorie ist die, in die $\Hom( ... )$ abbilden. Da $\Hom$-Mengen Mengen sind, ist die Ziel-Kategorie ebenfalls \textbf{Set}.

Beachte: Es ist etwas besonderes, dass wir $\Hom$-Mengen selber als Objekte einer Kategorie ansehen. Hierdurch erhaschen wir einen ersten Blick auf angereicherte Kategorien, bei denen die Hom's Objekte von (monoidalen) Kategorien sind.

%******************************************************************************************************
\subsection{Die Funktoren wirken auch auf Morphismen}
%******************************************************************************************************
Die oben aufgesagten 6 Funktoren wirken auch auf Morphismen. Wir werden uns die zwei des natürlichen Isomorphismus $\alpha$ stellvertretend für alle anderen genauer ansehen.

Sei
\begin{equation}
   f \colon X \to Y
\end{equation}
ein Morphismus in \textbf{Set}, also eine Funktion. Wir wollen nun wissen, was $\Hom( f \times B, C )$ ist. Dabei stellt sich zunächst die Frage, was $f \times B$ ist. Es ist der Funktor $\_ \times B$ angewandt auf $f$. (Dass  $\_ \times B$ ein Funktor ist, besprechen wir in einem anderen Video.)
\begin{align}
   f \times B \colon X \times B &\to Y \times B\\
   (x,b) &\mapsto (f(x), b).
\end{align}
Das müssen wir nun in den kontravarianten Hom-Funktor $\Hom( \_, C)$ einsetzen. Dieser wirkt auf Morphismen vermittels Prä-Komposition, hier zunächst mal auf $f$ angewandt:
\begin{align}
   \Hom(f, C) \colon \Hom( Y, C ) &\to \Hom( X, C)\\
   (h \colon Y \to C) &\mapsto (h \circ f \colon X \to C).
\end{align}
Jetzt wenden wir ihn auf $f \times B$ an:
\begin{align}
   \Hom(f \times B, C) \colon \Hom( Y \times B, C ) &\to \Hom( X \times B, C)\\
   (h \colon Y \times B \to C) &\mapsto (h \circ (f \times B) \colon X \times B \to C)\\
   ((y,b) \mapsto h(y, b)) &\mapsto ((x,b) \mapsto h(f(x),b)).
\end{align}

Nun kommen wir zu $\Hom( \_, \Hom( B, C))$.
\begin{align}
   \Hom( f, \Hom( B, C)) \colon \Hom(Y, \Hom( B, C)) &\to \Hom(X, \Hom( B, C))\\
   (h \colon Y \to \Hom( B, C) ) &\mapsto (h \circ f \colon X \to \Hom( B, C)).
\end{align}

Damit es wirklich Funktoren sind, müssen diese Abbildungen die Identitäten auf Identitäten abbilden und verträglich mit der Verknüpfung von Morphismen sein.

%******************************************************************************************************
\subsection{3 kommutative Diagramme}
%******************************************************************************************************
Das macht 3 kommutative Diagramme:
\begin{equation}
   \begin{tikzcd}
      \Hom( Y \times B, C ) \arrow[r, "\alpha_Y"] \arrow[d, "{\Hom( f \times B, C )}"]
      & \Hom( Y, \Hom( B, C)) \arrow[d, "{\Hom( f, \Hom( B, C))}"]\\
      \Hom( X \times B, C ) \arrow[r, "\alpha_X"]  
      & \Hom( X, \Hom( B, C))\\
   \end{tikzcd}
\end{equation}
\begin{equation}
   \begin{tikzcd}
      \Hom( A \times Y, C ) \arrow[r, "\beta_Y"] \arrow[d, "{\Hom( A \times f, C )}"]
      & \Hom( A, \Hom( Y, C)) \arrow[d, "{\Hom( A, \Hom( f, C))}"]\\
      \Hom( A \times X, C ) \arrow[r, "\beta_X"]  
      & \Hom( A, \Hom( X, C))\\
   \end{tikzcd}
\end{equation}
\begin{equation}
   \begin{tikzcd}
      \Hom( A \times B, X ) \arrow[r, "\gamma_X"] \arrow[d, "{\Hom( A \times B, f )}"]
      & \Hom( A, \Hom( B, X)) \arrow[d, "{\Hom( A, \Hom( B, f))}"]\\
      \Hom( A \times B, Y ) \arrow[r, "\gamma_Y"]  
      & \Hom( A, \Hom( B, Y))\\
   \end{tikzcd}.
\end{equation}

%******************************************************************************************************
\subsection{Ein paar Worte zum Wort natürlich}
%******************************************************************************************************
Wir sagen also in Wirklichkeit:
\begin{itemize}
   \item Wir haben zwei Kategorien
   \item In jeder der drei Variablen erhalten wir, bei fester Wahl der anderen beiden, zwei Funktoren (also insgesamt 6 und eigentlich sind es ganze Scharen von Funktoren)
   \item Dieser Funktor bildet auch Morphismen ab, und das so, dass Identitäten und Verknüpfungen erhalten bleiben
   \item Die Isomorphismen sind Isomorphismen in der Zielkategorie
   \item Die Verträglichkeitsbedingung für natürliche Transformationen ist für diese Isomorphismen und deren Inverse erfüllt
\end{itemize}

"`Natürlich"' ist ein kategorischer Begriff und erfordert eine präzise Angabe der vorliegenden Daten: Quell- und Ziel-Kategorie, Quell- und Ziel-Funktor sowie eine Schar von Morphismen.

Das harmlos daher kommende Wort sagt also ganz schön viel!

%******************************************************************************************************
\subsection{Und wozu ist das alles gut?}
%******************************************************************************************************
Seien $\m F, \m G \colon \m C \to \mathcal  D$ zwei Funktoren und $\alpha \m F \Rightarrow \m G$ eine natürliche Transformation sowie $(f \colon X \to Y) \in \mathcal C$ ein Morphismus in $\mathcal  C$. Dann können wir das kommutative Quadrat algebraisch als

\begin{equation}
   \alpha_Y \circ \m F(f) = \m G(f) \circ \alpha_X
\end{equation}
schreiben. Da ja $\m G = \alpha \m F$ ist können wir das knapper schreiben als
\begin{equation}
   \alpha( \m F(f) ) = (\alpha \m F)(f) \circ \alpha.
\end{equation}
Das heißt wir können die Bilder der Morphismen von $\mathcal C$ mit den Morphismen der natürlichen Transformation vertauschen.

Ähnlich können wir das Distributiv-Gesetz $a(x+y) = ax + ay$ schreiben als
\begin{equation}
   a \cdot ( +(x,y)) = +(a \cdot (x,y)).
\end{equation}
Vertauschbarkeit zeigt zum einen eine Verträglichkeit der zwei Operationen (irgendwie scheinen sie zusammen zu gehören) zum anderen erleichtert es das praktische Rechnen.

%******************************************************************************************************
%                                                                                                     *
\section{TODO}
%                                                                                                     *
%******************************************************************************************************
\begin{backup}
Noch zu erledigen sind
\begin{itemize}
   \item leer
\end{itemize}
\end{backup}

\begin{backup}
    (Zur Zeit nicht benötigter Inhalt)
\end{backup}

%******************************************************************************************************
%                                                                                                     *
\begin{thebibliography}{9}
%                                                                                                     *
%******************************************************************************************************
   \bibitem[Awodey2010]{Awodey}
      Steve Awode, \emph{Category Theory},
      2010 Oxford University Press, 978-0-19-923718-0 (ISBN)

   \bibitem[Bradley2020]{Bradley}
      Tai-Danae Bradley, \emph{Topology, A Categorical Approach},
      2020 MIT Press, 978-0-262-53935-7 (ISBN)

   \bibitem[LawvereSchanuel2009]{Lawvere}
      F. William Lawvere, Stephen H. Schanuel, \emph{Conceptual Mathematics: a First Introduction to Categories},
      2009 Cambridge University Press, 978-0-521-71916-2 (ISBN)


   \bibitem[MacLane1978]{MacLane}
      Saunders Mac Lane, \emph{Categories for the Working Mathematician},
      Springer-Verlag New York Inc., 978-0-387-98403-2 (ISBN)

   \bibitem[Rotman2009]{Rotman}
   	Joseph J. Rotman, \emph{An Introduction to Homological Algebra},
   	2009 Springer-Verlag New York Inc., 978-0-387-24527-0 (ISBN)
      
\end{thebibliography}

%******************************************************************************************************
%                                                                                                     *
\begin{large}
    \centerline{\textsc{Symbolverzeichnis}}
\end{large}
%                                                                                                     *
%******************************************************************************************************
\bigskip

\renewcommand*{\arraystretch}{1}

\begin{tabular}{ll}
    $A, B, C, \cdots, X, Y, Z$          & Objekte\\
    $\mathcal F,\mathcal G$             & Funktoren\\
    $f, g, h, r, s, \cdots$             & Homomorphismen\\
    $\mathcal C, \mathcal D, \mathcal E, \cdots$ & Kategorien\\
    \textbf{Set}                        & Die Kategorie der Mengen\\
    $\Hom( X, Y)$                       & Die Menge der Homomorphismen von $X$ nach $Y$\\
    $\alpha, \beta, \cdots$             & natürliche Transformationen\\
    $\mathcal C ^{\text{op}}$           & Duale Kategorie
\end{tabular}

\end{document}
