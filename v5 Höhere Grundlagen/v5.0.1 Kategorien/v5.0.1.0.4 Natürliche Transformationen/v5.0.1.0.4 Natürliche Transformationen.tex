%******************************************************** -*-LaTeX-*- ******************************
%                                                                                                  *
% v5.0.1.0.4 Natürliche Transformationen.tex                                                       *
%                                                                                                  *
% Copyright (C) 2023 Kategory GmbH \& Co. KG (joerg.kunze@kategory.de)                             *
%                                                                                                  *
% v5.0.1.0.4 Natürliche Transformationen is part of kategoryMathematik.                            *
%                                                                                                  *
% kategoryMathematik is free software: you can redistribute it and/or modify                       *
% it under the terms of the GNU General Public License as published by                             *
% the Free Software Foundation, either version 3 of the License, or                                *
% (at your option) any later version.                                                              *
%                                                                                                  *
% kategoryMathematik is distributed in the hope that it will be useful,                            *
% but WITHOUT ANY WARRANTY; without even the implied warranty of                                   *
% MERCHANTABILITY or FITNESS FOR A PARTICULAR PURPOSE.  See the                                    *
% GNU General Public License for more details.                                                     *
%                                                                                                  *
% You should have received a copy of the GNU General Public License                                *
% along with this program.  If not, see <http://www.gnu.org/licenses/>.                            *
%                                                                                                  *
%***************************************************************************************************

\documentclass[a4paper]{amsart}
% \documentclass[a4paper]{book}

%-----------------------------------------------------------------------------------------------------*
% package:                                                                                            *
%-----------------------------------------------------------------------------------------------------*
\usepackage{amssymb}
\usepackage{amsfonts}
\usepackage{amsmath}
\usepackage{amsthm}

\usepackage{mathabx}

\usepackage{a4wide} % a little bit smaller margins

\usepackage{graphicx}
\usepackage{hyperref}
\usepackage{algorithmic}
\usepackage{listings}
\usepackage{color}
\usepackage{colortbl}
\usepackage{sidecap}
\usepackage{comment}
\usepackage{tcolorbox}
\usepackage{collect}

\usepackage{upgreek}

% \usepackage{diagrams}

\usepackage[german]{babel}
\usepackage[none]{hyphenat}
\emergencystretch=4em

\usepackage[utf8]{inputenc} % to be able to use äöü as characters in text
\usepackage[T1]{fontenc} % to be able to use äöü in lables
\usepackage{lmodern}     % to avoid pixelation introduced by fontenc

\usepackage{hyperref}

\usepackage{tikz}
\usepackage{tikz-cd}
\usetikzlibrary{babel}

%-----------------------------------------------------------------------------------------------------*
% theorem:                                                                                            *
%-----------------------------------------------------------------------------------------------------*
\theoremstyle{definition}
\newtheorem{theorem}{Theorem}[subsection]

\newcommand{\myTheorem}[1]{%
  \newtheorem{jk#1}[theorem]{#1}
  \newenvironment{#1}[1]{%
    \expandafter\begin{jk#1} \expandafter\label{#1:##1}\textbf{(##1):}
  }{%
    \expandafter\end{jk#1}
  }
}

\myTheorem{Definition}
\myTheorem{Proposition}
\myTheorem{Theorem}
\myTheorem{Example}
\myTheorem{Remark}

\definecollection{jkjkFrage}
\newtheorem{jkFrage}[theorem]{Frage}
\newenvironment{Frage}[1]{%
  \expandafter\begin{jkFrage} \expandafter\label{Frage:#1}\textbf{(#1):}
  \begin{collect}{jkjkFrage}{}{}
    \item \ref{Frage:#1} #1
  \end{collect}
}{%
  \expandafter\end{jkFrage}
}

\newcommand{\myRef}[2]{[#1 \ref{#1:#2}, ``#2'']}

\renewcommand{\proofname}{Beweis}

%-----------------------------------------------------------------------------------------------------*
% operator:                                                                                           *
%-----------------------------------------------------------------------------------------------------*
\DeclareMathOperator{\End}{End}
\DeclareMathOperator{\Ker}{Ker}
\DeclareMathOperator{\Mat}{Mat}
\DeclareMathOperator{\rank}{rank}
\DeclareMathOperator{\ggT}{ggT}
\DeclareMathOperator{\len}{len}
\DeclareMathOperator{\ord}{ord}
\DeclareMathOperator{\kgV}{kgV}
\DeclareMathOperator{\id}{id}
\DeclareMathOperator{\red}{red}
\DeclareMathOperator{\supp}{supp}
\DeclareMathOperator{\Bild}{Bild}
\DeclareMathOperator{\Rang}{Rang}
\DeclareMathOperator{\Det}{Det}
\DeclareMathOperator{\Hom}{Hom}

\DeclareMathOperator{\sub}{sub}
\DeclareMathOperator{\blk}{blk}
\DeclareMathOperator{\minimal}{minimal}
\DeclareMathOperator{\maximal}{maximal}

\definecolor{mygreen}{rgb}{0,0.6,0}
\definecolor{mygray}{rgb}{0.5,0.5,0.5}
\definecolor{mymauve}{rgb}{0.58,0,0.82}

\lstset{ %
  backgroundcolor=\color{white},   % choose the background color
  basicstyle=\ttfamily\footnotesize,        % size of fonts used for the code
  breaklines=true,                 % automatic line breaking only at whitespace
  captionpos=b,                    % sets the caption-position to bottom
  commentstyle=\color{mygreen},    % comment style
  escapeinside={\%*}{*)},          % if you want to add LaTeX within your code
  keywordstyle=\color{blue},       % keyword style
  stringstyle=\color{mymauve},     % string literal style
  frame=single
}

\setcounter{MaxMatrixCols}{20}

%******************************************************************************************************
%                                                                                                     *
% definition:                                                                                         *
%                                                                                                     *
%******************************************************************************************************
\newcommand{\R}{\ensuremath{\mathbb{ R }}}
\newcommand{\Q}{\ensuremath{\mathbb{ Q }}}
\newcommand{\Z}{\ensuremath{\mathbb{ Z }}}
\newcommand{\N}{\ensuremath{\mathbb{ N }}}
\newcommand{\C}{\ensuremath{\mathbb{ C }}}
\newcommand{\A}{\ensuremath{\mathbb{ A }}}
\newcommand{\F}{\ensuremath{\mathbb{ F }}}
\newcommand{\K}{\ensuremath{\mathbb{ K }}}
\newcommand{\Pb}{\ensuremath{\mathbb{ P }}}

\newcommand{\M}{\ensuremath{\mathcal{ M }}}
\newcommand{\V}{\ensuremath{\mathcal{ V }}}

\newcommand{\AAA}{\ensuremath{\mathcal{ A }}}
\newcommand{\BB}{\ensuremath{\mathcal{ B }}}
\newcommand{\CC}{\ensuremath{\mathcal{ C }}}
\newcommand{\EE}{\ensuremath{\mathcal{ E }}}
\newcommand{\KK}{\ensuremath{\mathcal{ K }}}
\newcommand{\MM}{\ensuremath{\mathcal{ M }}}
\newcommand{\PP}{\ensuremath{\mathcal{ P }}}
\newcommand{\ZZ}{\ensuremath{\mathcal{ Z }}}

\newcommand{\imporant}[1]{ \textcolor{red}{\textbf{#1}} }

\newcommand{\bb}[1]{\mathbf{#1}}
\newcommand{\balpha}{\boldsymbol{\upalpha}}
\newcommand{\bbeta}{\boldsymbol{\upbeta}}
\newcommand{\bgamma}{\boldsymbol{\upgamma}}
\newcommand{\bdelta}{\boldsymbol{\delta}}
\newcommand{\bmu}{\boldsymbol{\upmu}}

\newcommand{\z}[1]{\Z_{#1}}
\newcommand{\e}[1]{\z{#1}^*}
\newcommand{\q}[1]{(\e{#1})^2}

\excludecomment{book}
\excludecomment{example}
\excludecomment{backup}

\begin{document}

%******************************************************************************************************
%                                                                                                     *
\begin{titlepage}
%                                                                                                     *
%******************************************************************************************************
% \vspace*{\fill}
\centering
{\huge
(Höhere Grundlagen) Kategorien\\[1cm]
\textbf{v5.0.1.0.4 Natürliche Transformationen}
}\\[1cm]

\textbf{Kategory GmbH \& Co. KG}\\
Präsentiert von Jörg Kunze\\
Copyright (C) 2023 Kategory GmbH \& Co. KG

\end{titlepage}

%\clearpage
%\setcounter{page}{2}
%
%\tableofcontents

\newpage

%******************************************************************************************************
%                                                                                                     *
\section*{Beschreibung}
%                                                                                                     *
%******************************************************************************************************

%******************************************************************************************************
\subsection*{Inhalt}
%******************************************************************************************************
Natürliche Transformationen sind Morphismen zwischen Funktoren. Dabei wird der Quell-Funktor mit den Mitteln, nämlich Morphismen, der Ziel-Kategorie in den Ziel-Funktor überführt. Wir können also sagen, dass die beiden Funktoren "`innerhalb"' der Ziel-Kategorie vergleichbar sind.

Diese sind uns bereits beim Thema Morphismen zwischen Morphismen aufgefallen als ein oft  (aber nicht immer) wiederkehrendes Prinzip.

Ein Funktor ist das Finden der Form oder Figur der Quell-Kategorie in der Ziel-Kategorie. Zwei Funktoren der selben Quell-Kategorie entspricht also dem Finden von zwei Figuren der selben Form in der Ziel-Kategorie. Falls eine natürliche Transformation von dem einen Funktor in den andern existiert, dann können wir die eine Figur innerhalb der Ziel-Kategorie in die andere überführen.

So wie wir zwei homotope Wege innerhalb des Trägerraumes ineinander überführen können. Natürliche Transformationen und Homotopien liegen also intuitiv sehr nahe bei einander.

Algebraisch ist eine natürliche Transformation eine Familie von Morphismen in der Ziel-Kategorie und zwar für jedes Objekt der Quell-Kategorie einer. Diese Morphismen müssen mit den Bilder der Morphismen der Quell-Kategorie in der Ziel-Kategorie kommutieren. 

Das führt dazu, dass die Funktoren von C nach D als OBJEKTE und die natürlichen Transformationen als HOMOMORPHISMEN die Axiome einer Kategorie erfüllen. Wir nennen die Funktorkategorie D hoch C. 

Somit können wir uns natürliche Transformationen auch als Kategorien-Homomorphismen denken. Dies ist allerdings insofern problematisch, als dann einige Objekte keine Mengen (also mathematische Objekte) sind. Wollen wir diese Intuition präzisieren, so bilden wir eine Kategorie der kleiner Kategorien. Dann stimmt auch dieser Begriff ohne Probleme.

Ein natürlicher Isomorphismus zwischen zwei Funktoren ist eine natürliche Transformation mit einer weiteren natürlichen Transformation, die das Inverse ist. Dem entspricht dann dem Finden von zwei identischen Figuren, die wir innerhalb der Kategorie ineinander überführen können.

%******************************************************************************************************
\subsection*{Präsentiert}
%******************************************************************************************************
Von Jörg Kunze

%******************************************************************************************************
\subsection*{Voraussetzungen}
%******************************************************************************************************
Axiome der Kategorien, Funktor.

%******************************************************************************************************
\subsection*{Text}
%******************************************************************************************************
Der Begleittext als PDF und als LaTeX findet sich unter
{\tiny
   \url{https://github.com/kategory/kategoryMathematik/tree/main/v5%20H%C3%B6here%20Grundlagen/v5.0.1%20Kategorien/v5.0.1.0.4%20Nat%C3%BCrliche%20Transformationen}
}

%******************************************************************************************************
\subsection*{Meine Videos}
%******************************************************************************************************
Siehe auch in den folgenden Videos:\\
v5.0.1.0.3 (Höher) Kategorien - Funktoren\\
\url{https://youtu.be/Ojf5LQGeyOU}\\
v5.0.1.0.3.5 (Höher) Kategorien - Kategorien von Homomorphismen\\
\url{https://youtu.be/v1F5BFH8nbo}\\
v5.0.1.0.3 (Höher) Kategorien - Funktoren\\
\url{https://youtu.be/Ojf5LQGeyOU}\\

%******************************************************************************************************
\subsection*{Quellen}
%******************************************************************************************************
Siehe auch in den folgenden Seiten:\\
\url{https://de.wikipedia.org/wiki/Nat%C3%BCrliche_Transformation}\\
\url{https://ncatlab.org/nlab/show/natural+transformation}\\
\url{https://en.wikipedia.org/wiki/Category_of_small_categories}\\
\url{https://ncatlab.org/nlab/show/small+category}\\
\url{https://de.wikipedia.org/wiki/Homotopie}\\
\url{https://ncatlab.org/nlab/show/natural+isomorphism}\\
\url{https://www.youtube.com/watch?v=Ir-DiecNUmA}
%******************************************************************************************************
\subsection*{Buch}
%******************************************************************************************************
Grundlage ist folgendes Buch:\\
"`Categories for the Working Mathematician"'\\
Saunders Mac Lane\\
1998 | 2nd ed. 1978\\
Springer-Verlag New York Inc.\\
978-0-387-98403-2 (ISBN)\\
{\tiny
   \url{https://www.amazon.de/Categories-Working-Mathematician-Graduate-Mathematics/dp/0387984038}}\\
\\
"`Topology, A Categorical Approach"'\\
Tai-Danae Bradley\\
2020 MIT Press\\
978-0-262-53935-7 (ISBN)\\ 
{\tiny
\url{https://www.lehmanns.de/shop/mathematik-informatik/52489766-9780262539357-topology}}\\
\\
Einige gut Erklärungen finden sich auch in den Einführenden Kapitel von\\
"`An Introduction to Homological Algebra"'\\
Joseph J. Rotman\\
2009 Springer-Verlag New York Inc.\\
978-0-387-24527-0 (ISBN)\\ 
{\tiny \url{https://www.lehmanns.de/shop/mathematik-informatik/6439666-9780387245270-an-introduction-to-homological-algebra}}\\

%******************************************************************************************************
\subsection*{Lizenz}
%******************************************************************************************************
Dieser Text und das Video sind freie Software. Sie können es unter den Bedingungen der 
GNU General Public License, wie von der Free Software Foundation veröffentlicht, weitergeben 
und/oder modifizieren, entweder gemäß Version 3 der Lizenz oder (nach Ihrer Option) jeder späteren Version.

Die Veröffentlichung von Text und Video erfolgt in der Hoffnung, dass es Ihnen von Nutzen sein wird, 
aber OHNE IRGENDEINE GARANTIE, sogar ohne die implizite Garantie der MARKTREIFE oder der 
VERWENDBARKEIT FÜR EINEN BESTIMMTEN ZWECK. Details finden Sie in der GNU General Public License.

Sie sollten ein Exemplar der GNU General Public License zusammen mit diesem Text erhalten haben 
(zu finden im selben Git-Projekt). 
Falls nicht, siehe \url{http://www.gnu.org/licenses/}.

\subsection*{Das Video}
%******************************************************************************************************
Das Video hierzu ist zu finden unter 
{\tiny
   \url{huhu}
}

%******************************************************************************************************
%                                                                                                     *
\section{Natürliche Transformation}
%                                                                                                     *
%******************************************************************************************************
Seien im Folgenden $\mathcal{C,D}$ zwei Kategorien und $\mathcal F,\mathcal G \colon \mathcal C \to \mathcal D$ zwei Funktoren mit $\mathcal C$ als Quell- und $\mathcal D$ als Zeil-Kategorie.

%******************************************************************************************************
\subsection{Definition einer natürlichen Transformation}
%******************************************************************************************************
\begin{Definition}{Natürliche Transformation}
   Eine \textbf{natürliche Transformation} 
   \begin{equation}
      \alpha \colon \mathcal F \Rightarrow \mathcal G
   \end{equation}
   ist eine Schar von Homomorphismen in der Ziel-Kategorie $\mathcal{D}$ mit Index die Objekte der Quell-Kategorie $\mathcal C$:
   \begin{equation}
      \left (\alpha_X \colon \mathcal F(X) \to \mathcal G(X) \right )_{X \in \mathcal C}.
   \end{equation}
   mit denen jeweils die Bilder der Objekte der Quell-Kategorie verbunden werden. Diese müssen mit den Bildern der Morphismen in dem Sinne verträglich sein, dass alle Diagramme der folgenden Art, d.h. für alle $f \colon X \to Y$, mit $f \in \mathcal C$, kommutativ sind:
   \begin{equation}\label{kommutativ}
      \begin{tikzcd}
         \mathcal F(X) \arrow[r, "\alpha_X"] \arrow[d, "\mathcal F(f)"]
            & \mathcal  G(X) \arrow[d, "\mathcal G(f)"]\\
         \mathcal F(Y) \arrow[r, "\alpha_Y"]  
            & \mathcal  G(Y)\\
      \end{tikzcd}
   \end{equation}
\end{Definition}

Eine Schar ist nur ein anderes Wort für Familie oder Funktion. Hier müssen wir auch Klassen-Funktionen zulassen, da die Klasse der Objekte der Quell-Kategorie durchaus eine echte Klasse sein kann. In unseren Gemälden sagen wir das so:

\begin{equation}
   \begin{tikzcd}
      \mathcal C 
         \arrow[r, bend left, "\mathcal F"{name=0}] 
         \arrow[r, bend right, swap, "\mathcal G"{name=1}] 
         \arrow[Rightarrow, from=0, to=1,shorten >= 6pt, yshift = -3pt, "\alpha"{yshift=3pt}]
      & \mathcal D
   \end{tikzcd}
\end{equation}
Wir haben nun drei Arten von Pfeilen: die zwischen Objekten, die zwischen Kategorien und die zwischen Funktoren. Die Pfeile zwischen Funktoren sind dabei eigentlich Scharen von Pfeilen zwischen Objekten.

%******************************************************************************************************
\subsection{Zweimaliges Finden einer Form}
%******************************************************************************************************
Sei $\mathcal C$ die Quell-Kategorie der Funktoren $\mathcal F$ und $\mathcal G$, die die selbe Ziel-Kategorie $\mathcal D$ haben. Dies können wir so beschreiben, dass $\mathcal C$ eine Form oder Figur ist, die wir in $\mathcal D$ wiederfinden. Bei zwei Funktoren handelt es sich also um ein zweimaliges Finden einer Form.

Sei z.B. $\mathcal C$ folgende Kategorie mit drei Objekten (wir lassen die Identitäten weg):
\begin{equation}
   \begin{tikzcd}
      X \arrow[dd] \arrow[dr] \\ 
                 & Y \arrow[dl]\\
      Z   
   \end{tikzcd}
\end{equation}

Die Ziel-Kategorie hat die Objekte $A,B,C,D,E,F$ und sei wie folgt. 
\begin{equation}
   \begin{tikzcd}
     A \arrow[dd] \arrow[dr] \arrow[dddrr, bend right] \\ 
      &B \arrow[dl] \arrow[ddr] &D \arrow[dd] \arrow[dr]\\
      C &&& E \arrow[dl]\\
     &&F
   \end{tikzcd}
\end{equation}

Betrachten wir nun die folgenden zwei Funktoren $\color{blue} \mathcal F$ und $\color{red} \mathcal G$
\begin{equation}
   \begin{tikzcd}
      X \arrow[dd] \arrow[dr] \arrow[rrr, blue, dashed, bend left, "\mathcal F"] \arrow[rrrrrd, red, dashed, bend right] &&& A \arrow[dd, blue] \arrow[dr, blue] \arrow[dddrr, bend right]\\ 
      & Y \arrow[dl] \arrow[rrr, blue, dashed, bend left] \arrow[rrrrrd, red, dashed, bend right] &&&B \arrow[dl, blue] \arrow[ddr] &D \arrow[dd, red] \arrow[dr, red]\\
      Z \arrow[rrr, blue, dashed, bend left] \arrow[rrrrrd, red, dashed, bend right, "\mathcal G"] &&&C &&& E \arrow[dl, red]\\
      &&&&&F
   \end{tikzcd}
\end{equation}
Diese finden beide die Dreiecksform als Teil der Ziel-Kategorie. Wir können aber keine natürliche Transformation finden, da Morphismen von $A \to D$, $B \to E$ und $C \to F$ fehlen. Es gibt ein Loch in der Ziel-Kategorie, welches das Überführen der einen gefundenen Form in die zweite innerhalb der Ziel-Kategorie verhindert.

Neben den Lücken, weil es gar keinen Morphismus gibt, kann es auch noch das Problem geben, dass die entstehenden Diagramme \eqref{kommutativ} nicht kommutativ sind.

Geben wir unserer Ziel-Kategorie wie im nächsten Gemälde gezeigt vier Morphismen, so existiert eine natürliche Transformation 
$\color{green} \mathcal F \Rightarrow \mathcal G$. 
\begin{equation}
	\begin{tikzcd}
		X \arrow[dd] \arrow[dr] \arrow[rrr, blue, dashed, bend left, "\mathcal F"{name=0}] \arrow[rrrrrd, red, dashed, bend right] &&& A \arrow[dd, blue] \arrow[dr, blue] \arrow[dddrr, bend right] \arrow[drr, green] \arrow[ddrrr, bend left]\\ 
		& Y \arrow[dl] \arrow[rrr, blue, dashed, bend left] \arrow[rrrrrd, red, dashed, bend right] &&&B \arrow[dl, blue] \arrow[ddr] \arrow[drr, green] &D \arrow[dd, red] \arrow[dr, red]\\
		Z \arrow[rrr, blue, dashed, bend left] \arrow[rrrrrd, red, dashed, bend right, "\mathcal G"{name=1}] &&&C  \arrow[drr, green] &&& E \arrow[dl, red]\\
		&&&&&F
		\arrow[Rightarrow, green, from=0, to=1,shorten >= 6pt, yshift = -3pt, "\alpha"{yshift=3pt}]
	\end{tikzcd}
\end{equation}

%******************************************************************************************************
\subsection{Das erinnert sehr an Homotopie}
%******************************************************************************************************
Das Loch oder die Lücke, die verhindert, dass wir eine Figur innerhalb des Ziels in die anderen überführen können, erinnert sehr an nicht homotope Kurven.

\begin{tikzpicture}
	\draw[fill=gray!30,even odd rule] (0,0) circle (3cm) (0,0) circle (1cm) 
		node {L};

	\draw[blue]  (0,2) circle (0.5 cm) node[shift={(  0:0.75)},blue ]{$A$};
	\draw[green] (2,0) circle (0.5 cm) node[shift={( 90:0.75)},green]{$B$};
	\draw[red]   (0,0) circle (1.25cm) node[shift={(180:1.50)},red  ]{$C$};
\end{tikzpicture}

In dieser Situation, in der der topologische Raum der Betrachtung der graue Bereich ist, kann der Quell-Kreis $\color{blue}A$ in den Ziel-Kreis $\color{green}B$ innerhalb der grauen Fläche deformiert werden. $\color{blue}A$ kann aber nicht in $\color{red}C$ überführt werden, da das Loch $L$ im Weg ist. $\color{blue}A$ und $\color{green}B$ sind homotop $\color{blue}A$ und $\color{red}C$ nicht.

Der fehlende Raum im Loch $L$ entspricht den fehlenden Morphismen, falls es keine natürliche Transformation gibt.

TODO: Funktor Kategorie

TODO: Ein paar Worte zum Wort natürlich

TODO: 

%******************************************************************************************************
%                                                                                                     *
\section{TODO}
%                                                                                                     *
%******************************************************************************************************
Noch zu erledigen sind
\begin{itemize}
   \item Hom(...) in $D^C$ schreiben wir Nat(...)
   \item Wie schreiben wir $D^C$ noch? Wenn es noch eine weitere Schreibweise gibt.
   \item Beispiel homotope und nicht homotope Kreise
   \item Natürlichkeit ist ein kategorieller Begriff und erfordert eine präzise Angabe der vorliegenden Daten: Quell, Ziel, Funktor und Schar von Morphismen.
   \item Kategorien von Prägarben
\end{itemize}



\begin{backup}
    (Zur Zeit nicht benötigter Inhalt)
\end{backup}

%******************************************************************************************************
%                                                                                                     *
\begin{thebibliography}{9}
%                                                                                                     *
%******************************************************************************************************

   \bibitem[MacLane1978]{MacLane}
      Saunders Mac Lane, \emph{Categories for the Working Mathematician},
      Springer-Verlag New York Inc., 978-0-387-98403-2 (ISBN)
      
   \bibitem[Bradley2020]{Bradley}
      Tai-Danae Bradley, \emph{Topology, A Categorical Approach},
      2020 MIT Press, 978-0-262-53935-7 (ISBN)

   \bibitem[Rotman2009]{Rotman}
   	Joseph J. Rotman, \emph{An Introduction to Homological Algebra},
   	2009 Springer-Verlag New York Inc., 978-0-387-24527-0 (ISBN)
      
\end{thebibliography}

%******************************************************************************************************
%                                                                                                     *
\begin{large}
    \centerline{\textsc{Symbolverzeichnis}}
\end{large}
%                                                                                                     *
%******************************************************************************************************
\bigskip

\renewcommand*{\arraystretch}{1}

\begin{tabular}{ll}
    $A, B, C, \cdots, X, Y, Z$          & Objekte\\
    $F,G,L,R$ & Funktoren\\
    $f, g, h, r, s, \cdots$   & Homomorphismen\\
    $\mathcal C, \mathcal D, \mathcal E, \cdots$ & Kategorien\\
    \textbf{Set} & Die Kategorie der Mengen\\
    $\Hom( X, Y)$ & Die Menge der Homomorphismen von $X$ nach $Y$\\
    $\K$         &Ein Körper\\
    $V, W, \cdots$ &$\K$-Vektorräume\\
    $\alpha, \beta, \cdots$ & Skalare des Vektorraums also Elemente aus $\K$\\
    $V^*, W^*, \cdots$ & Dualräume
\end{tabular}

\end{document}
