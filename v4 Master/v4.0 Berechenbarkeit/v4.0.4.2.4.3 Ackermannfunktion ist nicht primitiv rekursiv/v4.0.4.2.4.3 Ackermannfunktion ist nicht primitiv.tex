%******************************************************** -*-LaTeX-*- ******************************
%                                                                                                  *
% v4.0.4.2.4.3 (Master) Berechenbarkeit - Ackermannfunktion ist nicht primitiv rekursiv.tex        *
%                                                                                                  *
% Copyright (C) 2023 Kategory GmbH \& Co. KG (joerg.kunze@kategory.de)                             *
%                                                                                                  *
% v4.0.4.2.4.3 is part of kategoryMathematik.                                                      *
%                                                                                                  *
% kategoryMathematik is free software: you can redistribute it and/or modify                       *
% it under the terms of the GNU General Public License as published by                             *
% the Free Software Foundation, either version 3 of the License, or                                *
% (at your option) any later version.                                                              *
%                                                                                                  *
% kategoryMathematik is distributed in the hope that it will be useful,                            *
% but WITHOUT ANY WARRANTY; without even the implied warranty of                                   *
% MERCHANTABILITY or FITNESS FOR A PARTICULAR PURPOSE.  See the                                    *
% GNU General Public License for more details.                                                     *
%                                                                                                  *
% You should have received a copy of the GNU General Public License                                *
% along with this program.  If not, see <http://www.gnu.org/licenses/>.                            *
%                                                                                                  *
%***************************************************************************************************

\documentclass[a4paper]{amsart}
% \documentclass[a4paper]{book}

%-----------------------------------------------------------------------------------------------------*
% package:                                                                                            *
%-----------------------------------------------------------------------------------------------------*
\usepackage{amssymb}
\usepackage{amsfonts}
\usepackage{amsmath}
\usepackage{amsthm}

\usepackage{mathabx}

\usepackage{a4wide} % a little bit smaller margins

\usepackage{graphicx}
\usepackage{hyperref}
\usepackage{algorithmic}
\usepackage{listings}
\usepackage{color}
\usepackage{colortbl}
\usepackage{sidecap}
\usepackage{comment}
\usepackage{tcolorbox}
\usepackage{collect}

\usepackage{upgreek}


% \usepackage{diagrams}

\usepackage[german]{babel}
\usepackage[none]{hyphenat}
\emergencystretch=4em

\usepackage[utf8]{inputenc} % to be able to use äöü as characters in text
\usepackage[T1]{fontenc} % to be able to use äöü in lables
\usepackage{lmodern}     % to avoid pixelation introduced by fontenc

\usepackage{hyperref}

%-----------------------------------------------------------------------------------------------------*
% theorem:                                                                                            *
%-----------------------------------------------------------------------------------------------------*
\theoremstyle{definition}
\newtheorem{theorem}{Theorem}[subsection]

\newcommand{\myTheorem}[1]{%
  \newtheorem{jk#1}[theorem]{#1}
  \newenvironment{#1}[1]{%
    \expandafter\begin{jk#1} \expandafter\label{#1:##1}\textbf{(##1):}
  }{%
    \expandafter\end{jk#1}
  }
}

\myTheorem{Definition}
\myTheorem{Proposition}
\myTheorem{Theorem}
\myTheorem{Example}
\myTheorem{Remark}

\definecollection{jkjkFrage}
\newtheorem{jkFrage}[theorem]{Frage}
\newenvironment{Frage}[1]{%
  \expandafter\begin{jkFrage} \expandafter\label{Frage:#1}\textbf{(#1):}
  \begin{collect}{jkjkFrage}{}{}
    \item \ref{Frage:#1} #1
  \end{collect}
}{%
  \expandafter\end{jkFrage}
}

\newcommand{\myRef}[2]{[#1 \ref{#1:#2}, ``#2'']}

\renewcommand{\proofname}{Beweis}

%-----------------------------------------------------------------------------------------------------*
% operator:                                                                                           *
%-----------------------------------------------------------------------------------------------------*
\DeclareMathOperator{\End}{End}
\DeclareMathOperator{\Ker}{Ker}
\DeclareMathOperator{\Mat}{Mat}
\DeclareMathOperator{\rank}{rank}
\DeclareMathOperator{\ggT}{ggT}
\DeclareMathOperator{\len}{len}
\DeclareMathOperator{\ord}{ord}
\DeclareMathOperator{\kgV}{kgV}
\DeclareMathOperator{\id}{id}
\DeclareMathOperator{\red}{red}
\DeclareMathOperator{\supp}{supp}
\DeclareMathOperator{\Bild}{Bild}
\DeclareMathOperator{\Rang}{Rang}
\DeclareMathOperator{\Det}{Det}

\DeclareMathOperator{\sub}{sub}
\DeclareMathOperator{\blk}{blk}
\DeclareMathOperator{\minimal}{minimal}
\DeclareMathOperator{\maximal}{maximal}

\definecolor{mygreen}{rgb}{0,0.6,0}
\definecolor{mygray}{rgb}{0.5,0.5,0.5}
\definecolor{mymauve}{rgb}{0.58,0,0.82}

\lstset{ %
  backgroundcolor=\color{white},   % choose the background color
  basicstyle=\ttfamily\footnotesize,        % size of fonts used for the code
  breaklines=true,                 % automatic line breaking only at whitespace
  captionpos=b,                    % sets the caption-position to bottom
  commentstyle=\color{mygreen},    % comment style
  escapeinside={\%*}{*)},          % if you want to add LaTeX within your code
  keywordstyle=\color{blue},       % keyword style
  stringstyle=\color{mymauve},     % string literal style
  frame=single
}

\setcounter{MaxMatrixCols}{20}

%******************************************************************************************************
%                                                                                                     *
% definition:                                                                                         *
%                                                                                                     *
%******************************************************************************************************
\newcommand{\R}{\ensuremath{\mathbb{ R }}}
\newcommand{\Q}{\ensuremath{\mathbb{ Q }}}
\newcommand{\Z}{\ensuremath{\mathbb{ Z }}}
\newcommand{\N}{\ensuremath{\mathbb{ N }}}
\newcommand{\C}{\ensuremath{\mathbb{ C }}}
\newcommand{\A}{\ensuremath{\mathbb{ A }}}
\newcommand{\F}{\ensuremath{\mathbb{ F }}}
\newcommand{\K}{\ensuremath{\mathbb{ K }}}
\newcommand{\Pb}{\ensuremath{\mathbb{ P }}}

\newcommand{\M}{\ensuremath{\mathcal{ M }}}
\newcommand{\V}{\ensuremath{\mathcal{ V }}}

\newcommand{\AAA}{\ensuremath{\mathcal{ A }}}
\newcommand{\BB}{\ensuremath{\mathcal{ B }}}
\newcommand{\CC}{\ensuremath{\mathcal{ C }}}
\newcommand{\EE}{\ensuremath{\mathcal{ E }}}
\newcommand{\KK}{\ensuremath{\mathcal{ K }}}
\newcommand{\MM}{\ensuremath{\mathcal{ M }}}
\newcommand{\PP}{\ensuremath{\mathcal{ P }}}
\newcommand{\ZZ}{\ensuremath{\mathcal{ Z }}}

\newcommand{\imporant}[1]{ \textcolor{red}{\textbf{#1}} }

\newcommand{\bb}[1]{\mathbf{#1}}
\newcommand{\balpha}{\boldsymbol{\upalpha}}
\newcommand{\bbeta}{\boldsymbol{\upbeta}}
\newcommand{\bgamma}{\boldsymbol{\upgamma}}
\newcommand{\bdelta}{\boldsymbol{\delta}}
\newcommand{\bmu}{\boldsymbol{\upmu}}

\newcommand{\z}[1]{\Z_{#1}}
\newcommand{\e}[1]{\z{#1}^*}
\newcommand{\q}[1]{(\e{#1})^2}

\excludecomment{book}
\excludecomment{example}
\excludecomment{backup}

\begin{document}

%******************************************************************************************************
%                                                                                                     *
\begin{titlepage}
%                                                                                                     *
%******************************************************************************************************
% \vspace*{\fill}
\centering
{\huge
(Master) Berechenbarkeit\\[1cm]
\textbf{v4.0.4.2.4.3 Ackermannfunktion ist nicht primitiv rekursiv}
}\\[1cm]

\textbf{Kategory GmbH \& Co. KG}\\
Pr\"asentiert von Jörg Kunze

\end{titlepage}

%\clearpage
%\setcounter{page}{2}
%
%\tableofcontents

\newpage

%******************************************************************************************************
%                                                                                                     *
\section*{Beschreibung}
%                                                                                                     *
%******************************************************************************************************

%******************************************************************************************************
\subsection*{Inhalt}
%******************************************************************************************************
Die Ackermannfunktion ist nicht primitiv rekursiv. Wir haben aber schon gesehen, dass wir sie programmieren
können. Das ergibt den Kern dieser Vorlesung: \imporant{der Begriff der primitiven Rekursion reicht 
nicht aus, um
den Begriff der Berechenbarkeit zu formalisieren}. Und das ist der Grund, warum wir die $\mu$-Rekursion
eingeführt haben. Und es ist ein Grund, sich mit der Ackermannfunktion zu beschäftigen.

Für den Beweis zeigen wir, dass die Ackermannfunktion jede primitiv-rekursive Funktion majorisiert. Da
eine Funktion sich nicht selbst majorisieren kann, ist das gewünschte gezeigt. Wir tun dies induktiv
über die Definition von "`primitiv rekursiv"'.

Durch den Beweis bekommen wir auch ein Mittel, um aus dem Aufbau einer primitiv-rekursiven Funktionsdefinition
den Parameter $k$ zu errechnen, mit dem $A( k, \_ )$ größer ist als die definierte Funktion.

%******************************************************************************************************
\subsection*{Präsentiert}
%******************************************************************************************************
Von Jörg Kunze

%******************************************************************************************************
\subsection*{Voraussetzungen}
%******************************************************************************************************
Primitiv rekursive Funktionen, Ackermannfunktion und ihre Eigenschaften.

%******************************************************************************************************
\subsection*{Text}
%******************************************************************************************************
Der Begleittext als PDF und als LaTeX findet sich unter
\url{https://github.com/kategory/kategoryMathematik/tree/main/v4%20Master/v4.0%20Berechenbarkeit/v4.0.4.2.4.3%20Ackermannfunktion%20ist%20nicht%20primitiv%20rekursiv}.

%******************************************************************************************************
\subsection*{Quelltext}
%******************************************************************************************************
Der JavaScript-Quelltext ist hier zu finden:
{\tiny
	\url{https://github.com/kategory/kategoryEducation/blob/main/mathematics/computabilityInJavaScript/v4.0.4.2.4.3.2%20(Master)%20Berechenbarkeit%20-%20Ackermannfunktion%20ist%20nicht%20primitiv%20rekursiv%20in%20JavaScript.js}
}

%******************************************************************************************************
\subsection*{Meine Videos}
%******************************************************************************************************
Siehe auch in den folgenden Videos:\\
v4.0.4.2.4.2 (Master) Berechenbarkeit - Ackermannfunktion Eigenschaften\\
 \url{https://youtu.be/MzDubMBlQ20}\\
 \\
v4.0.4.2.4.1 (Master) Berechenbarkeit - Ackermannfunktion ist mu-rekursiv\\
 \url{https://youtu.be/cbZ8EJ4U-tg}\\
 \\
v4.0.4.2.4 (Master) Berechenbarkeit - Ackermannfunktionen\\
 \url{https://youtu.be/XE-7YdHi2Vw}\\
 \\
v4.0.4.2 (Master) Berechenbarkeit - Rekursive Funktionen\\
 \url{https://youtu.be/tFEn2GoLLEQ}

%******************************************************************************************************
\subsection*{Quellen}
%******************************************************************************************************
Siehe auch in den folgenden Seiten:\\
\url{https://de.wikipedia.org/wiki/Ackermannfunktion}\\
\url{https://en.wikipedia.org/wiki/Ackermann\_function}\\
\url{https://planetmath.org/ackermannfunctionisnotprimitiverecursive}\\
\url{https://planetmath.org/PropertiesOfAckermannFunction}

%******************************************************************************************************
\subsection*{Buch}
%******************************************************************************************************
Grundlage ist folgendes Buch:\\
Computability\\
A Mathematical Sketchbook\\
Douglas S. Bridges\\
Springer-Verlag New York Inc. 2013\\
978-1-4612-6925-0 (ISBN)

%******************************************************************************************************
\subsection*{Lizenz}
%******************************************************************************************************
Dieser Text und das Video sind freie Software. Sie können es unter den Bedingungen der 
GNU General Public License, wie von der Free Software Foundation veröffentlicht, weitergeben 
und/oder modifizieren, entweder gemäß Version 3 der Lizenz oder (nach Ihrer Option) jeder späteren Version.

Die Veröffentlichung von Text und Video erfolgt in der Hoffnung, dass es Ihnen von Nutzen sein wird, 
aber OHNE IRGENDEINE GARANTIE, sogar ohne die implizite Garantie der MARKTREIFE oder der 
VERWENDBARKEIT FÜR EINEN BESTIMMTEN ZWECK. Details finden Sie in der GNU General Public License.

Sie sollten ein Exemplar der GNU General Public License zusammen mit diesem Text erhalten haben 
(zu finden im selben Git-Projekt). 
Falls nicht, siehe \url{http://www.gnu.org/licenses/}.

\subsection*{Das Video}
%******************************************************************************************************
Das Video hierzu ist zu finden unter 
{\tiny
	\url{huhu}
}

%******************************************************************************************************
%                                                                                                     *
\section{Ackermannfunktion ist nicht primitiv rekursiv}
%                                                                                                     *
%******************************************************************************************************

%******************************************************************************************************
\subsection{Eigenschaften}
%******************************************************************************************************
Wir verwenden die in einem vorherigen Video gezeigten Definition und Eigenschaften er Ackermannfunktion
(hierbei seien $k, n \in \N$):

\begin{align*}
    &\text{(\textbf{D1})} && A( 0, n )                   &&:= n+1\\
    &\text{(\textbf{D2})} && A( k+1, 0 )                 &&:= A( k, 1 )\\
    &\text{(\textbf{D3})} && A( k+1, n+1 )               &&:= A( k, A(k+1, n) )\\
    &\text{(\textbf{E1})} && A \colon \N^2 \to \N        &&\text{ ist total}\\
    &\text{(\textbf{E2})} && A( 1, n )                   &&= n+2\\
    &\text{(\textbf{E3})} && A( 2, n )                   &&= 2n+3\\
    &\text{(\textbf{E4})} && A( k, n )                   &&> n\\
    &\text{(\textbf{E5})} && A( k, n+1 )                 &&> A( k, n )\\
    &\text{(\textbf{E6})} && A( k+1, n )                 &&> A( k, n )\\
    &\text{(\textbf{E7})} && A( k+1, n )                 &&> A( k, n+1 )\\
    &\text{(\textbf{E8})} && A( k+k'+2, n )              &&> A( k, A( k',n) )\\
    &\text{(\textbf{E9})} && A( \max \{k, k' \} + 4, n ) &&> A( k, n ) + A( k, n )
\end{align*}

Aus den Eigenschaften \textbf{E5} und \textbf{E6} folgt die strenge Monotonie in beiden Parametern:
\begin{align*}
    &\text{(\textbf{E5'})} && n >  m &&\Rightarrow &&A( k, n )  > A( k, m )\\
    &\text{(\textbf{E6'})} && l >  k &&\Rightarrow &&A( l, n )  > A( k, n )
\end{align*}


%******************************************************************************************************
\subsection{Primitiv rekursiv}
%******************************************************************************************************

%-----------------------------------------------------------------------------------------------------*
\begin{Definition}{Primitiv rekursiv}
    Wir definieren induktiv als \textbf{primitiv rekursiv}:
    
    \begin{itemize}
        \item (\textbf{PR1}): $f \circ g$, falls $f,g$ primitiv rekursiv sind
        \item (\textbf{PR2}): $h$, mit $h( 0, x_1, \cdots, x_s ) := f( x_1, \cdots, x_s )$ und
            $h$, mit $h( n+1, x_1, \cdots, x_s ) := g(n, h( n, x_1, \cdots, x_s ), x_1, \cdots, x_s)$,
             falls $f,g$ primitiv rekursiv sind 
        \item (\textbf{PR3}): Die konstanten Funktionen $c(n) := c$ für alle $c \in \N$ 
        \item (\textbf{PR4}): Die Projektionen $P_j(x_1, \cdots, x_s ) := x_j$
        \item (\textbf{PR5}): Die Nachfolger-Funktion $s(n) := n+1$.
    \end{itemize}
\end{Definition}

%******************************************************************************************************
\subsection{Die Ackermannfunktion ist nicht primitiv rekursiv}
%******************************************************************************************************
Im Folgenden ist $\bb{x} := (x_1, \cdots, x_s)$ und $|\bb{x}| := \max \{x_1, \cdots, x_s \}$.

%-----------------------------------------------------------------------------------------------------*
\begin{Definition}{majorisiert}
    Seien $g \colon \N^m \to \N$ und $h \colon \N^2 \to \N$, dann sagen wir $h$ \textbf{majorisiert} $g$
    und wir schreiben $h > g$,
    falls
    \begin{equation*}
        \exists k_g \forall \bb{x} |\bb{x}| > 1 \Rightarrow h( k_g,  |\bb{x}| ) > g(  \bb{x} ).
    \end{equation*}
   
\end{Definition}
Hier betrachten wir $h$ als Schar von Funktionen, parametrisiert über das erste Argument, und wir suchen
das Element der Schar, das größer ist als die zu majorisierende Funktion.

%-----------------------------------------------------------------------------------------------------*
\begin{Theorem}{keine Selbstmajorisierung}
    Sei $h \colon \N^2 \to \N$, dann gilt NICHT $h > h$.   
\end{Theorem}
\begin{proof}
    Siehe "`v4.0.4.2.4.2 (Master) Berechenbarkeit - Ackermannfunktion Eigenschaften"'. 
\end{proof}

%-----------------------------------------------------------------------------------------------------*
\begin{Theorem}{Ackermann majorisiert primitiv rekursiv}
    Sei $g \colon \N^s \to \N$ und $A$ die Ackermannfunktion, dann gilt $A > g$.   
\end{Theorem}
\begin{proof}
    Siehe weiter unten. 
\end{proof}

%-----------------------------------------------------------------------------------------------------*
\begin{Theorem}{Ackermann nicht primitiv rekursiv}
    Die Ackermannfunktion $A$ ist NICHT primitiv rekursiv. 
\end{Theorem}
\begin{proof}
    Wäre $A$ primitiv rekursiv so müsste nach \myRef{Theorem}{Ackermann majorisiert primitiv rekursiv} 
    gelten $A > A$, was aber nach \myRef{Theorem}{keine Selbstmajorisierung} nicht möglich ist.
\end{proof}

%******************************************************************************************************
\subsection{Beweis von \myRef{Theorem}{Ackermann majorisiert primitiv rekursiv} }
%******************************************************************************************************
Wir beweisen den Satz per vollständiger Induktion über den Aufbau der primitiv rekursiven Funktionen.

%-----------------------------------------------------------------------------------------------------*
\begin{Theorem}{PR1: Verknüpfung}
    Sei $A( k_f, |\bb{x}| ) > f( \bb{x} )$ und $A( k_g, |\bb{x}| ) > g( \bb{x} )$. Setze
    $\color{red}{k_{f \circ g} := k_f + k_g + 2}$. Dann gilt
    \begin{equation*}
        A( k_{f \circ g}, |\bb{x}| ) > f ( g (\bb{x})).
    \end{equation*}
\end{Theorem}
\begin{proof}
    \begin{align*}
        &f(g(n)) & < &&& A(k_f, g(\bb{x}))          &&\text{ nach Definition von } k_f\\
        &        & < &&& A(k_f, A(k_g, |\bb{x}|))   &&\text{ nach E5' und Definition von } k_g\\
        &        & < &&& A(k_f + k_g + 2, |\bb{x}|) &&\text{ nach E8 }
    \end{align*}
\end{proof}

%-----------------------------------------------------------------------------------------------------*
\begin{Theorem}{PR2: Rekursion}
    Sei $A( k_f, |\bb{x}| ) > f( \bb{x} )$ und 
    $A( k_g, \max \{ n, h( n,\bb{x} ), |\bb{x}|\} ) > g( n, h( n,\bb{x} ), \bb{x} )$. 
    Dann gilt für die wie in PR2 rekursiv definierte
    Funktion $h$ mit $\color{red}{k_{R(f,g)} := 5 + \max \{ k_f, k_g \}}$
    \begin{equation*}
        A( k_{R(f,g)}, |\bb{x}| ) > h( \bb{x}).
    \end{equation*}
\end{Theorem}
\begin{proof}
    Wir setzen $q := 1 + \max \{ k_f, k_g \}$ und beweisen zunächst, als Zwischenbehauptung, über vollständige Induktion
    folgende Abschätzung:
    \begin{equation}\label{zwischen}
        \boxed{h( n, \bb{x}) < A( q, n+ |\bb{x}| )}.
    \end{equation}
    Für $n=0$ haben wir (unter Nutzung von E6')
    \begin{equation*}
        h( 0, \bb{x}) = f( \bb{x} ) < A( k_f, |\bb{x}| ) < A( q, 0+|\bb{x}| ).
    \end{equation*}
    Als Vorbereitung für den $n+1$-Fall: Es gelten folgende beide Ungleichungen
    \begin{align*}
        &\max \{ n, |\bb{x}|\} & < &&& n + |\bb{x}|  &&< A(q, n + |\bb{x}|) &&\text{ nach E4 }\\
        &h( n,\bb{x} )         &   &&&               &&< A(q, n + |\bb{x}|) &&\text{ nach Induktionsvoraussetzung }
    \end{align*}
    Diese beiden Ungleichungen ergeben zusammen
    \begin{equation}\label{max}
        \boxed{\max \{ n, h( n,\bb{x} ), |\bb{x}|\} < A(q, n + |\bb{x}|)}.
    \end{equation}

    Für $n+1$ können wir nun wie folgt abschätzen:
    \begin{align*}
        &h(n+1,\bb{x}) & =   &&& g( n, h( n, \bb{x} ), \bb{x} )                 &&\text{ nach Definition von } h\\
        &              & <   &&& A( k_g, \max \{ n, h( n,\bb{x} ), |\bb{x}|\} ) &&\text{ nach Definition von } k_g\\
        &              & <   &&& A( k_g, A(q, n + |\bb{x}|) )                   &&\text{ nach \eqref{max} und E5' } \\
        &              & \le &&& A( q-1, A(q, n + |\bb{x}|) )                   &&\text{ nach Definition von $q$ und E6' }\\
        &              & =   &&& A( q, n + 1 + |\bb{x}|) )                      &&\text{ nach D3 },
    \end{align*}
    welches die Zwischenbehauptung beweist. Damit können wir nun beweisen:    
    \begin{align*}
        &h(n,\bb{x})   & <   &&& A( q, n+ |\bb{x}| )                                 &&\text{ nach Zwischenbehauptung \eqref{zwischen} }\\
        &              & <   &&& A( q, 2 \cdot \max \{ n, |\bb{x}|\} + 3 )           &&\text{ zunächst hätte +1 auch gereicht}\\
        &              & <   &&& A( k_g, A(2, \max \{ n, |\bb{x}|\} ))               &&\text{ nach E3 } \\
        &              & <   &&& A( q+4, \max \{ n, |\bb{x}|\} )                     &&\text{ nach E8 }\\
        &              & <   &&& A( 5 + \max \{ k_f, k_g \}, \max \{ n, |\bb{x}|\} ) &&\text{ nach Definition von } q
    \end{align*}
 
\end{proof}

%-----------------------------------------------------------------------------------------------------*
\begin{Theorem}{PR3: Konstante}
    Sei $c \in \N$ beliebig. Setze $\color{red}{k_c := c}$. Dann gilt
    \begin{equation*}
        A( k_c, n ) > c.
    \end{equation*}
\end{Theorem}
\begin{proof}
    Sei $c :=0$. Dann gilt
    \begin{equation*}
        A( 0, n ) = n+1 > 0 \text{ (die Gleichheit nach D1)}.
    \end{equation*}
    Sei nun $A( c, n ) > c$. Dann gilt
    \begin{equation*}
        A( c+1, n ) > A( c, n ) > c\text{ (nach E6 und Voraussetzung)}.
    \end{equation*}
    Da $A( c, n )$ mindestens $1$ größer ist als $c$, muss $A( c+1, n ) > c+1$ sein.
\end{proof}

%-----------------------------------------------------------------------------------------------------*
\begin{Theorem}{PR4: Projektion}
    Setze $\color{red}{k_{P_j} := 0}$. Dann gilt
    \begin{equation*}
        A( k_{P_j}, |\bb{x}| ) > P_j(\bb{x}).
    \end{equation*}
\end{Theorem}
\begin{proof}
    Sei $c :=0$. Dann gilt
    \begin{equation*}
       A( k_{P_j}, |\bb{x}| ) =  A( 0, |\bb{x}| ) = |\bb{x}| + 1 > x_j = P_j(\bb{x}).
    \end{equation*}
    Die zweite Gleichheit wegen D1.
\end{proof}

%-----------------------------------------------------------------------------------------------------*
\begin{Theorem}{PR5: Nachfolger}
    Setze $\color{red}{k_s := 1}$. Dann gilt
    \begin{equation*}
        A( k_s, n ) > s(n)).
    \end{equation*}
\end{Theorem}
\begin{proof}
    \begin{equation*}
        A( k_s, n ) =  A( 1, n ) = n+2 > n+1 = s(n).
    \end{equation*}
    Die zweite Gleichheit wegen E2.
\end{proof}


%******************************************************************************************************
%                                                                                                     *
\section{Schluss}
%                                                                                                     *
%******************************************************************************************************
\textbf{Der Begriff der primitiv rekursiven Funktionen kann den Begriff der Berechenbarkeit nicht formalisieren.}
Die Ackermannfunktion ist berechenbar. Dies "`wissen"' wir, da sie $\mu$-rekursiv ist und alle
$\mu$-rekursiven Funktionen berechenbar sind. Wir wissen es auch, weil wir ein JavaScript-Programm
zur Errechnung der Ackermannfunktion schreiben können, und dieses kann in eine Turing-Maschine 
überführt werden.

Dies ist ein Grund, sich mit der Ackermannfunktion zu beschäftigen. Ihr sehr schnelles Wachstum
und damit das der der Funktion $A(n) := A(n,n)$ ist ein weiterer Grund.

Dies ist auch der Grund, warum wir in der Theorie der rekursiven Funktionen den Begriff der
$\mu$-Rekursion benötigen.

\textbf{Wir können anhand der rekursiven Definition einer primitiv rekursiven Funktion das $k$ für
die Majorisierung durch $A(k,n)$ einfach berechnen.} Dafür müssen wir nur die entsprechenden 
Formeln für das $k$ in den Beweisschritten oben verwenden.
\begin{align*}
    &k_{f \circ g} &&:= k_f + k_g + 2           &&\text{(PR1)}\\
    &k_{R(f,g)}    &&:= 5 + \max \{ k_f, k_g \} &&\text{(PR2)}\\
    &k_c           &&:= c                       &&\text{(PR3)}\\
    &k_{P_j}       &&:= 0                       &&\text{(PR4)}\\
    &k_s           &&:= 1                       &&\text{(PR5)}
\end{align*}
Wir wissen, dass die Ackermannfunktion für großes $k$ abstrus schnell wächst. Mit dem hier gewählten
Ansatz benötigen wir z.B. für die konstante Funktion $f(n) := 100$ eine extrem schnell wachsende 
Funktion, und ich habe den Eindruck, wir schießen hier mit Kanonen auf Spatzen.


\begin{backup}
    (Zur Zeit nicht benötigter Inhalt)
\end{backup}

%******************************************************************************************************
%                                                                                                     *
\begin{thebibliography}{9}
%                                                                                                     *
%******************************************************************************************************

   \bibitem[Douglas2013]{Douglas}
      Douglas S. Bridges, \emph{Computability, A Mathematical Sketchbook},
      Springer, Berlin Heidelberg New York 2013, ISBN 978-1-4612-6925-0 (ISBN).

\end{thebibliography}

%******************************************************************************************************
%                                                                                                     *
\begin{large}
    \centerline{\textsc{Symbolverzeichnis}}
\end{large}
%                                                                                                     *
%******************************************************************************************************
\bigskip

\renewcommand*{\arraystretch}{1}

\begin{tabular}{ll}
    $\N$                    & Die Menge der natürlichen Zahlen (mit Null): $\{ 0, 1, 2, 3, \cdots \}$\\
    $n, k, l, s, i, x_i, q$ & Natürliche Zahlen\\
    $A( k, n )$             & Ackermannfunktion\\
    $f, g, h$               & Funktionen\\
    $s()$                   & Nachfolgerfunktion: $s(n) := n+1$\\
    $\bb{x}$                & Vektor natürlicher Zahlen $(x_1, \cdots, x_s)$\\
    $|\bb{x}|$              & $\max \{x_1, \cdots, x_s \}$\\
    $g > f$                 & Die Funktion $g$ majorisiert die Funktion $f$\\
    $P_j$                   & Die Projektionen $P_j(x_1, \cdots, x_s ) := x_j$
\end{tabular}

\end{document}
