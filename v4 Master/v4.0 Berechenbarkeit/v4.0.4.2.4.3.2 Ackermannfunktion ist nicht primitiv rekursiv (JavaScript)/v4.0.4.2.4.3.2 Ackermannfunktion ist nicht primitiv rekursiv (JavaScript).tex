%******************************************************** -*-LaTeX-*- ******************************
%                                                                                                  *
% v4.0.4.2.4.3.2 Ackermannfunktion ist nicht primitiv rekursiv (JavaScript).tex                    *
%                                                                                                  *
% Copyright (C) 2023 Kategory GmbH \& Co. KG (joerg.kunze@kategory.de)                             *
%                                                                                                  *
% v4.0.4.2.4.3.2 is part of kategoryMathematik.                                                    *
%                                                                                                  *
% kategoryMathematik is free software: you can redistribute it and/or modify                       *
% it under the terms of the GNU General Public License as published by                             *
% the Free Software Foundation, either version 3 of the License, or                                *
% (at your option) any later version.                                                              *
%                                                                                                  *
% kategoryMathematik is distributed in the hope that it will be useful,                            *
% but WITHOUT ANY WARRANTY; without even the implied warranty of                                   *
% MERCHANTABILITY or FITNESS FOR A PARTICULAR PURPOSE.  See the                                    *
% GNU General Public License for more details.                                                     *
%                                                                                                  *
% You should have received a copy of the GNU General Public License                                *
% along with this program.  If not, see <http://www.gnu.org/licenses/>.                            *
%                                                                                                  *
%***************************************************************************************************

\documentclass[a4paper]{amsart}
% \documentclass[a4paper]{book}

%-----------------------------------------------------------------------------------------------------*
% package:                                                                                            *
%-----------------------------------------------------------------------------------------------------*
\usepackage{amssymb}
\usepackage{amsfonts}
\usepackage{amsmath}
\usepackage{amsthm}

\usepackage{mathabx}

\usepackage{a4wide} % a little bit smaller margins

\usepackage{graphicx}
\usepackage{hyperref}
\usepackage{algorithmic}
\usepackage{listings}
\usepackage{color}
\usepackage{colortbl}
\usepackage{sidecap}
\usepackage{comment}
\usepackage{tcolorbox}
\usepackage{collect}

\usepackage{upgreek}


% \usepackage{diagrams}

\usepackage[german]{babel}
\usepackage[none]{hyphenat}
\emergencystretch=4em

\usepackage[utf8]{inputenc} % to be able to use äöü as characters in text
\usepackage[T1]{fontenc} % to be able to use äöü in lables
\usepackage{lmodern}     % to avoid pixelation introduced by fontenc

\usepackage{hyperref}

%-----------------------------------------------------------------------------------------------------*
% theorem:                                                                                            *
%-----------------------------------------------------------------------------------------------------*
\theoremstyle{definition}
\newtheorem{theorem}{Theorem}[subsection]

\newcommand{\myTheorem}[1]{%
  \newtheorem{jk#1}[theorem]{#1}
  \newenvironment{#1}[1]{%
    \expandafter\begin{jk#1} \expandafter\label{#1:##1}\textbf{(##1):}
  }{%
    \expandafter\end{jk#1}
  }
}

\myTheorem{Definition}
\myTheorem{Proposition}
\myTheorem{Theorem}
\myTheorem{Example}
\myTheorem{Remark}

\definecollection{jkjkFrage}
\newtheorem{jkFrage}[theorem]{Frage}
\newenvironment{Frage}[1]{%
  \expandafter\begin{jkFrage} \expandafter\label{Frage:#1}\textbf{(#1):}
  \begin{collect}{jkjkFrage}{}{}
    \item \ref{Frage:#1} #1
  \end{collect}
}{%
  \expandafter\end{jkFrage}
}

\newcommand{\myRef}[2]{[#1 \ref{#1:#2}, ``#2'']}

\renewcommand{\proofname}{Beweis}

%-----------------------------------------------------------------------------------------------------*
% operator:                                                                                           *
%-----------------------------------------------------------------------------------------------------*
\DeclareMathOperator{\End}{End}
\DeclareMathOperator{\Ker}{Ker}
\DeclareMathOperator{\Mat}{Mat}
\DeclareMathOperator{\rank}{rank}
\DeclareMathOperator{\ggT}{ggT}
\DeclareMathOperator{\len}{len}
\DeclareMathOperator{\ord}{ord}
\DeclareMathOperator{\kgV}{kgV}
\DeclareMathOperator{\id}{id}
\DeclareMathOperator{\red}{red}
\DeclareMathOperator{\supp}{supp}
\DeclareMathOperator{\Bild}{Bild}
\DeclareMathOperator{\Rang}{Rang}
\DeclareMathOperator{\Det}{Det}
\DeclareMathOperator{\Hom}{Hom}

\DeclareMathOperator{\sub}{sub}
\DeclareMathOperator{\blk}{blk}
\DeclareMathOperator{\minimal}{minimal}
\DeclareMathOperator{\maximal}{maximal}

\definecolor{mygreen}{rgb}{0,0.6,0}
\definecolor{mygray}{rgb}{0.5,0.5,0.5}
\definecolor{mymauve}{rgb}{0.58,0,0.82}

\lstset{ %
  backgroundcolor=\color{white},   % choose the background color
  basicstyle=\ttfamily\footnotesize,        % size of fonts used for the code
  breaklines=true,                 % automatic line breaking only at whitespace
  captionpos=b,                    % sets the caption-position to bottom
  commentstyle=\color{mygreen},    % comment style
  escapeinside={\%*}{*)},          % if you want to add LaTeX within your code
  keywordstyle=\color{blue},       % keyword style
  stringstyle=\color{mymauve},     % string literal style
  frame=single
}

\setcounter{MaxMatrixCols}{20}

%******************************************************************************************************
%                                                                                                     *
% definition:                                                                                         *
%                                                                                                     *
%******************************************************************************************************
\newcommand{\R}{\ensuremath{\mathbb{ R }}}
\newcommand{\Q}{\ensuremath{\mathbb{ Q }}}
\newcommand{\Z}{\ensuremath{\mathbb{ Z }}}
\newcommand{\N}{\ensuremath{\mathbb{ N }}}
\newcommand{\C}{\ensuremath{\mathbb{ C }}}
\newcommand{\A}{\ensuremath{\mathbb{ A }}}
\newcommand{\F}{\ensuremath{\mathbb{ F }}}
\newcommand{\K}{\ensuremath{\mathbb{ K }}}
\newcommand{\Pb}{\ensuremath{\mathbb{ P }}}

\newcommand{\M}{\ensuremath{\mathcal{ M }}}
\newcommand{\V}{\ensuremath{\mathcal{ V }}}

\newcommand{\AAA}{\ensuremath{\mathcal{ A }}}
\newcommand{\BB}{\ensuremath{\mathcal{ B }}}
\newcommand{\CC}{\ensuremath{\mathcal{ C }}}
\newcommand{\EE}{\ensuremath{\mathcal{ E }}}
\newcommand{\KK}{\ensuremath{\mathcal{ K }}}
\newcommand{\MM}{\ensuremath{\mathcal{ M }}}
\newcommand{\PP}{\ensuremath{\mathcal{ P }}}
\newcommand{\ZZ}{\ensuremath{\mathcal{ Z }}}

\newcommand{\imporant}[1]{ \textcolor{red}{\textbf{#1}} }

\newcommand{\bb}[1]{\mathbf{#1}}
\newcommand{\balpha}{\boldsymbol{\upalpha}}
\newcommand{\bbeta}{\boldsymbol{\upbeta}}
\newcommand{\bgamma}{\boldsymbol{\upgamma}}
\newcommand{\bdelta}{\boldsymbol{\delta}}
\newcommand{\bmu}{\boldsymbol{\upmu}}

\newcommand{\z}[1]{\Z_{#1}}
\newcommand{\e}[1]{\z{#1}^*}
\newcommand{\q}[1]{(\e{#1})^2}

\excludecomment{book}
\excludecomment{example}
\excludecomment{backup}

\begin{document}

%******************************************************************************************************
%                                                                                                     *
\begin{titlepage}
%                                                                                                     *
%******************************************************************************************************
% \vspace*{\fill}
\centering
{\huge
(Master) Berechenbarkeit\\[1cm]
\textbf{v4.0.4.2.4.3.2 Ackermannfunktion ist nicht primitiv rekursiv (JavaScript)}
}\\[1cm]

\textbf{Kategory GmbH \& Co. KG}\\
Pr\"asentiert von Jörg Kunze

\end{titlepage}

%\clearpage
%\setcounter{page}{2}
%
%\tableofcontents

\newpage

%******************************************************************************************************
%                                                                                                     *
\section*{Beschreibung}
%                                                                                                     *
%******************************************************************************************************

%******************************************************************************************************
\subsection*{Inhalt}
%******************************************************************************************************
Die Ackermannfunktion ist nicht primitiv rekursiv. Wir haben aber schon gesehen, dass wir sie programmieren können.

Der Beweis ist konstruktiv: wir geben zu jeder primitiv-rekursiven Funktion eine Ackermann-Funktion an, welche die primitiv-rekursiven Funktion majorisiert. 

Dies bilden wir programmatisch nach: wir Erweitern unser Programm zur Erzeugung primitiv-rekursiver Funktionen aus den Atomen um das Ermitteln und das Ausgeben der entsprechenden majorisierenden Ackermann-Funktion.

Was ist hier der Unterschied zwischen einer und der Ackermann-Funktion? Currying: Die Ackermann-Funktion ist eine Funktion mit zwei Argumenten. Wir betrachten den ersten als Parameter einer Schar von Funktionen in einer Variablen, indem wir zu jeder festen Wahl des ersten, nur den zweiten variieren.  

%******************************************************************************************************
\subsection*{Präsentiert}
%******************************************************************************************************
Von Jörg Kunze

%******************************************************************************************************
\subsection*{Voraussetzungen}
%******************************************************************************************************
Primitiv rekursive Funktionen, Ackermann-Funktion und ihre Eigenschaften, Ackermann-Funktion ist nicht primitiv-rekursiv.

%******************************************************************************************************
\subsection*{Text}
%******************************************************************************************************
Der Begleittext als PDF und als LaTeX findet sich unter
\url{https://github.com/kategory/kategoryMathematik/tree/main/v4%20Master/v4.0%20Berechenbarkeit/v4.0.4.2.4.3.2%20Ackermannfunktion%20ist%20nicht%20primitiv%20rekursiv%20(JavaScript)}.

%******************************************************************************************************
\subsection*{Quelltext}
%******************************************************************************************************
Der JavaScript-Quelltext ist hier zu finden:
{\tiny
   \url{https://github.com/kategory/kategoryEducation/blob/main/mathematics/computabilityInJavaScript/v4.0.4.2.4.3.2%20(Master)%20Berechenbarkeit%20-%20Ackermannfunktion%20ist%20nicht%20primitiv%20rekursiv%20in%20JavaScript.js}
}

%******************************************************************************************************
\subsection*{Meine Videos}
%******************************************************************************************************
Siehe auch in den folgenden Videos:\\
v4.0.4.2.4.3 (Master) Berechenbarkeit - Ackermannfunktion ist nicht primitiv rekursiv\\
 \url{https://youtu.be/7S2bJ6LGekM}\\
 \\
4.0.4.2.1 (Master) Berechenbarkeit - Rekursive Funktionen in JavaScript\\
\url{https://youtu.be/19a10fW1bLk}\\
\\
\\v4.0.4.2.4.2 (Master) Berechenbarkeit - Ackermannfunktion Eigenschaften\\
 \url{https://youtu.be/MzDubMBlQ20}\\
 \\
v4.0.4.2.4.1 (Master) Berechenbarkeit - Ackermannfunktion ist mu-rekursiv\\
 \url{https://youtu.be/cbZ8EJ4U-tg}\\
 \\
v4.0.4.2.4 (Master) Berechenbarkeit - Ackermannfunktionen\\
 \url{https://youtu.be/XE-7YdHi2Vw}\\
 \\
v4.0.4.2 (Master) Berechenbarkeit - Rekursive Funktionen\\
 \url{https://youtu.be/tFEn2GoLLEQ}

%******************************************************************************************************
\subsection*{Quellen}
%******************************************************************************************************
Siehe auch in den folgenden Seiten:\\
\url{https://de.wikipedia.org/wiki/Ackermannfunktion}\\
\url{https://en.wikipedia.org/wiki/Ackermann\_function}\\
\url{https://planetmath.org/ackermannfunctionisnotprimitiverecursive}\\
\url{https://planetmath.org/PropertiesOfAckermannFunction}

%******************************************************************************************************
\subsection*{Buch}
%******************************************************************************************************
Grundlage ist folgendes Buch:\\
Computability\\
A Mathematical Sketchbook\\
Douglas S. Bridges\\
Springer-Verlag New York Inc. 2013\\
978-1-4612-6925-0 (ISBN)

%******************************************************************************************************
%                                                                                                     *
\section{Ackermannfunktion ist nicht primitiv rekursiv}
%                                                                                                     *
%******************************************************************************************************

%******************************************************************************************************
\subsection{Ackermann-Funktion}
%******************************************************************************************************
%-----------------------------------------------------------------------------------------------------*
\begin{Definition}{Ackermann-Funktion}
   Die Ackermann-Funktion $A(k,n) \colon \N^2 \to \N$, mit $\N = \{ 0, 1, 2, 3, \cdot \}$, ist definiert durch
   \begin{alignat}{2}
      &A( 0, n ) &&:= n+1\\
      &A( k+1, 0 ) &&:= A( k, 1 )\\
      &A( k+1, n+1 ) &&:= A( k, A( k+1, n )).
   \end{alignat}
\end{Definition}

%******************************************************************************************************
\subsection{Currying}
%******************************************************************************************************
Wir machen aus der Funktion $A(k,n) \colon \N^2 \to \N$ für jedes $k$ eine Funktion:
\begin{alignat}{2}
   A(k, \_) \colon &\N \to &&\N\\
                   & n \mapsto &&A(k,n).
\end{alignat}
Dieses Vorgehen nennen wir Currying. Dass Currying liefert eine Bijektion:
\begin{alignat}{2}
   &\Hom( A \times B, C )       &&\cong \Hom( A, \Hom( B, C ))\\
   &[f \colon A \times B \to C] &&\mapsto [ f^* \colon A \to \Hom( B, C )]\\
   &[(a,b) \mapsto f(a,b)]      &&\mapsto \begin{bmatrix}
                             f^*(a) \colon &B &\to     &C\\
                                           &b &\mapsto &f(a,b)
   \end{bmatrix}.
\end{alignat}

%******************************************************************************************************
\subsection{Primitiv rekursiv}
%******************************************************************************************************

%-----------------------------------------------------------------------------------------------------*
\begin{Definition}{Primitiv rekursiv}
    Wir definieren induktiv als primitiv rekursiv:
	\begin{itemize}
		\item (\textbf{PR1}): $f \circ g$, falls $f,g$ primitiv rekursiv sind
		\item (\textbf{PR2}): $h$, mit $h( 0, x_1, \cdots, x_s ) := f( x_1, \cdots, x_s )$ und
			$h$, mit $h( n+1, x_1, \cdots, x_s ) := g(n, h( n, x_1, \cdots, x_s ), x_1, \cdots, x_s)$,
		falls $f,g$ primitiv rekursiv sind 
		\item (\textbf{PR3}): Die konstanten Funktionen $c(n) := c$ für alle $c \in \N$ 
		\item (\textbf{PR4}): Die Projektionen $P_j(x_1, \cdots, x_s ) := x_j$
		\item (\textbf{PR5}): Die Nachfolger-Funktion $s(n) := n+1$.
	\end{itemize}
\end{Definition}

%******************************************************************************************************
\subsection{Der Parameter}
%******************************************************************************************************
Gemäß des Beweises im Video "`v4.0.4.2.4.3 (Master) Berechenbarkeit - Ackermannfunktion ist nicht primitiv rekursiv"' haben wir folgende Tabelle für den Parameter $k$.

\begin{itemize}
	\item (\textbf{PR1}): $k_{f \circ g} := k_f + k_g + 2$
	\item (\textbf{PR2}): $k_{R(f,g)} := 5 + \max \{ k_f, k_g \}$ 
	\item (\textbf{PR3}): $k_c := c$
	\item (\textbf{PR4}): $k_{P_j} := 0$
	\item (\textbf{PR5}): $k_s := 1$
\end{itemize}

%******************************************************************************************************
%                                                                                                     *
\begin{thebibliography}{9}
%                                                                                                     *
%******************************************************************************************************

   \bibitem[Douglas2013]{Douglas}
      Douglas S. Bridges, \emph{Computability, A Mathematical Sketchbook},
      Springer, Berlin Heidelberg New York 2013, ISBN 978-1-4612-6925-0 (ISBN).

\end{thebibliography}

%******************************************************************************************************
%                                                                                                     *
\begin{large}
    \centerline{\textsc{Symbolverzeichnis}}
\end{large}
%                                                                                                     *
%******************************************************************************************************
\bigskip

\renewcommand*{\arraystretch}{1}

\begin{tabular}{ll}
    $\N$                    & Die Menge der natürlichen Zahlen (mit Null): $\{ 0, 1, 2, 3, \cdots \}$\\
    $n, k, l, s, i, x_i, q$ & Natürliche Zahlen\\
    $A( k, n )$             & Ackermannfunktion\\
    $f, g, h$               & Funktionen\\
    $s()$                   & Nachfolgerfunktion: $s(n) := n+1$\\
    $\bb{x}$                & Vektor natürlicher Zahlen $(x_1, \cdots, x_s)$\\
    $|\bb{x}|$              & $\max \{x_1, \cdots, x_s \}$\\
    $g > f$                 & Die Funktion $g$ majorisiert die Funktion $f$\\
    $P_j$                   & Die Projektionen $P_j(x_1, \cdots, x_s ) := x_j$
\end{tabular}

\end{document}
