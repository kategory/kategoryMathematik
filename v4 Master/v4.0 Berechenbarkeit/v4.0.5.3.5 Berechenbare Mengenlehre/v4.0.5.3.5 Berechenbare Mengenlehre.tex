%******************************************************** -*-LaTeX-*- ******************************
%                                                                                                  *
% v4.0.5.3.5 Berechenbare Mengenlehre.tex                                                          *
%                                                                                                  *
% Copyright (C) 2024 Kategory GmbH \& Co. KG (joerg.kunze@kategory.de)                             *
%                                                                                                  *
% v4.0.5.3.5 Berechenbare Mengenlehre is part of kategoryMathematik.                               *
%                                                                                                  *
% kategoryMathematik is free software: you can redistribute it and/or modify                       *
% it under the terms of the GNU General Public License as published by                             *
% the Free Software Foundation, either version 3 of the License, or                                *
% (at your option) any later version.                                                              *
%                                                                                                  *
% kategoryMathematik is distributed in the hope that it will be useful,                            *
% but WITHOUT ANY WARRANTY; without even the implied warranty of                                   *
% MERCHANTABILITY or FITNESS FOR A PARTICULAR PURPOSE.  See the                                    *
% GNU General Public License for more details.                                                     *
%                                                                                                  *
% You should have received a copy of the GNU General Public License                                *
% along with this program.  If not, see <http://www.gnu.org/licenses/>.                            *
%                                                                                                  *
%***************************************************************************************************

\documentclass[a4paper]{amsart}
% \documentclass[a4paper]{book}

%-----------------------------------------------------------------------------------------------------*
% package:                                                                                            *
%-----------------------------------------------------------------------------------------------------*
\usepackage{amssymb}
\usepackage{amsfonts}
\usepackage{amsmath}
\usepackage{amsthm}

\usepackage{mathabx}

\usepackage{a4wide} % a little bit smaller margins

\usepackage{graphicx}
\usepackage{hyperref}
\usepackage{algorithmic}
\usepackage{listings}
\usepackage{color}
\usepackage{colortbl}
\usepackage{sidecap}
\usepackage{comment}
\usepackage{tcolorbox}
\usepackage{collect}

\usepackage{upgreek}

% \usepackage{diagrams}

\usepackage[german]{babel}
\usepackage[none]{hyphenat}
\emergencystretch=4em

\usepackage[utf8]{inputenc} % to be able to use äöü as characters in text
\usepackage[T1]{fontenc} % to be able to use äöü in lables
\usepackage{lmodern}     % to avoid pixelation introduced by fontenc

\usepackage{hyperref}

\usepackage{tikz}
\usepackage{tikz-cd}
\usetikzlibrary{babel}

%-----------------------------------------------------------------------------------------------------*
% theorem:                                                                                            *
%-----------------------------------------------------------------------------------------------------*
\theoremstyle{definition}
\newtheorem{theorem}{Theorem}[subsection]

\newcommand{\myTheorem}[1]{%
  \newtheorem{jk#1}[theorem]{#1}
  \newenvironment{#1}[1]{%
    \expandafter\begin{jk#1} \expandafter\label{#1:##1}\textbf{(##1):}
  }{%
    \expandafter\end{jk#1}
  }
}

\myTheorem{Definition}
\myTheorem{Proposition}
\myTheorem{Satz}
\myTheorem{Theorem}
\myTheorem{Axiom}
\myTheorem{Beispiel}
\myTheorem{Anmerkung}

\definecollection{jkjkFrage}
\newtheorem{jkFrage}[theorem]{Frage}
\newenvironment{Frage}[1]{%
  \expandafter\begin{jkFrage} \expandafter\label{Frage:#1}\textbf{(#1):}
  \begin{collect}{jkjkFrage}{}{}
    \item \ref{Frage:#1} #1
  \end{collect}
}{%
  \expandafter\end{jkFrage}
}

\newcommand{\myRef}[2]{[#1 \ref{#1:#2}, ``#2'']}

\renewcommand{\proofname}{Beweis}

%-----------------------------------------------------------------------------------------------------*
% operator:                                                                                           *
%-----------------------------------------------------------------------------------------------------*
\DeclareMathOperator{\End}{End}
\DeclareMathOperator{\Ker}{Ker}
\DeclareMathOperator{\Mat}{Mat}
\DeclareMathOperator{\rank}{rank}
\DeclareMathOperator{\ggT}{ggT}
\DeclareMathOperator{\len}{len}
\DeclareMathOperator{\ord}{ord}
\DeclareMathOperator{\kgV}{kgV}
\DeclareMathOperator{\id}{id}
\DeclareMathOperator{\red}{red}
\DeclareMathOperator{\supp}{supp}
\DeclareMathOperator{\Bild}{Bild}
\DeclareMathOperator{\Rang}{Rang}
\DeclareMathOperator{\Det}{Det}
\DeclareMathOperator{\Hom}{Hom}
\DeclareMathOperator{\GL}{GL}

\DeclareMathOperator{\sub}{sub}
\DeclareMathOperator{\blk}{blk}
\DeclareMathOperator{\minimal}{minimal}
\DeclareMathOperator{\maximal}{maximal}

\definecolor{mygreen}{rgb}{0,0.6,0}
\definecolor{mygray}{rgb}{0.5,0.5,0.5}
\definecolor{mymauve}{rgb}{0.58,0,0.82}

\lstset{ %
  language=Java,
  backgroundcolor=\color{white},   % choose the background color
  basicstyle=\ttfamily\footnotesize,        % size of fonts used for the code
  breaklines=true,                 % automatic line breaking only at whitespace
  captionpos=b,                    % sets the caption-position to bottom
  commentstyle=\color{mygreen},    % comment style
  escapeinside={\%*}{*)},          % if you want to add LaTeX within your code
  keywordstyle=\color{blue},       % keyword style
  stringstyle=\color{mymauve},     % string literal style
  frame=single,
  morekeywords={function, let}
}

\setcounter{MaxMatrixCols}{20}

%******************************************************************************************************
%                                                                                                     *
% definition:                                                                                         *
%                                                                                                     *
%******************************************************************************************************
\newcommand{\R}{\ensuremath{\mathbb{ R }}}
\newcommand{\Q}{\ensuremath{\mathbb{ Q }}}
\newcommand{\Z}{\ensuremath{\mathbb{ Z }}}
\newcommand{\N}{\ensuremath{\mathbb{ N }}}
\newcommand{\C}{\ensuremath{\mathbb{ C }}}
\newcommand{\A}{\ensuremath{\mathbb{ A }}}
\newcommand{\F}{\ensuremath{\mathbb{ F }}}
\newcommand{\K}{\ensuremath{\mathbb{ K }}}
\newcommand{\Pb}{\ensuremath{\mathbb{ P }}}

\newcommand{\M}{\ensuremath{\mathcal{ M }}}
\newcommand{\V}{\ensuremath{\mathcal{ V }}}

\newcommand{\AAA}{\ensuremath{\mathcal{ A }}}
\newcommand{\BB}{\ensuremath{\mathcal{ B }}}
\newcommand{\CC}{\ensuremath{\mathcal{ C }}}
\newcommand{\DD}{\ensuremath{\mathcal{ D }}}
\newcommand{\EE}{\ensuremath{\mathcal{ E }}}
\newcommand{\FF}{\ensuremath{\mathcal{ F }}}
\newcommand{\KK}{\ensuremath{\mathcal{ K }}}
\newcommand{\MM}{\ensuremath{\mathcal{ M }}}
\newcommand{\PP}{\ensuremath{\mathcal{ P }}}
\newcommand{\ZZ}{\ensuremath{\mathcal{ Z }}}

\newcommand{\imporant}[1]{ \textcolor{red}{\textbf{#1}} }

\newcommand{\bb}[1]{\mathbf{#1}}
\newcommand{\balpha}{\boldsymbol{\upalpha}}
\newcommand{\bbeta}{\boldsymbol{\upbeta}}
\newcommand{\bgamma}{\boldsymbol{\upgamma}}
\newcommand{\bdelta}{\boldsymbol{\delta}}
\newcommand{\bmu}{\boldsymbol{\upmu}}

\newcommand{\z}[1]{\Z_{#1}}
\newcommand{\e}[1]{\z{#1}^*}
\newcommand{\q}[1]{(\e{#1})^2}
\newcommand{\m}{\mathcal}

\newcommand{\zb}{z.~B. }

\excludecomment{book}
\excludecomment{example}
\excludecomment{backup}

\begin{document}

%******************************************************************************************************
%                                                                                                     *
\begin{titlepage}
%                                                                                                     *
%******************************************************************************************************
% \vspace*{\fill}
\centering
{\huge
(Master) Berechenbarkeit\\[1cm]
\textbf{v4.0.5.3.5 Berechenbare Mengenlehre}
}\\[1cm]

\textbf{Kategory GmbH \& Co. KG}\\
Präsentiert von Jörg Kunze\\
Copyright (C) 2024 Kategory GmbH \& Co. KG

\end{titlepage}

%\clearpage
%\setcounter{page}{2}
%
%\tableofcontents

\newpage

%******************************************************************************************************
%                                                                                                     *
\section*{Beschreibung}
%                                                                                                     *
%******************************************************************************************************

\subsection*{Inhalt}
Berechenbare Mengenlehre (ein Begriff von mir) ist die übliche einfache Mengenlehre beschränkt auf die Menge der berechenbaren Mengen.

Wir untersuchen drei Arten berechenbarer Mengen: primitiv-rekursive, entscheidbare und listbare (rekursiv aufzählbare).

Es stellt sich heraus, dass primitiv-rekursive und entscheidbare Mengen je für sich eine boolesche Algebra sind, d.~h. sie sind abgeschlossen bezüglich Vereinigung, Schnitt und Komplement.

Die listbaren Mengen sind nicht abgeschlossen gegenüber dem Komplement. Somit sind diese bloß ein distributiver Verband. 

Die hinzunehmenden Einschränkungen (zu wenig Mengen oder keine boolesche Algebra) sind groß, wenn wir so etwas wie, nur rechentechnisch erreichbare Mengen existieren, erreichen wollen. 

\subsection*{Präsentiert}
Von Jörg Kunze

\subsection*{Voraussetzungen}
Berechenbare Funktionen, Alphabet

\subsection*{Text}
Der Begleittext als PDF und als LaTeX findet sich unter
{\tiny
   \url{https://github.com/kategory/kategoryMathematik/tree/main/v4%20Master/v4.0%20Berechenbarkeit/v4.0.5.3.5%20Berechenbare%20Mengenlehre}
}

\subsection*{Meine Videos}
Siehe auch in den folgenden Videos:\\
\\
v4.0.5.3.4 (Master) Berechenbarkeit - Entscheidbar, rekursiv aufzählbar\\
\url{https://youtu.be/X8kMHx3Zed8}\\
\\
v4.0.4 (Master) Berechenbarkeit - Berechenbare Funktionen\\
\url{https://youtu.be/tARmHFIP32o}

\subsection*{Quellen}
Siehe auch in den folgenden Seiten:\\
\url{https://de.wikipedia.org/wiki/Boolesche_Algebra}\\
\url{https://de.wikipedia.org/wiki/Verband_(Mathematik)}\\
\url{https://de.wikipedia.org/wiki/Distributiver_Verband}

\subsection*{Buch}
Grundlage ist folgendes Buch:\\
Computability\\
A Mathematical Sketchbook\\
Douglas S. Bridges\\
Springer-Verlag New York Inc. 2013\\
978-1-4612-6925-0 (ISBN)
\\
\\
Sehr schön aber dichter:\\
Turing Computability: Theory and Applications\\
Robert I. Soare\\
Springer-Verlag New York Inc. 2016\\
978-3-6423-1932-7 (ISBN)

\subsection*{Lizenz}
Dieser Text und das Video sind freie Software. Sie können es unter den Bedingungen der
GNU General Public License, wie von der Free Software Foundation veröffentlicht, weitergeben
und/oder modifizieren, entweder gemäß Version 3 der Lizenz oder (nach Ihrer Option) jeder späteren Version.

Die Veröffentlichung von Text und Video erfolgt in der Hoffnung, dass es Ihnen von Nutzen sein wird,
aber OHNE IRGENDEINE GARANTIE, sogar ohne die implizite Garantie der MARKTREIFE oder der
VERWENDBARKEIT FÜR EINEN BESTIMMTEN ZWECK. Details finden Sie in der GNU General Public License.

Sie sollten ein Exemplar der GNU General Public License zusammen mit diesem Text erhalten haben
(zu finden im selben Git-Projekt).
Falls nicht, siehe \url{http://www.gnu.org/licenses/}.

\subsection*{Das Video}
%******************************************************************************************************
Das Video hierzu ist zu finden unter
{\tiny
   \url{hhh}
}

%******************************************************************************************************
%                                                                                                     *
\section{v4.0.5.3.5 Berechenbare Mengenlehre}
%                                                                                                     *
%******************************************************************************************************

%******************************************************************************************************
\subsection{Boolesche Algebra}
%******************************************************************************************************
Wir betrachten die Menge der natürlichen Zahlen $\N .= \{ 0, 1, 2, 3, \cdots \}$. Die Potenzmenge von $\N$ ist die Menge aller Teilmengen $\PP(\N)$. Gemeint sind alle Teilmengen im Sinne von ZFC. Also
\begin{equation}\label{Potenzmenge}
   \PP(\N) := \{ x \mid x \subseteq \N \}.
\end{equation}
Diese Menge ist geordnet durch $\subseteq$, hat ein Minimum $\emptyset$ und ein Maximum $\N$. In dieser Ordnungsstruktur gibt es Operatoren (Verknüpfungen) $\cap, \cup, \overline{\phantom{x}}, \rightarrow, \times, \sqcup$. Dabei ist $\overline{\phantom{x}}$ das Komplement
\begin{equation}\label{Komplement}
   \overline{X} := \{ y \in \N \mid y \notin X\}. 
\end{equation}
$\rightarrow$ ist die Implikation, in der Kategorientheorie später auch Exponential genannt,
\begin{equation}\label{Implikation}
   X \rightarrow Y := \{ z \mid z \notin X \lor z \in Y\} = \{ z \mid z \in X \Rightarrow z \in Y\} = \overline{X} \cup Y.
\end{equation}
Für diese gilt übrigens die wichtige Beziehung (, die hier nicht relevant ist, aber ich liebe sie,)
\begin{equation}\label{key}
   (x \cap y ) \subseteq z \Leftrightarrow x \subseteq (y \rightarrow z). 
\end{equation}
$\times$ ist das kartesische Produkt und $\sqcup$ die disjunkte Vereinigung
\begin{equation}\label{disjunkteVereinigung}
   X \sqcup Y := \{ (x,0) \mid x \in X\} \cup \{ (y,1) \mid y \in Y\} = X \times \{0\} \cup Y \times \{1\}.
\end{equation} 

Damit wird $\PP(\N)$ zu einer \textbf{Booleschen Algebra}.

%******************************************************************************************************
\subsection{Arten von Mengen}
%******************************************************************************************************
Liste der Arten von Mengen, die wir betrachten
\begin{itemize}
   \item Entscheidbar durch primitiv-rekursive Funktionen.
   \item Entscheidbar
   \item Listbar (= rekursiv aufzählbar)
\end{itemize}

Frage: Welche davon sind bezüglich welcher Operationen abgeschlossen?

%******************************************************************************************************
\subsection{Primitiv-rekursiv}
%******************************************************************************************************
Schrittweise durchhangeln:

{\color{red}Grundsätzlich arbeiteten wir und hier Schritt für Schritt durch und erzeugen einen immer größer werdenden Katalog von Funktionen, auf die wir dann im Weiteren aufbauen können. Dies gemacht zu haben, setzen wir hier voraus.}

Seien $X, Y$ zwei primitiv-rekursive Mengen mit den primitiv-rekursiven Funktionen $x \colon X \to \{0, 1\}$ und $y \colon Y \to \{0, 1\}$mit
\begin{equation}\label{pr}
   a \in X \Leftrightarrow x(a) = 1.
\end{equation} 
und entsprechend für $Y$.
Wir wissen bereits, dass $+1$, Mal, Plus, Minus, Modulo und ähnliche Funktionen primitiv-rekursiv sind. Primitiv-rekursive Funktionen können verknüpft werden um wieder primitiv-rekursive Funktionen zu erhalten.
\begin{equation}\label{pSchnitt}
   x \cdot y
\end{equation}
ist offenbar eine charakteristische Funktion von $X \cap Y$.
\begin{equation}\label{pVereinigung}
   x + y - x \cdot y
\end{equation}
ist offenbar eine charakteristische Funktion von $X \cup Y$.

Die durch Rekursion definierte Funktion $c$ mit $c(0) := 1$ und $c(n+1) := 0$ ist offensichtlich primitiv-rekursiv. Sie hat $c(0) := 1 \land c(1) := 0$ dreht also $0$ und $1$ um.

Damit ist
\begin{equation}\label{pKomplement}
   c\circ x
\end{equation}
die charakteristische Funktion von $\overline X$.

Die anderen Operatoren gehen ähnlich.

{\color{red}Die primitiv-rekursiven Mengen bilden eine boolesche Unteralgebra von $\PP(\N)$.} 

%******************************************************************************************************
\subsection{Entscheidbar}
%******************************************************************************************************
Für entscheidbare Mengen geht das ganz ähnlich. Nur, dass wir hier Programme haben, die für jeden Input halten und wir daraus neue schaffen müssen, die wieder für jeden Input halten.

Z.~B. für $X \cap Y$ mit den beiden Programmen $x,y$ können wir folgendes Programm bilden:
\begin{lstlisting}
   function( n ) {
      if( x(n) && y(n) ) return 1;
      else return 0;
   }
\end{lstlisting}
Da $x$ und $y$ halten und wir keine weitere Schleife benutzen, hält auch dieses Programm.

{\color{red}Die entscheidbaren Mengen bilden eine boolesche Unteralgebra von $\PP(\N)$.} 

%******************************************************************************************************
\subsection{Listbar}
%******************************************************************************************************
Wenn es listbare Mengen gibt, die nicht entscheidbar sind, ist das Komplement nicht listbar. Diesen Zusammenhang haben wir schon bewiesen in v4.0.5.3.4 (Master) Berechenbarkeit - Entscheidbar, rekursiv aufzählbar, im Video \url{https://youtu.be/X8kMHx3Zed8}. Das es listbare nicht entscheidbare Mengen gibt haben wir noch nicht bewiesen, aber es gibt sie. Z.~B. die Menge der beweisbaren Sätze.

D.~h., dass die Menge der listbaren Mengen NICHT abgeschlossen bezüglich der Operatoren $\overline{\phantom{x}}$ und $\rightarrow$ ist. 

{\color{red}Die listbaren Mengen bilden KEINE boolesche Algebra.} 

Bezüglich welcher Operatoren ist die Menge der listbaren Mengen abgeschlossen? Antwort:
\begin{equation}\label{listbareOperatoren}
   \cap, \cup, \times, \sqcup
\end{equation}
Um das zu beweisen, müssen wir aus den Programmen für $X$ und $Y$ die entsprechenden Programme schnitzen, was recht einfach ist.

Die Existenz von $\cap$ und $\cup$ macht die Menge der listbaren Mengen zu einem Verband. Und da diese beiden Operationen auf Mengen distributiv sind gilt immerhin:

{\color{red}Die listbaren Mengen bilden einen distributiven Verband.}

%******************************************************************************************************
\subsection{Philosophie}
%******************************************************************************************************
Wir könnten an ZFC zweifeln und sagen, dass Mengen, die rechentechnisch/algorithmisch nicht erreichbar sind, gar nicht existieren.

Wenn wir uns auf die primitiv rekursiven beschränken haben wir ein nettes abgeschlossenes System, in der aber viele entscheidbare Mengen fehlen.

Wenn wir die entscheidbaren nehmen, haben wir das Problem, dass wir bei einigen Programmen nicht wissen, ob diese halten oder nicht. Außerdem gibt es interessante Mengen, wie die Menge der beweisbaren Sätze, die darin nicht enthalten sind.

Wenn wir alle listbaren nehmen, haben wir keine boolesche Algebra mehr.

Wenn das die Alternativen sind, bleibe ich bei ZFC.

\begin{backup}
%******************************************************************************************************
%                                                                                                     *
\section{TODO}
%                                                                                                     *
%******************************************************************************************************

\begin{itemize}
   \item Inkonsistenz, Unabhängigkeit, Unvollständigkeit
\end{itemize}

\begin{itemize}
     \item Überprüfe Symbolverzeichnis
\end{itemize}

Tafel:
\begin{itemize}
   \item Listbar
\end{itemize}

\end{backup}

\begin{backup}
    (Zur Zeit nicht benötigter Inhalt)
\end{backup}

%******************************************************************************************************
%                                                                                                     *
\begin{thebibliography}{9}
%                                                                                                     *
%******************************************************************************************************
   \bibitem[Douglas2013]{Douglas}
   Douglas S. Bridges, \emph{Computability, A Mathematical Sketchbook},
   Springer, Berlin Heidelberg New York 2013, ISBN 978-1-4612-6925-0 (ISBN).

\end{thebibliography}

%******************************************************************************************************
%                                                                                                     *
\begin{large}
    \centerline{\textsc{Symbolverzeichnis}}
\end{large}
%                                                                                                     *
%******************************************************************************************************
\bigskip

\renewcommand*{\arraystretch}{1}

\begin{tabular}{ll}
    $\N$                    & Die Menge der natürlichen Zahlen (mit Null): $\{ 0, 1, 2, 3, \cdots \}$\\
    $\PP(M)$                & Potenzmenge von $M$\\
    $n, k, l, s, i, x_i, q$ & Natürliche Zahlen\\
    $A( k, n )$             & Ackermannfunktion\\
    $f, g, h$               & Funktionen\\
    $s()$                   & Nachfolgerfunktion: $s(n) := n+1$

\end{tabular}

\end{document}
