%******************************************************** -*-LaTeX-*- ******************************
%                                                                                                  *
% v4.0.6.1 Turingmaschinen-Nummer = Sourcecode.tex                                                 *
%                                                                                                  *
% Copyright (C) 2025 Kategory GmbH \& Co. KG (joerg.kunze@kategory.de)                             *
%                                                                                                  *
% v4.0.6.1 Turingmaschinen-Nummer = Sourcecode is part of kategoryMathematik.                      *
%                                                                                                  *
% kategoryMathematik is free software: you can redistribute it and/or modify                       *
% it under the terms of the GNU General Public License as published by                             *
% the Free Software Foundation, either version 3 of the License, or                                *
% (at your option) any later version.                                                              *
%                                                                                                  *
% kategoryMathematik is distributed in the hope that it will be useful,                            *
% but WITHOUT ANY WARRANTY; without even the implied warranty of                                   *
% MERCHANTABILITY or FITNESS FOR A PARTICULAR PURPOSE.  See the                                    *
% GNU General Public License for more details.                                                     *
%                                                                                                  *
% You should have received a copy of the GNU General Public License                                *
% along with this program.  If not, see <http://www.gnu.org/licenses/>.                            *
%                                                                                                  *
%***************************************************************************************************

\documentclass[a4paper]{amsart}
% \documentclass[a4paper]{book}

%-----------------------------------------------------------------------------------------------------*
% package:                                                                                            *
%-----------------------------------------------------------------------------------------------------*
\usepackage{amssymb}
\usepackage{amsfonts}
\usepackage{amsmath}
\usepackage{amsthm}

\usepackage{mathabx}

\usepackage{a4wide} % a little bit smaller margins

\usepackage{graphicx}
\usepackage{hyperref}
\usepackage{algorithmic}
\usepackage{listings}
\usepackage{color}
\usepackage{colortbl}
\usepackage{sidecap}
\usepackage{comment}
\usepackage{tcolorbox}
\usepackage{collect}

\usepackage{upgreek}

% \usepackage{diagrams}

\usepackage[german]{babel}
\usepackage[none]{hyphenat}
\emergencystretch=4em

\usepackage[utf8]{inputenc} % to be able to use äöü as characters in text
\usepackage[T1]{fontenc} % to be able to use äöü in lables
\usepackage{lmodern}     % to avoid pixelation introduced by fontenc

\usepackage{hyperref}

\usepackage{tikz}
\usepackage{tikz-cd}
\usetikzlibrary{babel}

%-----------------------------------------------------------------------------------------------------*
% theorem:                                                                                            *
%-----------------------------------------------------------------------------------------------------*
\theoremstyle{definition}
\newtheorem{theorem}{Theorem}[subsection]

\newcommand{\myTheorem}[1]{%
  \newtheorem{jk#1}[theorem]{#1}
  \newenvironment{#1}[1]{%
    \expandafter\begin{jk#1} \expandafter\label{#1:##1}\textbf{(##1):}
  }{%
    \expandafter\end{jk#1}
  }
}

\myTheorem{Definition}
\myTheorem{Proposition}
\myTheorem{Satz}
\myTheorem{Theorem}
\myTheorem{Axiom}
\myTheorem{Beispiel}
\myTheorem{Anmerkung}

\definecollection{jkjkFrage}
\newtheorem{jkFrage}[theorem]{Frage}
\newenvironment{Frage}[1]{%
  \expandafter\begin{jkFrage} \expandafter\label{Frage:#1}\textbf{(#1):}
  \begin{collect}{jkjkFrage}{}{}
    \item \ref{Frage:#1} #1
  \end{collect}
}{%
  \expandafter\end{jkFrage}
}

\newcommand{\myRef}[2]{[#1 \ref{#1:#2}, ``#2'']}

\renewcommand{\proofname}{Beweis}

%-----------------------------------------------------------------------------------------------------*
% operator:                                                                                           *
%-----------------------------------------------------------------------------------------------------*
\DeclareMathOperator{\End}{End}
\DeclareMathOperator{\Ker}{Ker}
\DeclareMathOperator{\Mat}{Mat}
\DeclareMathOperator{\rank}{rank}
\DeclareMathOperator{\ggT}{ggT}
\DeclareMathOperator{\len}{len}
\DeclareMathOperator{\ord}{ord}
\DeclareMathOperator{\kgV}{kgV}
\DeclareMathOperator{\id}{id}
\DeclareMathOperator{\red}{red}
\DeclareMathOperator{\supp}{supp}
\DeclareMathOperator{\Bild}{Bild}
\DeclareMathOperator{\Rang}{Rang}
\DeclareMathOperator{\Det}{Det}
\DeclareMathOperator{\Hom}{Hom}
\DeclareMathOperator{\GL}{GL}

\DeclareMathOperator{\sub}{sub}
\DeclareMathOperator{\blk}{blk}
\DeclareMathOperator{\minimal}{minimal}
\DeclareMathOperator{\maximal}{maximal}

\definecolor{mygreen}{rgb}{0,0.6,0}
\definecolor{mygray}{rgb}{0.5,0.5,0.5}
\definecolor{mymauve}{rgb}{0.58,0,0.82}

\lstset{ %
  language=Java,
  backgroundcolor=\color{white},   % choose the background color
  basicstyle=\ttfamily\footnotesize,        % size of fonts used for the code
  breaklines=true,                 % automatic line breaking only at whitespace
  captionpos=b,                    % sets the caption-position to bottom
  commentstyle=\color{mygreen},    % comment style
  escapeinside={\%*}{*)},          % if you want to add LaTeX within your code
  keywordstyle=\color{blue},       % keyword style
  stringstyle=\color{mymauve},     % string literal style
  frame=single,
  morekeywords={function, let}
}

\setcounter{MaxMatrixCols}{20}

%******************************************************************************************************
%                                                                                                     *
% definition:                                                                                         *
%                                                                                                     *
%******************************************************************************************************
\newcommand{\R}{\ensuremath{\mathbb{ R }}}
\newcommand{\Q}{\ensuremath{\mathbb{ Q }}}
\newcommand{\Z}{\ensuremath{\mathbb{ Z }}}
\newcommand{\N}{\ensuremath{\mathbb{ N }}}
\newcommand{\C}{\ensuremath{\mathbb{ C }}}
\newcommand{\A}{\ensuremath{\mathbb{ A }}}
\newcommand{\F}{\ensuremath{\mathbb{ F }}}
\newcommand{\K}{\ensuremath{\mathbb{ K }}}
\newcommand{\Pb}{\ensuremath{\mathbb{ P }}}

\newcommand{\M}{\ensuremath{\mathcal{ M }}}
\newcommand{\V}{\ensuremath{\mathcal{ V }}}

\newcommand{\AAA}{\ensuremath{\mathcal{ A }}}
\newcommand{\BB}{\ensuremath{\mathcal{ B }}}
\newcommand{\CC}{\ensuremath{\mathcal{ C }}}
\newcommand{\DD}{\ensuremath{\mathcal{ D }}}
\newcommand{\EE}{\ensuremath{\mathcal{ E }}}
\newcommand{\FF}{\ensuremath{\mathcal{ F }}}
\newcommand{\KK}{\ensuremath{\mathcal{ K }}}
\newcommand{\MM}{\ensuremath{\mathcal{ M }}}
\newcommand{\PP}{\ensuremath{\mathcal{ P }}}
\newcommand{\ZZ}{\ensuremath{\mathcal{ Z }}}

\newcommand{\imporant}[1]{ \textcolor{red}{\textbf{#1}} }

\newcommand{\bb}[1]{\mathbf{#1}}
\newcommand{\balpha}{\boldsymbol{\upalpha}}
\newcommand{\bbeta}{\boldsymbol{\upbeta}}
\newcommand{\bgamma}{\boldsymbol{\upgamma}}
\newcommand{\bdelta}{\boldsymbol{\delta}}
\newcommand{\bmu}{\boldsymbol{\upmu}}

\newcommand{\z}[1]{\Z_{#1}}
\newcommand{\e}[1]{\z{#1}^*}
\newcommand{\q}[1]{(\e{#1})^2}
\newcommand{\m}{\mathcal}

\newcommand{\zb}{z.~B. }

\excludecomment{book}
\excludecomment{example}
\excludecomment{backup}

\begin{document}

%******************************************************************************************************
%                                                                                                     *
\begin{titlepage}
%                                                                                                     *
%******************************************************************************************************
% \vspace*{\fill}
\centering
{\huge
(Master) Berechenbarkeit\\[1cm]
\textbf{v4.0.6.1 Turingmaschinen-Nummer = Sourcecode}
}\\[1cm]

\textbf{Kategory GmbH \& Co. KG}\\
Präsentiert von Jörg Kunze\\
Copyright (C) 2024 Kategory GmbH \& Co. KG

\end{titlepage}

%\clearpage
%\setcounter{page}{2}
%
%\tableofcontents

\newpage

%******************************************************************************************************
%                                                                                                     *
\section*{Beschreibung}
%                                                                                                     *
%******************************************************************************************************

\subsection*{Inhalt}
Da Turing-Maschinen endliche Tupel von endliche  Mengen sind, können sie als Zahl codiert werden, und das so, dass aus der Zahl die Turing-Maschine wieder eindeutig rekonstruiert werden kann. Diese Turing-Maschinen-Nummer (descriptor number) ist wie der Sourcecode eines Programms. Der Sourcecode ist ein String (Zeichen-Kette), der das Programm repräsentiert.

Aus der Nummer / dem Sourcecode kann die Turing-Maschine / das Programm eindeutig rekonstruiert werden. 

Ob eine Zahl / ein String die Nummer / der Sourcecode einer Turing-Maschine / eines Programms ist, ist entscheidbar.

Mit diesem Werkzeug können wir nun Turing-Maschinen und Programme rekursiv aufzählen.

Der Datentyp der Nummer / des Sourcecodes ist mit Bedacht so gewählt, dass es der Datentyp der Ein- und Ausgabe-Parameter unserer Turing-Maschinen / Programme ist.

Damit können wir nun Turing-Maschinen / Programme mit Hilfe von Turing-Maschinen / Programmen bearbeiten.

Wir können nun (und erst jetzt) die Frage stellen: Ist das Prädikat "`die Maschine x hält bei Inout y"' entscheidbar?

\subsection*{Präsentiert}
Von Jörg Kunze

\subsection*{Voraussetzungen}
Berechenbare Funktionen, Alphabet

\subsection*{Text}
Der Begleittext als PDF und als LaTeX findet sich unter
{\tiny
   \url{https://github.com/kategory/kategoryMathematik/tree/main/v4%20Master/v4.0%20Berechenbarkeit/v4.0.5.3.5%20Berechenbare%20Mengenlehre}
}

\subsection*{Meine Videos}
Siehe auch in den folgenden Videos:\\
\\
v4.0.5.3.4 (Master) Berechenbarkeit - Entscheidbar, rekursiv aufzählbar\\
\url{https://youtu.be/X8kMHx3Zed8}\\
\\
v4.0.4 (Master) Berechenbarkeit - Berechenbare Funktionen\\
\url{https://youtu.be/tARmHFIP32o}

\subsection*{Quellen}
Siehe auch in den folgenden Seiten:\\
\url{https://de.wikipedia.org/wiki/Boolesche_Algebra}\\
\url{https://de.wikipedia.org/wiki/Verband_(Mathematik)}\\
\url{https://de.wikipedia.org/wiki/Distributiver_Verband}

\subsection*{Buch}
Grundlage ist folgendes Buch:\\
Computability\\
A Mathematical Sketchbook\\
Douglas S. Bridges\\
Springer-Verlag New York Inc. 2013\\
978-1-4612-6925-0 (ISBN)
\\
\\
Sehr schön aber dichter:\\
Turing Computability: Theory and Applications\\
Robert I. Soare\\
Springer-Verlag New York Inc. 2016\\
978-3-6423-1932-7 (ISBN)

\subsection*{Lizenz}
Dieser Text und das Video sind freie Software. Sie können es unter den Bedingungen der
GNU General Public License, wie von der Free Software Foundation veröffentlicht, weitergeben
und/oder modifizieren, entweder gemäß Version 3 der Lizenz oder (nach Ihrer Option) jeder späteren Version.

Die Veröffentlichung von Text und Video erfolgt in der Hoffnung, dass es Ihnen von Nutzen sein wird,
aber OHNE IRGENDEINE GARANTIE, sogar ohne die implizite Garantie der MARKTREIFE oder der
VERWENDBARKEIT FÜR EINEN BESTIMMTEN ZWECK. Details finden Sie in der GNU General Public License.

Sie sollten ein Exemplar der GNU General Public License zusammen mit diesem Text erhalten haben
(zu finden im selben Git-Projekt).
Falls nicht, siehe \url{http://www.gnu.org/licenses/}.

\subsection*{Das Video}
%******************************************************************************************************
Das Video hierzu ist zu finden unter
{\tiny
   \url{hhh}
}

%******************************************************************************************************
%                                                                                                     *
\section{v4.0.6.1 Turingmaschinen-Nummer = Sourcecode}
%                                                                                                     *
%******************************************************************************************************

%******************************************************************************************************
\subsection{xxx}
%******************************************************************************************************
Wir betrachten ... 

\begin{backup}
%******************************************************************************************************
%                                                                                                     *
\section{TODO}
%                                                                                                     *
%******************************************************************************************************

\begin{itemize}
   \item Inkonsistenz, Unabhängigkeit, Unvollständigkeit
\end{itemize}

\begin{itemize}
     \item Überprüfe Symbolverzeichnis
\end{itemize}

Tafel:
\begin{itemize}
   \item Listbar
\end{itemize}

\end{backup}

\begin{backup}
    (Zur Zeit nicht benötigter Inhalt)
\end{backup}

%******************************************************************************************************
%                                                                                                     *
\begin{thebibliography}{9}
%                                                                                                     *
%******************************************************************************************************
   \bibitem[Douglas2013]{Douglas}
   Douglas S. Bridges, \emph{Computability, A Mathematical Sketchbook},
   Springer, Berlin Heidelberg New York 2013, ISBN 978-1-4612-6925-0 (ISBN).

\end{thebibliography}

%******************************************************************************************************
%                                                                                                     *
\begin{large}
    \centerline{\textsc{Symbolverzeichnis}}
\end{large}
%                                                                                                     *
%******************************************************************************************************
\bigskip

\renewcommand*{\arraystretch}{1}

\begin{tabular}{ll}
    $\N$                    & Die Menge der natürlichen Zahlen (mit Null): $\{ 0, 1, 2, 3, \cdots \}$\\
    $\PP(M)$                & Potenzmenge von $M$\\
    $n, k, l, s, i, x_i, q$ & Natürliche Zahlen\\
    $A( k, n )$             & Ackermannfunktion\\
    $f, g, h$               & Funktionen\\
    $s()$                   & Nachfolgerfunktion: $s(n) := n+1$

\end{tabular}

\end{document}
