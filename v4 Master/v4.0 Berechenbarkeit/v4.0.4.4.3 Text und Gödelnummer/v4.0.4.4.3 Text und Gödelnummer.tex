%******************************************************** -*-LaTeX-*- ******************************
%                                                                                                  *
% v4.0.4.4.3 Text und Gödelnummer.tex                                                              *
%                                                                                                  *
% Copyright (C) 2024 Kategory GmbH \& Co. KG (joerg.kunze@kategory.de)                             *
%                                                                                                  *
% v4.0.4.4.3 Text und Gödelnummer is part of kategoryMathematik.                                   *
%                                                                                                  *
% kategoryMathematik is free software: you can redistribute it and/or modify                       *
% it under the terms of the GNU General Public License as published by                             *
% the Free Software Foundation, either version 3 of the License, or                                *
% (at your option) any later version.                                                              *
%                                                                                                  *
% kategoryMathematik is distributed in the hope that it will be useful,                            *
% but WITHOUT ANY WARRANTY; without even the implied warranty of                                   *
% MERCHANTABILITY or FITNESS FOR A PARTICULAR PURPOSE.  See the                                    *
% GNU General Public License for more details.                                                     *
%                                                                                                  *
% You should have received a copy of the GNU General Public License                                *
% along with this program.  If not, see <http://www.gnu.org/licenses/>.                            *
%                                                                                                  *
%***************************************************************************************************

\documentclass[a4paper]{amsart}
% \documentclass[a4paper]{book}

%-----------------------------------------------------------------------------------------------------*
% package:                                                                                            *
%-----------------------------------------------------------------------------------------------------*
\usepackage{amssymb}
\usepackage{amsfonts}
\usepackage{amsmath}
\usepackage{amsthm}

\usepackage{mathabx}

\usepackage{a4wide} % a little bit smaller margins

\usepackage{graphicx}
\usepackage{hyperref}
\usepackage{algorithmic}
\usepackage{listings}
\usepackage{color}
\usepackage{colortbl}
\usepackage{sidecap}
\usepackage{comment}
\usepackage{tcolorbox}
\usepackage{collect}

\usepackage{upgreek}

% \usepackage{diagrams}

\usepackage[german]{babel}
\usepackage[none]{hyphenat}
\emergencystretch=4em

\usepackage[utf8]{inputenc} % to be able to use äöü as characters in text
\usepackage[T1]{fontenc} % to be able to use äöü in lables
\usepackage{lmodern}     % to avoid pixelation introduced by fontenc

\usepackage{hyperref}

\usepackage{tikz}
\usepackage{tikz-cd}
\usetikzlibrary{babel}

%-----------------------------------------------------------------------------------------------------*
% theorem:                                                                                            *
%-----------------------------------------------------------------------------------------------------*
\theoremstyle{definition}
\newtheorem{theorem}{Theorem}[subsection]

\newcommand{\myTheorem}[1]{%
  \newtheorem{jk#1}[theorem]{#1}
  \newenvironment{#1}[1]{%
    \expandafter\begin{jk#1} \expandafter\label{#1:##1}\textbf{(##1):}
  }{%
    \expandafter\end{jk#1}
  }
}

\myTheorem{Definition}
\myTheorem{Proposition}
\myTheorem{Satz}
\myTheorem{Theorem}
\myTheorem{Axiom}
\myTheorem{Beispiel}
\myTheorem{Anmerkung}

\definecollection{jkjkFrage}
\newtheorem{jkFrage}[theorem]{Frage}
\newenvironment{Frage}[1]{%
  \expandafter\begin{jkFrage} \expandafter\label{Frage:#1}\textbf{(#1):}
  \begin{collect}{jkjkFrage}{}{}
    \item \ref{Frage:#1} #1
  \end{collect}
}{%
  \expandafter\end{jkFrage}
}

\newcommand{\myRef}[2]{[#1 \ref{#1:#2}, ``#2'']}

\renewcommand{\proofname}{Beweis}

%-----------------------------------------------------------------------------------------------------*
% operator:                                                                                           *
%-----------------------------------------------------------------------------------------------------*
\DeclareMathOperator{\End}{End}
\DeclareMathOperator{\Ker}{Ker}
\DeclareMathOperator{\Mat}{Mat}
\DeclareMathOperator{\rank}{rank}
\DeclareMathOperator{\ggT}{ggT}
\DeclareMathOperator{\len}{len}
\DeclareMathOperator{\ord}{ord}
\DeclareMathOperator{\kgV}{kgV}
\DeclareMathOperator{\id}{id}
\DeclareMathOperator{\red}{red}
\DeclareMathOperator{\supp}{supp}
\DeclareMathOperator{\Bild}{Bild}
\DeclareMathOperator{\Rang}{Rang}
\DeclareMathOperator{\Det}{Det}
\DeclareMathOperator{\Hom}{Hom}
\DeclareMathOperator{\GL}{GL}

\DeclareMathOperator{\sub}{sub}
\DeclareMathOperator{\blk}{blk}
\DeclareMathOperator{\minimal}{minimal}
\DeclareMathOperator{\maximal}{maximal}

\definecolor{mygreen}{rgb}{0,0.6,0}
\definecolor{mygray}{rgb}{0.5,0.5,0.5}
\definecolor{mymauve}{rgb}{0.58,0,0.82}

\lstset{ %
  backgroundcolor=\color{white},   % choose the background color
  basicstyle=\ttfamily\footnotesize,        % size of fonts used for the code
  breaklines=true,                 % automatic line breaking only at whitespace
  captionpos=b,                    % sets the caption-position to bottom
  commentstyle=\color{mygreen},    % comment style
  escapeinside={\%*}{*)},          % if you want to add LaTeX within your code
  keywordstyle=\color{blue},       % keyword style
  stringstyle=\color{mymauve},     % string literal style
  frame=single
}

\setcounter{MaxMatrixCols}{20}

%******************************************************************************************************
%                                                                                                     *
% definition:                                                                                         *
%                                                                                                     *
%******************************************************************************************************
\newcommand{\R}{\ensuremath{\mathbb{ R }}}
\newcommand{\Q}{\ensuremath{\mathbb{ Q }}}
\newcommand{\Z}{\ensuremath{\mathbb{ Z }}}
\newcommand{\N}{\ensuremath{\mathbb{ N }}}
\newcommand{\C}{\ensuremath{\mathbb{ C }}}
\newcommand{\A}{\ensuremath{\mathbb{ A }}}
\newcommand{\F}{\ensuremath{\mathbb{ F }}}
\newcommand{\K}{\ensuremath{\mathbb{ K }}}
\newcommand{\Pb}{\ensuremath{\mathbb{ P }}}

\newcommand{\M}{\ensuremath{\mathcal{ M }}}
\newcommand{\V}{\ensuremath{\mathcal{ V }}}

\newcommand{\AAA}{\ensuremath{\mathcal{ A }}}
\newcommand{\BB}{\ensuremath{\mathcal{ B }}}
\newcommand{\CC}{\ensuremath{\mathcal{ C }}}
\newcommand{\DD}{\ensuremath{\mathcal{ D }}}
\newcommand{\EE}{\ensuremath{\mathcal{ E }}}
\newcommand{\FF}{\ensuremath{\mathcal{ F }}}
\newcommand{\KK}{\ensuremath{\mathcal{ K }}}
\newcommand{\MM}{\ensuremath{\mathcal{ M }}}
\newcommand{\PP}{\ensuremath{\mathcal{ P }}}
\newcommand{\ZZ}{\ensuremath{\mathcal{ Z }}}

\newcommand{\imporant}[1]{ \textcolor{red}{\textbf{#1}} }

\newcommand{\bb}[1]{\mathbf{#1}}
\newcommand{\balpha}{\boldsymbol{\upalpha}}
\newcommand{\bbeta}{\boldsymbol{\upbeta}}
\newcommand{\bgamma}{\boldsymbol{\upgamma}}
\newcommand{\bdelta}{\boldsymbol{\delta}}
\newcommand{\bmu}{\boldsymbol{\upmu}}

\newcommand{\z}[1]{\Z_{#1}}
\newcommand{\e}[1]{\z{#1}^*}
\newcommand{\q}[1]{(\e{#1})^2}
\newcommand{\m}{\mathcal}

\newcommand{\zb}{z.~B. }

\excludecomment{book}
\excludecomment{example}
\excludecomment{backup}

\begin{document}

%******************************************************************************************************
%                                                                                                     *
\begin{titlepage}
%                                                                                                     *
%******************************************************************************************************
% \vspace*{\fill}
\centering
{\huge
(Master) Berechenbarkeit\\[1cm]
\textbf{v4.0.4.4.3 Text und Gödelnummer}
}\\[1cm]

\textbf{Kategory GmbH \& Co. KG}\\
Präsentiert von Jörg Kunze\\
Copyright (C) 2024 Kategory GmbH \& Co. KG

\end{titlepage}

%\clearpage
%\setcounter{page}{2}
%
%\tableofcontents

\newpage

%******************************************************************************************************
%                                                                                                     *
\section*{Beschreibung}
%                                                                                                     *
%******************************************************************************************************

%******************************************************************************************************
\subsection*{Inhalt}
%******************************************************************************************************
Texte, definiert als endliche Folgen von Zeichen aus einem Alphabet, kommen in der Mathematik vor, wenn wir Formeln, Aussagen, Beweise oder Algorithmen mit mathematischen Methoden untersuchen. Ein Satz ist in dem Zusammenhang eine Folge von Zeichen, der einer bestimmten Syntax gehorcht.

Eine Gödelnummer ist ein Code, der aus einem Text oder einem Tupel von natürlichen Zahlen, was eigentlich auch wieder eine Art Text ist, eine natürliche Zahl macht. Umkehrbar. Damit können wir Funktionen, die auf Texten oder auf Tupeln arbeiten, auf Funktionen von $\N$ nach $\N$ zurückführen.

Wir wollen die üblichen Formelmanipulationen, die wir in der Mathematik benötigen, wie Termumformungen, Variablen einsetzen, Prüfen, ob ein Text (eine Zeichenkette) ein gültiger Term, Ausdruck, Satz oder Beweis ist, berechenbar haben.

Dafür haben wir zwei Möglichkeiten:

1. Wir bauen eine Variante der Berechenbarkeitstheorie auf, die auf Texten statt auf den natürlichen Zahlen beruht.

2. Wir kodieren Texte als Zahlen auf eine Weise, dass die oben genannten Funktionen berechenbar werden.

Wir gehen den zweiten Weg, denn er hat zwei entscheidende Vorteile:

1. Die Berechenbarkeitstheorie bleibt erheblich einfacher.

2. In den meisten Axiomen-Systemen, die wir untersuchen, werden wir in der Lage sein, mit Zahlen zu rechnen. Damit stehen uns in all diesen Systemen die Ergebnisse einer auf Zahlen basierenden Berechenbarkeitstheorie zur Verfügung.

Auch Turingmaschinen können gödelisiert werden, d.~h. so durch Zahlen repräsentiert werden, dass deren Manipulation und deren Laufenlassen berechenbare Funktionen werden.

Eine Wunsch-Definition von Gödelnummer ist eine Funktion von der Ausgangsmenge in $\N$ mit bestimmten Eigenschaften. Die Bilder dieses Funktion heißen auch Codes. Die Ausgangsmenge ist \zb die Menge der Wörter über einem Alphabet oder die Menge der Tupel über $\N$. Die Eigenschaften sind:
\begin{itemize}
   \item Injektiv
   \item Berechenbar
   \item Die Frage, ob eine Zahl im Bild ist, ist berechenbar
   \item Die auf dem Bild definierte Umkehrfunktion ist ebenfalls berechenbar.
\end{itemize}
Mit anderen Worten können wir zwischen Wörtern/Tupeln und deren Codes verlustfrei mit Hilfe eines Computerprogramms hin und her rechnen.

Dies ist allerdings keine mathematische Definition sondern Poesie:
Die Forderung der Berechenbarkeit der Funktion und deren Umkehrung ist nämlich genau genommen Unfug, da Berechenbarkeit ja nur von $\N^m \to \N$ definiert ist. Wir wollen ja umgekehrt erst durch die Definition die Berechenbarkeit auf Texte erweitern. 

In Wirklichkeit ist es so, dass wir auf Computern einen Text immer schon als Zahl haben, nämlich als Folge von Nullen und Einsen. Genauer, haben wir zunächst nur die Zahl. Dann definieren wir eine berechenbare Funktion $z(n,m) \colon \N \times \N \to M$, die aus der Zahl $n$, die wir als Text auffassen, das $m$-te Zeichen extrahiert. Hier ist $M \subset \N$ eine endliche Teilmenge von $\N$. $M$ ist unser Alphabet. Die Zahlen aus $M$ schreiben wir als abcdefg\dots.

$z(n, \_)$ ist damit eine Folge von Zeichen. $z( \_, \_)$ als Funktion vom linken Parameter ist damit eine Funktion von $\N$ in die Menge er endlichen Folgen von Zeichen. Deren Umkehrfunktion ist die Gödelnummer, mit der wir, wenn wir vor dem Bildschirm hocken und auf die Tastatur einhacken, immer schon rechnen.

Wir gehen immer davon aus, dass auf die eine oder andere Art Berechenbarkeitstheorie und ihre Ergebnisse auf Texte und die Textfunktionen übertragen wurden.

%******************************************************************************************************
\subsection*{Präsentiert}
%******************************************************************************************************
Von Jörg Kunze

%******************************************************************************************************
\subsection*{Voraussetzungen}
%******************************************************************************************************
Berechenbare Funktionen, Alphabet

%******************************************************************************************************
\subsection*{Text}
%******************************************************************************************************
Der Begleittext als PDF und als LaTeX findet sich unter
{\tiny
   \url{https://github.com/kategory/kategoryMathematik/tree/main/v4%20Master/v4.0%20Berechenbarkeit/v4.0.4.4.3%20Text%20und%20G%C3%B6delnummer}
}

%******************************************************************************************************
\subsection*{Meine Videos}
%******************************************************************************************************
Siehe auch in den folgenden Videos:\\
\\
v4.0.2 (Master) Berechenbarkeit - Alphabete und Sprachen\\
\url{https://youtu.be/cwlU8m9ldbA}\\
\\
v4.0.4 (Master) Berechenbarkeit - Berechenbare Funktionen\\
\url{https://youtu.be/tARmHFIP32o}

%******************************************************************************************************
\subsection*{Quellen}
%******************************************************************************************************
Siehe auch in den folgenden Seiten:\\
\url{https://de.wikipedia.org/wiki/G%C3%B6delnummer}\\
\url{https://en.wikipedia.org/wiki/G%C3%B6del_numbering}\\
\url{https://ncatlab.org/nlab/show/computability}

%******************************************************************************************************
\subsection*{Buch}
%******************************************************************************************************
Grundlage ist folgendes Buch:\\
Computability\\
A Mathematical Sketchbook\\
Douglas S. Bridges\\
Springer-Verlag New York Inc. 2013\\
978-1-4612-6925-0 (ISBN)

%******************************************************************************************************
\subsection*{Lizenz}
%******************************************************************************************************
Dieser Text und das Video sind freie Software. Sie können es unter den Bedingungen der
GNU General Public License, wie von der Free Software Foundation veröffentlicht, weitergeben
und/oder modifizieren, entweder gemäß Version 3 der Lizenz oder (nach Ihrer Option) jeder späteren Version.

Die Veröffentlichung von Text und Video erfolgt in der Hoffnung, dass es Ihnen von Nutzen sein wird,
aber OHNE IRGENDEINE GARANTIE, sogar ohne die implizite Garantie der MARKTREIFE oder der
VERWENDBARKEIT FÜR EINEN BESTIMMTEN ZWECK. Details finden Sie in der GNU General Public License.

Sie sollten ein Exemplar der GNU General Public License zusammen mit diesem Text erhalten haben
(zu finden im selben Git-Projekt).
Falls nicht, siehe \url{http://www.gnu.org/licenses/}.

\subsection*{Das Video}
%******************************************************************************************************
Das Video hierzu ist zu finden unter
{\tiny
   \url{hhh}
}

%******************************************************************************************************
%                                                                                                     *
\section{Text und Gödelnummer}
%                                                                                                     *
%******************************************************************************************************
%******************************************************************************************************
\subsection{Texte}
%******************************************************************************************************


%******************************************************************************************************
\subsection{Gödelnummern: der Versuch einer Definition}
%******************************************************************************************************


%******************************************************************************************************
\subsection{Gödelnummern für Tupel aus $\N$}
%******************************************************************************************************

Die Umkehrfunktion ist nicht berechenbar, weil ...

%******************************************************************************************************
\subsection{Berechenbare Textfunktionen}
%******************************************************************************************************

%******************************************************************************************************
\subsection{Gödelnummern für Texte}
%******************************************************************************************************

Definition über die Forderung, dass die Funktionen von oben berechenbar werden sollen.


%******************************************************************************************************
\subsection{Gödelnummern für Turingmaschinen}
%******************************************************************************************************








Gödelnummern arithmetisieren das Ausgangsmaterial



Wenn wir ein System mit Hilfe von Gödelnummern arithmetisieren, wissen wir, dass es immer dann zur Verfügung stehen kann, wenn die natürlichen Zahlen zur Verfügung stehen. Dies ist eine recht milde Forderung, die in den meisten Axiomen-Systemen, die wir betrachten, erfüllt ist.

Siehe: https://plato.stanford.edu/entries/goedel-incompleteness/sup1.html  


Beispiel: aus Text per Funktionen erstes Zeichen ermitteln

\begin{backup}
%******************************************************************************************************
%                                                                                                     *
\section{TODO}
%                                                                                                     *
%******************************************************************************************************
\begin{itemize}
     \item Überprüfe Symbolverzeichnis
\end{itemize}


\end{backup}

\begin{backup}
    (Zur Zeit nicht benötigter Inhalt)
\end{backup}

%******************************************************************************************************
%                                                                                                     *
\begin{thebibliography}{9}
%                                                                                                     *
%******************************************************************************************************
   \bibitem[Douglas2013]{Douglas}
   Douglas S. Bridges, \emph{Computability, A Mathematical Sketchbook},
   Springer, Berlin Heidelberg New York 2013, ISBN 978-1-4612-6925-0 (ISBN).

\end{thebibliography}

%******************************************************************************************************
%                                                                                                     *
\begin{large}
    \centerline{\textsc{Symbolverzeichnis}}
\end{large}
%                                                                                                     *
%******************************************************************************************************
\bigskip

\renewcommand*{\arraystretch}{1}

\begin{tabular}{ll}
    $\N$                    & Die Menge der natürlichen Zahlen (mit Null): $\{ 0, 1, 2, 3, \cdots \}$\\
    $n, k, l, s, i, x_i, q$ & Natürliche Zahlen\\
    $A( k, n )$             & Ackermannfunktion\\
    $f, g, h$               & Funktionen\\
    $s()$                   & Nachfolgerfunktion: $s(n) := n+1$

\end{tabular}

\end{document}
