%******************************************************** -*-LaTeX-*- ******************************
%                                                                                                  *
% 3.0.2.0.5.5 Dualität in der Algebra der Mengen.tex                                               *
%                                                                                                  *
% Copyright (C) 2025 Kategory GmbH \& Co. KG (joerg.kunze@kategory.de)                             *
%                                                                                                  *
% 3.0.2.0.5.5 Dualität in der Algebra der Mengen is part of kategoryMathematik.                    *
%                                                                                                  *
% kategoryMathematik is free software: you can redistribute it and/or modify                       *
% it under the terms of the GNU General Public License as published by                             *
% the Free Software Foundation, either version 3 of the License, or                                *
% (at your option) any later version.                                                              *
%                                                                                                  *
% kategoryMathematik is distributed in the hope that it will be useful,                            *
% but WITHOUT ANY WARRANTY; without even the implied warranty of                                   *
% MERCHANTABILITY or FITNESS FOR A PARTICULAR PURPOSE.  See the                                    *
% GNU General Public License for more details.                                                     *
%                                                                                                  *
% You should have received a copy of the GNU General Public License                                *
% along with this program.  If not, see <http://www.gnu.org/licenses/>.                            *
%                                                                                                  *
%***************************************************************************************************

\documentclass[a4paper]{amsart}
% \documentclass[a4paper]{book}

%-----------------------------------------------------------------------------------------------------*
% package:                                                                                            *
%-----------------------------------------------------------------------------------------------------*
\usepackage{amssymb}
\usepackage{amsfonts}
\usepackage{amsmath}
\usepackage{amsthm}

\usepackage{mathabx}

\usepackage{a4wide} % a little bit smaller margins

\usepackage{graphicx}
\usepackage{hyperref}
\usepackage{algorithmic}
\usepackage{listings}
\usepackage{color}
\usepackage{colortbl}
\usepackage{sidecap}
\usepackage{comment}
\usepackage{tcolorbox}
\usepackage{collect}

\usepackage{upgreek}

% \usepackage{diagrams}

\usepackage[german]{babel}
\usepackage[none]{hyphenat}
\emergencystretch=4em

\usepackage[utf8]{inputenc} % to be able to use äöü as characters in text
\usepackage[T1]{fontenc} % to be able to use äöü in lables
\usepackage{lmodern}     % to avoid pixelation introduced by fontenc

\usepackage{hyperref}

\usepackage{tikz}
\usepackage{tikz-cd}
\usetikzlibrary{babel}

\usepackage{leftindex}

\usepackage{xcolor}
\pagecolor{black}
\color{white}

%-----------------------------------------------------------------------------------------------------*
% theorem:                                                                                            *
%-----------------------------------------------------------------------------------------------------*
\theoremstyle{definition}
\newtheorem{theorem}{Theorem}[subsection]

\newcommand{\myTheorem}[1]{%
  \newtheorem{jk#1}[theorem]{#1}
  \newenvironment{#1}[1]{%
    \expandafter\begin{jk#1} \expandafter\label{#1:##1}\textbf{(##1):}
  }{%
    \expandafter\end{jk#1}
  }
}

\myTheorem{Definition}
\myTheorem{Proposition}
\myTheorem{Theorem}
\myTheorem{Satz}
\myTheorem{Example}
\myTheorem{Remark}

\definecollection{jkjkFrage}
\newtheorem{jkFrage}[theorem]{Frage}
\newenvironment{Frage}[1]{%
  \expandafter\begin{jkFrage} \expandafter\label{Frage:#1}\textbf{(#1):}
  \begin{collect}{jkjkFrage}{}{}
    \item \ref{Frage:#1} #1
  \end{collect}
}{%
  \expandafter\end{jkFrage}
}

\newcommand{\myRef}[2]{[#1 \ref{#1:#2}, ``#2'']}

\renewcommand{\proofname}{Beweis}

%-----------------------------------------------------------------------------------------------------*
% operator:                                                                                           *
%-----------------------------------------------------------------------------------------------------*
\DeclareMathOperator{\End}{End}
\DeclareMathOperator{\Ker}{Ker}
\DeclareMathOperator{\Mat}{Mat}
\DeclareMathOperator{\rank}{rank}
\DeclareMathOperator{\ggT}{ggT}
\DeclareMathOperator{\len}{len}
\DeclareMathOperator{\ord}{ord}
\DeclareMathOperator{\kgV}{kgV}
\DeclareMathOperator{\id}{id}
\DeclareMathOperator{\red}{red}
\DeclareMathOperator{\supp}{supp}
\DeclareMathOperator{\Bild}{Bild}
\DeclareMathOperator{\Urbild}{Urbild}
\DeclareMathOperator{\Rang}{Rang}
\DeclareMathOperator{\Det}{Det}
\DeclareMathOperator{\Hom}{Hom}

\DeclareMathOperator{\sub}{sub}
\DeclareMathOperator{\blk}{blk}
\DeclareMathOperator{\minimal}{minimal}
\DeclareMathOperator{\maximal}{maximal}

\definecolor{mygreen}{rgb}{0,0.6,0}
\definecolor{mygray}{rgb}{0.5,0.5,0.5}
\definecolor{mymauve}{rgb}{0.58,0,0.82}

\lstset{ %
  backgroundcolor=\color{white},   % choose the background color
  basicstyle=\ttfamily\footnotesize,        % size of fonts used for the code
  breaklines=true,                 % automatic line breaking only at whitespace
  captionpos=b,                    % sets the caption-position to bottom
  commentstyle=\color{mygreen},    % comment style
  escapeinside={\%*}{*)},          % if you want to add LaTeX within your code
  keywordstyle=\color{blue},       % keyword style
  stringstyle=\color{mymauve},     % string literal style
  frame=single
}

\setcounter{MaxMatrixCols}{20}

%******************************************************************************************************
%                                                                                                     *
% definition:                                                                                         *
%                                                                                                     *
%******************************************************************************************************
\newcommand{\R}{\ensuremath{\mathbb{ R }}}
\newcommand{\Q}{\ensuremath{\mathbb{ Q }}}
\newcommand{\Z}{\ensuremath{\mathbb{ Z }}}
\newcommand{\N}{\ensuremath{\mathbb{ N }}}
\newcommand{\C}{\ensuremath{\mathbb{ C }}}
\newcommand{\A}{\ensuremath{\mathbb{ A }}}
\newcommand{\F}{\ensuremath{\mathbb{ F }}}
\newcommand{\K}{\ensuremath{\mathbb{ K }}}
\newcommand{\Pb}{\ensuremath{\mathbb{ P }}}

\newcommand{\M}{\ensuremath{\mathcal{ M }}}
\newcommand{\V}{\ensuremath{\mathcal{ V }}}

\newcommand{\AAA}{\ensuremath{\mathcal{ A }}}
\newcommand{\BB}{\ensuremath{\mathcal{ B }}}
\newcommand{\CC}{\ensuremath{\mathcal{ C }}}
\newcommand{\EE}{\ensuremath{\mathcal{ E }}}
\newcommand{\KK}{\ensuremath{\mathcal{ K }}}
\newcommand{\MM}{\ensuremath{\mathcal{ M }}}
\newcommand{\PP}{\ensuremath{\mathcal{ P }}}
\newcommand{\ZZ}{\ensuremath{\mathcal{ Z }}}

\newcommand{\imporant}[1]{ \textcolor{red}{\textbf{#1}} }

\newcommand{\bb}[1]{\mathbf{#1}}
\newcommand{\balpha}{\boldsymbol{\upalpha}}
\newcommand{\bbeta}{\boldsymbol{\upbeta}}
\newcommand{\bgamma}{\boldsymbol{\upgamma}}
\newcommand{\bdelta}{\boldsymbol{\delta}}
\newcommand{\bmu}{\boldsymbol{\upmu}}

\newcommand{\z}[1]{\Z_{#1}}
\newcommand{\e}[1]{\z{#1}^*}
\newcommand{\q}[1]{(\e{#1})^2}

\newcommand{\zb}{z.~B. }

\excludecomment{book}
\excludecomment{example}
\excludecomment{backup}

\begin{document}

%******************************************************************************************************
%                                                                                                     *
\begin{titlepage}
%                                                                                                     *
%******************************************************************************************************
% \vspace*{\fill}
\centering
{\huge
(Bachelor) Anfang\\[1cm]
\textbf{3.0.2.0.5.5 Dualität in der Algebra der Mengen}
}\\[1cm]

\textbf{Kategory GmbH \& Co. KG}\\
Präsentiert von Jörg Kunze\\
Copyright (C) 2025 Kategory GmbH \& Co. KG

\end{titlepage}

%\clearpage
%\setcounter{page}{2}
%
%\tableofcontents

\newpage

%******************************************************************************************************
%                                                                                                     *
\section*{Beschreibung}
%                                                                                                     *
%******************************************************************************************************

\subsection*{Inhalt}
Wenn wir uns in der Mengenlehre auf ein "`Universum"' beschränken und damit der Komplement-Operator aus einer Menge wieder eine Menge macht, erhalten eine Selbst-Dualität.

Das Universum ist hier ganz locker als irgendeine Menge zu sehen, ohne dass wir besondere Abgeschlossenheitseigenschaften fordern würden.

Der Komplement-Operator wird dann relativ zu dieser Menge definiert: das Komplement einer Teilmenge des Universums ist dann die Menge aller Elemente aus dem Universum, die nicht in der Ausgangsmenge drinne sind.

Dann sind die leere Menge und das Universum sowie Vereinigung und Schnitt dual zueinander: drehen wir in einer gültigen Formel alles um, erhalten wir erneut eine gültige Formel: Formelfabrik.

Da wir hier Aussagen über Formeln machen, statt über mathematische Objekte, betreiben wir dabei in Wirklichkeit Meta-Mathematik.

\subsection*{Präsentiert}
Von Jörg Kunze

\subsection*{Voraussetzungen}
Mengen, Vereinigung, Schnitt, Funktion, Familie, Indexschreibweise.

\subsection*{Text}
Der Begleittext als PDF und als LaTeX findet sich unter
{\tiny
   \url{https://github.com/kategory/kategoryMathematik/tree/main/v3%20Bachelor/v3.0%20Anfang/v3.0.2.0.5.5%20Dualit%C3%A4t%20in%20der%20Algebra%20der%20Mengen}
}

\subsection*{Meine Videos}
Siehe auch in den folgenden Videos:\\
\\
v3.0.1.2.3 (Bachelor) Die Axiome - Klassen Lehre\\
\url{https://youtu.be/2heuG7w-rnQ}\\
\\
v3.0.1.2.5 (Bachelor) Die Axiome - Mengen konstruieren\\
\url{https://youtu.be/LkWDcR5EHp4}\\

\subsection*{Quellen}
Siehe auch in den folgenden Seiten:\\
\url{https://de.wikipedia.org/wiki/Dualit%C3%A4t_(Verbandstheorie)}\\
\url{https://de.wikipedia.org/wiki/Mengenlehre}

\subsection*{Buch}
Grundlage ist folgendes Buch:\\
"`Grundwissen Mathematikstudium"'\\
Tilo Arens, Rolf Busam, Frank Hettlich, Christian Karpfinger, Hellmuth Stachel \\
2022\\
Springer-Verlag\\
978-3-662-63312-0 (ISBN)\\
{\tiny
   \url{https://www.lehmanns.de/shop/mathematik-informatik/56427740-9783662633120-grundwissen-mathematikstudium}}\\
\\

\subsection*{Lizenz}
Dieser Text und das Video sind freie Software. Sie können es unter den Bedingungen der
GNU General Public License, wie von der Free Software Foundation veröffentlicht, weitergeben
und/oder modifizieren, entweder gemäß Version 3 der Lizenz oder (nach Ihrer Option) jeder späteren Version.

Die Veröffentlichung von Text und Video erfolgt in der Hoffnung, dass es Ihnen von Nutzen sein wird,
aber OHNE IRGENDEINE GARANTIE, sogar ohne die implizite Garantie der MARKTREIFE oder der
VERWENDBARKEIT FÜR EINEN BESTIMMTEN ZWECK. Details finden Sie in der GNU General Public License.

Sie sollten ein Exemplar der GNU General Public License zusammen mit diesem Text erhalten haben
(zu finden im selben Git-Projekt).
Falls nicht, siehe \url{http://www.gnu.org/licenses/}.

\subsection*{Das Video}
%******************************************************************************************************
Das Video hierzu ist zu finden unter
{\tiny
   \url{huhu}
}

%******************************************************************************************************
%                                                                                                     *
\section{3.0.2.0.5.5 Dualität in der Algebra der Mengen}
%                                                                                                     *
%******************************************************************************************************

%******************************************************************************************************
\subsection{Universum}
%******************************************************************************************************
Hier arbeiten wir mit einem sehr einfachen (billigen) Universum $U$. Wir betrachten alle Teilmengen von $U$ also die Potenzmenge $\PP(U)$. Wir verlangen für diesen Abschnitt keinerlei zusätzliche Abgeschlossenheitseigenschaften. So muss $U$ \zb nicht notwendig abgeschlossen gegenüber Produktbildungen $X \times Y$ sein.

Ab jetzt und bis zum Ende von Sektion "`3.0.2.0.5.5 Dualität in der Algebra der Mengen"' sei eine Menge $U$ beliebig und fest vorgegeben.
 
%******************************************************************************************************
\subsection{Komplement}
%******************************************************************************************************
\begin{Definition}{Komplement}
   Wenn $T$ eine Klasse ist, dann ist deren \textbf{Komplement} definiert als folgende Klasse:
   \begin{equation}\label{KomplementFormel}
      T^c := \{ x \mid x \notin T \}.
   \end{equation}
   $ T^c $ ist in der Regel keine Menge, auch dann nicht wenn $T$ eine ist.
\end{Definition}

\begin{Definition}{relatives Komplement}
   Wenn $A$ eine Teilmenge von $U$ ist, dann ist deren \textbf{relatives Komplement} definiert als folgende Teilmenge von $U$:
   \begin{equation}\label{KomplementRelativFormel}
      \overline{A} := \{ x \in U \mid x \notin A \}.
   \end{equation}
   $ \overline{A} $ ist nach dem Aussonderungsaxiom eine Menge.
\end{Definition}

Es gilt klarerweise: $ \overline{A} $ ist Teilmenge von $U$ und ist damit selber in unserem Universum.
\begin{equation}
   \overline{A} \subseteq U.
\end{equation}

Eigentlich müssten wir in der Schreibweise $ \overline{A} $ noch das $U$ irgendwo unterbringen. Das macht aber alle Formeln blöde unübersichtlich und da $U$ fest ist, verzichten wir darauf.

Die Schreibweisen $\overline{A}, A^c$ werden beide mal für das Komplement und mal für das relative Komplement verwendet. Die hier getroffene Festlegung ist keinesfalls universell.

Es gilt:
\begin{equation}\label{Alles}
   A \cup \overline{A} = U.
\end{equation}

Eine für das Folgende wichtige Erkenntnis ist
\begin{equation}\label{ALle}
   \{\overline A \mid A \subseteq U \} = \{A \mid A \subseteq U \}
\end{equation}
oder umgangssprachlich: Wenn die Variable $A$ über alle Teilmengen von $U$ "`läuft"', dann läuft auch der Ausdruck $\overline A$ über alle Teilmengen von $U$. Die Menge in \ref{ALle} ist natürlich nichts  anderes als $\PP( U )$. 

%******************************************************************************************************
\subsection{Duale Ausdrücke}
%******************************************************************************************************
Es gilt:
\begin{alignat}{5}
   &\overline{ \overline{A} } &&= A\\
   &\overline{ \emptyset    } &&= U\\
   &\overline{ U            } &&= \emptyset\\
   &\overline{ A \cup B     } &&= \overline A \cap \overline B\\
   &\overline{ A \cap B     } &&= \overline A \cup \overline B
\end{alignat}
sowie
\begin{equation}\label{dualSubset}
   A \subseteq B \Rightarrow \overline B \subseteq \overline A.
\end{equation}

Die Beweise sind einfach. Als Beispiel nehmen wir $\overline{ A \cap B     } = \overline A \cup \overline B$:
\begin{alignat}{5}
   &\overline{ A \cap B     }\\
   &&&= \{x \in U \mid x \notin A \cap B \} \\
   &&&= \{x \in U \mid \lnot (x \in A \cap B) \} \\
   &&&= \{x \in U \mid \lnot (x \in A \land x \in B) \} \\
   &&&= \{x \in U \mid (\lnot x \in A) \lor (\lnot x \in B) \} \\
   &&&= \{x \in U \mid (x \notin A) \lor ( x \notin B) \} \\
   &&&= \{x \in U \mid x \notin A \} \cup \{x \in U \mid   x \notin B \} \\
   &&&= \overline A \cup \overline  B
\end{alignat}

Wir sehen in diesem Beweis schön eine Korrespondenz zwischen mengentheoretischen und logischen Ausdrücken, eine Korrespondenz, die ein eigenes Video verdient hat.

%******************************************************************************************************
\subsection{Formelfabrik}
%******************************************************************************************************
Die obigen Umformungen lassen sich auch rekursiv auf komplexere Ausdrücke anwenden. Haben wir dann Formeln der Art 
\begin{alignat}{5}
   &\Phi( A,B,C, ...) &&=         &&\, \Psi( A, B, C, ... ) \text{ oder }\\
   &\Phi( A,B,C, ...) &&\subseteq &&\, \Psi( A, B, C, ... ),
\end{alignat}
so gelten auch die Formeln
\begin{alignat}{5}
   &\overline\Phi( \overline A,\overline B,\overline C, ...) &&=         &&\, \overline \Psi( \overline A, \overline B, \overline C, ... ) \text{ oder }\\
   &\overline\Psi( \overline A,\overline B,\overline C, ...) &&\subseteq &&\, \overline \Phi( \overline A, \overline B, \overline C, ... ).
\end{alignat}

$\overline\Phi$ entsteht durch die Transformation
\begin{alignat}{5}
   &\emptyset &&\rightarrow U\\
   &U         &&\rightarrow \emptyset\\
   &\cap      &&\rightarrow \cup\\
   &\cup      &&\rightarrow \cap
\end{alignat}

Gilt \zb für drei bestimmte Mengen $A \subseteq (B \cap \overline  C)$ so gilt automatisch auch $(\overline B \cap C)  \subseteq \overline A$.

%******************************************************************************************************
\subsection{Allgemeingültige Formeln}
%******************************************************************************************************
Wir betrachten nun Formeln, die für alle Mengen gelten, also allgemeingültige Formeln. Wie \zb
\begin{equation}\label{Dist}
   A \cap (B \cup C) = (A \cap B) \cup (A \cap C).
\end{equation}
Hier ist ja eigentlich gemeint
\begin{equation}\label{DistForall}
   \forall A, B, C  \in U \colon A \cap (B \cup C) = (A \cap B) \cup (A \cap C).
\end{equation}
Wenn wir das Dual bilden
\begin{equation}\label{DistDual}
   \overline A \cup (\overline B \cap \overline C) = (\overline A \cup \overline B) \cap (\overline A \cup \overline C),
\end{equation}
so "`laufen"' $\overline A, \overline B, \overline C$ wieder über alle Teilmengen von $U$ und es gilt somit
\begin{equation}\label{Distcup}
   A \cup (B \cap C) = (A \cup B) \cap (A \cup C),
\end{equation}

Bei allgemeingültigen Formeln können wir uns bei der Dualisierung also auf $\cap, \cup, \subseteq, \emptyset$ und $U$ beschränken, die Variablen können wir unberührt lassen.

%******************************************************************************************************
\subsection{Meta-Mathematik}
%******************************************************************************************************
Bei der Formelfabrik arbeiten wir nicht mit Mengen sondern mit Formeln, also Zeichenketten. Der Satz "`allgemeingültige mengentheoretische Aussagen bleiben bei Dualisierung wahr"' ist eine Satz über Sätze. Irgendwie wenden wir hier mathematische Methoden auf unser Geschriebenes an.

Das liegt alles außerhalb von ZFC und ist damit Meta-Mathematik. Diese präzise innerhalb von ZFC zu formulieren ist Aufgabe der mathematischen Logik. 

\begin{backup}
%******************************************************************************************************
%                                                                                                     *
\section{TODO}
%                                                                                                     *
%******************************************************************************************************
Noch zu erledigen sind
\begin{itemize}
   \item leer
\end{itemize}
\end{backup}

\begin{backup}
    (Zur Zeit nicht benötigter Inhalt)
\end{backup}

%******************************************************************************************************
%                                                                                                     *
\begin{thebibliography}{9}
%                                                                                                     *
%******************************************************************************************************

   \bibitem[ArensBusamHettlichKarpfingerStachel2022]{Grundwissen}
      Tilo Arens, Rolf Busam, Frank Hettlich, Christian Karpfinger, Hellmuth Stachel, \emph{Grundwissen Mathematikstudium},
      Springer, 978-3-662-63312-0 (ISBN)

\end{thebibliography}

%******************************************************************************************************
%                                                                                                     *
\begin{large}
    \centerline{\textsc{Symbolverzeichnis}}
\end{large}
%                                                                                                     *
%******************************************************************************************************
\bigskip

\renewcommand*{\arraystretch}{1}

\begin{tabular}{ll}
    $\emptyset$             & die leere Menge $\{\}$\\
    $X, Y, \cdots$          & Mengen\\
    $x, y, \cdots$             & Elemente\\
    $F$                 & eine Funktion\\
    $R$             & eine Relation

\end{tabular}

\end{document}
