%******************************************************** -*-LaTeX-*- ******************************
%                                                                                                  *
% v3.0.2.0.3 Relationen und Funktionen - Leere Funktion.tex                                        *
%                                                                                                  *
% Copyright (C) 2023 Kategory GmbH \& Co. KG (joerg.kunze@kategory.de)                             *
%                                                                                                  *
% v3.0.2.0.3 Relationen und Funktionen - Leere Funktion is part of kategoryMathematik.             *
%                                                                                                  *
% kategoryMathematik is free software: you can redistribute it and/or modify                       *
% it under the terms of the GNU General Public License as published by                             *
% the Free Software Foundation, either version 3 of the License, or                                *
% (at your option) any later version.                                                              *
%                                                                                                  *
% kategoryMathematik is distributed in the hope that it will be useful,                            *
% but WITHOUT ANY WARRANTY; without even the implied warranty of                                   *
% MERCHANTABILITY or FITNESS FOR A PARTICULAR PURPOSE.  See the                                    *
% GNU General Public License for more details.                                                     *
%                                                                                                  *
% You should have received a copy of the GNU General Public License                                *
% along with this program.  If not, see <http://www.gnu.org/licenses/>.                            *
%                                                                                                  *
%***************************************************************************************************

\documentclass[a4paper]{amsart}
% \documentclass[a4paper]{book}

%-----------------------------------------------------------------------------------------------------*
% package:                                                                                            *
%-----------------------------------------------------------------------------------------------------*
\usepackage{amssymb}
\usepackage{amsfonts}
\usepackage{amsmath}
\usepackage{amsthm}

\usepackage{mathabx}

\usepackage{a4wide} % a little bit smaller margins

\usepackage{graphicx}
\usepackage{hyperref}
\usepackage{algorithmic}
\usepackage{listings}
\usepackage{color}
\usepackage{colortbl}
\usepackage{sidecap}
\usepackage{comment}
\usepackage{tcolorbox}
\usepackage{collect}

\usepackage{upgreek}

% \usepackage{diagrams}

\usepackage[german]{babel}
\usepackage[none]{hyphenat}
\emergencystretch=4em

\usepackage[utf8]{inputenc} % to be able to use äöü as characters in text
\usepackage[T1]{fontenc} % to be able to use äöü in lables
\usepackage{lmodern}     % to avoid pixelation introduced by fontenc

\usepackage{hyperref}

\usepackage{tikz}
\usepackage{tikz-cd}
\usetikzlibrary{babel}

%-----------------------------------------------------------------------------------------------------*
% theorem:                                                                                            *
%-----------------------------------------------------------------------------------------------------*
\theoremstyle{definition}
\newtheorem{theorem}{Theorem}[subsection]

\newcommand{\myTheorem}[1]{%
  \newtheorem{jk#1}[theorem]{#1}
  \newenvironment{#1}[1]{%
    \expandafter\begin{jk#1} \expandafter\label{#1:##1}\textbf{(##1):}
  }{%
    \expandafter\end{jk#1}
  }
}

\myTheorem{Definition}
\myTheorem{Proposition}
\myTheorem{Theorem}
\myTheorem{Example}
\myTheorem{Remark}

\definecollection{jkjkFrage}
\newtheorem{jkFrage}[theorem]{Frage}
\newenvironment{Frage}[1]{%
  \expandafter\begin{jkFrage} \expandafter\label{Frage:#1}\textbf{(#1):}
  \begin{collect}{jkjkFrage}{}{}
    \item \ref{Frage:#1} #1
  \end{collect}
}{%
  \expandafter\end{jkFrage}
}

\newcommand{\myRef}[2]{[#1 \ref{#1:#2}, ``#2'']}

\renewcommand{\proofname}{Beweis}

%-----------------------------------------------------------------------------------------------------*
% operator:                                                                                           *
%-----------------------------------------------------------------------------------------------------*
\DeclareMathOperator{\End}{End}
\DeclareMathOperator{\Ker}{Ker}
\DeclareMathOperator{\Mat}{Mat}
\DeclareMathOperator{\rank}{rank}
\DeclareMathOperator{\ggT}{ggT}
\DeclareMathOperator{\len}{len}
\DeclareMathOperator{\ord}{ord}
\DeclareMathOperator{\kgV}{kgV}
\DeclareMathOperator{\id}{id}
\DeclareMathOperator{\red}{red}
\DeclareMathOperator{\supp}{supp}
\DeclareMathOperator{\Bild}{Bild}
\DeclareMathOperator{\Urbild}{Urbild}
\DeclareMathOperator{\Rang}{Rang}
\DeclareMathOperator{\Det}{Det}
\DeclareMathOperator{\Hom}{Hom}

\DeclareMathOperator{\sub}{sub}
\DeclareMathOperator{\blk}{blk}
\DeclareMathOperator{\minimal}{minimal}
\DeclareMathOperator{\maximal}{maximal}

\definecolor{mygreen}{rgb}{0,0.6,0}
\definecolor{mygray}{rgb}{0.5,0.5,0.5}
\definecolor{mymauve}{rgb}{0.58,0,0.82}

\lstset{ %
  backgroundcolor=\color{white},   % choose the background color
  basicstyle=\ttfamily\footnotesize,        % size of fonts used for the code
  breaklines=true,                 % automatic line breaking only at whitespace
  captionpos=b,                    % sets the caption-position to bottom
  commentstyle=\color{mygreen},    % comment style
  escapeinside={\%*}{*)},          % if you want to add LaTeX within your code
  keywordstyle=\color{blue},       % keyword style
  stringstyle=\color{mymauve},     % string literal style
  frame=single
}

\setcounter{MaxMatrixCols}{20}

%******************************************************************************************************
%                                                                                                     *
% definition:                                                                                         *
%                                                                                                     *
%******************************************************************************************************
\newcommand{\R}{\ensuremath{\mathbb{ R }}}
\newcommand{\Q}{\ensuremath{\mathbb{ Q }}}
\newcommand{\Z}{\ensuremath{\mathbb{ Z }}}
\newcommand{\N}{\ensuremath{\mathbb{ N }}}
\newcommand{\C}{\ensuremath{\mathbb{ C }}}
\newcommand{\A}{\ensuremath{\mathbb{ A }}}
\newcommand{\F}{\ensuremath{\mathbb{ F }}}
\newcommand{\K}{\ensuremath{\mathbb{ K }}}
\newcommand{\Pb}{\ensuremath{\mathbb{ P }}}

\newcommand{\M}{\ensuremath{\mathcal{ M }}}
\newcommand{\V}{\ensuremath{\mathcal{ V }}}

\newcommand{\AAA}{\ensuremath{\mathcal{ A }}}
\newcommand{\BB}{\ensuremath{\mathcal{ B }}}
\newcommand{\CC}{\ensuremath{\mathcal{ C }}}
\newcommand{\EE}{\ensuremath{\mathcal{ E }}}
\newcommand{\KK}{\ensuremath{\mathcal{ K }}}
\newcommand{\MM}{\ensuremath{\mathcal{ M }}}
\newcommand{\PP}{\ensuremath{\mathcal{ P }}}
\newcommand{\ZZ}{\ensuremath{\mathcal{ Z }}}

\newcommand{\imporant}[1]{ \textcolor{red}{\textbf{#1}} }

\newcommand{\bb}[1]{\mathbf{#1}}
\newcommand{\balpha}{\boldsymbol{\upalpha}}
\newcommand{\bbeta}{\boldsymbol{\upbeta}}
\newcommand{\bgamma}{\boldsymbol{\upgamma}}
\newcommand{\bdelta}{\boldsymbol{\delta}}
\newcommand{\bmu}{\boldsymbol{\upmu}}

\newcommand{\z}[1]{\Z_{#1}}
\newcommand{\e}[1]{\z{#1}^*}
\newcommand{\q}[1]{(\e{#1})^2}

\excludecomment{book}
\excludecomment{example}
\excludecomment{backup}

\begin{document}

%******************************************************************************************************
%                                                                                                     *
\begin{titlepage}
%                                                                                                     *
%******************************************************************************************************
% \vspace*{\fill}
\centering
{\huge
(Bachelor) Anfang\\[1cm]
\textbf{v3.0.2.0.3 Relationen und Funktionen - Leere Funktion}
}\\[1cm]

\textbf{Kategory GmbH \& Co. KG}\\
Präsentiert von Jörg Kunze\\
Copyright (C) 2024 Kategory GmbH \& Co. KG

\end{titlepage}

%\clearpage
%\setcounter{page}{2}
%
%\tableofcontents

\newpage

%******************************************************************************************************
%                                                                                                     *
\section*{Beschreibung}
%                                                                                                     *
%******************************************************************************************************

%******************************************************************************************************
\subsection*{Inhalt}
%******************************************************************************************************
Leere Funktionen entstehen, wenn die Quelle leer ist. 

Von der leeren Menge gibt es genau eine Funktion, die leere Funktion. Und die leere Funktion existiert nur, wenn die Quelle leer ist. Es gibt keine Funktion, auch nicht die leere, von einer nicht-leeren Menge in die leere.

Relationen sind Teilmengen des kartesischen Produktes von Quelle und Ziel. Ist eine der beiden Mengen leer, ist die leere Menge die einzige Teilmenge dieses Produktes. Aber auch, wenn beide Mengen ungleich leer sind, ist die leere Menge eine Teilmenge des kartesischen Produktes. Diese Teilmenge nennen wir die leere Relation. Sie ist eine Relation wie alle anderen auch, mit allen Rechten.

Funktionen sind spezielle Relationen. Frage: ist die leere Relation eine Funktion?

Ja, und zwar genau dann, wenn die Quelle leer ist, da dann die Bedingung dafür, eine Funktion zu sein, eine leere Wahrheit ist.

Andererseits gibt es keine Funktion von einer nicht-leeren Quelle in ein leeres Ziel. 

Somit können wir sagen:
Wenn die Quelle leer ist, ENTHÄLT die Menge der Funktionen von der Quelle in das Ziel die leeren Menge. Wenn die Quelle nicht-leer das Ziel aber leer ist, IST die Menge der Funktionen von der Quelle in das Ziel die leeren Menge.

%******************************************************************************************************
\subsection*{Präsentiert}
%******************************************************************************************************
Von Jörg Kunze

%******************************************************************************************************
\subsection*{Voraussetzungen}
%******************************************************************************************************
Relation, Funktion, Teilmenge, leere Wahrheit.

%******************************************************************************************************
\subsection*{Text}
%******************************************************************************************************
Der Begleittext als PDF und als LaTeX findet sich unter
{\tiny
   \url{}
}

%******************************************************************************************************
\subsection*{Meine Videos}
%******************************************************************************************************
Siehe auch in den folgenden Videos:\\
v3.0.2.0.1 (Bachelor) Relationen und Funktionen - Surjektiv, injektiv, bijektiv\\
\url{https://youtu.be/8YFNEWZBpWc}\\
v3.0.2 (Bachelor) Relationen und Funktionen\\
\url{https://youtu.be/qjhNZXFAYEM}

%******************************************************************************************************
\subsection*{Quellen}
%******************************************************************************************************
Siehe auch in den folgenden Seiten:\\
\url{https://de.wikipedia.org/wiki/Relation_(Mathematik)}\\
\url{https://jcsites.juniata.edu/faculty/rhodes/ida/relations.html}

%******************************************************************************************************
\subsection*{Buch}
%******************************************************************************************************
Grundlage ist folgendes Buch:\\
"`Grundwissen Mathematikstudium"'\\
Tilo Arens, Rolf Busam, Frank Hettlich, Christian Karpfinger, Hellmuth Stachel \\
2022\\
Springer-Verlag\\
978-3-662-63312-0 (ISBN)\\
{\tiny
   \url{https://www.lehmanns.de/shop/mathematik-informatik/56427740-9783662633120-grundwissen-mathematikstudium}}\\
\\

%******************************************************************************************************
\subsection*{Lizenz}
%******************************************************************************************************
Dieser Text und das Video sind freie Software. Sie können es unter den Bedingungen der 
GNU General Public License, wie von der Free Software Foundation veröffentlicht, weitergeben 
und/oder modifizieren, entweder gemäß Version 3 der Lizenz oder (nach Ihrer Option) jeder späteren Version.

Die Veröffentlichung von Text und Video erfolgt in der Hoffnung, dass es Ihnen von Nutzen sein wird, 
aber OHNE IRGENDEINE GARANTIE, sogar ohne die implizite Garantie der MARKTREIFE oder der 
VERWENDBARKEIT FÜR EINEN BESTIMMTEN ZWECK. Details finden Sie in der GNU General Public License.

Sie sollten ein Exemplar der GNU General Public License zusammen mit diesem Text erhalten haben 
(zu finden im selben Git-Projekt). 
Falls nicht, siehe \url{http://www.gnu.org/licenses/}.

\subsection*{Das Video}
%******************************************************************************************************
Das Video hierzu ist zu finden unter 
{\tiny
   \url{huhu}
}

%******************************************************************************************************
%                                                                                                     *
\section{Geometrisches Bild von Relationen und Funktionen}
%                                                                                                     *
%******************************************************************************************************

Zur Förderung der Intuition und um verschiedene Aspekte von Relationen und damit auch Funktionen zu visualisieren, stellen wir diese hier graphisch in drei Weisen dar. Die Nummern beziehen sich auf die händischen Skizzen auf den beiden eingefügten Bildern.



\begin{backup}
%******************************************************************************************************
%                                                                                                     *
\section{TODO}
%                                                                                                     *
%******************************************************************************************************
Noch zu erledigen sind
\begin{itemize}
   \item leer
\end{itemize}
\end{backup}

\begin{backup}
    (Zur Zeit nicht benötigter Inhalt)
\end{backup}

%******************************************************************************************************
%                                                                                                     *
\begin{thebibliography}{9}
%                                                                                                     *
%******************************************************************************************************

   \bibitem[ArensBusamHettlichKarpfingerStachel2022]{Grundwissen}
      Tilo Arens, Rolf Busam, Frank Hettlich, Christian Karpfinger, Hellmuth Stachel, \emph{Grundwissen Mathematikstudium},
      Springer, 978-3-662-63312-0 (ISBN)
      
\end{thebibliography}

%******************************************************************************************************
%                                                                                                     *
\begin{large}
    \centerline{\textsc{Symbolverzeichnis}}
\end{large}
%                                                                                                     *
%******************************************************************************************************
\bigskip

\renewcommand*{\arraystretch}{1}

\begin{tabular}{ll}
    $X, Y, \cdots$          & Mengen\\
    $x, y, \cdots$             & Elemente\\
    $R$             & Eine Relation
   
\end{tabular}

\end{document}
