%******************************************************** -*-LaTeX-*- ******************************
%                                                                                                  *
% v1.2.1.2 Vergleichen.tex                                                                         *
%                                                                                                  *
% Copyright (C) 2025 Kategory GmbH \& Co. KG (joerg.kunze@kategory.de)                             *
%                                                                                                  *
% v1.2.1.2 Vergleichen is part of kategoryMathematik.                                              *
%                                                                                                  *
% kategoryMathematik is free software: you can redistribute it and/or modify                       *
% it under the terms of the GNU General Public License as published by                             *
% the Free Software Foundation, either version 3 of the License, or                                *
% (at your option) any later version.                                                              *
%                                                                                                  *
% kategoryMathematik is distributed in the hope that it will be useful,                            *
% but WITHOUT ANY WARRANTY; without even the implied warranty of                                   *
% MERCHANTABILITY or FITNESS FOR A PARTICULAR PURPOSE.  See the                                    *
% GNU General Public License for more details.                                                     *
%                                                                                                  *
% You should have received a copy of the GNU General Public License                                *
% along with this program.  If not, see <http://www.gnu.org/licenses/>.                            *
%                                                                                                  *
%***************************************************************************************************

\documentclass[a4paper]{amsart}
% \documentclass[a4paper]{book}

%-----------------------------------------------------------------------------------------------------*
% package:                                                                                            *
%-----------------------------------------------------------------------------------------------------*
\usepackage{amssymb}
\usepackage{amsfonts}
\usepackage{amsmath}
\usepackage{amsthm}

\usepackage{mathabx}

\usepackage{a4wide} % a little bit smaller margins

\usepackage{graphicx}
\usepackage{hyperref}
\usepackage{algorithmic}
\usepackage{listings}
\usepackage{color}
\usepackage{colortbl}
\usepackage{sidecap}
\usepackage{comment}
\usepackage{tcolorbox}
\usepackage{collect}

\usepackage{upgreek}

% \usepackage{diagrams}

\usepackage[german]{babel}
\usepackage[none]{hyphenat}
\emergencystretch=4em

\usepackage[utf8]{inputenc} % to be able to use äöü as characters in text
\usepackage[T1]{fontenc} % to be able to use äöü in lables
\usepackage{lmodern}     % to avoid pixelation introduced by fontenc

\usepackage{hyperref}

\usepackage{tikz}
\usepackage{tikz-cd}
\usetikzlibrary{babel}

%-----------------------------------------------------------------------------------------------------*
% theorem:                                                                                            *
%-----------------------------------------------------------------------------------------------------*
\theoremstyle{definition}
\newtheorem{theorem}{Theorem}[subsection]

\newcommand{\myTheorem}[1]{%
  \newtheorem{jk#1}[theorem]{#1}
  \newenvironment{#1}[1]{%
    \expandafter\begin{jk#1} \expandafter\label{#1:##1}\textbf{(##1):}
  }{%
    \expandafter\end{jk#1}
  }
}

\myTheorem{Definition}
\myTheorem{Proposition}
\myTheorem{Satz}
\myTheorem{Theorem}
\myTheorem{Axiom}
\myTheorem{Beispiel}
\myTheorem{Anmerkung}

\definecollection{jkjkFrage}
\newtheorem{jkFrage}[theorem]{Frage}
\newenvironment{Frage}[1]{%
  \expandafter\begin{jkFrage} \expandafter\label{Frage:#1}\textbf{(#1):}
  \begin{collect}{jkjkFrage}{}{}
    \item \ref{Frage:#1} #1
  \end{collect}
}{%
  \expandafter\end{jkFrage}
}

\newcommand{\myRef}[2]{[#1 \ref{#1:#2}, ``#2'']}

\renewcommand{\proofname}{Beweis}

%-----------------------------------------------------------------------------------------------------*
% operator:                                                                                           *
%-----------------------------------------------------------------------------------------------------*
\DeclareMathOperator{\End}{End}
\DeclareMathOperator{\Ker}{Ker}
\DeclareMathOperator{\Mat}{Mat}
\DeclareMathOperator{\rank}{rank}
\DeclareMathOperator{\ggT}{ggT}
\DeclareMathOperator{\len}{len}
\DeclareMathOperator{\ord}{ord}
\DeclareMathOperator{\kgV}{kgV}
\DeclareMathOperator{\id}{id}
\DeclareMathOperator{\red}{red}
\DeclareMathOperator{\supp}{supp}
\DeclareMathOperator{\Bild}{Bild}
\DeclareMathOperator{\Rang}{Rang}
\DeclareMathOperator{\Det}{Det}
\DeclareMathOperator{\Hom}{Hom}
\DeclareMathOperator{\GL}{GL}

\DeclareMathOperator{\sub}{sub}
\DeclareMathOperator{\blk}{blk}
\DeclareMathOperator{\minimal}{minimal}
\DeclareMathOperator{\maximal}{maximal}

\DeclareMathOperator{\Dom}{Dom}
\DeclareMathOperator{\Cod}{Cod}
\DeclareMathOperator{\Obj}{Obj}

\definecolor{mygreen}{rgb}{0,0.6,0}
\definecolor{mygray}{rgb}{0.5,0.5,0.5}
\definecolor{mymauve}{rgb}{0.58,0,0.82}

\lstset{ %
  backgroundcolor=\color{white},   % choose the background color
  basicstyle=\ttfamily\footnotesize,        % size of fonts used for the code
  breaklines=true,                 % automatic line breaking only at whitespace
  captionpos=b,                    % sets the caption-position to bottom
  commentstyle=\color{mygreen},    % comment style
  escapeinside={\%*}{*)},          % if you want to add LaTeX within your code
  keywordstyle=\color{blue},       % keyword style
  stringstyle=\color{mymauve},     % string literal style
  frame=single
}

\setcounter{MaxMatrixCols}{20}

%******************************************************************************************************
%                                                                                                     *
% definition:                                                                                         *
%                                                                                                     *
%******************************************************************************************************
\newcommand{\R}{\ensuremath{\mathbb{ R }}}
\newcommand{\Q}{\ensuremath{\mathbb{ Q }}}
\newcommand{\Z}{\ensuremath{\mathbb{ Z }}}
\newcommand{\N}{\ensuremath{\mathbb{ N }}}
\newcommand{\C}{\ensuremath{\mathbb{ C }}}
\newcommand{\A}{\ensuremath{\mathbb{ A }}}
\newcommand{\F}{\ensuremath{\mathbb{ F }}}
\newcommand{\K}{\ensuremath{\mathbb{ K }}}
\newcommand{\Pb}{\ensuremath{\mathbb{ P }}}

\newcommand{\M}{\ensuremath{\mathcal{ M }}}
\newcommand{\V}{\ensuremath{\mathcal{ V }}}

\newcommand{\AAA}{\ensuremath{\mathcal{ A }}}
\newcommand{\BB}{\ensuremath{\mathcal{ B }}}
\newcommand{\CC}{\ensuremath{\mathcal{ C }}}
\newcommand{\DD}{\ensuremath{\mathcal{ D }}}
\newcommand{\EE}{\ensuremath{\mathcal{ E }}}
\newcommand{\FF}{\ensuremath{\mathcal{ F }}}
\newcommand{\KK}{\ensuremath{\mathcal{ K }}}
\newcommand{\MM}{\ensuremath{\mathcal{ M }}}
\newcommand{\PP}{\ensuremath{\mathcal{ P }}}
\newcommand{\ZZ}{\ensuremath{\mathcal{ Z }}}

\newcommand{\Set}{\text{\textbf{Set}}}
\newcommand{\Ab}{\text{\textbf{Ab}}}

\newcommand{\imporant}[1]{ \textcolor{red}{\textbf{#1}} }

\newcommand{\bb}[1]{\mathbf{#1}}
\newcommand{\balpha}{\boldsymbol{\upalpha}}
\newcommand{\bbeta}{\boldsymbol{\upbeta}}
\newcommand{\bgamma}{\boldsymbol{\upgamma}}
\newcommand{\bdelta}{\boldsymbol{\delta}}
\newcommand{\bmu}{\boldsymbol{\upmu}}

\newcommand{\z}[1]{\Z_{#1}}
\newcommand{\e}[1]{\z{#1}^*}
\newcommand{\q}[1]{(\e{#1})^2}
\newcommand{\m}{\mathcal}

\excludecomment{book}
\excludecomment{example}
\excludecomment{backup}

\newcommand{\zb}{z.~B. }

\begin{document}

%******************************************************************************************************
%                                                                                                     *
\begin{titlepage}
%                                                                                                     *
%******************************************************************************************************
% \vspace*{\fill}
\centering
{\huge
(Grund) Packungs-Rechnen\\[1cm]
\textbf{v1.2.1.2 Vergleichen}
}\\[1cm]

\textbf{Kategory GmbH \& Co. KG}\\
Präsentiert von Jörg Kunze\\
Copyright (C) 2025 Kategory GmbH \& Co. KG

\end{titlepage}

%\clearpage
%\setcounter{page}{2}
%
%\tableofcontents

\newpage

%******************************************************************************************************
%                                                                                                     *
\section*{Beschreibung}
%                                                                                                     *
%******************************************************************************************************

\subsection*{Inhalt}
Beim Vergleichen von in Packungen organisierten Dingen vergleichen wir zunächst nur die (vollständig gefüllten) Packungen selber und nur, wenn diese gleich sind, auch noch die einzelnen.

Wie immer beim Vergleich, müssen wir die Packungen oder einzelnen nicht zählen, wir können sie zuordnen und sehen, wo was übrig bleibt.

Wir können auf diese Weise recht effizient Zahlen bis hundert vergleichen.

Das ist aber nicht der Punkt, denn diese Effizienz haben wir auch bei der Blockschreibweise oder, wenn wir für je zehn ein X nehmen.

Der Knackpunkt bei Packungen ist: wir rechnen erst mit den Packungen, dann mit den einzelnen. Wir teilen also unsere Arbeit in zwei Abschnitte. Und wir betrachten die einzelnen als angebrochene Packung. Auf diese Weise werden wir auch Addition und Subtraktion schneller bewältigen können.

\subsection*{Präsentiert}
Von Jörg Kunze

\subsection*{Voraussetzungen}
Zählen, ein wenig rechnen mit Strichen und Punkten, Zählen mit Packungen

\subsection*{Text}
Der Begleittext als PDF und als LaTeX findet sich unter\\
{\tiny\url{https://github.com/kategory/kategoryMathematik/tree/main/v1%20Grund/v1.2%20Grundrechenarten/v1.2.1%20Packungs-Rechnen/v1.2.1.2%20Vergleichen}}.

\subsection*{Meine Videos}
Siehe auch in den folgenden Videos:\\
v1.2.1.1 (Grund) Packung - Zählen\\
\url{https://youtu.be/PG8WMmemEYc}\\
\\
v1.1.2.5.6.1 (Grund) Zahlen - Darstellung\\
\url{https://youtu.be/t8cyZevFWFs}\\
\\
v1.1.2.5.6 (Grund) Zahlen - Pack Schreibweise\\
\url{https://youtu.be/OGXoLiBL2MQ}\\
\\
v1.1.2.4 (Grund) Zählen\\
\url{ttps://youtu.be/I6iIG2ZtPCU}\\

\subsection*{Quellen}
Siehe auch in den folgenden Seiten:\\
(Ich habe bisher nichts gefunden.)

\subsection*{Buch}
Grundlage ist folgendes Buch:\\
"`Basiswissen Grundschule – Mathematik"'\\
Ute Müller-Wolfangel, Beate Schreiber\\
2014\\
Bibliographisches Institut\\
978-3-411-72063-7 (ISBN)
\\
{\tiny\url{https://www.lehmanns.de/shop/schulbuch-lexikon-woerterbuch/28535581-9783411720637-basiswissen-grundschule-mathematik}}

\subsection*{Lizenz}
Dieser Text und das Video sind freie Software. Sie können es unter den Bedingungen der
GNU General Public License, wie von der Free Software Foundation veröffentlicht, weitergeben
und/oder modifizieren, entweder gemäß Version 3 der Lizenz oder (nach Ihrer Option) jeder späteren Version.

Die Veröffentlichung von Text und Video erfolgt in der Hoffnung, dass es Ihnen von Nutzen sein wird,
aber OHNE IRGENDEINE GARANTIE, sogar ohne die implizite Garantie der MARKTREIFE oder der
VERWENDBARKEIT FÜR EINEN BESTIMMTEN ZWECK. Details finden Sie in der GNU General Public License.

Sie sollten ein Exemplar der GNU General Public License zusammen mit diesem Text erhalten haben
(zu finden im selben Git-Projekt).
Falls nicht, siehe \url{http://www.gnu.org/licenses/}.

%******************************************************************************************************
\subsection*{Das Video}
%******************************************************************************************************
Das Video hierzu ist zu finden unter \url{Ups}

%******************************************************************************************************
%                                                                                                     *
\section{v1.2.1.2 Vergleichen}
%                                                                                                     *
%******************************************************************************************************

%******************************************************************************************************
\subsection{Vergleichen}
%******************************************************************************************************
1. Schritt: wir vergleichen zunächst nur die vollen Packungen, durch Zuordnung, Untereinanderschreiben oder, wer will, Zählen. Ist eine der beiden Zahlen mehr Schachteln ist sie größer und wir sind fertig.

2. Wenn die Zahl der Schachteln gleich ist, vergleichen wir die einzelnen. Hat dann eine der beiden Zahlen mehr einzelne, ist sie größer. Wenn es auch gleich viele einzelnen sind, sind beide Zahlen die selben.

Hat eine Zahl keine Schachteln die andere aber schon, ist die mit Schachteln die größere, weil $0$ kleiner als jede zahl ab $1$ ist.

Haben beide Zahlen keine Schachteln, können wir direkt mit Schritt 2 weitermachen. Genauer: Wir sehen sehr schnell, dass beide Zahlen gleich viele Schachteln haben, nämlich Null.

%******************************************************************************************************
\subsection{Blöcke}
%******************************************************************************************************
Der Vergleich geht genauso mit Zehner-Blöcken oder X als Zehn. 

Wir wollen uns die Packung immer als gefüllt mit 10 Kugeln vorstellen. Und die einzelnen als angebrochene Packung. Dies ist eine Vorbereitung für das Rechnen mit Packungen, was dadurch einfacher wird. Beim Rechnen werden wir manchmal volle Packungen anbrechen und angebrochene vervollständigen.

%******************************************************************************************************
\subsection{Strategie: erst die vollständigen Packungen dann die einzelnen}
%******************************************************************************************************
Neben der Vorstellung der einzelnen als angebrochene Packung mit einem einfachen Blick dafür, wie viele zur vollen Packung fehlen ist dies auch ein erster Schritt zu einer Strategie für das Rechnen mit großen Zahlen: erst die vollständigen Packungen dann die einzelnen in den angebrochenen Packungen.

\begin{backup}
   (Zur Zeit nicht benötigter Inhalt)
\end{backup}

%******************************************************************************************************
%                                                                                                     *
\begin{thebibliography}{9}
%                                                                                                     *
%******************************************************************************************************
   \bibitem [MüllerWolfangel2014]{MüllerWolfangel}
      Ute Müller-Wolfangel, Beate Schreiber \emph{Basiswissen Grundschule – Mathematik}
      Bibliographisches Institut 2014, 978-3-411-72063-7 (ISBN)

\end{thebibliography}

%******************************************************************************************************
%                                                                                                     *
\begin{large}
    \centerline{\textsc{Symbolverzeichnis}}
\end{large}
%                                                                                                     *
%******************************************************************************************************
\bigskip

\renewcommand*{\arraystretch}{1}

\begin{tabular}{ll}
    $0,1,2,3,4,5,6,7,8,9$          & Ziffern\\
    $\infty$                       & Unendlich
\end{tabular}

\end{document}
