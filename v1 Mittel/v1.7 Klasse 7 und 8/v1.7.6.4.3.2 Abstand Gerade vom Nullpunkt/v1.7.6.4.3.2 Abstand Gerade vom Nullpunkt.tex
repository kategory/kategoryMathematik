%******************************************************** -*-LaTeX-*- ******************************
%                                                                                                  *
% v1.7.6.4.3.2 Abstand Gerade vom Nullpunkt.tex                                                    *
%                                                                                                  *
% Copyright (C) 2023 Kategory GmbH \& Co. KG (joerg.kunze@kategory.de)                             *
%                                                                                                  *
% v1.7.6.4.3.2 is part of kategoryMathematik.                                                      *
%                                                                                                  *
% kategoryMathematik is free software: you can redistribute it and/or modify                       *
% it under the terms of the GNU General Public License as published by                             *
% the Free Software Foundation, either version 3 of the License, or                                *
% (at your option) any later version.                                                              *
%                                                                                                  *
% kategoryMathematik is distributed in the hope that it will be useful,                            *
% but WITHOUT ANY WARRANTY; without even the implied warranty of                                   *
% MERCHANTABILITY or FITNESS FOR A PARTICULAR PURPOSE.  See the                                    *
% GNU General Public License for more details.                                                     *
%                                                                                                  *
% You should have received a copy of the GNU General Public License                                *
% along with this program.  If not, see <http://www.gnu.org/licenses/>.                            *
%                                                                                                  *
%***************************************************************************************************

\documentclass[a4paper]{amsart}
% \documentclass[a4paper]{book}

%-----------------------------------------------------------------------------------------------------*
% package:                                                                                            *
%-----------------------------------------------------------------------------------------------------*
\usepackage{amssymb}
\usepackage{amsfonts}
\usepackage{amsmath}
\usepackage{amsthm}

\usepackage{mathabx}

\usepackage{a4wide} % a little bit smaller margins

\usepackage{graphicx}
\usepackage{hyperref}
\usepackage{algorithmic}
\usepackage{listings}
\usepackage{color}
\usepackage{colortbl}
\usepackage{sidecap}
\usepackage{comment}
\usepackage{tcolorbox}
\usepackage{collect}

\usepackage{upgreek}

% \usepackage{diagrams}

\usepackage[german]{babel}
\usepackage[none]{hyphenat}
\emergencystretch=4em

\usepackage[utf8]{inputenc} % to be able to use äöü as characters in text
\usepackage[T1]{fontenc} % to be able to use äöü in lables
\usepackage{lmodern}     % to avoid pixelation introduced by fontenc

\usepackage{hyperref}

\usepackage{tikz}
\usepackage{tikz-cd}
\usetikzlibrary{babel}

%-----------------------------------------------------------------------------------------------------*
% theorem:                                                                                            *
%-----------------------------------------------------------------------------------------------------*
\theoremstyle{definition}
\newtheorem{theorem}{Theorem}[subsection]

\newcommand{\myTheorem}[1]{%
  \newtheorem{jk#1}[theorem]{#1}
  \newenvironment{#1}[1]{%
    \expandafter\begin{jk#1} \expandafter\label{#1:##1}\textbf{(##1):}
  }{%
    \expandafter\end{jk#1}
  }
}

\myTheorem{Definition}
\myTheorem{Proposition}
\myTheorem{Theorem}
\myTheorem{Example}
\myTheorem{Remark}

\definecollection{jkjkFrage}
\newtheorem{jkFrage}[theorem]{Frage}
\newenvironment{Frage}[1]{%
  \expandafter\begin{jkFrage} \expandafter\label{Frage:#1}\textbf{(#1):}
  \begin{collect}{jkjkFrage}{}{}
    \item \ref{Frage:#1} #1
  \end{collect}
}{%
  \expandafter\end{jkFrage}
}

\newcommand{\myRef}[2]{[#1 \ref{#1:#2}, ``#2'']}

\renewcommand{\proofname}{Beweis}

%-----------------------------------------------------------------------------------------------------*
% operator:                                                                                           *
%-----------------------------------------------------------------------------------------------------*
\DeclareMathOperator{\End}{End}
\DeclareMathOperator{\Ker}{Ker}
\DeclareMathOperator{\Mat}{Mat}
\DeclareMathOperator{\rank}{rank}
\DeclareMathOperator{\ggT}{ggT}
\DeclareMathOperator{\len}{len}
\DeclareMathOperator{\ord}{ord}
\DeclareMathOperator{\kgV}{kgV}
\DeclareMathOperator{\id}{id}
\DeclareMathOperator{\red}{red}
\DeclareMathOperator{\supp}{supp}
\DeclareMathOperator{\Bild}{Bild}
\DeclareMathOperator{\Rang}{Rang}
\DeclareMathOperator{\Det}{Det}
\DeclareMathOperator{\Hom}{Hom}

\DeclareMathOperator{\sub}{sub}
\DeclareMathOperator{\blk}{blk}
\DeclareMathOperator{\minimal}{minimal}
\DeclareMathOperator{\maximal}{maximal}

\definecolor{mygreen}{rgb}{0,0.6,0}
\definecolor{mygray}{rgb}{0.5,0.5,0.5}
\definecolor{mymauve}{rgb}{0.58,0,0.82}

\lstset{ %
  backgroundcolor=\color{white},   % choose the background color
  basicstyle=\ttfamily\footnotesize,        % size of fonts used for the code
  breaklines=true,                 % automatic line breaking only at whitespace
  captionpos=b,                    % sets the caption-position to bottom
  commentstyle=\color{mygreen},    % comment style
  escapeinside={\%*}{*)},          % if you want to add LaTeX within your code
  keywordstyle=\color{blue},       % keyword style
  stringstyle=\color{mymauve},     % string literal style
  frame=single
}

\setcounter{MaxMatrixCols}{20}

%******************************************************************************************************
%                                                                                                     *
% definition:                                                                                         *
%                                                                                                     *
%******************************************************************************************************
\newcommand{\R}{\ensuremath{\mathbb{ R }}}
\newcommand{\Q}{\ensuremath{\mathbb{ Q }}}
\newcommand{\Z}{\ensuremath{\mathbb{ Z }}}
\newcommand{\N}{\ensuremath{\mathbb{ N }}}
\newcommand{\C}{\ensuremath{\mathbb{ C }}}
\newcommand{\A}{\ensuremath{\mathbb{ A }}}
\newcommand{\F}{\ensuremath{\mathbb{ F }}}
\newcommand{\K}{\ensuremath{\mathbb{ K }}}
\newcommand{\Pb}{\ensuremath{\mathbb{ P }}}

\newcommand{\M}{\ensuremath{\mathcal{ M }}}
\newcommand{\V}{\ensuremath{\mathcal{ V }}}

\newcommand{\AAA}{\ensuremath{\mathcal{ A }}}
\newcommand{\BB}{\ensuremath{\mathcal{ B }}}
\newcommand{\CC}{\ensuremath{\mathcal{ C }}}
\newcommand{\EE}{\ensuremath{\mathcal{ E }}}
\newcommand{\KK}{\ensuremath{\mathcal{ K }}}
\newcommand{\MM}{\ensuremath{\mathcal{ M }}}
\newcommand{\PP}{\ensuremath{\mathcal{ P }}}
\newcommand{\ZZ}{\ensuremath{\mathcal{ Z }}}

\newcommand{\imporant}[1]{ \textcolor{red}{\textbf{#1}} }

\newcommand{\bb}[1]{\mathbf{#1}}
\newcommand{\balpha}{\boldsymbol{\upalpha}}
\newcommand{\bbeta}{\boldsymbol{\upbeta}}
\newcommand{\bgamma}{\boldsymbol{\upgamma}}
\newcommand{\bdelta}{\boldsymbol{\delta}}
\newcommand{\bmu}{\boldsymbol{\upmu}}

\newcommand{\z}[1]{\Z_{#1}}
\newcommand{\e}[1]{\z{#1}^*}
\newcommand{\q}[1]{(\e{#1})^2}

\excludecomment{book}
\excludecomment{example}
\excludecomment{backup}

\begin{document}

%******************************************************************************************************
%                                                                                                     *
\begin{titlepage}
%                                                                                                     *
%******************************************************************************************************
% \vspace*{\fill}
\centering
{\huge
(Mittel) Geometrie\\[1cm]
\textbf{v1.7.6.4.3.2 Abstand Gerade vom Nullpunkt}
}\\[1cm]

\textbf{Kategory GmbH \& Co. KG}\\
Präsentiert von Jörg Kunze\\
Copyright (C) 2023 Kategory GmbH \& Co. KG

\end{titlepage}

%\clearpage
%\setcounter{page}{2}
%
%\tableofcontents

\newpage

%******************************************************************************************************
%                                                                                                     *
\section*{Beschreibung}
%                                                                                                     *
%******************************************************************************************************

%******************************************************************************************************
\subsection*{Inhalt}
%******************************************************************************************************
Der Abstand einer Geraden von einem Punkt muss zunächst mal definiert werden! Es ist beileibe nicht so, dass dieser Begriff logisch abgeleitet werden kann. Wir definieren ihn als den kleinsten Abstand eines Punktes auf der Geraden zu diesem Punkt.

Hier kümmern wir uns speziell um den Abstand einer Geraden vom Nullpunkt, weil dann die Rechnerei ein wenig übersichtlicher bleibt.

Den Abstand von einem Punkt zu einem anderen haben wir schon definiert. Über die Definition des Pythagoras. Die Formel enthält eine Wurzel. Damit ist das Finden des Minimums nicht ganz einfach. Wegen der Monotonie der Wurzel, die auf der der Quadratischen Funktion basiert, können wir aber auch den Term unterhalb der Wurzel minimieren. Das ist ein quadratischer Term.

Also eine Parabel. Das Minimum nimmt eine Parabel an ihrem Scheitelpunkt ein. Die Formel dafür kennen wir, und sie ist überraschend einfach.

Mit ihr berechnen wir neben dem oben definierten Abstand zu einem Punkt auch die Koordinaten des Punktes auf der Geraden, der das Minimum realisiert. 

Abstand und Koordinaten berechnen wir dann auch noch in dem Spezialfall des Nullpunktes, da damit die Formeln nochmal erheblich einfacher werden.

%******************************************************************************************************
\subsection*{Präsentiert}
%******************************************************************************************************
Von Jörg Kunze

%******************************************************************************************************
\subsection*{Voraussetzungen}
%******************************************************************************************************
Koordinaten-System, Geradengleichung, Abstand zwischen zwei Punkten, Parabel, Lösung der quadratischen Gleichung, Formel für den Scheitelpunkt einer Parabel.

%******************************************************************************************************
\subsection*{Text}
%******************************************************************************************************
Der Begleittext als PDF und als LaTeX findet sich unter
{\tiny
   \url{https://github.com/kategory/kategoryMathematik/tree/main/v1.6%20Mittel/v1.7.6%20Geometrie/v1.7.6.4.3.2%20Abstand%20Gerade%20vom%20Nullpunkt}
}

%******************************************************************************************************
\subsection*{Meine Videos}
%******************************************************************************************************
Siehe auch in den folgenden Videos:\\
\\
v1.7.6.4.2 (Mittel) Geometrie - Satz des Pythagoras\\
\url{https://youtu.be/mTMOtXfUQzQ}\\
\\
v1.7.6.4.3 (Mittel) Geometrie - Abstand im Koordinatensystem\\
\url{https://youtu.be/HEdolewfn78}\\
\\
v1.7.4.7 (Mittel) abc-Formel\\
\url{https://youtu.be/tEXLpYN7O-0}

%******************************************************************************************************
\subsection*{Quellen}
%******************************************************************************************************
Siehe auch in den folgenden Seiten:\\
\\
\url{https://de.wikipedia.org/wiki/Satz_des_Pythagoras}\\
\url{https://de.wikipedia.org/wiki/Euklidischer_Abstand}\\
\url{https://de.wikipedia.org/wiki/Geradengleichung}

%******************************************************************************************************
\subsection*{Buch}
%******************************************************************************************************
Grundlage ist folgendes Buch:\\

"`KomplettWissen Mathematik Gymnasium Klasse 5-10"'\\
Klett Lerntraining bei PONS\\
978-3-12-926097-5 (ISBN)\\
{\tiny
   \url{https://www.lehmanns.de/shop/schulbuch-lexikon-woerterbuch/35031626-9783129260975-komplettwissen-mathematik-gymnasium-klasse-5-10
   }
}\\
\\
"`Grundlagen der ebenen Geometrie"'
Hendrik Kasten, Denis Vogel\\
Springer Berlin\\
978-3-662-57620-5 (ISBN)\\
{\tiny
   \url{
      https://www.lehmanns.de/shop/mathematik-informatik/43394438-9783662576205-grundlagen-der-ebenen-geometrie
   }
}

%******************************************************************************************************
\subsection*{Lizenz}
%******************************************************************************************************
Dieser Text und das Video sind freie Software. Sie können es unter den Bedingungen der 
GNU General Public License, wie von der Free Software Foundation veröffentlicht, weitergeben 
und/oder modifizieren, entweder gemäß Version 3 der Lizenz oder (nach Ihrer Option) jeder späteren Version.

Die Veröffentlichung von Text und Video erfolgt in der Hoffnung, dass es Ihnen von Nutzen sein wird, 
aber OHNE IRGENDEINE GARANTIE, sogar ohne die implizite Garantie der MARKTREIFE oder der 
VERWENDBARKEIT FÜR EINEN BESTIMMTEN ZWECK. Details finden Sie in der GNU General Public License.

Sie sollten ein Exemplar der GNU General Public License zusammen mit diesem Text erhalten haben 
(zu finden im selben Git-Projekt). 
Falls nicht, siehe \url{http://www.gnu.org/licenses/}.

\subsection*{Das Video}
%******************************************************************************************************
Das Video hierzu ist zu finden unter 
{\tiny
   \url{}
}

%******************************************************************************************************
%                                                                                                     *
\section{Abstand Gerade vom Nullpunkt}
%                                                                                                     *
%******************************************************************************************************

%******************************************************************************************************
\subsection{Abstand Punkt vom Nullpunkt}
%******************************************************************************************************
Der Abstand eines Punktes $(x_0, y_0)$ vom Nullpunkt ist \emph{definiert} über den Pythagoras als
$\sqrt{x_0^2 + y_0^2}$.

\begin{tikzpicture}[scale=3]
   \draw[->, thick] (0, -0.5) -- (0, 1.5) node(yaxis)[above]{$y$};
   \draw[->, thick] (-0.5, 0) -- (2, 0) node(xaxis)[right]{$x$};

   \draw (1, -0.05) node[below]{$1$} -- (1, 0.05);
   \draw (-0.05, 1) node[left]{$1$} -- (0.05, 1);
   
   \fill (3/4,3/8) coordinate (c) circle (0.03);
   \draw[dashed] (yaxis |- c) node[left]{$y_0 = \frac{3}{8} = 0.375$} -- (c);
   \draw[dashed] (xaxis -| c) node[below,align=left]{$x_0 = \frac{3}{4}$\\$= 0.75$} -- (c);
   
   \node[red] at (1, 0.5) {$\sqrt{x_0^2 + y_0^2} = \sqrt{\frac{45}{64}} \approx 0.838525 \dots$};
   
   \draw[dashed, red] (c)  -- (0,0);
\end{tikzpicture}

%******************************************************************************************************
\subsection{Abstand eines Punktes einer Geraden vom Nullpunkt}
%******************************************************************************************************
Liegt der Punkt auf der Geraden $y = mx + q$ so ist die y-Koordinate zum x-Wer $x_0$ gegeben durch  
$mx_0 + q$.
Der Abstand eines Punktes $(x_0, mx_0 + q)$ auf der Geraden vom Nullpunkt ist damit wieder \emph{definiert} über den Pythagoras als
$\sqrt{x_0^2 + (mx_0 + q)^2} = \sqrt{(m^2+1)x_0^2 + 2mqx_0 + q^2}$. Den zweiten Term haben wir durch die binomische Formel und anschließendes Zusammenfassen erhalten.

\begin{tikzpicture}[scale=3]
   \draw[->, thick] (0, -0.5) -- (0, 1.5) node(yaxis)[above]{$y$};
   \draw[->, thick] (-0.5, 0) -- (2, 0) node(xaxis)[right]{$x$};
   
   \draw (1, -0.05) node[below]{$1$} -- (1, 0.05);
   \draw (-0.05, 1) node[left]{$1$} -- (0.05, 1);
   
   \fill (3/4,3/8) coordinate (c) circle (0.03);
   \draw[dashed] (yaxis |- c) node[left]{$mx_0 + q$} -- (c);
   \draw[dashed] (xaxis -| c) node[below,align=left]{$x_0$} -- (c);
   
   \node[red] at (1.3, 0.5) {$\sqrt{(m^2+1)x_0^2 + 2mqx_0 + q^2}$};
   
   \draw[blue] (-1/2,1) -- (2,-1/4);
   
   \draw[dashed] (yaxis |- c) node[left]{$mx_0+q$} -- (c);
   \draw[dashed] (xaxis -| c) node[below]{$x_0$} -- (c);
   
   \draw[dashed, red] (c)  -- (0,0);
\end{tikzpicture}

Die blaue Gerade in der Zeichnung ist $y = -\frac{1}{2}x + \frac{3}{4}$, der Punkt ist wie oben $\left( \frac{3}{4}, \frac{3}{8}\right)$.

%******************************************************************************************************
\subsection{Abstand einer Geraden vom Nullpunkt}
%******************************************************************************************************
Den Abstand einer Geraden vom Nullpunkt können wir nicht ermitteln, solange wir diesen Begriff nicht definiert haben. 

%------------------------------------------------------------------------------------------------------
\begin{Definition}{Abstand einer Geraden vom Nullpunkt}
   Der \textbf{Abstand einer Geraden vom Nullpunkt} ist das Minimum der Abstände aller Punkte auf der Geraden vom Nullpunkt. Also der Abstand des Punktes auf der Geraden, der dem Nullpunkt am nähesten kommt.
\end{Definition}

\begin{tikzpicture}[scale=3]
   \draw[->, thick] (0, -0.5) -- (0, 1.5) node(yaxis)[above]{$y$};
   \draw[->, thick] (-0.5, 0) -- (2, 0) node(xaxis)[right]{$x$};
   
   \draw (1, -0.05) node[below]{$1$} -- (1, 0.05);
   \draw (-0.05, 1) node[left]{$1$} -- (0.05, 1);
   
   \fill (3/4,3/8) coordinate (c) circle (0.03);
   \draw[dashed] (yaxis |- c) node[left]{$mx_0 + q$} -- (c);
   \draw[dashed] (xaxis -| c) node[below,align=left]{$x_0$} -- (c);
   \draw[dashed, blue] (c)  -- (0,0);
   
   \draw[blue] (-1/2,1) -- (2,-1/4);
   
   \draw[dashed] (yaxis |- c) node[left]{$mx_0+q$} -- (c);
   \draw[dashed] (xaxis -| c) node[below]{$x_0$} -- (c);

   \coordinate (minimum) at (3/10,3/5);
   \fill[red] (minimum) circle (0.03); 
   \node[right, red] at (0.4,0.7){
      nähester Punkt $(\frac{3}{10}, \frac{3}{5}) = (0.3, 0.6)$; 
   };
   \node[right, red] at (0.4,0.9){
      Abstand = $\sqrt{\frac{9}{20}} = \frac{3}{10}\sqrt{5} \approx 0,6708203932499369...$
   };
   \draw[dashed, red] (minimum)  -- (0,0);
\end{tikzpicture}

Um die $x$-Koordinate dieses Punktes zu finden, müssen wir den Abstand minimieren. Wir müssen also den Term $\sqrt{(m^2+1)x^2 + 2mqx + q^2}$ minimieren. Da die Wurzelfunktion streng monoton steigend ist, im Sinne von $x_1 < x_2 \Leftrightarrow \sqrt{x_1} > \sqrt{x_2}$ für alle $x_1, x_2 \ge 0$, reicht es den Term unter dem Wurzel zu minimieren, also
\begin{equation}
   (m^2+1)x^2 + 2mqx + q^2.
\end{equation}

Da $m,1,2,q$ Konstanten sind, handelt es sich hierbei um eine Parabel. Da immer gilt $m^2 \ge 0$, gilt auch immer $m^2 + 1 > 0$. Es handelt sich also um eine nach oben offene Parabel, welche ihr Minimum am Scheitelpunkt annimmt.

\begin{tikzpicture}[scale=3]
   \draw[->, thick] (0, -0.5) -- (0, 2.5) node(yaxis)[above]{$y$};
   \draw[->, thick] (-1, 0) -- (2, 0) node(xaxis)[right]{$x$};
   
   \draw (1, -0.05) node[below]{$1$} -- (1, 0.05);
   \draw (-0.05, 1) node[left]{$1$} -- (0.05, 1);
   
   \draw[blue] (-1,5/4) -- (2,-1/4);
   
   \coordinate (minimum) at (3/10,3/5);
   \fill (minimum) circle (0.03); 
   \draw[dashed] (minimum)  -- (0,0);

	\draw[domain = -1:1.5, variable = \x, red] 
      plot ({\x}, {1.25*\x*\x - 0.75*\x + 0.5625});
   \node[right, red] at (1.5, 2.2) {$a x^2 + b x + c$};    

	\draw[domain = -1:2, variable = \x, blue, dashed] 
      plot ({\x}, {sqrt( 1.25*\x*\x - 0.75*\x + 0.5625 )});    
   \node[right, blue] at (1.3, 1.2) {$\sqrt{a x^2 + b x + c}$};    

    \coordinate (parabel) at (3/10,9/20);
    \fill[red] (parabel) circle (0.03); 

    \draw[dashed,red] (yaxis |- parabel) node[left]{Minimum: $\frac{q^2}{m^2+1}$} -- (parabel);

    \draw[dashed,red] (xaxis -| minimum) node[below]{$-\frac{b}{2a}$} -- (3/10, 1.1);
    
    \draw[dashed,green] (-1,1) -- (2,1);
\end{tikzpicture}

In dem Bild oben ist die blau gestrichelte Linie der Abstand der Geraden vom Nullpunkt in Abhängigkeit von $x$:
\begin{equation}
   y = \sqrt{ \frac{5}{4}x^2 - \frac{3}{4}x + \frac{9}{16} }.
\end{equation}
Die rot gestrichelte Parabel ist dieser Abstand zum Quadrat:
\begin{equation}
   y = \frac{5}{4}x^2 - \frac{3}{4}x + \frac{9}{16}.
\end{equation}
Der Scheitelpunkt der Parabel ist bei:
\begin{equation}
   \left( \frac{9}{20}, \frac{3}{10} \right ).
\end{equation}
Diese beiden Kurven scheiden sich bei $y=1$ (grün gestrichelte Linie), da $1 = \sqrt 1$.

Generell ist die $x$-Koordinate des Scheitelpunktes der Parabel $a x^2 + b x + c$ gegeben durch
\begin{equation}
   -\frac{b}{2a}.
\end{equation}
Erinnerung: das ist genau der Teil in der abc-Formel, der vor dem plus-minus-Wurzel steht:
\begin{equation}
   -\frac{b}{2a} \pm \sqrt{\left ( \frac{b}{2a} \right )^2 - \frac{c}{a} }.
\end{equation}
 
Hier haben wir
\begin{align}
   a &:= m^2 + 1\\
   b &:= 2mq\\
   c &:= q^2.
\end{align}
Also  ist die $x$-Koordinate des Scheitelpunktes unserer speziellen Parabel und damit die $x$-Koordinate des Punktes auf der Geraden mit minimalem Abstand gegeben durch:
\begin{equation}
   x_0 = -\frac{2mq}{2(m^2+1)} = \color{red}-\frac{mq}{m^2+1}.
\end{equation}

Der $y$-Wert ergibt sich durch Einsetzen in die Geradengleichung
\begin{equation}
   y_0 = m(-\frac{mq}{m^2+1}) + q = -\frac{m^2q}{m^2 + 1} + \frac{q(m^2 + 1)}{m^2 + 1}
   = \color{red} \frac{q}{m^2+1}.
\end{equation} 

Der Abstand vom Nullpunkt, endlich, ist durch Einsetzen in $\sqrt{x_0^2 + y_0^2}$:
\begin{equation}
   \sqrt{\left(-\frac{mq}{m^2+1} \right)^2 + \left(\frac{q}{m^2+1} \right)^2} =
   \sqrt{\frac{m^2q^2 + q^2}{(m^2+1)^2}} = 
   \color{red} \sqrt{\frac{q^2}{m^2+1}}.
\end{equation}

Wir haben auch
\begin{equation}
   x_0 = -my_0,
\end{equation}
ein nach all den komplizierten Formeln erstaunlich einfacher Zusammenhang.

In dem Beispiel $y = -\frac{1}{2}x + \frac{5}{2}$ in der Zeichnung oben ist der Abstand $\sqrt{5}$ und der näheste Punkt ist $(1,2)$.

%******************************************************************************************************
%                                                                                                     *
\section{Schluss}
%                                                                                                     *
%******************************************************************************************************
Der Abstand der Geraden $y=mx+q$ zum Nullpunkt ist 
\begin{equation}
	\sqrt{\frac{q^2}{m^2+1}}.
\end{equation}
Der näheste Punkt ist
\begin{equation}
	\left( -\frac{mq}{m^2+1}, \frac{q}{m^2+1} \right).
\end{equation}

Es gibt einen einfachen Zusammenhang zwischen der $x$- und der $y$-Koordinate $x_0$ und $y_0$ des Punktes mit dem minimalen Abstand:
\begin{equation}
   x_0 = -my_0.
\end{equation}

%******************************************************************************************************
%                                                                                                     *
\section{TODO:}
%                                                                                                     *
%******************************************************************************************************
\begin{itemize}
   \item Tafelbild auf Blatt Papier
\end{itemize}

\begin{backup}
    (Zur Zeit nicht benötigter Inhalt)
\end{backup}

%******************************************************************************************************
%                                                                                                     *
\begin{thebibliography}{9}
%                                                                                                     *
%******************************************************************************************************

   \bibitem[Klett2016]{Klett}
      Klett Lerntraining, \emph{Komplett Wissen Mathematik Gymnasium 5-10},
      Klett Lerntraining, 978-3-12-926097-5 (ISBN)
   
   \bibitem[KastenVogel2018]{Kasten}
      Hendrik Kasten, Denis Vogel , \emph{Grundlagen der ebenen Geometrie},
      2018 Springer Berlin, 978-3-662-57620-5 (ISBN)

\end{thebibliography}

%******************************************************************************************************
%                                                                                                     *
\begin{large}
    \centerline{\textsc{Symbolverzeichnis}}
\end{large}
%                                                                                                     *
%******************************************************************************************************
\bigskip

\renewcommand*{\arraystretch}{1}

\begin{tabular}{ll}
    $x_0, y_0$          & $x$- und $y$-Koordinate eines Punktes\\
    $m$                 & Steigung einer Geraden\\
    $q$                 & Achsenabschnitt einer Geraden\\
    $a, b, c$           & Koeffizienten einer Parabel

\end{tabular}

\end{document}
