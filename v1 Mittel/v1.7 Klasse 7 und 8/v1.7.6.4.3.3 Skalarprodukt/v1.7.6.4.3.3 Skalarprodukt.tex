%******************************************************** -*-LaTeX-*- ******************************
%                                                                                                  *
% v1.7.6.4.3.3 Skalarprodukt.tex                                                                   *
%                                                                                                  *
% Copyright (C) 2023 Kategory GmbH \& Co. KG (joerg.kunze@kategory.de)                             *
%                                                                                                  *
% v1.7.6.4.3.3 Skalarprodukt is part of kategoryMathematik.                                        *
%                                                                                                  *
% kategoryMathematik is free software: you can redistribute it and/or modify                       *
% it under the terms of the GNU General Public License as published by                             *
% the Free Software Foundation, either version 3 of the License, or                                *
% (at your option) any later version.                                                              *
%                                                                                                  *
% kategoryMathematik is distributed in the hope that it will be useful,                            *
% but WITHOUT ANY WARRANTY; without even the implied warranty of                                   *
% MERCHANTABILITY or FITNESS FOR A PARTICULAR PURPOSE.  See the                                    *
% GNU General Public License for more details.                                                     *
%                                                                                                  *
% You should have received a copy of the GNU General Public License                                *
% along with this program.  If not, see <http://www.gnu.org/licenses/>.                            *
%                                                                                                  *
%***************************************************************************************************

\documentclass[a4paper]{amsart}
% \documentclass[a4paper]{book}

%-----------------------------------------------------------------------------------------------------*
% package:                                                                                            *
%-----------------------------------------------------------------------------------------------------*
\usepackage{amssymb}
\usepackage{amsfonts}
\usepackage{amsmath}
\usepackage{amsthm}

\usepackage{mathabx}

\usepackage{a4wide} % a little bit smaller margins

\usepackage{empheq}

\usepackage{graphicx}
\usepackage{hyperref}
\usepackage{algorithmic}
\usepackage{listings}
\usepackage{color}
\usepackage{colortbl}
\usepackage{sidecap}
\usepackage{comment}
\usepackage{tcolorbox}
\usepackage{collect}

\usepackage{upgreek}

% \usepackage{diagrams}

\usepackage[german]{babel}
\usepackage[none]{hyphenat}
\emergencystretch=4em

\usepackage[utf8]{inputenc} % to be able to use äöü as characters in text
\usepackage[T1]{fontenc} % to be able to use äöü in lables
\usepackage{lmodern}     % to avoid pixelation introduced by fontenc

\usepackage{hyperref}

\usepackage{tikz}
\usepackage{tikz-cd}
\usetikzlibrary{babel}
\usetikzlibrary{calc}

%-----------------------------------------------------------------------------------------------------*
% theorem:                                                                                            *
%-----------------------------------------------------------------------------------------------------*
\theoremstyle{definition}
\newtheorem{theorem}{Theorem}[subsection]

\newcommand{\myTheorem}[1]{%
  \newtheorem{jk#1}[theorem]{#1}
  \newenvironment{#1}[1]{%
    \expandafter\begin{jk#1} \expandafter\label{#1:##1}\textbf{(##1):}
  }{%
    \expandafter\end{jk#1}
  }
}

\myTheorem{Definition}
\myTheorem{Proposition}
\myTheorem{Satz}
\myTheorem{Theorem}
\myTheorem{Example}
\myTheorem{Remark}

\definecollection{jkjkFrage}
\newtheorem{jkFrage}[theorem]{Frage}
\newenvironment{Frage}[1]{%
  \expandafter\begin{jkFrage} \expandafter\label{Frage:#1}\textbf{(#1):}
  \begin{collect}{jkjkFrage}{}{}
    \item \ref{Frage:#1} #1
  \end{collect}
}{%
  \expandafter\end{jkFrage}
}

\newcommand{\myRef}[2]{[#1 \ref{#1:#2}, ``#2'']}

\renewcommand{\proofname}{Beweis}

%-----------------------------------------------------------------------------------------------------*
% operator:                                                                                           *
%-----------------------------------------------------------------------------------------------------*
\DeclareMathOperator{\End}{End}
\DeclareMathOperator{\Ker}{Ker}
\DeclareMathOperator{\Mat}{Mat}
\DeclareMathOperator{\rank}{rank}
\DeclareMathOperator{\ggT}{ggT}
\DeclareMathOperator{\len}{len}
\DeclareMathOperator{\ord}{ord}
\DeclareMathOperator{\kgV}{kgV}
\DeclareMathOperator{\id}{id}
\DeclareMathOperator{\red}{red}
\DeclareMathOperator{\supp}{supp}
\DeclareMathOperator{\Bild}{Bild}
\DeclareMathOperator{\Rang}{Rang}
\DeclareMathOperator{\Det}{Det}
\DeclareMathOperator{\Hom}{Hom}

\DeclareMathOperator{\sub}{sub}
\DeclareMathOperator{\blk}{blk}
\DeclareMathOperator{\minimal}{minimal}
\DeclareMathOperator{\maximal}{maximal}

\definecolor{mygreen}{rgb}{0,0.6,0}
\definecolor{mygray}{rgb}{0.5,0.5,0.5}
\definecolor{mymauve}{rgb}{0.58,0,0.82}

\lstset{ %
  backgroundcolor=\color{white},   % choose the background color
  basicstyle=\ttfamily\footnotesize,        % size of fonts used for the code
  breaklines=true,                 % automatic line breaking only at whitespace
  captionpos=b,                    % sets the caption-position to bottom
  commentstyle=\color{mygreen},    % comment style
  escapeinside={\%*}{*)},          % if you want to add LaTeX within your code
  keywordstyle=\color{blue},       % keyword style
  stringstyle=\color{mymauve},     % string literal style
  frame=single
}

\setcounter{MaxMatrixCols}{20}

%******************************************************************************************************
%                                                                                                     *
% definition:                                                                                         *
%                                                                                                     *
%******************************************************************************************************
\newcommand{\R}{\ensuremath{\mathbb{ R }}}
\newcommand{\Q}{\ensuremath{\mathbb{ Q }}}
\newcommand{\Z}{\ensuremath{\mathbb{ Z }}}
\newcommand{\N}{\ensuremath{\mathbb{ N }}}
\newcommand{\C}{\ensuremath{\mathbb{ C }}}
\newcommand{\A}{\ensuremath{\mathbb{ A }}}
\newcommand{\F}{\ensuremath{\mathbb{ F }}}
\newcommand{\K}{\ensuremath{\mathbb{ K }}}
\newcommand{\Pb}{\ensuremath{\mathbb{ P }}}

\newcommand{\M}{\ensuremath{\mathcal{ M }}}
\newcommand{\V}{\ensuremath{\mathcal{ V }}}

\newcommand{\AAA}{\ensuremath{\mathcal{ A }}}
\newcommand{\BB}{\ensuremath{\mathcal{ B }}}
\newcommand{\CC}{\ensuremath{\mathcal{ C }}}
\newcommand{\EE}{\ensuremath{\mathcal{ E }}}
\newcommand{\KK}{\ensuremath{\mathcal{ K }}}
\newcommand{\MM}{\ensuremath{\mathcal{ M }}}
\newcommand{\PP}{\ensuremath{\mathcal{ P }}}
\newcommand{\ZZ}{\ensuremath{\mathcal{ Z }}}

\newcommand{\imporant}[1]{ \textcolor{red}{\textbf{#1}} }

\newcommand{\bb}[1]{\mathbf{#1}}
\newcommand{\balpha}{\boldsymbol{\upalpha}}
\newcommand{\bbeta}{\boldsymbol{\upbeta}}
\newcommand{\bgamma}{\boldsymbol{\upgamma}}
\newcommand{\bdelta}{\boldsymbol{\delta}}
\newcommand{\bmu}{\boldsymbol{\upmu}}

\newcommand{\z}[1]{\Z_{#1}}
\newcommand{\e}[1]{\z{#1}^*}
\newcommand{\q}[1]{(\e{#1})^2}

\excludecomment{book}
\excludecomment{example}
\excludecomment{backup}

\begin{document}

%******************************************************************************************************
%                                                                                                     *
\begin{titlepage}
%                                                                                                     *
%******************************************************************************************************
% \vspace*{\fill}
\centering
{\huge
(Mittel) Geometrie\\[1cm]
\textbf{v1.7.6.4.3.3 Skalarprodukt}
}\\[1cm]

\textbf{Kategory GmbH \& Co. KG}\\
Präsentiert von Jörg Kunze\\
Copyright (C) 2023 Kategory GmbH \& Co. KG

\end{titlepage}

%\clearpage
%\setcounter{page}{2}
%
%\tableofcontents

\newpage

%******************************************************************************************************
%                                                                                                     *
\section*{Beschreibung}
%                                                                                                     *
%******************************************************************************************************

%******************************************************************************************************
\subsection*{Inhalt}
%******************************************************************************************************
Das Skalarprodukt ist bei Dreiecken im Koordinatensystem das Hindernis (Obstruktion) gegen die oder der Fehler bei der Anwendung des Pythagoras.

Das Skalarprodukt als Summe der Produkte der Koordinaten-Versätze der Seiten mit einer recht einfachen Formel zu berechnen. Die Koordinaten-Versätze einer Strecke sind die Differenzen zwischen jeweils den beiden x- und y-Koordinaten des Anfangs- und End-Punktes der Seite. Wegen des Vorzeichens betrachten wir hier gerichtete Seiten.

Das Skalarprodukt ist der nummerische Fehler bei der Anwendung der Formel von Pythagoras in beliebigen Dreiecken. Anders gesagt ist es ein weiterer Term in einer verallgemeinerten Formel des Pythagoras, die in allen Dreiecken gilt.

Wieder anders gesagt, ist es Indikator für die Gültigkeit der Formel des Pythagoras: genau dann, wenn das Skalarprodukt Null ist, gilt der Satz des Pythagoras.

Die beiden äquivalenten Aussagen "`das Skalarprodukt ist Null"' und "`der Satz des Pythagoras ist gültig"' sind damit zwei gleichwertige Definitionen von "`senkrecht"'.

%******************************************************************************************************
\subsection*{Präsentiert}
%******************************************************************************************************
Von Jörg Kunze

%******************************************************************************************************
\subsection*{Voraussetzungen}
%******************************************************************************************************
Koordinaten-System, Satz des Pythagoras, Abstand im Koordinatensystem, Rechnen mit Buchstaben.

%******************************************************************************************************
\subsection*{Text}
%******************************************************************************************************
Der Begleittext als PDF und als LaTeX findet sich unter
{\tiny
   \url{https://github.com/kategory/kategoryMathematik/tree/main/v1.6%20Mittel/v1.7.6%20Geometrie/v1.7.6.4.3.3%20Skalarprodukt}
}

%******************************************************************************************************
\subsection*{Meine Videos}
%******************************************************************************************************
Siehe auch in den folgenden Videos:\\
\\
v1.7.6.4.2 (Mittel) Geometrie - Satz des Pythagoras\\
\url{https://youtu.be/mTMOtXfUQzQ}\\
\\
v1.7.6.4.3 (Mittel) Geometrie - Abstand im Koordinatensystem\\
\url{https://youtu.be/HEdolewfn78}

%******************************************************************************************************
\subsection*{Quellen}
%******************************************************************************************************
Siehe auch in den folgenden Seiten:\\
\\
\url{https://de.wikipedia.org/wiki/Satz_des_Pythagoras}\\
\url{https://de.wikipedia.org/wiki/Euklidischer_Abstand}\\
\url{https://de.wikipedia.org/wiki/Skalarprodukt}\\
\url{https://www.schuelerhilfe.de/online-lernen/1-mathematik/719-skalarprodukt}\\

%******************************************************************************************************
\subsection*{Buch}
%******************************************************************************************************
Grundlage ist folgendes Buch:\\

"`KomplettWissen Mathematik Gymnasium Klasse 5-10"'\\
Klett Lerntraining bei PONS\\
978-3-12-926097-5 (ISBN)\\
{\tiny
   \url{https://www.lehmanns.de/shop/schulbuch-lexikon-woerterbuch/35031626-9783129260975-komplettwissen-mathematik-gymnasium-klasse-5-10
   }
}\\
\\
"`Grundlagen der ebenen Geometrie"'
Hendrik Kasten, Denis Vogel\\
Springer Berlin\\
978-3-662-57620-5 (ISBN)\\
{\tiny
   \url{
      https://www.lehmanns.de/shop/mathematik-informatik/43394438-9783662576205-grundlagen-der-ebenen-geometrie
   }
}

%******************************************************************************************************
\subsection*{Lizenz}
%******************************************************************************************************
Dieser Text und das Video sind freie Software. Sie können es unter den Bedingungen der 
GNU General Public License, wie von der Free Software Foundation veröffentlicht, weitergeben 
und/oder modifizieren, entweder gemäß Version 3 der Lizenz oder (nach Ihrer Option) jeder späteren Version.

Die Veröffentlichung von Text und Video erfolgt in der Hoffnung, dass es Ihnen von Nutzen sein wird, 
aber OHNE IRGENDEINE GARANTIE, sogar ohne die implizite Garantie der MARKTREIFE oder der 
VERWENDBARKEIT FÜR EINEN BESTIMMTEN ZWECK. Details finden Sie in der GNU General Public License.

Sie sollten ein Exemplar der GNU General Public License zusammen mit diesem Text erhalten haben 
(zu finden im selben Git-Projekt). 
Falls nicht, siehe \url{http://www.gnu.org/licenses/}.

\subsection*{Das Video}
%******************************************************************************************************
Das Video hierzu ist zu finden unter 
{\tiny
   \url{https://youtu.be/PMYHObLI54Q}
}

%******************************************************************************************************
%                                                                                                     *
\section{Skalarprodukt}
%                                                                                                     *
%******************************************************************************************************

%******************************************************************************************************
\subsection{Definition des Skalarproduktes}
%******************************************************************************************************
Sei im Folgenden gegeben ein Dreieck $ABC$ mit den Punkten $A =(A_x,A_y), B =(B_x,B_y), C =(C_x,C_y)$ und den Seiten $a = \overrightarrow{AB}, b = \overrightarrow{BC}, c = \overrightarrow{CA}$.

\begin{tikzpicture}[scale=3]
   \draw[->, thick] (0, -0.5) -- (0, 1.5) node(yaxis)[above]{$y$};
   \draw[->, thick] (-0.5, 0) -- (2, 0) node(xaxis)[right]{$x$};

   \draw (1, -0.05) node[below]{$1$} -- (1, 0.05);
   \draw (-0.05, 1) node[left]{$1$} -- (0.05, 1);
   
   \fill (3/4,3/8) coordinate (A) circle (0.03) node[left]{A};
   \fill (4/4,8/8) coordinate (B) circle (0.03) node[above]{B};
   \fill (6/4,4/8) coordinate (C) circle (0.03) node[right]{C};

   \draw (A) -- node[left] {a} (B);
   \draw (B) -- node[right] {b} (C);
   \draw (C) -- node[below] {c} (A);

   \draw[dashed] (A |- xaxis)  -- (A);
   \draw[dashed] (B |- xaxis)  -- (B);

   \draw[dashed] (yaxis |- A)  -- (A);
   \draw[dashed] (yaxis |- B)  -- (B);

   \node[red] at (0.1, 11/16) {$a_y$};
   \node[red] at (7/8, 0.1) {$a_x$};
\end{tikzpicture}

\begin{Definition}{Der Koordinaten-Versatz der Seiten}
   Der \textbf{Koordinaten-Versatz der Seiten} ist definiert durch:
   \begin{alignat}{4}
      &a_x &&:= B_x-A_x, \quad &a_y &&:= B_y-A_y\\
      &b_x &&:= C_x-B_x, \quad &b_y &&:= C_y-B_y\\
      &c_x &&:= A_x-C_x, \quad &c_y &&:= A_y-C_y.
   \end{alignat}
\end{Definition}

Diese Koordinaten-Versätze tauchen ganz natürlich bei der Berechnung der Länge der Seiten auf.
Das Quadrat der Länge der Seite $a$ wird mit $a^2 := |a|^2$ bezeichnet und ist \emph{definiert} als Abstand der Punkte $A$ und $B$ über die Formel des Pythagoras als $a^2 := (B_x-A_x)^2 + (B_y-A_y)^2$. Somit können wir mit Hilfe der Koordinaten-Versätze der Seiten schreiben:
\begin{empheq}[box=\fbox]{alignat=3}
   &a^2 &&= a_x^2 &&+ a_y^2\\ 
   &b^2 &&= b_x^2 &&+ b_y^2\\
   &c^2 &&= c_x^2 &&+ c_y^2.  
\end{empheq}

Aus $B_x-A_x + C_x-B_x + A_x-C_x = 0$ und $B_y-A_y + C_y-B_y + A_y-C_y = 0$ ergibt sich sofort
\begin{empheq}[box=\fbox]{alignat=2}
      &a_x + b_x + c_x &&= 0\\
      &a_y + b_y + c_y &&= 0,
\end{empheq}
welches auch optisch im Koordinatensystem-Bild leicht nachzuvollziehen ist.

Quadrieren wir die Gleichung $a_x + b_x = -c_x$ erhalten wir $(a_x + b_x)^2 = c_x^2$ und genauso für die $y$-Koordinate. Damit können wir folgendermaßen rechnen:
\begin{equation}
   c_x^2 + c_y^2 = (a_x^2 + 2a_xb_x + b_x^2) + (a_y^2 + 2a_yb_y + b_y^2) = 
   a_x^2 + a_y^2 + 2(a_xb_x + a_yb_y) + b_x^2 + b_y^2.
\end{equation}
Wir erhalten die wichtige Formel
\begin{empheq}[box=\fbox]{equation}
   c^2 = a^2 + 2(a_xb_x + a_yb_y) + b^2.
\end{empheq}

Das $2(a_xb_x + a_yb_y)$ ist die Größe, die, wenn ungleich Null, verhindert, dass die Formel des Pythagoras in einem allgemeinen Dreieck gilt.
\begin{Definition}{Skalarprodukt}
   Das \textbf{Skalarprodukt} der Seiten $a$ und $b$ ist definiert durch:
   \begin{equation}
      ab := a \cdot b := \langle a, b \rangle := a_xb_x + a_yb_y.
   \end{equation}
   Wir werden meistens $ab$ schreiben. 
\end{Definition}

Damit erhalten wir den
\begin{Satz}{Verallgemeinerter Pythagoras}
   In einem allgemeinem Dreieck mit den Seiten $a, b, c$ gilt
   \begin{empheq}[box=\fbox]{equation}
      c^2 = a^2 + 2ab + b^2.
   \end{empheq}
\end{Satz}
Beachte: Die Buchstaben hier sind keine Zahlen sondern Seiten, wobei allerdings $a^2, b^2, c^2$ über die Längen der Seiten definiert sind: $|a|^2, |b|^2, |c|^2$. Das erinnert frappant an die binomische Formel, wenn wir $c = a+b$ schreiben würden.

Damit verhinder das Skalarprodukt $ab$, wenn es ungleich Null ist, die Gültigkeit des Pythagoras. Das Skalarprodukt ist damit das (oder ein Maß für das) Hindernis für die Gültigkeit. Es ist ein Maß für die Obstruktion des Pythagoras.

Mit einem anderen Blickwinkel ist das Skalarprodukt ein Maß für den Fehler des Pythagoras oder nochmal leicht anders sein Korrekturterm für den allgemeinen Fall.

Ist eine Seite null lang, ist also z.B. $B =C$, dann ist $b_x = b_y = 0$ und damit dann $ab = a_xb_x + a_yb_y = a_x \cdot 0 + a_y \cdot 0 = 0$ also gilt für das Skalarprodukt, wenn wir etwas sallop $0$ für die null-lange Seite schreiben
\begin{empheq}[box=\fbox]{equation}
   a \cdot 0 = 0.
\end{empheq}

Falls die Seiten $a, b$ mit den Koordinatenachsen liegen, also $a$ parallel zur $x$- und $b$ parallal zur $y$-Achse, dann gilt $a_x = b_y = 0$ und somit $ab = a_xb_x + a_yb_y = 0 \cdot b_x + a_y \cdot 0 = 0$.
\begin{tikzpicture}[scale=3]
   \draw[->, thick] (0, -0.5) -- (0, 1.5) node(yaxis)[above]{$y$};
   \draw[->, thick] (-0.5, 0) -- (2, 0) node(xaxis)[right]{$x$};
   
   \draw (1, -0.05) node[below]{$1$} -- (1, 0.05);
   \draw (-0.05, 1) node[left]{$1$} -- (0.05, 1);
   
   \fill (3/4,3/8) coordinate (A) circle (0.03) node[left]{A};
   \fill (3/4,8/8) coordinate (B) circle (0.03) node[above]{B};
   \fill (6/4,8/8) coordinate (C) circle (0.03) node[right]{C};
   
   \draw (A) -- node[left] {a} (B);
   \draw (B) -- node[above] {b} (C);
   \draw (C) -- node[below] {c} (A);
   
   \draw[dashed] (A |- xaxis)  -- (A);
   \draw[dashed] (B |- xaxis)  -- (B);
   
   \draw[dashed] (yaxis |- A)  -- (A);
   \draw[dashed] (yaxis |- B)  -- (B);
   
   \node[red] at (0.1, 11/16) {$a_y$};
   \node[red] at (7/8, 0.1) {$a_x=0$};
\end{tikzpicture}

Für solche Dreiecke gilt der Pythagoras aufgrund unserer Definition der Länge von $c$ und sie ist offensichtlich konsistent mit der neuen Formel.

Das Skalarprodukt einer Seite mit sich selbst ist $a \cdot a = a_xa_x + a_ya_y = a_x^2 + a_y^2 = |a|^2$. Damit ist $a^2$ das selbe egal, ob wir es als Quadrat der Länge oder als Skalarprodukt mit sich selbst auffassen.

Wir haben den 
\begin{Satz}{Gültigkeit Pythagoras}
   In einem allgemeinem Dreieck mit den Seiten $a, b, c$ gilt
   \begin{empheq}[box=\fbox]{equation}
      \text{Pythagoras} \Leftrightarrow ab=0.
   \end{empheq}
\end{Satz}

Das wird uns zu folgender Definition verleiten:
\begin{Definition}{Senkrecht}
   Die beiden Seiten $a, b$ eines allgemeines Dreieck mit den Seiten $a, b, c$ sind  \textbf{senkrecht} zueinander, genau dann wenn ihr Skalarprodukt Null ist:
   \begin{empheq}[box=\fbox]{equation}
      a \perp b :\Leftrightarrow ab = 0.
   \end{empheq}
\end{Definition}

%******************************************************************************************************
%                                                                                                     *
\section{TODO:}
%                                                                                                     *
%******************************************************************************************************
\begin{backup}
    (Zur Zeit nicht benötigter Inhalt)
\end{backup}

%******************************************************************************************************
%                                                                                                     *
\begin{thebibliography}{9}
%                                                                                                     *
%******************************************************************************************************

   \bibitem[Klett2016]{Klett}
      Klett Lerntraining, \emph{Komplett Wissen Mathematik Gymnasium 5-10},
      Klett Lerntraining, 978-3-12-926097-5 (ISBN)
      
   \bibitem[KastenVogel2018]{Kasten}
       Hendrik Kasten, Denis Vogel , \emph{Grundlagen der ebenen Geometrie},
      2018 Springer Berlin, 978-3-662-57620-5 (ISBN)

\end{thebibliography}

%******************************************************************************************************
%                                                                                                     *
\begin{large}
    \centerline{\textsc{Symbolverzeichnis}}
\end{large}
%                                                                                                     *
%******************************************************************************************************
\bigskip

\renewcommand*{\arraystretch}{1}

\begin{tabular}{ll}
    $a,b,c, \cdots$               &Seiten\\
    $a_x,a_y$                   &Koordinatenversätze der Seite $a$\\
    $|a|$                       &Länge der Seite $a$\\
    $x, y$                      &Koordinatenachsen\\
    $A, B, C, \cdots$          &Punkte\\
    $A_x, A_y$                 &Koordinaten des Punktes $A$\\
    $\perp$                    &Senkrecht
\end{tabular}

\end{document}
