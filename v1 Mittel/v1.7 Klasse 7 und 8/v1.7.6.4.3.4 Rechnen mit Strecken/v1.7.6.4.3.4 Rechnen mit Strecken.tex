%******************************************************** -*-LaTeX-*- ******************************
%                                                                                                  *
% v1.7.6.4.3.4 Rechnen mit Strecken.tex                                                            *
%                                                                                                  *
% Copyright (C) 2024 Kategory GmbH \& Co. KG (joerg.kunze@kategory.de)                             *
%                                                                                                  *
% v1.7.6.4.3.4 Rechnen mit Strecken is part of kategoryMathematik.                                 *
%                                                                                                  *
% kategoryMathematik is free software: you can redistribute it and/or modify                       *
% it under the terms of the GNU General Public License as published by                             *
% the Free Software Foundation, either version 3 of the License, or                                *
% (at your option) any later version.                                                              *
%                                                                                                  *
% kategoryMathematik is distributed in the hope that it will be useful,                            *
% but WITHOUT ANY WARRANTY; without even the implied warranty of                                   *
% MERCHANTABILITY or FITNESS FOR A PARTICULAR PURPOSE.  See the                                    *
% GNU General Public License for more details.                                                     *
%                                                                                                  *
% You should have received a copy of the GNU General Public License                                *
% along with this program.  If not, see <http://www.gnu.org/licenses/>.                            *
%                                                                                                  *
%***************************************************************************************************

\documentclass[a4paper]{amsart}
% \documentclass[a4paper]{book}

%-----------------------------------------------------------------------------------------------------*
% package:                                                                                            *
%-----------------------------------------------------------------------------------------------------*
\usepackage{amssymb}
\usepackage{amsfonts}
\usepackage{amsmath}
\usepackage{amsthm}

\usepackage{mathabx}

\usepackage{a4wide} % a little bit smaller margins

\usepackage{empheq}

\usepackage{graphicx}
\usepackage{hyperref}
\usepackage{algorithmic}
\usepackage{listings}
\usepackage{color}
\usepackage{colortbl}
\usepackage{sidecap}
\usepackage{comment}
\usepackage{tcolorbox}
\usepackage{collect}

\usepackage{upgreek}

% \usepackage{diagrams}

\usepackage[german]{babel}
\usepackage[none]{hyphenat}
\emergencystretch=4em

\usepackage[utf8]{inputenc} % to be able to use äöü as characters in text
\usepackage[T1]{fontenc} % to be able to use äöü in lables
\usepackage{lmodern}     % to avoid pixelation introduced by fontenc

\usepackage{hyperref}

\usepackage{tikz}
\usepackage{tikz-cd}
\usetikzlibrary{babel}
\usetikzlibrary{calc}

%-----------------------------------------------------------------------------------------------------*
% theorem:                                                                                            *
%-----------------------------------------------------------------------------------------------------*
\theoremstyle{definition}
\newtheorem{theorem}{Theorem}[subsection]

\newcommand{\myTheorem}[1]{%
  \newtheorem{jk#1}[theorem]{#1}
  \newenvironment{#1}[1]{%
    \expandafter\begin{jk#1} \expandafter\label{#1:##1}\textbf{(##1):}
  }{%
    \expandafter\end{jk#1}
  }
}

\myTheorem{Definition}
\myTheorem{Proposition}
\myTheorem{Satz}
\myTheorem{Theorem}
\myTheorem{Example}
\myTheorem{Remark}

\definecollection{jkjkFrage}
\newtheorem{jkFrage}[theorem]{Frage}
\newenvironment{Frage}[1]{%
  \expandafter\begin{jkFrage} \expandafter\label{Frage:#1}\textbf{(#1):}
  \begin{collect}{jkjkFrage}{}{}
    \item \ref{Frage:#1} #1
  \end{collect}
}{%
  \expandafter\end{jkFrage}
}

\newcommand{\myRef}[2]{[#1 \ref{#1:#2}, ``#2'']}

\renewcommand{\proofname}{Beweis}

%-----------------------------------------------------------------------------------------------------*
% operator:                                                                                           *
%-----------------------------------------------------------------------------------------------------*
\DeclareMathOperator{\End}{End}
\DeclareMathOperator{\Ker}{Ker}
\DeclareMathOperator{\Mat}{Mat}
\DeclareMathOperator{\rank}{rank}
\DeclareMathOperator{\ggT}{ggT}
\DeclareMathOperator{\len}{len}
\DeclareMathOperator{\ord}{ord}
\DeclareMathOperator{\kgV}{kgV}
\DeclareMathOperator{\id}{id}
\DeclareMathOperator{\red}{red}
\DeclareMathOperator{\supp}{supp}
\DeclareMathOperator{\Bild}{Bild}
\DeclareMathOperator{\Rang}{Rang}
\DeclareMathOperator{\Det}{Det}
\DeclareMathOperator{\Hom}{Hom}

\DeclareMathOperator{\sub}{sub}
\DeclareMathOperator{\blk}{blk}
\DeclareMathOperator{\minimal}{minimal}
\DeclareMathOperator{\maximal}{maximal}

\definecolor{mygreen}{rgb}{0,0.6,0}
\definecolor{mygray}{rgb}{0.5,0.5,0.5}
\definecolor{mymauve}{rgb}{0.58,0,0.82}

\lstset{ %
  backgroundcolor=\color{white},   % choose the background color
  basicstyle=\ttfamily\footnotesize,        % size of fonts used for the code
  breaklines=true,                 % automatic line breaking only at whitespace
  captionpos=b,                    % sets the caption-position to bottom
  commentstyle=\color{mygreen},    % comment style
  escapeinside={\%*}{*)},          % if you want to add LaTeX within your code
  keywordstyle=\color{blue},       % keyword style
  stringstyle=\color{mymauve},     % string literal style
  frame=single
}

\setcounter{MaxMatrixCols}{20}

%******************************************************************************************************
%                                                                                                     *
% definition:                                                                                         *
%                                                                                                     *
%******************************************************************************************************
\newcommand{\R}{\ensuremath{\mathbb{ R }}}
\newcommand{\Q}{\ensuremath{\mathbb{ Q }}}
\newcommand{\Z}{\ensuremath{\mathbb{ Z }}}
\newcommand{\N}{\ensuremath{\mathbb{ N }}}
\newcommand{\C}{\ensuremath{\mathbb{ C }}}
\newcommand{\A}{\ensuremath{\mathbb{ A }}}
\newcommand{\F}{\ensuremath{\mathbb{ F }}}
\newcommand{\K}{\ensuremath{\mathbb{ K }}}
\newcommand{\Pb}{\ensuremath{\mathbb{ P }}}

\newcommand{\M}{\ensuremath{\mathcal{ M }}}
\newcommand{\V}{\ensuremath{\mathcal{ V }}}

\newcommand{\AAA}{\ensuremath{\mathcal{ A }}}
\newcommand{\BB}{\ensuremath{\mathcal{ B }}}
\newcommand{\CC}{\ensuremath{\mathcal{ C }}}
\newcommand{\EE}{\ensuremath{\mathcal{ E }}}
\newcommand{\KK}{\ensuremath{\mathcal{ K }}}
\newcommand{\MM}{\ensuremath{\mathcal{ M }}}
\newcommand{\PP}{\ensuremath{\mathcal{ P }}}
\newcommand{\ZZ}{\ensuremath{\mathcal{ Z }}}

\newcommand{\imporant}[1]{ \textcolor{red}{\textbf{#1}} }

\newcommand{\bb}[1]{\mathbf{#1}}
\newcommand{\balpha}{\boldsymbol{\upalpha}}
\newcommand{\bbeta}{\boldsymbol{\upbeta}}
\newcommand{\bgamma}{\boldsymbol{\upgamma}}
\newcommand{\bdelta}{\boldsymbol{\delta}}
\newcommand{\bmu}{\boldsymbol{\upmu}}

\newcommand{\z}[1]{\Z_{#1}}
\newcommand{\e}[1]{\z{#1}^*}
\newcommand{\q}[1]{(\e{#1})^2}

\excludecomment{book}
\excludecomment{example}
\excludecomment{backup}

\begin{document}

%******************************************************************************************************
%                                                                                                     *
\begin{titlepage}
%                                                                                                     *
%******************************************************************************************************
% \vspace*{\fill}
\centering
{\huge
(Mittel) Geometrie\\[1cm]
\textbf{v1.7.6.4.3.4 Rechnen mit Strecken}
}\\[1cm]

\textbf{Kategory GmbH \& Co. KG}\\
Präsentiert von Jörg Kunze\\
Copyright (C) 2024 Kategory GmbH \& Co. KG

\end{titlepage}

%\clearpage
%\setcounter{page}{2}
%
%\tableofcontents

\newpage

%******************************************************************************************************
%                                                                                                     *
\section*{Beschreibung}
%                                                                                                     *
%******************************************************************************************************

%******************************************************************************************************
\subsection*{Inhalt}
%******************************************************************************************************
Das Rechnen mit Strecken im Koordinatensystem hat viele aber nicht alle Eigenschaften vom Rechnen mit Zahlen. Wir können Strecken addieren, die Addition ist assoziativ. Wir können sie multiplizieren. Es gilt Distributivgesetz. Es gibt Strecken der Länge Null, die sich auch wie eine Null verhalten. Und es gibt sogar die negative Strecke zu einer Strecke, sodass wir bei einer Addition auf Null kommen.

Die beiden wichtigsten Unterschiede:
\begin{enumerate}
   \item Nur bestimmte Paare von Strecken können addiert werden. Wir sprechen von einer partiellen Verknüpfung.
   \item Die Multiplikation fällt aus dem Bereich heraus: das Produkt zweier Strecken ist keine Strecke, sondern eine Zahl.
\end{enumerate} 

Diese Unterschiede sind es, die mich davon abhalten, es Zahlbereich zu nennen.

Wichtig ist, dass wir den Strecken dabei eine Richtung geben: sie gehen vom Startpunkt zum Endpunkt. Wir sollten sie vielleicht besser als Pfeile schreiben.

Mit diesem Instrument haben wir einen wunderschöne Korrespondenz zwischen Geometrie und Algebra beim erweiterten (oder verallgemeinerten) Pythagoras.

Es ist aber ein schon in der Mittelstufe der Mathematik also recht früh auftauchender Blick auf eine sehr wichtige Struktur der höheren Mathematik: Kategorien, und da wir auch die negativen Strecken haben sogar ein: Gruppoide. Die Objekte sind die Punkte des Raumes. Je zwei Punkte haben genau einen Morphismus: die Strecke zwischen ihnen. Die Verknüpfung ist die Addition der Strecken. Die Identitäten sind die Null-Strecken.

%******************************************************************************************************
\subsection*{Präsentiert}
%******************************************************************************************************
Von Jörg Kunze

%******************************************************************************************************
\subsection*{Voraussetzungen}
%******************************************************************************************************
Koordinaten-System, Satz des Pythagoras, Abstand im Koordinatensystem, Rechnen mit Buchstaben, Skalarprodukt

%******************************************************************************************************
\subsection*{Text}
%******************************************************************************************************
Der Begleittext als PDF und als LaTeX findet sich unter
{\tiny
   \url{}
}

%******************************************************************************************************
\subsection*{Meine Videos}
%******************************************************************************************************
Siehe auch in den folgenden Videos:\\
\\
v1.7.6.4.2 (Mittel) Geometrie - Satz des Pythagoras\\
\url{https://youtu.be/mTMOtXfUQzQ}\\
\\
v1.7.6.4.3 (Mittel) Geometrie - Abstand im Koordinatensystem\\
\url{https://youtu.be/HEdolewfn78}
\\
\\v1.7.6.4.3.2 (Mittel) Abstand Gerade vom Nullpunkt
\url{https://youtu.be/_qN6Z-a72ok}
\\
\\v1.7.6.4.3.3 (Mittel) Geometrie - Skalarprodukt
\url{https://youtu.be/PMYHObLI54Q}

%******************************************************************************************************
\subsection*{Quellen}
%******************************************************************************************************
Siehe auch in den folgenden Seiten:\\
\\
\url{https://de.wikipedia.org/wiki/Satz_des_Pythagoras}\\
\url{https://de.wikipedia.org/wiki/Euklidischer_Abstand}\\
\url{https://de.wikipedia.org/wiki/Skalarprodukt}\\
\url{https://www.schuelerhilfe.de/online-lernen/1-mathematik/719-skalarprodukt}\\
\url{https://de.wikipedia.org/wiki/Kategorientheorie#Kategorie}\\
\url{https://de.wikipedia.org/wiki/Gruppoid_(Kategorientheorie)}\\

%******************************************************************************************************
\subsection*{Buch}
%******************************************************************************************************
Grundlage ist folgendes Buch:\\

"`KomplettWissen Mathematik Gymnasium Klasse 5-10"'\\
Klett Lerntraining bei PONS\\
978-3-12-926097-5 (ISBN)\\
{\tiny
   \url{https://www.lehmanns.de/shop/schulbuch-lexikon-woerterbuch/35031626-9783129260975-komplettwissen-mathematik-gymnasium-klasse-5-10
   }
}\\
\\
"`Grundlagen der ebenen Geometrie"'
Hendrik Kasten, Denis Vogel\\
Springer Berlin\\
978-3-662-57620-5 (ISBN)\\
{\tiny
   \url{
      https://www.lehmanns.de/shop/mathematik-informatik/43394438-9783662576205-grundlagen-der-ebenen-geometrie
   }
}

%******************************************************************************************************
\subsection*{Lizenz}
%******************************************************************************************************
Dieser Text und das Video sind freie Software. Sie können es unter den Bedingungen der 
GNU General Public License, wie von der Free Software Foundation veröffentlicht, weitergeben 
und/oder modifizieren, entweder gemäß Version 3 der Lizenz oder (nach Ihrer Option) jeder späteren Version.

Die Veröffentlichung von Text und Video erfolgt in der Hoffnung, dass es Ihnen von Nutzen sein wird, 
aber OHNE IRGENDEINE GARANTIE, sogar ohne die implizite Garantie der MARKTREIFE oder der 
VERWENDBARKEIT FÜR EINEN BESTIMMTEN ZWECK. Details finden Sie in der GNU General Public License.

Sie sollten ein Exemplar der GNU General Public License zusammen mit diesem Text erhalten haben 
(zu finden im selben Git-Projekt). 
Falls nicht, siehe \url{http://www.gnu.org/licenses/}.

\subsection*{Das Video}
%******************************************************************************************************
Das Video hierzu ist zu finden unter 
{\tiny
   \url{upps}
}

%******************************************************************************************************
%                                                                                                     *
\section{Skalarprodukt}
%                                                                                                     *
%******************************************************************************************************

%******************************************************************************************************
\subsection{Gerichtete Strecken}
%******************************************************************************************************
Wichtig ist, dass wir den Strecken dabei eine Richtung geben: sie gehen vom Startpunkt zum Endpunkt. Wir sollten sie vielleicht besser als Pfeile schreiben und zeichnen.

%******************************************************************************************************
\subsection{Definition des Skalarproduktes}
%******************************************************************************************************
Sei im Folgenden gegeben ein Dreieck $ABC$ mit den Punkten $A =(A_x,A_y), B =(B_x,B_y), C =(C_x,C_y)$ und den Seiten $a = \overrightarrow{AB}, b = \overrightarrow{BC}, c = \overrightarrow{CA}$.

\begin{tikzpicture}[scale=3]
   \draw[->, thick] (0, -0.5) -- (0, 1.5) node(yaxis)[above]{$y$};
   \draw[->, thick] (-0.5, 0) -- (2, 0) node(xaxis)[right]{$x$};

   \draw (1, -0.05) node[below]{$1$} -- (1, 0.05);
   \draw (-0.05, 1) node[left]{$1$} -- (0.05, 1);
   
   \fill (3/4,3/8) coordinate (A) circle (0.01) node[left]{A};
   \fill (4/4,8/8) coordinate (B) circle (0.01) node[above]{B};
   \fill (6/4,8/8) coordinate (C) circle (0.01) node[right]{C};
   \fill (7/4,4/8) coordinate (D) circle (0.01) node[right]{D};

   \draw[-Latex] (A) -- node[left] {a} (B);
   \draw[-Latex] (C) -- node[right] {c} (D);

   \draw[dashed] (A |- xaxis)  -- (A);

   \draw[dashed] (yaxis |- A)  -- (A);

   \node[red] at (0.1, 9/16) {$a_y$};
   \node[red] at (7/8, 0.1) {$a_x$};
\end{tikzpicture}

\begin{Definition}{Der Koordinaten-Versatz der Seiten}
   Der \textbf{Koordinaten-Versatz der Seiten} ist definiert durch:
   \begin{alignat}{4}
      &a_x &&:= B_x-A_x, \quad &a_y &&:= B_y-A_y\\
      &c_x &&:= D_x-C_x, \quad &c_y &&:= D_y-C_y.
   \end{alignat}
\end{Definition}

\begin{Definition}{Skalarprodukt}
   Das \textbf{Skalarprodukt} der Seiten $a$ und $b$ ist definiert durch:
   \begin{equation}
      ac := a \cdot c := \langle a, c \rangle := a_xc_x + a_yc_y.
   \end{equation}
   Wir werden meistens $ab$ schreiben. 
\end{Definition}

Ausführlich geschrieben ist
\begin{empheq}[box=\fbox]{equation}
   a \cdot b = (B_x-A_x)(D_x-C_x) + (B_y-A_y)(D_y-C_y).
\end{empheq}

%******************************************************************************************************
\subsection{Addition}
%******************************************************************************************************
Wir betrachten \textbf{gerichtete Strecken} $\overrightarrow{AB}$. Eine solche gerichtete Strecke hat einen \textbf{Anfang} (im Beispiel $A$) und ein \textbf{Ende} (hier $B$).
\begin{Definition}{Addition gerichteter Strecken}
   Für zwei Strecken $\overrightarrow{AB}$ und $\overrightarrow{BC}$, bei denen das Ende der ersten (linken) Strecke gleich dem Anfang der zweiten (rechten) Strecke ist, definieren wir deren Summe wie folgt:
   \begin{equation}
      \overrightarrow{AB} + \overrightarrow{BC} = \overrightarrow{AC}.
   \end{equation}
\end{Definition}

\begin{tikzpicture}[scale=3]
   \draw[->, thick] (0, -0.5) -- (0, 1.5) node(yaxis)[above]{$y$};
   \draw[->, thick] (-0.5, 0) -- (2, 0) node(xaxis)[right]{$x$};
   
   \draw (1, -0.05) node[below]{$1$} -- (1, 0.05);
   \draw (-0.05, 1) node[left]{$1$} -- (0.05, 1);
   
   \fill (3/4,3/8) coordinate (A) circle (0.01) node[left]{A};
   \fill (4/4,8/8) coordinate (B) circle (0.01) node[above]{B};
   \fill (6/4,4/8) coordinate (C) circle (0.01) node[right]{C};
   
   \draw[-Latex] (A) -- node[left] {a} (B);
   \draw[-Latex] (B) -- node[right] {b} (C);
   \draw[-Latex, red] (A) -- node[below] {c = a+b} (C);
\end{tikzpicture}

Die Summe entspricht dem kürzesten Weg zum Endpunkt eines Pfades aus aneinander gehängten Stücken.

\begin{Satz}{Addition von Strecken ist assoziativ}
   Seien $a, b, c$ drei Strecken, dann können wir die Summe $(a+b)+c$ genau dann bilden, wenn wir die Summe $a+(b+c)$ bilden können, in welchem Fall dann gilt:
   \begin{equation}
      (a+b)+c = a+(b+c).
   \end{equation}
\end{Satz}
\begin{proof}
   Wenn wir $a+b$ bilden können sind $a, b$ von der Form $\overrightarrow{AB}$ und $\overrightarrow{BC}$. Dann ist $a + b = \overrightarrow{AC}$. Wenn wir auch noch $(a+b)+c$ bilden können ist $c$ von der Form $\overrightarrow{CD}$. Dann können wir aber auch $b + c = \overrightarrow{BD}$ rechen und die Bedingungen für die Addition mit $a$ sind erfüllt. Der umgekehrte Weg geht genauso.
   Mit den Bezeichnungen von oben gilt dann:
   \begin{alignat}{4}
      &(a&&+b)&&+c &&= \\
      &(\overrightarrow{AB} &&+ \overrightarrow{BC}) &&+ \overrightarrow{CD} &&=\\
      &\overrightarrow{AC} && &&+ \overrightarrow{CD} &&=\\
      &\overrightarrow{AD} && && &&=\\
      &\overrightarrow{AB} && &&+ \overrightarrow{BD} &&=\\
      &\overrightarrow{AB} &&+ (\overrightarrow{BC} &&+ \overrightarrow{CD}) &&=\\
      &a&&+(b&&+c)
   \end{alignat}
\end{proof}

\begin{tikzpicture}[scale=3]
   \draw[->, thick] (0, -0.5) -- (0, 1.5) node(yaxis)[above]{$y$};
   \draw[->, thick] (-0.5, 0) -- (2, 0) node(xaxis)[right]{$x$};
   
   \draw (1, -0.05) node[below]{$1$} -- (1, 0.05);
   \draw (-0.05, 1) node[left]{$1$} -- (0.05, 1);
   
   \fill (3/4,3/8) coordinate (A) circle (0.01) node[left]{A};
   \fill (4/4,8/8) coordinate (B) circle (0.01) node[above]{B};
   \fill (6/4,8/8) coordinate (C) circle (0.01) node[right]{C};
   \fill (7/4,4/8) coordinate (D) circle (0.01) node[right]{D};
   
   \draw[-Latex] (A) -- node[left] {a} (B);
   \draw[-Latex] (B) -- node[above] {b} (C);
   \draw[-Latex] (C) -- node[right] {c} (D);

   \draw[-Latex, blue] (A) -- node[above] {a+b} (C);
   \draw[-Latex, blue] (B) -- node[below] {b+c} (D);
   
   \draw[-Latex, red] (A) -- node[below, yshift=-10] {(a+b)+c = a + (b+c)} (D);
\end{tikzpicture}

\begin{Definition}{Null-Strecke}
   Strecken der Form $\overrightarrow{AA}$, wobei $A$ ein beliebiger Punkt der Ebene ist, heißen \textbf{Null-Strecke}. Wir definieren
   \begin{equation}
      0_A := \overrightarrow{AA}.
   \end{equation}   
\end{Definition}

\begin{Satz}{Länge von Null-Strecken ist Null}
   \begin{equation}
      |0_A| = 0.
   \end{equation}
\end{Satz}
\begin{proof}
   Allgemein ist die Länge der Strecke $\overrightarrow{AB}$, wo $A = (A_x, A_y)$ und $B= (B_x,B_y)$ sind, ist definiert als (Definition von Pythagoras):
   \begin{equation}
      |\overrightarrow{AB}| := \sqrt{ (B_x-A_x)^2 + (B_y-A_y)^2}.
   \end{equation}
   Im Falle einer Null-Strecke ist $A_x = B_x$ und $A_y = B_y$. Die Terme heben sich gegenseitig auf: $B_x-A_x = 0$ und $B_y-A_y = 0$.
\end{proof}

\begin{Definition}{Inverse Strecke}
   Sei $\overrightarrow{AB}$, dann ist die \textbf{inverse Strecke} definiert durch:
   \begin{equation}
      - \overrightarrow{AB} := \overrightarrow{BA}.
   \end{equation}   
\end{Definition}

\begin{Satz}{Inverse Strecke ist inverses}
   \begin{equation}
      \overrightarrow{AB} + \overrightarrow{BA} = 0_A.
   \end{equation}
\end{Satz}
\begin{proof}
   Nach der Definition der Addition (linkes $=$) und der Definition der Null-Strecken (rechtes $=$) ist
   \begin{equation}
      \overrightarrow{AB} + \overrightarrow{BA} = \overrightarrow{AA} = 0_A.
   \end{equation}
\end{proof}

\begin{Satz}{Koordinaten Versatz ist additiv}
   Seien $a,b,c$ gerichtete Strecken mit $a + b = c$, dann gilt für die Koordinaten-Versätze:
   \begin{equation}
      a_x + b_x = c_x \text{ und } a_y + b_y = c_y.
   \end{equation}
\end{Satz}
\begin{proof}
   \begin{alignat}{4}
      &a_x         &&+ b_x       &&=\\
      &(B_x - A_x) &&+ (C_x-B_x) &&=\\
      &C_x + B_x   &&- B_x - A_x &&=\\
      &C_x         &&+ A_x       &&=\\
      &c_x
   \end{alignat}
   und genauso für die $y$-Koordinate.
\end{proof}

Dieses Phänomen, dass sich $B$ als Ende der einen und Anfang der anderen Strecke beim Rechnen aufhebt, kommt überall in der Mathematik immer wieder vor. Insbesondere baut die homologische Algebra, ein Thema des weit vorgeschrittenen Studiums, unter anderem darauf auf.

%******************************************************************************************************
\subsection{Multiplikation}
%******************************************************************************************************
Das Skalarprodukt von oben definiert eine Multiplikation von Strecken. Die Strecken müssen dabei \emph{nicht} verbunden oder irgendwie zueinander positioniert sein. Die Multiplikation verhält sich zur Addition fast wie bei normalen Zahlen.

\begin{Satz}{Kommutativgesetz}
   \begin{equation}
      ab = ba.
   \end{equation}
\end{Satz}
\begin{proof}
   Nach \myRef{Definition}{Skalarprodukt} gilt
   \begin{equation}
      ab = a_xb_x + a_yb_y = b_xa_x + b_ya_y = ba.
   \end{equation}
   Hier haben wir das Kommutativgesetz für \R\ verwendet. Wir können also sagen, das Kommutativgesetz für \R\ schlägt auf Skalarprodukte durch.
\end{proof}

\begin{Satz}{Mal Null ist Null}
   \begin{equation}
      a \cdot 0_D = 0.
   \end{equation}
\end{Satz}
\begin{proof}
   Sei $a = \overrightarrow{AB}$, dann gilt
   \begin{equation}
      a \cdot 0_D = (B_x-A_x)(D_x-D_x) + (B_y-A_y)(D_y-D_y) = (B_x-A_x) \cdot 0 + (B_y-A_y) \cdot 0 = 0.
   \end{equation}
\end{proof}

\begin{Satz}{Inverse Strecke ist inverses}
   \begin{equation}
      \overrightarrow{AB} + \overrightarrow{BA} = 0_A.
   \end{equation}
\end{Satz}
\begin{proof}
   Nach der Definition der Addition (linkes $=$) und der Definition der Null-Strecken (rechtes $=$) ist
   \begin{equation}
      \overrightarrow{AB} + \overrightarrow{BA} = \overrightarrow{AA} = 0_A.
   \end{equation}
\end{proof}

\begin{Satz}{Distributivgesetz}
   Seien $a, b, d$ gerichtete Strecken, so dass $a$ und $B$ addiert werden können, dann gilt
   \begin{equation}
      (a + b) \cdot d = a \cdot d + b \cdot d.
   \end{equation}
\end{Satz}
\begin{proof}
   Wir benutzen \myRef{Satz}{Koordinaten Versatz ist additiv}:
   \begin{alignat}{4}
      &(a+b) \cdot d && &&= \\
      &(a_x+b_x)d_x &&+ (a_y + b_y)d_y &&=\\
      &a_x d_y +b_xd_x &&+ a_y d_y + b_yd_y &&=\\
      &(a_x d_y +a_y d_y) &&+ (b_xd_x + b_y d_y) &&=\\
      &ad &&+ bd
   \end{alignat}
\end{proof}

Wir haben in \myRef{Satz}{Länge von Null-Strecken ist Null} bereits gezeigt, dass Null-Strecken die Länge Null haben. Dies können wir nun mit \myRef{Satz}{Mal Null ist Null} einfacher beweisen:
\begin{equation}
   |0_A| = \sqrt{0_A \cdot 0_A} = \sqrt 0 = 0.
\end{equation}

%******************************************************************************************************
\subsection{Pythagoras}
%******************************************************************************************************
Der in einem vorherigen Video schon bewiesenen verallgemeinerten Satz des Pythagoras können wir nun nochmal und diesmal rein algebraisch beweisen.
\begin{Satz}{Verallgemeinerter Pythagoras}
   In einem allgemeinem Dreieck mit den Seiten $a, b, c$ gilt
   \begin{empheq}[box=\fbox]{equation}
      c^2 = a^2 + 2ab + b^2.
   \end{empheq}
\end{Satz}
\begin{proof}
   Wir benutzen zweimal den \myRef{Satz}{Distributivgesetz} und einmal \myRef{Satz}{Kommutativgesetz}:
   \begin{alignat}{4}
      &c^2 = (a+b)^2 = (a+b)(a+b) &&=\\
      &(a+b)a + (a+b)b &&=\\
      &a^2 + ba + ab + b^2 &&=\\
      &c^2 = a^2 + 2ab + b^2.
   \end{alignat}
\end{proof}

Erinnerung: Damit verhindert das Skalarprodukt $ab$, wenn es ungleich Null ist, die Gültigkeit des Pythagoras. Das Skalarprodukt ist damit das (oder ein Maß für das) Hindernis für die Gültigkeit. Es ist ein Maß für die Obstruktion des Pythagoras. Mit einem anderen Blickwinkel ist das Skalarprodukt ein Maß für den Fehler des Pythagoras oder nochmal leicht anders sein Korrekturterm für den allgemeinen Fall.


%******************************************************************************************************
\subsection{Aber, ist es eine Zahl}
%******************************************************************************************************
Als mathematische Frage ist sie völliger Unsinn, solange wir nicht definiert haben, was eine Zahl ist.

Intuitiv und emotional können wir aber feststellen: wie bei Zahlen können wir addieren, multiplizieren und es gelten die üblichen Rechenregel (assoziativ, kommutativ, distributiv). Es gibt Nullen und minus.

Aber: wir können nicht jedes Paar von Strecken addieren. Die Multiplikation von zwei Strecken ergibt keine Strecke. Diese Unterschiede sind es, die mich davon abhalten, es Zahlbereich zu nennen. Die entdeckte Struktur ist aber einem Zahlbereich sehr ähnlich.

%******************************************************************************************************
\subsection{Ausblick}
%******************************************************************************************************
Mit diesem Instrument haben wir einen wunderschöne Korrespondenz zwischen Geometrie und Algebra beim erweiterten (oder verallgemeinerten) Pythagoras.

Es ist ein schon in der Mittelstufe der Mathematik also recht früh auftauchender Blick auf eine sehr wichtige Struktur der höheren Mathematik: Kategorien, und da wir auch die negativen Strecken haben sogar ein: Gruppoide. Die Objekte sind die Punkte des Raumes. Je zwei Punkte haben genau einen Morphismus: die Strecke zwischen ihnen. Die Verknüpfung ist die Addition der Strecken. Die Identitäten sind die Null-Strecken.

%******************************************************************************************************
%                                                                                                     *
\section{TODO:}
%                                                                                                     *
%******************************************************************************************************
\begin{backup}
    (Zur Zeit nicht benötigter Inhalt)
\end{backup}

%******************************************************************************************************
%                                                                                                     *
\begin{thebibliography}{9}
%                                                                                                     *
%******************************************************************************************************

   \bibitem[Klett2016]{Klett}
      Klett Lerntraining, \emph{Komplett Wissen Mathematik Gymnasium 5-10},
      Klett Lerntraining, 978-3-12-926097-5 (ISBN)
      
   \bibitem[KastenVogel2018]{Kasten}
       Hendrik Kasten, Denis Vogel , \emph{Grundlagen der ebenen Geometrie},
      2018 Springer Berlin, 978-3-662-57620-5 (ISBN)

\end{thebibliography}

%******************************************************************************************************
%                                                                                                     *
\begin{large}
    \centerline{\textsc{Symbolverzeichnis}}
\end{large}
%                                                                                                     *
%******************************************************************************************************
\bigskip

\renewcommand*{\arraystretch}{1}

\begin{tabular}{ll}
    $a,b,c, \cdots$               &Seiten\\
    $a_x,a_y$                   &Koordinatenversätze der Seite $a$\\
    $|a|$                       &Länge der Seite $a$\\
    $x, y$                      &Koordinatenachsen\\
    $A, B, C, \cdots$          &Punkte\\
    $A_x, A_y$                 &Koordinaten des Punktes $A$\\
    $\overrightarrow{AB}$      &gerichtete Strecke von $A$ nach $B$\\
    $\perp$                    &Senkrecht
\end{tabular}

\end{document}
