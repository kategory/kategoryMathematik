%******************************************************** -*-LaTeX-*- ******************************
%                                                                                                  *
% v1.7.6.5.1 Winkelzwänge.tex                                                                      *
%                                                                                                  *
% Copyright (C) 2025 Kategory GmbH \& Co. KG (joerg.kunze@kategory.de)                             *
%                                                                                                  *
% v1.7.6.5.1 Winkelzwänge is part of kategoryMathematik.                                           *
%                                                                                                  *
% kategoryMathematik is free software: you can redistribute it and/or modify                       *
% it under the terms of the GNU General Public License as published by                             *
% the Free Software Foundation, either version 3 of the License, or                                *
% (at your option) any later version.                                                              *
%                                                                                                  *
% kategoryMathematik is distributed in the hope that it will be useful,                            *
% but WITHOUT ANY WARRANTY; without even the implied warranty of                                   *
% MERCHANTABILITY or FITNESS FOR A PARTICULAR PURPOSE.  See the                                    *
% GNU General Public License for more details.                                                     *
%                                                                                                  *
% You should have received a copy of the GNU General Public License                                *
% along with this program.  If not, see <http://www.gnu.org/licenses/>.                            *
%                                                                                                  *
%***************************************************************************************************

\documentclass[a4paper]{amsart}
% \documentclass[a4paper]{book}

%-----------------------------------------------------------------------------------------------------*
% package:                                                                                            *
%-----------------------------------------------------------------------------------------------------*
\usepackage{amssymb}
\usepackage{amsfonts}
\usepackage{amsmath}
\usepackage{amsthm}

\usepackage{mathabx}

\usepackage{a4wide} % a little bit smaller margins

\usepackage{empheq}

\usepackage{graphicx}
\usepackage{hyperref}
\usepackage{algorithmic}
\usepackage{listings}
\usepackage{color}
\usepackage{colortbl}
\usepackage{sidecap}
\usepackage{comment}
\usepackage{tcolorbox}
\usepackage{collect}

\usepackage{upgreek}

% \usepackage{diagrams}

\usepackage[german]{babel}
\usepackage[none]{hyphenat}
\emergencystretch=4em

\usepackage[utf8]{inputenc} % to be able to use äöü as characters in text
\usepackage[T1]{fontenc} % to be able to use äöü in lables
\usepackage{lmodern}     % to avoid pixelation introduced by fontenc

\usepackage{hyperref}

\usepackage{tikz}
\usepackage{tikz-cd}
\usetikzlibrary{babel}
\usetikzlibrary{calc}
\usetikzlibrary{patterns}

%-----------------------------------------------------------------------------------------------------*
% theorem:                                                                                            *
%-----------------------------------------------------------------------------------------------------*
\theoremstyle{definition}
\newtheorem{theorem}{Theorem}[subsection]

\newcommand{\myTheorem}[1]{%
  \newtheorem{jk#1}[theorem]{#1}
  \newenvironment{#1}[1]{%
    \expandafter\begin{jk#1} \expandafter\label{#1:##1}\textbf{(##1):}
  }{%
    \expandafter\end{jk#1}
  }
}

\myTheorem{Definition}
\myTheorem{Proposition}
\myTheorem{Satz}
\myTheorem{Theorem}
\myTheorem{Example}
\myTheorem{Remark}

\definecollection{jkjkFrage}
\newtheorem{jkFrage}[theorem]{Frage}
\newenvironment{Frage}[1]{%
  \expandafter\begin{jkFrage} \expandafter\label{Frage:#1}\textbf{(#1):}
  \begin{collect}{jkjkFrage}{}{}
    \item \ref{Frage:#1} #1
  \end{collect}
}{%
  \expandafter\end{jkFrage}
}

\newcommand{\myRef}[2]{[#1 \ref{#1:#2}, ``#2'']}

\renewcommand{\proofname}{Beweis}

%-----------------------------------------------------------------------------------------------------*
% operator:                                                                                           *
%-----------------------------------------------------------------------------------------------------*
\DeclareMathOperator{\End}{End}
\DeclareMathOperator{\Ker}{Ker}
\DeclareMathOperator{\Mat}{Mat}
\DeclareMathOperator{\rank}{rank}
\DeclareMathOperator{\ggT}{ggT}
\DeclareMathOperator{\len}{len}
\DeclareMathOperator{\ord}{ord}
\DeclareMathOperator{\kgV}{kgV}
\DeclareMathOperator{\id}{id}
\DeclareMathOperator{\red}{red}
\DeclareMathOperator{\supp}{supp}
\DeclareMathOperator{\Bild}{Bild}
\DeclareMathOperator{\Rang}{Rang}
\DeclareMathOperator{\Det}{Det}
\DeclareMathOperator{\Hom}{Hom}

\DeclareMathOperator{\sub}{sub}
\DeclareMathOperator{\blk}{blk}
\DeclareMathOperator{\minimal}{minimal}
\DeclareMathOperator{\maximal}{maximal}

\definecolor{mygreen}{rgb}{0,0.6,0}
\definecolor{mygray}{rgb}{0.5,0.5,0.5}
\definecolor{mymauve}{rgb}{0.58,0,0.82}

\lstset{ %
  backgroundcolor=\color{white},   % choose the background color
  basicstyle=\ttfamily\footnotesize,        % size of fonts used for the code
  breaklines=true,                 % automatic line breaking only at whitespace
  captionpos=b,                    % sets the caption-position to bottom
  commentstyle=\color{mygreen},    % comment style
  escapeinside={\%*}{*)},          % if you want to add LaTeX within your code
  keywordstyle=\color{blue},       % keyword style
  stringstyle=\color{mymauve},     % string literal style
  frame=single
}

\setcounter{MaxMatrixCols}{20}

%******************************************************************************************************
%                                                                                                     *
% definition:                                                                                         *
%                                                                                                     *
%******************************************************************************************************
\newcommand{\R}{\ensuremath{\mathbb{ R }}}
\newcommand{\Q}{\ensuremath{\mathbb{ Q }}}
\newcommand{\Z}{\ensuremath{\mathbb{ Z }}}
\newcommand{\N}{\ensuremath{\mathbb{ N }}}
\newcommand{\C}{\ensuremath{\mathbb{ C }}}
\newcommand{\A}{\ensuremath{\mathbb{ A }}}
\newcommand{\F}{\ensuremath{\mathbb{ F }}}
\newcommand{\K}{\ensuremath{\mathbb{ K }}}
\newcommand{\Pb}{\ensuremath{\mathbb{ P }}}

\newcommand{\M}{\ensuremath{\mathcal{ M }}}
\newcommand{\V}{\ensuremath{\mathcal{ V }}}

\newcommand{\AAA}{\ensuremath{\mathcal{ A }}}
\newcommand{\BB}{\ensuremath{\mathcal{ B }}}
\newcommand{\CC}{\ensuremath{\mathcal{ C }}}
\newcommand{\EE}{\ensuremath{\mathcal{ E }}}
\newcommand{\KK}{\ensuremath{\mathcal{ K }}}
\newcommand{\MM}{\ensuremath{\mathcal{ M }}}
\newcommand{\PP}{\ensuremath{\mathcal{ P }}}
\newcommand{\ZZ}{\ensuremath{\mathcal{ Z }}}

\newcommand{\imporant}[1]{ \textcolor{red}{\textbf{#1}} }

\newcommand{\bb}[1]{\mathbf{#1}}
\newcommand{\balpha}{\boldsymbol{\upalpha}}
\newcommand{\bbeta}{\boldsymbol{\upbeta}}
\newcommand{\bgamma}{\boldsymbol{\upgamma}}
\newcommand{\bdelta}{\boldsymbol{\delta}}
\newcommand{\bmu}{\boldsymbol{\upmu}}

\newcommand{\z}[1]{\Z_{#1}}
\newcommand{\e}[1]{\z{#1}^*}
\newcommand{\q}[1]{(\e{#1})^2}

\excludecomment{book}
\excludecomment{example}
\excludecomment{backup}

\begin{document}

%******************************************************************************************************
%                                                                                                     *
\begin{titlepage}
%                                                                                                     *
%******************************************************************************************************
% \vspace*{\fill}
\centering
{\huge
(Mittel) Geometrie\\[1cm]
\textbf{v1.7.6.5.1 Winkelzwänge}
}\\[1cm]

\textbf{Kategory GmbH \& Co. KG}\\
Präsentiert von Jörg Kunze\\
Copyright (C) 2025 Kategory GmbH \& Co. KG

\end{titlepage}

%\clearpage
%\setcounter{page}{2}
%
%\tableofcontents

\newpage

%******************************************************************************************************
%                                                                                                     *
\section*{Beschreibung}
%                                                                                                     *
%******************************************************************************************************


%******************************************************************************************************
\subsection*{Inhalt}
%******************************************************************************************************
Wir wissen nun, was senkrecht heißt. Wir haben mehrere äquivalente Definitionen von senkrecht, 90 Grad.

Wir nennen einen Winkel stumpf, wenn er größer, und spitz, wenn er kleiner als 90 Grad ist. Wir nennen ihn zusätzlich ultra-spitz, wenn er kleiner als 60 Grad ist.

Da wir immer drei Winkel im Dreieck haben und die Summe immer 180 Grad ist gibt es eine Reihe von Zwängen, denen die Winkel eines Dreiecks unterliegen.

Auf der anderen Seite, gibt es zu je 3 Zahlen größer 0, deren Summe 180 ist, Dreiecke mit diesen 3 Zahlen als Winkel.

%******************************************************************************************************
\subsection*{Präsentiert}
%******************************************************************************************************
Von Jörg Kunze

%******************************************************************************************************
\subsection*{Voraussetzungen}
%******************************************************************************************************
Punkte, Strecken, Dreiecke, Winkel, Winkelsumme im Dreieck

%******************************************************************************************************
\subsection*{Text}
%******************************************************************************************************
Der Begleittext als PDF und als LaTeX findet sich unter
{\tiny
   \url{https://github.com/kategory/kategoryMathematik/tree/main/v1%20Mittel/v1.7%20Klasse%207%20und%208/v1.7.6.5%20Winkel%20vs%20Seitenl%C3%A4nge}
}

%******************************************************************************************************
\subsection*{Meine Videos}
%******************************************************************************************************
Siehe auch in den folgenden Videos:\\
\\
v1.7.6.0.1 (Mittel) Geometrie - Punkt, Gerade, Ebene, Kreis, Winkel\\
\url{https://youtu.be/sM064kyF67Y}\\
\\
v1.7.6.2 (Mittel) Geometrie - Winkelsummen - Winkelsummen im Dreieck ist 180\\
\url{https://youtu.be/ERo4638IY68}\\
\\
\\v1.7.6.3 (Mittel) Geometrie - Besondere Dreiecke
\url{https://youtu.be/TrUNSDnTISs}
\\
\\v1.7.6.4.3.5 (Mittel) Geometrie - Senkrecht
\url{https://youtu.be/3e7ScXE-qBQ}\\
\\
\\v1.7.6.5 (Mittel) Geometrie - Winkel vs Seitenlänge
\url{https://youtu.be/64glnLLQMp4}

%******************************************************************************************************
\subsection*{Quellen}
%******************************************************************************************************
Siehe auch in den folgenden Seiten:\\
\url{https://de.wikipedia.org/wiki/Winkel#Arten_von_Winkeln}\\
\url{https://de.wikipedia.org/wiki/Stumpfer_Winkel}\\
\url{https://de.wikipedia.org/wiki/Dreieck}\\
\url{https://de.schubu.org/p305/die-winkelsumme}

%******************************************************************************************************
\subsection*{Buch}
%******************************************************************************************************
Grundlage ist folgendes Buch:\\

"`KomplettWissen Mathematik Gymnasium Klasse 5-10"'\\
Klett Lerntraining bei PONS\\
978-3-12-926097-5 (ISBN)\\
{\tiny
   \url{https://www.lehmanns.de/shop/schulbuch-lexikon-woerterbuch/35031626-9783129260975-komplettwissen-mathematik-gymnasium-klasse-5-10
   }
}\\
\\
"`Grundlagen der ebenen Geometrie"'
Hendrik Kasten, Denis Vogel\\
Springer Berlin\\
978-3-662-57620-5 (ISBN)\\
{\tiny
   \url{
      https://www.lehmanns.de/shop/mathematik-informatik/43394438-9783662576205-grundlagen-der-ebenen-geometrie
   }
}

%******************************************************************************************************
\subsection*{Lizenz}
%******************************************************************************************************
Dieser Text und das Video sind freie Software. Sie können es unter den Bedingungen der 
GNU General Public License, wie von der Free Software Foundation veröffentlicht, weitergeben 
und/oder modifizieren, entweder gemäß Version 3 der Lizenz oder (nach Ihrer Option) jeder späteren Version.

Die Veröffentlichung von Text und Video erfolgt in der Hoffnung, dass es Ihnen von Nutzen sein wird, 
aber OHNE IRGENDEINE GARANTIE, sogar ohne die implizite Garantie der MARKTREIFE oder der 
VERWENDBARKEIT FÜR EINEN BESTIMMTEN ZWECK. Details finden Sie in der GNU General Public License.

Sie sollten ein Exemplar der GNU General Public License zusammen mit diesem Text erhalten haben 
(zu finden im selben Git-Projekt). 
Falls nicht, siehe \url{http://www.gnu.org/licenses/}.

\subsection*{Das Video}
%******************************************************************************************************
Das Video hierzu ist zu finden unter 
{\tiny
   \url{upps}
}

%******************************************************************************************************
%                                                                                                     *
\section{v1.7.6.5.1 Winkelzwänge}
%                                                                                                     *
%******************************************************************************************************

%******************************************************************************************************
\subsection{Senkrecht, spitz, stumpf}
%******************************************************************************************************
Ein Winkel heißt
\begin{itemize}
   \item stumpf, wenn er $> 90$
   \item recht, wenn er $= 90$
   \item spitz, wenn er $< 90$
   \item ultra-spitz, wenn er $< 60$
\end{itemize}
ist. Ein ultra-spitzer Winkel ist also auch spitz. Den Begriff ultra-spitz habe ich selber erfunden und keine Alternative gefunden.

%******************************************************************************************************
\subsection{Zwänge}
%******************************************************************************************************
Für die 3 Winkel eines beliebigen Dreiecks gilt:
\begin{itemize}
   \item Jeder $< 180$
   \item Die Summe von je zwei $< 180$
   \item Höchstens einer $> 90$ (höchsten ein stumpfer)
   \item Höchstens einer $= 90$ (höchsten ein rechter)
   \item Höchstens zwei $> 60$
   \item Höchstens zwei $< 60$ (höchsten zwei ultra-spitze)
\end{itemize}

%******************************************************************************************************
\subsection{Zwänge im rechtwinkligen Dreieck}
%******************************************************************************************************
Für die 2 weiteren Winkel eines beliebigen rechtwinkligen Dreiecks gilt darüber hinaus:
\begin{itemize}
   \item Jeder $< 90$ (keine stumpfen)
   \item Kein $= 90$ (nur ein rechter)
   \item Höchstens ein $> 45$
   \item Höchstens ein $< 45$ 
\end{itemize}

%******************************************************************************************************
\subsection{Winkel berechnen}
%******************************************************************************************************
Wenn zwei Winkel bekannt sind, können wir den dritten berechnen. 
\begin{equation}\label{berechnen}
   w_3 = 180 - (w_1+w_2).
\end{equation}

Im rechtwinkligen können wir, wenn einer bekannt ist, den anderen berechnen, denn die Summe der beiden weiteren Winkel muss 90 sein.
\begin{equation}\label{berechnenRecht}
   w_2 = 90 - w_1.
\end{equation}

%******************************************************************************************************
\subsection{Konstruktion}
%******************************************************************************************************
Wenn die Summe von 3 Zahlen, die größer als 0 sind, 180 ergibt, gibt es auch ein Dreieck mit diesen Zahlen als Winkel. Beweis über ein Diagramm mit Parallelen. 

%******************************************************************************************************
%                                                                                                     *
\section{TODO:}
%                                                                                                     *
%******************************************************************************************************
\begin{backup}
    (Zur Zeit nicht benötigter Inhalt)
\end{backup}

%******************************************************************************************************
%                                                                                                     *
\begin{thebibliography}{9}
%                                                                                                     *
%******************************************************************************************************

   \bibitem[Klett2016]{Klett}
      Klett Lerntraining, \emph{Komplett Wissen Mathematik Gymnasium 5-10},
      Klett Lerntraining, 978-3-12-926097-5 (ISBN)
      
   \bibitem[KastenVogel2018]{Kasten}
       Hendrik Kasten, Denis Vogel , \emph{Grundlagen der ebenen Geometrie},
      2018 Springer Berlin, 978-3-662-57620-5 (ISBN)

\end{thebibliography}

%******************************************************************************************************
%                                                                                                     *
\begin{large}
    \centerline{\textsc{Symbolverzeichnis}}
\end{large}
%                                                                                                     *
%******************************************************************************************************
\bigskip

\renewcommand*{\arraystretch}{1}

\begin{tabular}{ll}
    $\vec a,\vec b, \vec c, \cdots$               &Seiten\\
    $a_x,a_y$                   &Koordinatenversätze der Seite $a$\\
    $|a|$                       &Länge der Seite $a$\\
    $m$                         &Steigung der Gerade\\
    $q$                         &Achsenabschnitt der Gerade\\
    $x, y$                      &Koordinatenachsen\\
    $A, B, C, \cdots$          &Punkte\\
    $A_x, A_y$                 &Koordinaten des Punktes $A$\\
    $\overrightarrow{AB}$      &gerichtete Strecke von $A$ nach $B$\\
    $\perp$                    &Senkrecht
\end{tabular}

\end{document}
