%******************************************************** -*-LaTeX-*- ******************************
%                                                                                                  *
% v1.7.6.4.3.5 Senkrecht.tex                                                                       *
%                                                                                                  *
% Copyright (C) 2024 Kategory GmbH \& Co. KG (joerg.kunze@kategory.de)                             *
%                                                                                                  *
% v1.7.6.4.3.5 Senkrecht is part of kategoryMathematik.                                            *
%                                                                                                  *
% kategoryMathematik is free software: you can redistribute it and/or modify                       *
% it under the terms of the GNU General Public License as published by                             *
% the Free Software Foundation, either version 3 of the License, or                                *
% (at your option) any later version.                                                              *
%                                                                                                  *
% kategoryMathematik is distributed in the hope that it will be useful,                            *
% but WITHOUT ANY WARRANTY; without even the implied warranty of                                   *
% MERCHANTABILITY or FITNESS FOR A PARTICULAR PURPOSE.  See the                                    *
% GNU General Public License for more details.                                                     *
%                                                                                                  *
% You should have received a copy of the GNU General Public License                                *
% along with this program.  If not, see <http://www.gnu.org/licenses/>.                            *
%                                                                                                  *
%***************************************************************************************************

\documentclass[a4paper]{amsart}
% \documentclass[a4paper]{book}

%-----------------------------------------------------------------------------------------------------*
% package:                                                                                            *
%-----------------------------------------------------------------------------------------------------*
\usepackage{amssymb}
\usepackage{amsfonts}
\usepackage{amsmath}
\usepackage{amsthm}

\usepackage{mathabx}

\usepackage{a4wide} % a little bit smaller margins

\usepackage{empheq}

\usepackage{graphicx}
\usepackage{hyperref}
\usepackage{algorithmic}
\usepackage{listings}
\usepackage{color}
\usepackage{colortbl}
\usepackage{sidecap}
\usepackage{comment}
\usepackage{tcolorbox}
\usepackage{collect}

\usepackage{upgreek}

% \usepackage{diagrams}

\usepackage[german]{babel}
\usepackage[none]{hyphenat}
\emergencystretch=4em

\usepackage[utf8]{inputenc} % to be able to use äöü as characters in text
\usepackage[T1]{fontenc} % to be able to use äöü in lables
\usepackage{lmodern}     % to avoid pixelation introduced by fontenc

\usepackage{hyperref}

\usepackage{tikz}
\usepackage{tikz-cd}
\usetikzlibrary{babel}
\usetikzlibrary{calc}
\usetikzlibrary{patterns}

%-----------------------------------------------------------------------------------------------------*
% theorem:                                                                                            *
%-----------------------------------------------------------------------------------------------------*
\theoremstyle{definition}
\newtheorem{theorem}{Theorem}[subsection]

\newcommand{\myTheorem}[1]{%
  \newtheorem{jk#1}[theorem]{#1}
  \newenvironment{#1}[1]{%
    \expandafter\begin{jk#1} \expandafter\label{#1:##1}\textbf{(##1):}
  }{%
    \expandafter\end{jk#1}
  }
}

\myTheorem{Definition}
\myTheorem{Proposition}
\myTheorem{Satz}
\myTheorem{Theorem}
\myTheorem{Example}
\myTheorem{Remark}

\definecollection{jkjkFrage}
\newtheorem{jkFrage}[theorem]{Frage}
\newenvironment{Frage}[1]{%
  \expandafter\begin{jkFrage} \expandafter\label{Frage:#1}\textbf{(#1):}
  \begin{collect}{jkjkFrage}{}{}
    \item \ref{Frage:#1} #1
  \end{collect}
}{%
  \expandafter\end{jkFrage}
}

\newcommand{\myRef}[2]{[#1 \ref{#1:#2}, ``#2'']}

\renewcommand{\proofname}{Beweis}

%-----------------------------------------------------------------------------------------------------*
% operator:                                                                                           *
%-----------------------------------------------------------------------------------------------------*
\DeclareMathOperator{\End}{End}
\DeclareMathOperator{\Ker}{Ker}
\DeclareMathOperator{\Mat}{Mat}
\DeclareMathOperator{\rank}{rank}
\DeclareMathOperator{\ggT}{ggT}
\DeclareMathOperator{\len}{len}
\DeclareMathOperator{\ord}{ord}
\DeclareMathOperator{\kgV}{kgV}
\DeclareMathOperator{\id}{id}
\DeclareMathOperator{\red}{red}
\DeclareMathOperator{\supp}{supp}
\DeclareMathOperator{\Bild}{Bild}
\DeclareMathOperator{\Rang}{Rang}
\DeclareMathOperator{\Det}{Det}
\DeclareMathOperator{\Hom}{Hom}

\DeclareMathOperator{\sub}{sub}
\DeclareMathOperator{\blk}{blk}
\DeclareMathOperator{\minimal}{minimal}
\DeclareMathOperator{\maximal}{maximal}

\definecolor{mygreen}{rgb}{0,0.6,0}
\definecolor{mygray}{rgb}{0.5,0.5,0.5}
\definecolor{mymauve}{rgb}{0.58,0,0.82}

\lstset{ %
  backgroundcolor=\color{white},   % choose the background color
  basicstyle=\ttfamily\footnotesize,        % size of fonts used for the code
  breaklines=true,                 % automatic line breaking only at whitespace
  captionpos=b,                    % sets the caption-position to bottom
  commentstyle=\color{mygreen},    % comment style
  escapeinside={\%*}{*)},          % if you want to add LaTeX within your code
  keywordstyle=\color{blue},       % keyword style
  stringstyle=\color{mymauve},     % string literal style
  frame=single
}

\setcounter{MaxMatrixCols}{20}

%******************************************************************************************************
%                                                                                                     *
% definition:                                                                                         *
%                                                                                                     *
%******************************************************************************************************
\newcommand{\R}{\ensuremath{\mathbb{ R }}}
\newcommand{\Q}{\ensuremath{\mathbb{ Q }}}
\newcommand{\Z}{\ensuremath{\mathbb{ Z }}}
\newcommand{\N}{\ensuremath{\mathbb{ N }}}
\newcommand{\C}{\ensuremath{\mathbb{ C }}}
\newcommand{\A}{\ensuremath{\mathbb{ A }}}
\newcommand{\F}{\ensuremath{\mathbb{ F }}}
\newcommand{\K}{\ensuremath{\mathbb{ K }}}
\newcommand{\Pb}{\ensuremath{\mathbb{ P }}}

\newcommand{\M}{\ensuremath{\mathcal{ M }}}
\newcommand{\V}{\ensuremath{\mathcal{ V }}}

\newcommand{\AAA}{\ensuremath{\mathcal{ A }}}
\newcommand{\BB}{\ensuremath{\mathcal{ B }}}
\newcommand{\CC}{\ensuremath{\mathcal{ C }}}
\newcommand{\EE}{\ensuremath{\mathcal{ E }}}
\newcommand{\KK}{\ensuremath{\mathcal{ K }}}
\newcommand{\MM}{\ensuremath{\mathcal{ M }}}
\newcommand{\PP}{\ensuremath{\mathcal{ P }}}
\newcommand{\ZZ}{\ensuremath{\mathcal{ Z }}}

\newcommand{\imporant}[1]{ \textcolor{red}{\textbf{#1}} }

\newcommand{\bb}[1]{\mathbf{#1}}
\newcommand{\balpha}{\boldsymbol{\upalpha}}
\newcommand{\bbeta}{\boldsymbol{\upbeta}}
\newcommand{\bgamma}{\boldsymbol{\upgamma}}
\newcommand{\bdelta}{\boldsymbol{\delta}}
\newcommand{\bmu}{\boldsymbol{\upmu}}

\newcommand{\z}[1]{\Z_{#1}}
\newcommand{\e}[1]{\z{#1}^*}
\newcommand{\q}[1]{(\e{#1})^2}

\excludecomment{book}
\excludecomment{example}
\excludecomment{backup}

\begin{document}

%******************************************************************************************************
%                                                                                                     *
\begin{titlepage}
%                                                                                                     *
%******************************************************************************************************
% \vspace*{\fill}
\centering
{\huge
(Mittel) Geometrie\\[1cm]
\textbf{v1.7.6.4.3.5 Senkrecht}
}\\[1cm]

\textbf{Kategory GmbH \& Co. KG}\\
Präsentiert von Jörg Kunze\\
Copyright (C) 2024 Kategory GmbH \& Co. KG

\end{titlepage}

%\clearpage
%\setcounter{page}{2}
%
%\tableofcontents

\newpage

%******************************************************************************************************
%                                                                                                     *
\section*{Beschreibung}
%                                                                                                     *
%******************************************************************************************************


%******************************************************************************************************
\subsection*{Inhalt}
%******************************************************************************************************
Senkrecht ist ein Konzept, für das es etliche Definitionen gibt. Keine der Definitionen ist die eine richtige. Alle sind sie äquivalent. Der gemeinsame Geist dahinter, ist das Konzept Senkrecht.

Bevor wir über Senkrecht reden können, bevor wir es untersuchen können, müssen wir es definieren. Die verschiedenen Vorstellungen von uns, was Senkrecht ist, kann helfen, eine Definition zu finden. Sie reicht aber für einen präzisen mathematischen Diskurs nicht aus.

Der sogenannte Satz des Pythagoras ist eine dieser Definitionen. Ein Dreieck wir dann senkrecht genannt, wenn der Satz des Pythagoras gilt.

Mit Hilfe des Rechnens mit Strecken und insbesondere dem Skalarprodukt kann die Äquivalenz der hier vorgestellten Definitionen gut gezeigt werden.

%******************************************************************************************************
\subsection*{Präsentiert}
%******************************************************************************************************
Von Jörg Kunze

%******************************************************************************************************
\subsection*{Voraussetzungen}
%******************************************************************************************************
Koordinaten-System, Satz des Pythagoras, Abstand im Koordinatensystem, Rechnen mit Buchstaben, Skalarprodukt, Rechnen mit Strecken

%******************************************************************************************************
\subsection*{Text}
%******************************************************************************************************
Der Begleittext als PDF und als LaTeX findet sich unter
{\tiny
   \url{https://github.com/kategory/kategoryMathematik/tree/main/v1%20Mittel/v1.7%20Klasse%207%20und%208/v1.7.6.4.3.5%20Senkrecht}
}

%******************************************************************************************************
\subsection*{Meine Videos}
%******************************************************************************************************
Siehe auch in den folgenden Videos:\\
\\
v1.7.6.4.2 (Mittel) Geometrie - Satz des Pythagoras\\
\url{https://youtu.be/mTMOtXfUQzQ}\\
\\
v1.7.6.4.3 (Mittel) Geometrie - Abstand im Koordinatensystem\\
\url{https://youtu.be/HEdolewfn78}
\\
\\v1.7.6.4.3.2 (Mittel) Abstand Gerade vom Nullpunkt
\url{https://youtu.be/_qN6Z-a72ok}
\\
\\v1.7.6.4.3.3 (Mittel) Geometrie - Skalarprodukt
\url{https://youtu.be/PMYHObLI54Q}
\\
\\v1.7.6.4.3.4 (Mittel) Geometrie - Rechnen mit Strecken
\url{https://youtu.be/hYRLizedcVM}

%******************************************************************************************************
\subsection*{Quellen}
%******************************************************************************************************
Siehe auch in den folgenden Seiten:\\
\\
\url{https://de.wikipedia.org/wiki/Orthogonalit%C3%A4t}\\
\url{https://de.wikipedia.org/wiki/Satz_des_Pythagoras}\\
\url{https://de.wikipedia.org/wiki/Euklidischer_Abstand}\\
\url{https://de.wikipedia.org/wiki/Skalarprodukt}\\
\url{https://www.schuelerhilfe.de/online-lernen/1-mathematik/719-skalarprodukt}\\
\url{https://de.wikipedia.org/wiki/Scheitelpunkt}

%******************************************************************************************************
\subsection*{Buch}
%******************************************************************************************************
Grundlage ist folgendes Buch:\\

"`KomplettWissen Mathematik Gymnasium Klasse 5-10"'\\
Klett Lerntraining bei PONS\\
978-3-12-926097-5 (ISBN)\\
{\tiny
   \url{https://www.lehmanns.de/shop/schulbuch-lexikon-woerterbuch/35031626-9783129260975-komplettwissen-mathematik-gymnasium-klasse-5-10
   }
}\\
\\
"`Grundlagen der ebenen Geometrie"'
Hendrik Kasten, Denis Vogel\\
Springer Berlin\\
978-3-662-57620-5 (ISBN)\\
{\tiny
   \url{
      https://www.lehmanns.de/shop/mathematik-informatik/43394438-9783662576205-grundlagen-der-ebenen-geometrie
   }
}

%******************************************************************************************************
\subsection*{Lizenz}
%******************************************************************************************************
Dieser Text und das Video sind freie Software. Sie können es unter den Bedingungen der 
GNU General Public License, wie von der Free Software Foundation veröffentlicht, weitergeben 
und/oder modifizieren, entweder gemäß Version 3 der Lizenz oder (nach Ihrer Option) jeder späteren Version.

Die Veröffentlichung von Text und Video erfolgt in der Hoffnung, dass es Ihnen von Nutzen sein wird, 
aber OHNE IRGENDEINE GARANTIE, sogar ohne die implizite Garantie der MARKTREIFE oder der 
VERWENDBARKEIT FÜR EINEN BESTIMMTEN ZWECK. Details finden Sie in der GNU General Public License.

Sie sollten ein Exemplar der GNU General Public License zusammen mit diesem Text erhalten haben 
(zu finden im selben Git-Projekt). 
Falls nicht, siehe \url{http://www.gnu.org/licenses/}.

\subsection*{Das Video}
%******************************************************************************************************
Das Video hierzu ist zu finden unter 
{\tiny
   \url{upps}
}

%******************************************************************************************************
%                                                                                                     *
\section{Senkrecht}
%                                                                                                     *
%******************************************************************************************************

%******************************************************************************************************
\subsection{Das Konzept Senkrecht}
%******************************************************************************************************
Wir stellen hier 5 äquivalente Definitionen des Begriffs Senkrecht vor. Zunächst ist jede Definition so gut wie die andere. Den gemeinsam sich dahinter stehenden Geist, das Wesen von Senkrecht, nenne ich Konzept Senkrecht.

\begin{enumerate}
   \item Der kürzeste Abstand eines Punktes zu einer Geraden liegt auf einer senkrechten Geraden.
   \item Zu einer Gerade mit Steigung $m$ ist eine Gerade mit Steigung $-1/m$ senkrecht (sofern $m \ne 0$).
   \item Zwei Seiten eines Dreiecks sind senkrecht, wenn der Satz des Pythagoras gilt (mit der dritten Seite als Hypotenuse).
   \item Zwei Strecken sind Senkrecht, wenn deren Skalarprodukt 0 ist.
   \item Eine Gerade ist senkrecht zu einer zweiten, wenn deren Punkte gleich weit weg von links und rechts sind.
\end{enumerate}

%******************************************************************************************************
\subsection{Die Definitionen sind äquivalent}
%******************************************************************************************************
%------------------------------------------------------------------------------------------------------
\subsubsection{Äquivalenz zu (1), minimaler Abstand}

Wir betrachten den Fall des Nullpunktes. Für andere Punkte geht alles genauso, nur die Formeln werden alle unübersichtlicher. Der Abstand des Nullpunktes zu einem Punkt auf der Geraden $y = mx + q$ mit $x$-Koordinate $x$ ist 
\begin{equation}
   \sqrt{(m^2+1)x^2 + 2mqx + q^2},
\end{equation}
wie wir z.~B. dem Video "`v1.7.6.4.3.2 (Mittel) Geometrie - Abstand einer Geraden vom Nullpunkt"' entnehmen können. Um das Minimum zu finden, genügt es den Ausdruck unter der Wurzel, der Abstand zum Quadrat zu minimieren.
Diesen betrachten wir als Funktion in $x$, also als Parabel, und schreiben sie nun um mit der quadratischen Ergänzung:
\begin{equation}
   \left( x \sqrt{m^2+1} + \frac{mq}{\sqrt{m^2+1}} \right)^2 - \frac{m^2q^2}{m^2+1} + q^2.
\end{equation}
Die Terme hinter dem Quadrat zusammenfassend erhalten wir schließlich die Scheitelpunktform der Parabel:
\begin{equation}
   \boxed{
      \left( x \sqrt{m^2+1} + \frac{mq}{\sqrt{m^2+1}} \right)^2 + \frac{q^2}{m^2+1}.
   }
\end{equation}
Das Minimum dieses Ausdrucks ist $\frac{q^2}{m^2+1}$ und es wird erreicht, wenn der Term im Quadrat $0$ wird. Das führt zu dem im obengenannten Video bereits gezeigten minimalen Abstand $\sqrt{\frac{q^2}{m^2+1}}$ angenommen an der Stelle $x = -\frac{mq}{m^2+1}$.

Anders gesagt ist
\begin{equation}
   \boxed{
      \Delta := \left( x \sqrt{m^2+1} + \frac{mq}{\sqrt{m^2+1}} \right)^2 .
   }
\end{equation}
der \textbf{Fehler} vom tatsächlichen zum minimalen Abstandquadrat.

Wir möchten nun den Ausdruck im Quadrat als Skalarprodukt darstellen. Dazu suchen wir zunächst eine Strecke auf der Geraden $y = mx+q$. Dazu nehmen wir die beiden Punkte an den Stellen $x=0$ und $x=1$ auf dieser Geraden. Es handelt sich um die Punkte $(0,q)$ und $(1,m+q)$. Wir bezeichnen die Strecke von
$(0,q)$ nach $(1,m+q)$ mit $\vec d$. Die Koordinatenversätze sind
\begin{alignat}{3}
   &d_x &&= 1\\
   &d_y &&= m.
\end{alignat}
Für die Länge dieser Strecke gilt nach unserer Definition des Pythagoras
\begin{equation}
   \boxed{|\vec d| = \sqrt{m^2+1}}.
\end{equation}
Der Punkt an der Stelle $x$ auf der Geraden hat die Koordinaten $(x, mx+q)$. Diese Strecke nennen wir $\vec a$, und da die Strecke bei $0$ anfängt, sind die Koordinatenversätze gleich den Koordinaten des zweiten Punktes:
\begin{alignat}{3}
   &a_x &&= x\\
   &a_y &&= mx+q.
\end{alignat}
Damit gilt
\begin{alignat}{4}
   &\vec a \cdot \vec d &&= x \cdot 1 + (mx+q) \cdot m\\
   &                    &&= x + m^2x+mq\\
   &                    &&= (m^2+1)x +mq.
\end{alignat}
Diesen letzten Ausdruck wollen wir mit dem Wert im Quadrat oben vergleichen und kommen auf:
\begin{equation}
   \sqrt{m^2+1} \left(x \sqrt{m^2+1} + \frac{mq}{\sqrt{m^2+1}} \right) = (m^2+1)x +mq.
\end{equation}
Schon jetzt sehen wir, dass das Skalarprodukt $0$ ist, genau dann wenn das Minimum des Abstandquadrates erreicht wird, wenn also die Fehler $\Delta$ zu $0$ wird.

Betrachten wir statt dessen die Strecke $\vec e$, deren Längen mit $1/\sqrt{m^2+1}$ multipliziert werden. Diese hat ihren ersten Punkt wieder den Achsenabschnitt der Geraden, also $(0,q)$. Der zweite Punkt ist allerdings an der Stelle $x=1/\sqrt{m^2+1}$ und ist damit der Punkt $(1/\sqrt{m^2+1}, m/\sqrt{m^2+1} + q)$ mit den Koordinatenversätzen
\begin{alignat}{3}
   &e_x &&= 1/\sqrt{m^2+1}\\
   &e_y &&= m/\sqrt{m^2+1}.
\end{alignat}
Die Länge dieser Strecke ist
\begin{alignat}{3}
   &(1/\sqrt{m^2+1})^2 + (m/\sqrt{m^2+1})^2 = 1/(m^2+1) + m^2/(m^2+1) = \\
   &(m^2 +1)/(m^2 +1) = 1.
\end{alignat}
Es handelt sich also um eine Strecke der Länge $1$, die auf der Geraden liegt.
Die Rechnung sparen ich mir. Es ergibt sich:

\begin{equation}
   \vec a \cdot \vec e = \left(x \sqrt{m^2+1} + \frac{mq}{\sqrt{m^2+1}} \right),
\end{equation}
und damit
\begin{equation}
   \boxed{
      \left( \vec a \cdot \vec e \right)^2 = \Delta.
   }
\end{equation}

\noindent\fbox{%
   \parbox{\textwidth}{%
      Das Quadrat des Skalarproduktes einer Strecke vom Null-Punkt zu einer Geraden mit einer Strecke der Länge $1$ auf der Geraden misst den Fehler zum minimalen Abstandsquadrat $\Delta$.
   }%
}

\begin{tikzpicture}[scale=3]
   \draw[->, thick] (0, -0.5) -- (0, 1.5) node(yaxis)[above]{$y$};
   \draw[->, thick] (-1, 0) -- (2, 0) node(xaxis)[right]{$x$};
   
   \draw (1, -0.05) node[below]{$1$} -- (1, 0.05);
   \draw (-0.05, 1) node[left]{$1$} -- (0.05, 1);
   
   \draw[blue] (-1,5/4) -- (2,-1/4);
   
   \coordinate (minimum) at (3/10,3/5);
   \fill (minimum) circle (0.03); 
   \draw[dashed] (minimum)  -- (0,0);
   
   \draw[fill=none](0,0) circle ( {sqrt(9/20)} ); 
   
   \fill[green] (3/4,3/8) coordinate (c) circle (0.03);

   \draw[black] (0,0) -- (c) -- (3/4+3/8,3/8-3/4) -- (3/8,-3/4) -- (0,0);

   \draw[green] (0,0) -- node[above] {$\vec a$} (c) -- (3/4+3/8,3/8-3/4) -- (3/8,-3/4) -- (0,0);
   
   \draw[->, thick, red] (c) -- node[above] {$\vec e$} (14/8,-1/8);
   
   \fill[red] ({sqrt(48/75)*3/4},{sqrt(48/75)*3/8}) coordinate (x) circle (0.03);

   \draw (x) -- 
      ({sqrt(48/75)*(3/4+3/8)},{sqrt(48/75)*(3/8-3/4)}) -- 
      ({sqrt(48/75)*3/8},{sqrt(48/75)*(-3/4)}) -- 
      (0,0);
      
   \fill[pattern=north west lines, pattern color=red] (x) -- 
      ({sqrt(48/75)*(3/4+3/8)},{sqrt(48/75)*(3/8-3/4)}) -- 
      ({sqrt(48/75)*3/8},{sqrt(48/75)*(-3/4)}) -- 
      (3/8,-3/4) --
      (3/4+3/8,3/8-3/4) --
      (c) --
      (x);
      
   \node[text width=3cm, color=red] at (1.5,-4/8) {$\Delta = \left( \vec a \cdot \vec e \right)^2$};
   
\end{tikzpicture}

%------------------------------------------------------------------------------------------------------
\subsubsection{Äquivalenz zu (2), Steigung ist $-1/m$}
Die Steigung einer Geraden ergibt sich aus beliebigen zwei verschiedenen Punkten $(x_0,y_0)$ und $(x_1, y_2)$ zu
\begin{equation}
   m = \frac{y_1 - y_0}{x_1 - x_0}.
\end{equation}
Für eine Strecke $\vec a$ auf der Geraden mit den Koordinatenversätzen $a_x$ und $a_y$ gilt demnach
\begin{equation}
   m = \frac{a_y}{a_x}.
\end{equation}
Für eine zweite Gerade $y = m'x+q'$ und einer Strecke $\vec b$ auf ihr gilt genauso
\begin{equation}
   m' = \frac{b_y}{b_x}.
\end{equation}
Damit haben wir
\begin{alignat}{4}
   &\vec a \cdot \vec b &&= 0       &&\Leftrightarrow\\
   &a_xb_x + a_yb_y      &&= 0       &&\Leftrightarrow\\
   &a_xb_x               &&= -a_yb_y &&\Leftrightarrow\\
   &\frac{a_x}{a_y}      &&= -\frac{b_y}{b_x} &&\Leftrightarrow\\
   &m                    &&= \frac{1}{-m'}
\end{alignat}
Bei der Umformung zur vierten Zeile haben wir durch $a_y$ und durch $b_x$ geteilt. Das funktioniert so nur, wenn diese beiden Werte nicht $0$ sind.

\noindent\fbox{%
   \parbox{\textwidth}{%
      Die Definition über die Steigung von Geraden funktioniert nicht für alle Strecken. Die des Skalarproduktes aber immer.
   }%
}

%------------------------------------------------------------------------------------------------------
\subsubsection{Äquivalenz zu (3), es gilt der Pythagoras}
Sei ein Dreieck gegeben mit den Seiten $\vec a, \vec b, \vec c$ und zwar so gerichtet, dass $\vec a + \vec b = \vec c$ gilt. Wie in dem Video "`v1.7.6.4.3.4 (Mittel) Geometrie - Rechnen mit Strecken"' beschrieben gilt dann 
\begin{equation}
   \vec c ^2 = (\vec a + \vec b)^2 = \vec a^2 + 2\vec a \vec b + \vec b^2.
\end{equation}
Dies ist die binomische Formel für das Rechnen mit Strecken. Wir sehen sofort, dass
\begin{equation}
   \vec c ^2 = \vec a^2 + \vec b^2 \Leftrightarrow 2\vec a \vec b = 0.
\end{equation}

%------------------------------------------------------------------------------------------------------
\subsubsection{Äquivalenz zu (4), gleich weit weg}
Bilden wir links und rechts von $\vec a$ ein Dreieck mit einer Strecke der Länge $1$ auf der Geraden, die wir $\vec e_-$ und $\vec e_+$ nennen, so erhalten wir, wegen $\vec e_- = -\vec e_+$
\begin{alignat}{5}
   &(\vec a - \vec e)^2 = (\vec c_-)^2\\
   &(\vec a + \vec e)^2 = (\vec c_+)^2
\end{alignat}
Ausmultiplizieren führt zu:
\begin{alignat}{5}
   &\vec a^2 -2 \vec a \cdot \vec e + \vec e^2 = (\vec c_-)^2\\
   &\vec a^2 +2 \vec a \cdot \vec e + \vec e^2 = (\vec c_+)^2
\end{alignat}
Hier können wir ablesen:
\begin{alignat}{5}
   &(\vec c_+)^2 = (\vec c_-)^2 \Leftrightarrow 2 \vec a \cdot \vec e = 0.
\end{alignat}

\begin{tikzpicture}[scale=3]
   \draw[->, thick] (0, -0.5) -- (0, 1.5) node(yaxis)[above]{$y$};
   \draw[->, thick] (-1, 0) -- (2, 0) node(xaxis)[right]{$x$};
   
   \draw (1, -0.05) node[below]{$1$} -- (1, 0.05);
   \draw (-0.05, 1) node[left]{$1$} -- (0.05, 1);
   
   \draw[blue] (-1,5/4) -- (2,-1/4);
   
   \coordinate (minimum) at (3/10,3/5);
   \fill (minimum) circle (0.03); 
   \draw[dashed] (minimum)  -- (0,0);
   
   \draw[fill=none](0,0) circle ( {sqrt(9/20)} ); 
   
   \fill[green] (3/4,3/8) coordinate (c) circle (0.03);
   
   \draw[->, thick, red] (c) -- node[above] {$\vec e_+$} (14/8,-1/8) coordinate (cp);
   \draw[->, thick, red] (c) -- node[above] {$\vec e_-$} (-2/8, 7/8) coordinate (cm);
   
   \draw[green] (0,0) -- node[above] {$\vec a$} (c);
    
   \draw[blue] (0,0) -- node[below] {$c_+$} (cp);
   \draw[blue] (0,0) -- node[left]  {$c_-$} (cm);
\end{tikzpicture}

%******************************************************************************************************
%                                                                                                     *
\section{TODO:}
%                                                                                                     *
%******************************************************************************************************
\begin{backup}
    (Zur Zeit nicht benötigter Inhalt)
\end{backup}

%******************************************************************************************************
%                                                                                                     *
\begin{thebibliography}{9}
%                                                                                                     *
%******************************************************************************************************

   \bibitem[Klett2016]{Klett}
      Klett Lerntraining, \emph{Komplett Wissen Mathematik Gymnasium 5-10},
      Klett Lerntraining, 978-3-12-926097-5 (ISBN)
      
   \bibitem[KastenVogel2018]{Kasten}
       Hendrik Kasten, Denis Vogel , \emph{Grundlagen der ebenen Geometrie},
      2018 Springer Berlin, 978-3-662-57620-5 (ISBN)

\end{thebibliography}

%******************************************************************************************************
%                                                                                                     *
\begin{large}
    \centerline{\textsc{Symbolverzeichnis}}
\end{large}
%                                                                                                     *
%******************************************************************************************************
\bigskip

\renewcommand*{\arraystretch}{1}

\begin{tabular}{ll}
    $\vec a,\vec b, \vec c, \cdots$               &Seiten\\
    $a_x,a_y$                   &Koordinatenversätze der Seite $a$\\
    $|a|$                       &Länge der Seite $a$\\
    $m$                         &Steigung der Gerade\\
    $q$                         &Achsenabschnitt der Gerade\\
    $x, y$                      &Koordinatenachsen\\
    $A, B, C, \cdots$          &Punkte\\
    $A_x, A_y$                 &Koordinaten des Punktes $A$\\
    $\overrightarrow{AB}$      &gerichtete Strecke von $A$ nach $B$\\
    $\perp$                    &Senkrecht
\end{tabular}

\end{document}
