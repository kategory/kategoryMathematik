%******************************************************** -*-LaTeX-*- ******************************
%                                                                                                  *
% v5.1.1.1.2.2 Gruppen-Darstellung = Gruppenring-Modul.tex                                         *
%                                                                                                  *
% Copyright (C) 2024 Kategory GmbH \& Co. KG (joerg.kunze@kategory.de)                             *
%                                                                                                  *
% v5.1.1.1.2.2 Gruppen-Darstellung = Gruppenring-Modul is part of kategoryMathematik.              *
%                                                                                                  *
% kategoryMathematik is free software: you can redistribute it and/or modify                       *
% it under the terms of the GNU General Public License as published by                             *
% the Free Software Foundation, either version 3 of the License, or                                *
% (at your option) any later version.                                                              *
%                                                                                                  *
% kategoryMathematik is distributed in the hope that it will be useful,                            *
% but WITHOUT ANY WARRANTY; without even the implied warranty of                                   *
% MERCHANTABILITY or FITNESS FOR A PARTICULAR PURPOSE.  See the                                    *
% GNU General Public License for more details.                                                     *
%                                                                                                  *
% You should have received a copy of the GNU General Public License                                *
% along with this program.  If not, see <http://www.gnu.org/licenses/>.                            *
%                                                                                                  *
%***************************************************************************************************

\documentclass[a4paper]{amsart}
% \documentclass[a4paper]{book}

%-----------------------------------------------------------------------------------------------------*
% package:                                                                                            *
%-----------------------------------------------------------------------------------------------------*
\usepackage{amssymb}
\usepackage{amsfonts}
\usepackage{amsmath}
\usepackage{amsthm}

\usepackage{mathabx}

\usepackage{a4wide} % a little bit smaller margins

\usepackage{graphicx}
\usepackage{hyperref}
\usepackage{algorithmic}
\usepackage{listings}
\usepackage{color}
\usepackage{colortbl}
\usepackage{sidecap}
\usepackage{comment}
\usepackage{tcolorbox}
\usepackage{collect}

\usepackage{upgreek}

% \usepackage{diagrams}

\usepackage[german]{babel}
\usepackage[none]{hyphenat}
\emergencystretch=4em

\usepackage[utf8]{inputenc} % to be able to use äöü as characters in text
\usepackage[T1]{fontenc} % to be able to use äöü in lables
\usepackage{lmodern}     % to avoid pixelation introduced by fontenc

\usepackage{hyperref}

\usepackage{tikz}
\usepackage{tikz-cd}
\usetikzlibrary{babel}

%-----------------------------------------------------------------------------------------------------*
% theorem:                                                                                            *
%-----------------------------------------------------------------------------------------------------*
\theoremstyle{definition}
\newtheorem{theorem}{Theorem}[subsection]

\newcommand{\myTheorem}[1]{%
  \newtheorem{jk#1}[theorem]{#1}
  \newenvironment{#1}[1]{%
    \expandafter\begin{jk#1} \expandafter\label{#1:##1}\textbf{(##1):}
  }{%
    \expandafter\end{jk#1}
  }
}

\myTheorem{Definition}
\myTheorem{Proposition}
\myTheorem{Satz}
\myTheorem{Theorem}
\myTheorem{Axiom}
\myTheorem{Beispiel}
\myTheorem{Anmerkung}

\definecollection{jkjkFrage}
\newtheorem{jkFrage}[theorem]{Frage}
\newenvironment{Frage}[1]{%
  \expandafter\begin{jkFrage} \expandafter\label{Frage:#1}\textbf{(#1):}
  \begin{collect}{jkjkFrage}{}{}
    \item \ref{Frage:#1} #1
  \end{collect}
}{%
  \expandafter\end{jkFrage}
}

\newcommand{\myRef}[2]{[#1 \ref{#1:#2}, ``#2'']}

\renewcommand{\proofname}{Beweis}

%-----------------------------------------------------------------------------------------------------*
% operator:                                                                                           *
%-----------------------------------------------------------------------------------------------------*
\DeclareMathOperator{\End}{End}
\DeclareMathOperator{\Ker}{Ker}
\DeclareMathOperator{\Mat}{Mat}
\DeclareMathOperator{\rank}{rank}
\DeclareMathOperator{\ggT}{ggT}
\DeclareMathOperator{\len}{len}
\DeclareMathOperator{\ord}{ord}
\DeclareMathOperator{\kgV}{kgV}
\DeclareMathOperator{\id}{id}
\DeclareMathOperator{\red}{red}
\DeclareMathOperator{\supp}{supp}
\DeclareMathOperator{\Bild}{Bild}
\DeclareMathOperator{\Rang}{Rang}
\DeclareMathOperator{\Det}{Det}
\DeclareMathOperator{\Hom}{Hom}
\DeclareMathOperator{\GL}{GL}

\DeclareMathOperator{\sub}{sub}
\DeclareMathOperator{\blk}{blk}
\DeclareMathOperator{\minimal}{minimal}
\DeclareMathOperator{\maximal}{maximal}

\definecolor{mygreen}{rgb}{0,0.6,0}
\definecolor{mygray}{rgb}{0.5,0.5,0.5}
\definecolor{mymauve}{rgb}{0.58,0,0.82}

\lstset{ %
  backgroundcolor=\color{white},   % choose the background color
  basicstyle=\ttfamily\footnotesize,        % size of fonts used for the code
  breaklines=true,                 % automatic line breaking only at whitespace
  captionpos=b,                    % sets the caption-position to bottom
  commentstyle=\color{mygreen},    % comment style
  escapeinside={\%*}{*)},          % if you want to add LaTeX within your code
  keywordstyle=\color{blue},       % keyword style
  stringstyle=\color{mymauve},     % string literal style
  frame=single
}

\setcounter{MaxMatrixCols}{20}

%******************************************************************************************************
%                                                                                                     *
% definition:                                                                                         *
%                                                                                                     *
%******************************************************************************************************
\newcommand{\R}{\ensuremath{\mathbb{ R }}}
\newcommand{\Q}{\ensuremath{\mathbb{ Q }}}
\newcommand{\Z}{\ensuremath{\mathbb{ Z }}}
\newcommand{\N}{\ensuremath{\mathbb{ N }}}
\newcommand{\C}{\ensuremath{\mathbb{ C }}}
\newcommand{\A}{\ensuremath{\mathbb{ A }}}
\newcommand{\F}{\ensuremath{\mathbb{ F }}}
\newcommand{\K}{\ensuremath{\mathbb{ K }}}
\newcommand{\Pb}{\ensuremath{\mathbb{ P }}}

\newcommand{\M}{\ensuremath{\mathcal{ M }}}
\newcommand{\V}{\ensuremath{\mathcal{ V }}}

\newcommand{\AAA}{\ensuremath{\mathcal{ A }}}
\newcommand{\BB}{\ensuremath{\mathcal{ B }}}
\newcommand{\CC}{\ensuremath{\mathcal{ C }}}
\newcommand{\DD}{\ensuremath{\mathcal{ D }}}
\newcommand{\EE}{\ensuremath{\mathcal{ E }}}
\newcommand{\FF}{\ensuremath{\mathcal{ F }}}
\newcommand{\KK}{\ensuremath{\mathcal{ K }}}
\newcommand{\MM}{\ensuremath{\mathcal{ M }}}
\newcommand{\PP}{\ensuremath{\mathcal{ P }}}
\newcommand{\ZZ}{\ensuremath{\mathcal{ Z }}}

\newcommand{\Set}{\text{\textbf{Set}} }


\newcommand{\imporant}[1]{ \textcolor{red}{\textbf{#1}} }

\newcommand{\bb}[1]{\mathbf{#1}}
\newcommand{\balpha}{\boldsymbol{\upalpha}}
\newcommand{\bbeta}{\boldsymbol{\upbeta}}
\newcommand{\bgamma}{\boldsymbol{\upgamma}}
\newcommand{\bdelta}{\boldsymbol{\delta}}
\newcommand{\bmu}{\boldsymbol{\upmu}}

\newcommand{\z}[1]{\Z_{#1}}
\newcommand{\e}[1]{\z{#1}^*}
\newcommand{\q}[1]{(\e{#1})^2}
\newcommand{\m}{\mathcal}

\newcommand{\zz}[1]{\ensuremath{\Z /#1\Z}}

\excludecomment{book}
\excludecomment{example}
\excludecomment{backup}

\newcommand{\zb}{z.~B. }

\begin{document}

%******************************************************************************************************
%                                                                                                     *
\begin{titlepage}
%                                                                                                     *
%******************************************************************************************************
% \vspace*{\fill}
\centering
{\huge
(Höhere Grundlagen) Homologische Algebra\\[1cm]
\textbf{v5.1.1.1.2.2 Gruppen-Darstellung = Gruppenring-Modul}
}\\[1cm]

\textbf{Kategory GmbH \& Co. KG}\\
Präsentiert von Jörg Kunze\\
Copyright (C) 2024 Kategory GmbH \& Co. KG

\end{titlepage}

%\clearpage
%\setcounter{page}{2}
%
%\tableofcontents

\newpage

%******************************************************************************************************
%                                                                                                     *
\section*{Beschreibung}
%                                                                                                     *
%******************************************************************************************************

%******************************************************************************************************
\subsection*{Inhalt}
%******************************************************************************************************
Eine Gruppen-Darstellung findet in gewisser Weise die Gruppe in der Gruppe der invertierbaren Matrizen mit Einträgen in einem kommutativen Ring $R$ wieder. $R$ kann auch ein Körper sein.

Genauer ist eine Gruppen-Darstellung ein Gruppen-Homomorphismus in die Gruppe der invertierbaren Matrizen mit Einträgen in $R$.

Die Gruppe der invertierbaren Matrizen mit Einträgen in einem Körper als Zielgruppe ist die allgemeine lineare Gruppe GL($V$), wo $V$ der entsprechende Vektorraum ist.

Der Gruppen-Ring $RG$ einer Gruppe $G$ über einem Ring $R$ ist der Polynomring mit Exponenten in der Gruppe und Koeffizienten im Ring $R$.

Die Gruppen-Darstellungen der Gruppe werden durch lineare Erweiterung zu Ring-Darstellungen des Gruppen-Rings, welche wiederum Moduln über dem Gruppen-Ring sind. 

Also: Gruppen-Darstellung ist Gruppenring-Modul.

%******************************************************************************************************
\subsection*{Präsentiert}
%******************************************************************************************************
Von Jörg Kunze

%******************************************************************************************************
\subsection*{Voraussetzungen}
%******************************************************************************************************
Ringe, Gruppen, lineare Algebra, Ring-Darstellung

%******************************************************************************************************
\subsection*{Text}
%******************************************************************************************************
Der Begleittext als PDF und als LaTeX findet sich unter
{\tiny
   \url{https://github.com/kategory/kategoryMathematik/tree/main/v5%20H%C3%B6here%20Grundlagen/v5.1%20Homologische%20Algebra/v5.1.1.1.2.2%20Gruppen-Darstellung%20%3D%20Gruppenring-Modul}
}

%******************************************************************************************************
\subsection*{Meine Videos}
%******************************************************************************************************
Siehe auch in den folgenden Videos:\\
\\
v5.1.1.1 (Höher) Homologische Algebra - Moduln\\
\url{https://youtu.be/JY43_07kNmA}
\\
v5.1.1.1.2 (Höher) Homologische Algebra - Ring-Darstellung = Modul\\
\url{https://youtu.be/FBQAdljFVF4}

%******************************************************************************************************
\subsection*{Quellen}
%******************************************************************************************************
Siehe auch in den folgenden Seiten:\\
\url{https://de.wikipedia.org/wiki/Darstellung_(Gruppe)}\\
\url{https://de.wikipedia.org/wiki/Polynomring}\\
\url{https://de.wikipedia.org/wiki/Monoidring}

%******************************************************************************************************
\subsection*{Buch}
%******************************************************************************************************
Grundlage ist folgendes Buch:\\
"`An Introduction to Homological Algebra"'\\
Joseph J. Rotman\\
2009\\
Springer-Verlag New York Inc.\\
978-0-387-24527-0 (ISBN)\\
{\tiny
   \url{https://www.lehmanns.de/shop/mathematik-informatik/6439666-9780387245270-an-introduction-to-homological-algebra}}\\
\\
Oft zitiert:\\
"`An Introduction to Homological Algebra"'\\
Charles A. Weibel\\
1995\\
Cambridge University Press\\
978-0-521-55987-4 (ISBN)\\
{\tiny
   \url{https://www.lehmanns.de/shop/mathematik-informatik/510327-9780521559874-an-introduction-to-homological-algebra}}\\
\\
Ohne Kategorien-Theorie:\\
"`Algébre 10. Algèbre homologique"'\\
Nicolas Bourbaki\\
1980\\
Springer-Verlag \\
978-3-540-34492-6 (ISBN)\\
{\tiny
   \url{https://www.lehmanns.de/shop/mathematik-informatik/7416782-9783540344926-algebre}}

%******************************************************************************************************
\subsection*{Lizenz}
%******************************************************************************************************
Dieser Text und das Video sind freie Software. Sie können es unter den Bedingungen der
GNU General Public License, wie von der Free Software Foundation veröffentlicht, weitergeben
und/oder modifizieren, entweder gemäß Version 3 der Lizenz oder (nach Ihrer Option) jeder späteren Version.

Die Veröffentlichung von Text und Video erfolgt in der Hoffnung, dass es Ihnen von Nutzen sein wird,
aber OHNE IRGENDEINE GARANTIE, sogar ohne die implizite Garantie der MARKTREIFE oder der
VERWENDBARKEIT FÜR EINEN BESTIMMTEN ZWECK. Details finden Sie in der GNU General Public License.

Sie sollten ein Exemplar der GNU General Public License zusammen mit diesem Text erhalten haben
(zu finden im selben Git-Projekt).
Falls nicht, siehe \url{http://www.gnu.org/licenses/}.

\subsection*{Das Video}
%******************************************************************************************************
Das Video hierzu ist zu finden unter
{\tiny
   \url{uups}
}

%******************************************************************************************************
%                                                                                                     *
\section{v5.1.1.1.2.2 Gruppen-Darstellung = Gruppenring-Modul}
%                                                                                                     *
%******************************************************************************************************

%******************************************************************************************************
\subsection{Polynom-Ring vs Monoid-Ring}
%******************************************************************************************************
Die folgende Konstruktion entsteht bei einer genauen Analyse dessen, was wir bei einem Polynom-Ring mit Koeffizienten in $R$ machen. Hier ist das verwendete Monoid einfach nur $\N$. Bei einem Polynomring über $\Z$ haben wir sowohl als Koeffizienten als auch als Exponenten einfache Zahlen, was deren Rollen nicht so deutlich getrennt aussehen lässt. 

$x$ wird im Folgenden als Träger einer Schreibweise verwendet, die an Variablen und Unbekannte erinnern soll. Seien ein Monoid (oder Gruppe) $G$ und ein Ring $R$ gegeben. Für jedes $g \in G$ und $r \in R$ definieren wir das \textbf{Monom}
\begin{equation}
   rx^g.
\end{equation}
Wenn $e$ das neutrale Element des Monoids $G$ ist, schreiben wir $x = x^e$.
Die Elemente des zu definierenden \textbf{Monoid-Ring}s $RG$ sind die Summen mit $n \in \N, r_i \in R$ und $g_i \in G$.
\begin{equation}\label{polynomRingElement}
   \sum_{i=1}^n r_i x^{g_i}
\end{equation}
Es handelt sich also um endliche Linearkombinationen von Potenzen mit Exponenten in $G$ und Koeffizienten in $R$.

Addition und Multiplikation werden nun schrittweise eingeführt: Monome können zusammengefasst werden, wenn sie den selben Exponenten haben.
\begin{equation}\label{AdditionMonom}
   r_1x^g + r_2x^g := (r_1 + r_2)x^g.
\end{equation}
Monome werden multipliziert, indem die Koeffizienten im Ring multipliziert und die Exponenten im Monoid verknüpft werden.
\begin{equation}\label{MultiplikationMonom}
   r_1x^{g_1} \cdot r_2x^{g_2} := r_1r_2x^{g_1\asterisk g_2}.
\end{equation}

Additive Kombinationen von Monomen werden addiert, indem Monome mit dem selben Exponenten zusammengefasst werden.

Für die Multiplikation von Elementen des Polynom-Rings wird das Distributiv-Gesetz verlangt. Damit kann die Multiplikation von Elementen abgeleitet werden: ausmultiplizieren (Distributiv-Gesetz anwenden) und zusammenfassen. Das ergibt im Ergebnis:

\begin{equation}\label{MultiplikationPolynom}
   \left(\sum_{i=1}^n r_i x^{g_i}\right) \cdot \left(\sum_{j=1}^m s_j x^{h_j}\right) :=
   \sum_{z \in \left\{g_i \asterisk h_j \big| \substack{i = 1, \cdots n, \\j = 1, \cdots m} \right\} } 
   \left( \sum_{\substack{i,j \text{ mit } \\g_i \asterisk h_j = z}} r_i s_j \right) x^z.
\end{equation} 
Also die Summe über alle Exponenten, die bei den Kombinaten der Monome entstehen können, und für jeden Exponenten die Summe der Produkte der Koeffizienten, deren Exponenten diesen Exponenten erzeugen.
Die Formel sieht schlimmer aus als sie ist. Wenn wir von Hand ausmultiplizieren und zusammenfassen, entsteht aber genau das.

%******************************************************************************************************
\subsection{Gruppen-Ring}
%******************************************************************************************************
Sei $R$ ein Ring und $G$ eine Gruppe. Der \textbf{Gruppen-Ring} $RG$ ist der Monoid-Ring $RG$, bei dem $G$ die Rolle des Monoids der Exponenten übernimmt.

%******************************************************************************************************
\subsection{Gruppen-Darstellung}
%******************************************************************************************************
Eine Gruppen-Darstellung ist ein Gruppen-Homomorphismus in die Gruppe der invertierbaren  $(n \times n)$-Matrizen mit Einträgen in $R$. Diese invertierbaren Matrizen sind lineare Automorphismen von $R^n$.

%******************************************************************************************************
\subsection{Schreibweise}
%******************************************************************************************************
Wenn bei einem Gruppen-Ring die Elemente der Gruppe und die des Rings nicht verwechselt werden können schreiben wir auch kürzer:
\begin{equation}\label{multiplyByRing}
   rg := rx^g.
\end{equation}

Ist $\phi \colon G \to Mat_n(R)$, so schreiben wir für ein Element \( g \in G \) und \( \vec r \in R^n \)
\begin{equation}\label{gvruppeAlsMal}
   g \vec r := \phi(g)(\vec r).
\end{equation}
Manchmal bleiben wir auch bei der Exponentialschreibweise
\begin{equation}\label{gruppeAlsHoch}
   \vec r^{\,g} := \phi(g)(\vec r).
\end{equation}

%******************************************************************************************************
\subsection{Gruppenaktion \( \rightarrow \) Ring-Darstellung}
%******************************************************************************************************
Die obige Schreibweise unterstützt die Sichtweise, dass Vermöge der Darstellung die Gruppenelemente auf die Elemente von \( R^n \) wirken. Die Elemente des Gruppen-Rings sind bei dieser Sichtweise Linearkombinationen von Gruppen-Aktionen.

Ein Polynom kann dann auf \( \vec r \in R^n \) angewandt werden:
\begin{equation}\label{polynomAlsFunktion}
   \sum_{i=1}^n r_i x^{g_i}
\end{equation}
ist die Funktion mit
\begin{equation}\label{polynomAktion}
   \boxed{
      \left( \sum_{i=1}^n r_i x^{g_i} \right) (\vec r) := \sum_{i=1}^n r_i \vec r^{\,g_i} := \sum_{i=1}^n r_i \phi(g_i)(\vec r).
   }
\end{equation}
d.~h. wir wenden erst alle \( g_i \) auf das $\vec r$ an und linearkombinieren daraufhin die Ergebnisse mit den Koeffizienten \( r_i \).

Es ist leicht zu sehen, dass die so entstehenden Funktionen Endomorphismen von \( R^n \) sind. Damit haben wir eine Darstellung des Gruppen-Rings.

%******************************************************************************************************
\subsection{Erweiterung der Skalare}
%******************************************************************************************************
Die Gruppen-Elemente können auch als neue Skalare angesehen werden, mit denen wir die Elemente von \( R^n \) multiplizieren können. Wenn wir mit den originalen und den neuen Skalaren gemischt rechnen und dabei die Rechenregeln von Ringen anwenden, entstehen genau die Elemente des Gruppen-Rings als neuer Skalarbereich.

%******************************************************************************************************
\subsection{Gruppen-Darstellung \( \rightarrow \) Gruppenring-Modul}
%******************************************************************************************************
Über den obigen Zusammenhang, wissen wir nun, dass zu einer Gruppen-Darstellung ein Gruppen-Ring gehört, welcher die Gruppen-Aktion zu einer Ring-Darstellung erweitert. Wir wissen bereits, dass zu einer Ring-Darstellung eines Rings \( R \) ein \( R \)-Modul gehört. Damit haben wir zu jeder Gruppen-Darstellung ein \( RG \)-Modul.

Die Skalar-Multiplikation ist dabei durch \eqref{polynomAktion} gegeben.

\begin{backup}
%******************************************************************************************************
%                                                                                                     *
\section{TODO}
%                                                                                                     *
%******************************************************************************************************
\begin{itemize}
     \item Überprüfe Symbolverzeichnis
\end{itemize}


\end{backup}

\begin{backup}
    (Zur Zeit nicht benötigter Inhalt)
\end{backup}

%******************************************************************************************************
%                                                                                                     *
\begin{thebibliography}{9}
%                                                                                                     *
%******************************************************************************************************
  \bibitem[Rotman2009]{Rotman}
   	Joseph J. Rotman, \emph{An Introduction to Homological Algebra},
   	2009 Springer-Verlag New York Inc., 978-0-387-24527-0 (ISBN)

   \bibitem[Bourbaki1970]{A1-3}
      Nicolas Bourbaki, \emph{Algébre 1-3},
      2006 Springer-Verlag, 978-3-540-33849-9 (ISBN)

   \bibitem[Bourbaki1980]{A10}
      Nicolas Bourbaki, \emph{Algébre 10. Algèbre homologique},
      2006 Springer-Verlag, 978-3-540-34492-6 (ISBN)

   \bibitem[MacLane1978]{MacLane}
      Saunders Mac Lane, \emph{Categories for the Working Mathematician},
      Springer-Verlag New York Inc., 978-0-387-98403-2 (ISBN)

\end{thebibliography}

%******************************************************************************************************
%                                                                                                     *
\begin{large}
    \centerline{\textsc{Symbolverzeichnis}}
\end{large}
%                                                                                                     *
%******************************************************************************************************
\bigskip

\renewcommand*{\arraystretch}{1}

\begin{tabular}{ll}
    $R$                                 & Ein kommutativer Ring mit Eins\\
    $G$                                 & Eine Gruppe, nicht notwendig abelsch\\
    $\asterisk$                         & Verknüpfung der Gruppe $G$\\
    $n\Z$                               & Das Ideal aller Vielfachen von $n$ in $\Z$\\
    $\zz{n}$                            & Der Restklassenring modulo $n$\\
    $\K$                                & Ein Körper\\
    $\vec x, \vec y$                    & Elemente eines Moduls, wenn wir sie von den Skalaren abheben wollen. \\
                                        & Entspricht in etwa Vektoren\\
    $\vec r$                            & Element von $R^n$\\
    $\phi$                              & Gruppen-Homomorphismus
    
\end{tabular}

\end{document}
