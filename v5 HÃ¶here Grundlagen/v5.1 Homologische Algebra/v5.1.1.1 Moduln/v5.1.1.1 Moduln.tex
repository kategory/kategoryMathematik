%******************************************************** -*-LaTeX-*- ******************************
%                                                                                                  *
% v5.1.1.1 Moduln.tex                                                                              *
%                                                                                                  *
% Copyright (C) 2024 Kategory GmbH \& Co. KG (joerg.kunze@kategory.de)                             *
%                                                                                                  *
% v5.1.1.1 Moduln is part of kategoryMathematik.                                                   *
%                                                                                                  *
% kategoryMathematik is free software: you can redistribute it and/or modify                       *
% it under the terms of the GNU General Public License as published by                             *
% the Free Software Foundation, either version 3 of the License, or                                *
% (at your option) any later version.                                                              *
%                                                                                                  *
% kategoryMathematik is distributed in the hope that it will be useful,                            *
% but WITHOUT ANY WARRANTY; without even the implied warranty of                                   *
% MERCHANTABILITY or FITNESS FOR A PARTICULAR PURPOSE.  See the                                    *
% GNU General Public License for more details.                                                     *
%                                                                                                  *
% You should have received a copy of the GNU General Public License                                *
% along with this program.  If not, see <http://www.gnu.org/licenses/>.                            *
%                                                                                                  *
%***************************************************************************************************

\documentclass[a4paper]{amsart}
% \documentclass[a4paper]{book}

%-----------------------------------------------------------------------------------------------------*
% package:                                                                                            *
%-----------------------------------------------------------------------------------------------------*
\usepackage{amssymb}
\usepackage{amsfonts}
\usepackage{amsmath}
\usepackage{amsthm}

\usepackage{mathabx}

\usepackage{a4wide} % a little bit smaller margins

\usepackage{graphicx}
\usepackage{hyperref}
\usepackage{algorithmic}
\usepackage{listings}
\usepackage{color}
\usepackage{colortbl}
\usepackage{sidecap}
\usepackage{comment}
\usepackage{tcolorbox}
\usepackage{collect}

\usepackage{upgreek}

% \usepackage{diagrams}

\usepackage[german]{babel}
\usepackage[none]{hyphenat}
\emergencystretch=4em

\usepackage[utf8]{inputenc} % to be able to use äöü as characters in text
\usepackage[T1]{fontenc} % to be able to use äöü in lables
\usepackage{lmodern}     % to avoid pixelation introduced by fontenc

\usepackage{hyperref}

\usepackage{tikz}
\usepackage{tikz-cd}
\usetikzlibrary{babel}

%-----------------------------------------------------------------------------------------------------*
% theorem:                                                                                            *
%-----------------------------------------------------------------------------------------------------*
\theoremstyle{definition}
\newtheorem{theorem}{Theorem}[subsection]

\newcommand{\myTheorem}[1]{%
  \newtheorem{jk#1}[theorem]{#1}
  \newenvironment{#1}[1]{%
    \expandafter\begin{jk#1} \expandafter\label{#1:##1}\textbf{(##1):}
  }{%
    \expandafter\end{jk#1}
  }
}

\myTheorem{Definition}
\myTheorem{Proposition}
\myTheorem{Satz}
\myTheorem{Theorem}
\myTheorem{Axiom}
\myTheorem{Beispiel}
\myTheorem{Anmerkung}

\definecollection{jkjkFrage}
\newtheorem{jkFrage}[theorem]{Frage}
\newenvironment{Frage}[1]{%
  \expandafter\begin{jkFrage} \expandafter\label{Frage:#1}\textbf{(#1):}
  \begin{collect}{jkjkFrage}{}{}
    \item \ref{Frage:#1} #1
  \end{collect}
}{%
  \expandafter\end{jkFrage}
}

\newcommand{\myRef}[2]{[#1 \ref{#1:#2}, ``#2'']}

\renewcommand{\proofname}{Beweis}

%-----------------------------------------------------------------------------------------------------*
% operator:                                                                                           *
%-----------------------------------------------------------------------------------------------------*
\DeclareMathOperator{\End}{End}
\DeclareMathOperator{\Ker}{Ker}
\DeclareMathOperator{\Mat}{Mat}
\DeclareMathOperator{\rank}{rank}
\DeclareMathOperator{\ggT}{ggT}
\DeclareMathOperator{\len}{len}
\DeclareMathOperator{\ord}{ord}
\DeclareMathOperator{\kgV}{kgV}
\DeclareMathOperator{\id}{id}
\DeclareMathOperator{\red}{red}
\DeclareMathOperator{\supp}{supp}
\DeclareMathOperator{\Bild}{Bild}
\DeclareMathOperator{\Rang}{Rang}
\DeclareMathOperator{\Det}{Det}
\DeclareMathOperator{\Hom}{Hom}
\DeclareMathOperator{\GL}{GL}

\DeclareMathOperator{\sub}{sub}
\DeclareMathOperator{\blk}{blk}
\DeclareMathOperator{\minimal}{minimal}
\DeclareMathOperator{\maximal}{maximal}

\definecolor{mygreen}{rgb}{0,0.6,0}
\definecolor{mygray}{rgb}{0.5,0.5,0.5}
\definecolor{mymauve}{rgb}{0.58,0,0.82}

\lstset{ %
  backgroundcolor=\color{white},   % choose the background color
  basicstyle=\ttfamily\footnotesize,        % size of fonts used for the code
  breaklines=true,                 % automatic line breaking only at whitespace
  captionpos=b,                    % sets the caption-position to bottom
  commentstyle=\color{mygreen},    % comment style
  escapeinside={\%*}{*)},          % if you want to add LaTeX within your code
  keywordstyle=\color{blue},       % keyword style
  stringstyle=\color{mymauve},     % string literal style
  frame=single
}

\setcounter{MaxMatrixCols}{20}

%******************************************************************************************************
%                                                                                                     *
% definition:                                                                                         *
%                                                                                                     *
%******************************************************************************************************
\newcommand{\R}{\ensuremath{\mathbb{ R }}}
\newcommand{\Q}{\ensuremath{\mathbb{ Q }}}
\newcommand{\Z}{\ensuremath{\mathbb{ Z }}}
\newcommand{\N}{\ensuremath{\mathbb{ N }}}
\newcommand{\C}{\ensuremath{\mathbb{ C }}}
\newcommand{\A}{\ensuremath{\mathbb{ A }}}
\newcommand{\F}{\ensuremath{\mathbb{ F }}}
\newcommand{\K}{\ensuremath{\mathbb{ K }}}
\newcommand{\Pb}{\ensuremath{\mathbb{ P }}}

\newcommand{\M}{\ensuremath{\mathcal{ M }}}
\newcommand{\V}{\ensuremath{\mathcal{ V }}}

\newcommand{\AAA}{\ensuremath{\mathcal{ A }}}
\newcommand{\BB}{\ensuremath{\mathcal{ B }}}
\newcommand{\CC}{\ensuremath{\mathcal{ C }}}
\newcommand{\DD}{\ensuremath{\mathcal{ D }}}
\newcommand{\EE}{\ensuremath{\mathcal{ E }}}
\newcommand{\FF}{\ensuremath{\mathcal{ F }}}
\newcommand{\KK}{\ensuremath{\mathcal{ K }}}
\newcommand{\MM}{\ensuremath{\mathcal{ M }}}
\newcommand{\PP}{\ensuremath{\mathcal{ P }}}
\newcommand{\ZZ}{\ensuremath{\mathcal{ Z }}}

\newcommand{\Set}{\text{\textbf{Set}} }


\newcommand{\imporant}[1]{ \textcolor{red}{\textbf{#1}} }

\newcommand{\bb}[1]{\mathbf{#1}}
\newcommand{\balpha}{\boldsymbol{\upalpha}}
\newcommand{\bbeta}{\boldsymbol{\upbeta}}
\newcommand{\bgamma}{\boldsymbol{\upgamma}}
\newcommand{\bdelta}{\boldsymbol{\delta}}
\newcommand{\bmu}{\boldsymbol{\upmu}}

\newcommand{\z}[1]{\Z_{#1}}
\newcommand{\e}[1]{\z{#1}^*}
\newcommand{\q}[1]{(\e{#1})^2}
\newcommand{\m}{\mathcal}

\excludecomment{book}
\excludecomment{example}
\excludecomment{backup}

\newcommand{\zb}{z.~B. }

\begin{document}

%******************************************************************************************************
%                                                                                                     *
\begin{titlepage}
%                                                                                                     *
%******************************************************************************************************
% \vspace*{\fill}
\centering
{\huge
(Höhere Grundlagen) Homologische Algebra\\[1cm]
\textbf{v5.1.1.1 Moduln}
}\\[1cm]

\textbf{Kategory GmbH \& Co. KG}\\
Präsentiert von Jörg Kunze\\
Copyright (C) 2024 Kategory GmbH \& Co. KG

\end{titlepage}

%\clearpage
%\setcounter{page}{2}
%
%\tableofcontents

\newpage

%******************************************************************************************************
%                                                                                                     *
\section*{Beschreibung}
%                                                                                                     *
%******************************************************************************************************

%******************************************************************************************************
\subsection*{Inhalt}
%******************************************************************************************************
Moduln sind die prototypischen Beispiele in der homologischen Algebra. Genauer sind es die Kategorien der R-Moduln, für jeden kommutativen Ring mit Eins eine Kategorie. Der eigentliche Schauplatz der homologischen Algebra sind die abelschen Kategorien. So dass wir sagen können, die homologische Algebra ist die Untersuchung von abelschen Kategorien.

Die Kategorien der R-Moduln sind abelsche Kategorien: welche Eigenschaften das sind und der Nachweis ist Teil dieses Kurses.

R-Moduln sind Vektorräume, in denen wir statt einen Körper nur einen kommutativen Ring mit Eins voraussetzen. Das heißt wir haben eine kommutative Gruppe (wir sagen eine additive Gruppe) mit einer Skalarmultiplikation, welche assoziativ und distributiv ist und wo die Multiplikation mit Eins nichts macht.

Die simple Tatsache, dass wir nicht mehr zu jedem Skalar ein multiplikativ Inverses (noch $0 \ne 1$) fordern, hat umfangreiche Konsequenzen für die entstehenden Phänomene. Es gibt erheblich schrägere Phänomene als bei den Vektorräumen.

Auf der anderen Seite funktioniert ein großer Teil der linearen Algebra auch in R-Moduln, so dass man lineare Algebra da studieren sollte, mit Anmerkungen, welche Sätze nur in Vektorräumen gelten. So verfährt \zb "`Algèbre"' Chapitre 2 von Nicolas Bourbaki \cite{A1-3}

Phänomene sind beispielsweise
\begin{itemize}
   \item Der Ring $R$ kann Nullteiler haben
   \item In einer linearen Gleichung können wir nicht einfach durch einen Koeffizienten teilen, um ihn zu $1$ zu machen
   \item Echte Teilmengen von $R$, die aber ungleich dem Null-Modul $\{0\}$ sind, können $R$-Moduln sein
   \item Ein $R$-Modul kann stärker aufgewickelt sein als $R$ selber (in dem Sinne, indem der Körper $\Z / 7 \Z$ aufgewickelt ist.) So ist \zb $\Z / 6 \Z$ ein $\Z$-Modul, obwohl $\Z$ selber gar nicht aufgewickelt ist.
   \item Es gibt $R$-Moduln, die keine Basis haben.
\end{itemize}

Da abelsche Gruppen das selbe ist wie $\Z$-Moduln, sind $R$-Moduln gleichzeitig die Verallgemeinerung von Vektorräumen und abelschen Gruppen.

%******************************************************************************************************
\subsection*{Präsentiert}
%******************************************************************************************************
Von Jörg Kunze

%******************************************************************************************************
\subsection*{Voraussetzungen}
%******************************************************************************************************
Kategorien-Theorie, Ringe, Gruppen, lineare Algebra

%******************************************************************************************************
\subsection*{Text}
%******************************************************************************************************
Der Begleittext als PDF und als LaTeX findet sich unter
{\tiny
   \url{}
}

%******************************************************************************************************
\subsection*{Meine Videos}
%******************************************************************************************************
Siehe auch in den folgenden Videos:\\
\\
v5.0.1.0.1 (Höher) Kategorien - Axiome für Kategorien\\
\url{https://youtu.be/X8v5Kyly0KI}\\
\\
v5.0.1.0.2 (Höher) Kategorien - Kategorien\\
\url{https://youtu.be/sIaKt-Wxlog}\\
\\
v5.0.1.0.3 (Höher) Kategorien - Funktoren\\
\url{https://youtu.be/Ojf5LQGeyOU}\\
\\
v5.0.1.0.6 (Höher) Kategorien - Mathematische Grundlagen\\
\url{https://youtu.be/ezW54mnzHMw}

%******************************************************************************************************
\subsection*{Quellen}
%******************************************************************************************************
Siehe auch in den folgenden Seiten:\\
\url{https://de.wikipedia.org/wiki/Abelsche_Kategorie}\\
\url{https://de.wikipedia.org/wiki/Homologische_Algebra}\\
\url{https://de.wikipedia.org/wiki/Modul_(Mathematik)}\\
\url{https://de.wikipedia.org/wiki/Vektorraum}\\
\url{}\\
\url{}\\
\url{}\\
\url{}\\
\url{}\\
\url{}\\


%******************************************************************************************************
\subsection*{Buch}
%******************************************************************************************************
Grundlage ist folgendes Buch:\\
"`Categories for the Working Mathematician"'\\
Saunders Mac Lane\\
1998 | 2nd ed. 1978\\
Springer-Verlag New York Inc.\\
978-0-387-98403-2 (ISBN)\\
{\tiny
   \url{https://www.amazon.de/Categories-Working-Mathematician-Graduate-Mathematics/dp/0387984038}}\\

Gut für die kategorische Sichtweise ist:\\
"`Topology, A Categorical Approach"'\\
Tai-Danae Bradley\\
2020 MIT Press\\
978-0-262-53935-7 (ISBN)\\
{\tiny
\url{https://www.lehmanns.de/shop/mathematik-informatik/52489766-9780262539357-topology}}\\

Einige gut Erklärungen finden sich auch in den Einführenden Kapitel von:\\
"`An Introduction to Homological Algebra"'\\
Joseph J. Rotman\\
2009 Springer-Verlag New York Inc.\\
978-0-387-24527-0 (ISBN)\\
{\tiny \url{https://www.lehmanns.de/shop/mathematik-informatik/6439666-9780387245270-an-introduction-to-homological-algebra}}\\

Etwas weniger umfangreich und weniger tiefgehend aber gut motivierend ist:
"`Category Theory"'\\
Steve Awodey\\
2010 Oxford University Press\\
978-0-19-923718-0 (ISBN)\\
{\tiny\url{https://www.lehmanns.de/shop/mathematik-informatik/9478288-9780199237180-category-theory}}\\

Mit noch weniger Mathematik und die Konzepte motivierend ist:
"`Conceptual Mathematics: a First Introduction to Categories"'\\
F. William Lawvere, Stephen H. Schanuel\\
2009 Cambridge University Press\\
978-0-521-71916-2 (ISBN)\\
{\tiny\url{https://www.lehmanns.de/shop/mathematik-informatik/8643555-9780521719162-conceptual-mathematics}}

%******************************************************************************************************
\subsection*{Lizenz}
%******************************************************************************************************
Dieser Text und das Video sind freie Software. Sie können es unter den Bedingungen der
GNU General Public License, wie von der Free Software Foundation veröffentlicht, weitergeben
und/oder modifizieren, entweder gemäß Version 3 der Lizenz oder (nach Ihrer Option) jeder späteren Version.

Die Veröffentlichung von Text und Video erfolgt in der Hoffnung, dass es Ihnen von Nutzen sein wird,
aber OHNE IRGENDEINE GARANTIE, sogar ohne die implizite Garantie der MARKTREIFE oder der
VERWENDBARKEIT FÜR EINEN BESTIMMTEN ZWECK. Details finden Sie in der GNU General Public License.

Sie sollten ein Exemplar der GNU General Public License zusammen mit diesem Text erhalten haben
(zu finden im selben Git-Projekt).
Falls nicht, siehe \url{http://www.gnu.org/licenses/}.

\subsection*{Das Video}
%******************************************************************************************************
Das Video hierzu ist zu finden unter
{\tiny
   \url{XXX}
}

%******************************************************************************************************
%                                                                                                     *
\section{Mathematische Grundlagen der Kategorien}
%                                                                                                     *
%******************************************************************************************************

%******************************************************************************************************
\subsection{Kleine Kategorien}
%******************************************************************************************************

%------------------------------------------------------------------------------------------------------
\subsubsection{Quasiordnung}
\begin{Definition}{Quasiordnung}
   Eine \textbf{Quasiorndung} oder \textbf{Präordnung} ist eine Menge $C$ zusammen mit einer Relation $\le$, so dass für alle $x, y, z \in C$ gilt
   \begin{alignat}{1}
      &x \le x\\
      &x \le y \land y \le z \Rightarrow x \le z
   \end{alignat}
\end{Definition}

Nun definieren wir für alle $x, y, z \in C$
\begin{equation}
   \Hom(x,y) := \begin{cases}
      \{ (x,y) \} & \text{ falls } x \le y\\
      \emptyset & \text{ sonst }
   \end{cases}
\end{equation}
mit $(y,z) \circ (x,y) := (x,z)$. Dies macht die Quasiordnung zu einer Kategorie. Wählen wir $C$ aus unserem Universum, ist $C$ eine kleine Kategorie, da dann die Menge der Objekte (gleich der Menge der Elemente) und die Menge der Morphismen (gleich der Menge der Paare $(x,y)$ mit $x \le y$) klein sind.

%------------------------------------------------------------------------------------------------------
\subsubsection{Monoid}
Sei $M$ ein Monoid dessen Menge von Elementen, die wir hier mit $|M|$ bezeichnen, klein ist, also ein Element des Universums ist. Wir bilden eine Kategorie mit nur einem Objekt. Es spielt keine Rolle, was wir als Objekt auswählen, solange die Wahl im Universum bleibt. Wir können $\emptyset$ oder auch $M$ selber nehmen. Wir nehmen $M$. Da wir nur ein Objekt $M$ haben gibt es nur eine einzige Hom-Menge, nämlich $\Hom( M, M )$. Wir definieren $\Hom( M, M ) := |M|$. Die Verknüpfung von Morphismen sei dabei die Multiplikation im Monoid. Das neutrale Element übernimmt die Rolle der Identität.

Damit ist das Monoid eine kleine Kategorie.

%------------------------------------------------------------------------------------------------------
\subsubsection{Graph}
Eine Kategorie ist ein gerichteter Graph mit mindestens einer Kante von jedem Knoten zu sich selber, auf dessen Menge der Kanten eine bestimmte Verknüpfung definiert ist.

Aus jedem gerichteten Graph kann durch Aufnahme der Verknüpfungen eine Kategorie erzeugt werden.

Ich gezeichnetes Diagramm entspricht einer endlichen Kategorie.

%******************************************************************************************************
\subsection{Vokabeln}
%******************************************************************************************************
Wählen wir ein, sagen wir mal Grothendieck-, Universum $U$, können wir folgende Namenshierarchie bilden:
\begin{itemize}
   \item $x$ kleine Menge: $x \in U$
   \item $\CC$ kleine Kategorie: $\operatorname{Obj}(\CC)$ und $\Hom(\CC)$ sind klein
   \item $\CC$ große Kategorie: die Objekte und Hom-Mengen sind klein
   \item $\CC$ Mengen-Kategorie: $\operatorname{Obj}(\CC)$ und $\Hom(\CC)$ sind Mengen 
   \item $\CC$ Klassen-Kategorie: $\operatorname{Obj}(\CC)$ und $\Hom(\CC)$ sind Klassen
   \item $\CC$ Meta-Kategorie: Objekte und Morphismen sind eigenständige Dinge einer Prädikatenlogik. Wir scheren uns nicht um deren Implementierung in ZFC
\end{itemize}
Die Namen Mengen-Kategorie und Klassen-Kategorie sind Eigenkreationen von mir. So wie wir in diesem Kurs auch Klassen-Relation und Klassen-Funktion sagen, wenn die Klasse der Paare, die die Relation oder Funktion ausmachen, nicht notwendig eine Menge ist. Bei \cite{MacLane} werden Klassen-Kategorie und Meta-Kategorie zusammengefasst zu Meta-Kategorie.

%******************************************************************************************************
\subsection{Große Kategorien}
%******************************************************************************************************
\begin{itemize}
   \item Ab: kleine ablesche Gruppen
   \item Ring: kleine Ringe 
   \item Gruppe: kleine Gruppen
   \item R-Moduln: kleine R-Moduln 
   \item Top: kleine topologische Räume mit stetigen Funktionen 
   \item Toph: kleine topologische Räume mit Homotopie-Klassen von stetigen Funktionen
   \item Set: kleine Mengen mit Funktionen
   \item Rel: kleine Mengen mit Relationen
\end{itemize}

Hier wird jeweils gefordert, dass die Grundmengen klein sind. Die Menge der Morphismen und im Falle der topologischen Räume die Topologie (als Menge der offenen Mengen) sind dann aufgrund der Abgeschlossenheit eines Universums auch klein.

%******************************************************************************************************
\subsection{Konkrete Kategorien}
%******************************************************************************************************
\begin{Definition}{Konkrete Kategorie}
   Ein Paar $(\CC, V)$ aus einer Kategorie $\CC$ und einem treuen Funktor $V \colon \CC \to \Set$ heißt \textbf{konkrete Kategorie}. Dabei ist ein Funktor \textbf{treu}, wenn er injektiv auf Hom-Mengen operiert. Der Funktor $V$ heißt \textbf{Vergiss-Funktor}. Eine Kategorie heißt \textbf{konkretisierbar}, falls es einen Vergiss-Funktor wie oben gibt.
\end{Definition}

Beachte: es wird nicht die Injektivität der Abbildung der Objekte verlangt. Das heißt, für zwei Objekte $X \ne Y$ aus $\CC$ kann durchaus $V(X) = V(Y)$ gelten. Der folgende Satz zeigt, dass wir dies stets heilen können.

\begin{Satz}{konkrete Kategorien sind Unter-Kategorien von Set}
   $\CC$ ist konkretisierbar genau dann, wenn sie isomorph zu einer Unter-Kategorie von \Set ist. 
\end{Satz}
\begin{proof}
   Idee aus\\
   {\tiny\url{https://math.stackexchange.com/questions/3317487/is-every-concretizable-category-equivalent-to-a-subcategory-of-the-category-of-s}}
   
   Sei $(\CC, V)$ eine konkrete Kategorie. Wir wollen einen neuen Funktor $V'$ bauen. Die Zielmengen sollen nun eindeutig sein. Um dies zu erreichen wollen wir $X$ als neues Element in $V(X)$ aufnehmen. 
   
   $V(X) \cup \{X\}$ könnte aber schiefgehen, da $X$ schon in $V(X)$ liegen könnte. Also brauchen wir ein Element, was in einer Form $X$ enthält und nicht in $V(X)$ enthalten sein kann. Wir definieren
   \begin{equation}
      V'(X) := V(X) \cup \{ (X, V(X)) \}
   \end{equation}
   
   Jetzt müssen wir noch zeigen, dass wenn $X \ne Y$, nicht doch $V'(X) = V'(Y)$ entstehen könnte, weil $\{ (X, V(X)) \} \in Y$ und $\{ (Y, V(Y)) \} \in X$. Wir arbeiten mit der Definition $(a,b) := \{\{a\}, \{a, b\}\}$. Mit dem Fall, den wir ausschließen wollen, könnten wir damit eine zyklische Element-Kette bilden ($x_1 \in x_2 \in \cdots \in x_1$ ), welches gegen das Fundierungs-Axiom aus ZFC verstößt. Aus dem selben Grund kann übrigens auch $(X, V(X))$ nicht in $X$ liegen.
   
   Die Bilder der Morphismen $f \colon X \to Y$ werden so definiert, das sie auf den Elementen von $V(X)$ die Werte von $V(f)$ bekommen und auf dem neuen Element das neue:

   \begin{equation}
     V'(f)(x) := \begin{cases}
        f(x)      & \text{falls } x \in X\\
        (Y, V(Y)) & \text{falls } x = (X, V(X)).
     \end{cases}
   \end{equation}

   Dass Unter-Kategorien von \Set konkretisierbar sind, ist klar, da der Einbettungs-Funktor hier die Rolle des Vergiss-Funktors übernimmt. 
\end{proof}

\begin{backup}
%******************************************************************************************************
%                                                                                                     *
\section{TODO}
%                                                                                                     *
%******************************************************************************************************
\begin{itemize}
     \item Überprüfe Symbolverzeichnis
\end{itemize}


\end{backup}

\begin{backup}
    (Zur Zeit nicht benötigter Inhalt)
\end{backup}

%******************************************************************************************************
%                                                                                                     *
\begin{thebibliography}{9}
%                                                                                                     *
%******************************************************************************************************
  \bibitem[Rotman2009]{Rotman}
   	Joseph J. Rotman, \emph{An Introduction to Homological Algebra},
   	2009 Springer-Verlag New York Inc., 978-0-387-24527-0 (ISBN)

   \bibitem[Bourbaki1970]{A1-3}
      Nicolas Bourbaki, \emph{Algébre 1-3},
      2006 Springer-Verlag, 978-3-540-33849-9 (ISBN)

   \bibitem[Bourbaki1980]{A10}
      Nicolas Bourbaki, \emph{Algébre 10. Algèbre homologique},
      2006 Springer-Verlag, 978-3-540-34492-6 (ISBN)
      
   \bibitem[MacLane1978]{MacLane}
      Saunders Mac Lane, \emph{Categories for the Working Mathematician},
      Springer-Verlag New York Inc., 978-0-387-98403-2 (ISBN)

\end{thebibliography}

%******************************************************************************************************
%                                                                                                     *
\begin{large}
    \centerline{\textsc{Symbolverzeichnis}}
\end{large}
%                                                                                                     *
%******************************************************************************************************
\bigskip

\renewcommand*{\arraystretch}{1}

\begin{tabular}{ll}
    $P(x)$                              & ein Prädikat\\
    $A, B, C, \cdots, X, Y, Z$          & Objekte\\
    $F,G$                               & Funktoren\\
    $V, V'$                             & Vergiss-Funktoren\\
    $f, g, h, r, s, \cdots$             & Homomorphismen\\
    $\mathcal C, \mathcal D, \mathcal E, \cdots$ & Kategorien\\
    \textbf{Set}                        & Die Kategorie der Mengen\\
    $\Hom( X, Y)$                       & Die Menge der Homomorphismen von $X$ nach $Y$\\
    $\alpha, \beta, \cdots$             & natürliche Transformationen oder Ordinalzahlen\\
    $\mathcal C ^{\text{op}}$           & Duale Kategorie\\
    $\DD^\CC$                           & Funktorkategorie\\
    $\textbf{Ring}, \textbf{Gruppe}$    & Kategorie der Ringe und der Gruppen\\
    $U, U', U''$                        & Universen\\
    $V_\alpha$                          & eine Menge der Von-Neumann-Hierarchie

\end{tabular}

\end{document}
