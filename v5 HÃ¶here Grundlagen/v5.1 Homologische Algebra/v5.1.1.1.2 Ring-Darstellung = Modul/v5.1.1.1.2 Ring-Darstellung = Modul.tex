%******************************************************** -*-LaTeX-*- ******************************
%                                                                                                  *
% v5.1.1.1.2 Ring-Darstellung = Modul.tex                                                          *
%                                                                                                  *
% Copyright (C) 2024 Kategory GmbH \& Co. KG (joerg.kunze@kategory.de)                             *
%                                                                                                  *
% v5.1.1.1.2 Ring-Darstellung = Modul is part of kategoryMathematik.                               *
%                                                                                                  *
% kategoryMathematik is free software: you can redistribute it and/or modify                       *
% it under the terms of the GNU General Public License as published by                             *
% the Free Software Foundation, either version 3 of the License, or                                *
% (at your option) any later version.                                                              *
%                                                                                                  *
% kategoryMathematik is distributed in the hope that it will be useful,                            *
% but WITHOUT ANY WARRANTY; without even the implied warranty of                                   *
% MERCHANTABILITY or FITNESS FOR A PARTICULAR PURPOSE.  See the                                    *
% GNU General Public License for more details.                                                     *
%                                                                                                  *
% You should have received a copy of the GNU General Public License                                *
% along with this program.  If not, see <http://www.gnu.org/licenses/>.                            *
%                                                                                                  *
%***************************************************************************************************

\documentclass[a4paper]{amsart}
% \documentclass[a4paper]{book}

%-----------------------------------------------------------------------------------------------------*
% package:                                                                                            *
%-----------------------------------------------------------------------------------------------------*
\usepackage{amssymb}
\usepackage{amsfonts}
\usepackage{amsmath}
\usepackage{amsthm}

\usepackage{mathabx}

\usepackage{a4wide} % a little bit smaller margins

\usepackage{graphicx}
\usepackage{hyperref}
\usepackage{algorithmic}
\usepackage{listings}
\usepackage{color}
\usepackage{colortbl}
\usepackage{sidecap}
\usepackage{comment}
\usepackage{tcolorbox}
\usepackage{collect}

\usepackage{upgreek}

% \usepackage{diagrams}

\usepackage[german]{babel}
\usepackage[none]{hyphenat}
\emergencystretch=4em

\usepackage[utf8]{inputenc} % to be able to use äöü as characters in text
\usepackage[T1]{fontenc} % to be able to use äöü in lables
\usepackage{lmodern}     % to avoid pixelation introduced by fontenc

\usepackage{hyperref}

\usepackage{tikz}
\usepackage{tikz-cd}
\usetikzlibrary{babel}

%-----------------------------------------------------------------------------------------------------*
% theorem:                                                                                            *
%-----------------------------------------------------------------------------------------------------*
\theoremstyle{definition}
\newtheorem{theorem}{Theorem}[subsection]

\newcommand{\myTheorem}[1]{%
  \newtheorem{jk#1}[theorem]{#1}
  \newenvironment{#1}[1]{%
    \expandafter\begin{jk#1} \expandafter\label{#1:##1}\textbf{(##1):}
  }{%
    \expandafter\end{jk#1}
  }
}

\myTheorem{Definition}
\myTheorem{Proposition}
\myTheorem{Satz}
\myTheorem{Theorem}
\myTheorem{Axiom}
\myTheorem{Beispiel}
\myTheorem{Anmerkung}

\definecollection{jkjkFrage}
\newtheorem{jkFrage}[theorem]{Frage}
\newenvironment{Frage}[1]{%
  \expandafter\begin{jkFrage} \expandafter\label{Frage:#1}\textbf{(#1):}
  \begin{collect}{jkjkFrage}{}{}
    \item \ref{Frage:#1} #1
  \end{collect}
}{%
  \expandafter\end{jkFrage}
}

\newcommand{\myRef}[2]{[#1 \ref{#1:#2}, ``#2'']}

\renewcommand{\proofname}{Beweis}

%-----------------------------------------------------------------------------------------------------*
% operator:                                                                                           *
%-----------------------------------------------------------------------------------------------------*
\DeclareMathOperator{\End}{End}
\DeclareMathOperator{\Ker}{Ker}
\DeclareMathOperator{\Mat}{Mat}
\DeclareMathOperator{\rank}{rank}
\DeclareMathOperator{\ggT}{ggT}
\DeclareMathOperator{\len}{len}
\DeclareMathOperator{\ord}{ord}
\DeclareMathOperator{\kgV}{kgV}
\DeclareMathOperator{\id}{id}
\DeclareMathOperator{\red}{red}
\DeclareMathOperator{\supp}{supp}
\DeclareMathOperator{\Bild}{Bild}
\DeclareMathOperator{\Rang}{Rang}
\DeclareMathOperator{\Det}{Det}
\DeclareMathOperator{\Hom}{Hom}
\DeclareMathOperator{\GL}{GL}

\DeclareMathOperator{\sub}{sub}
\DeclareMathOperator{\blk}{blk}
\DeclareMathOperator{\minimal}{minimal}
\DeclareMathOperator{\maximal}{maximal}

\definecolor{mygreen}{rgb}{0,0.6,0}
\definecolor{mygray}{rgb}{0.5,0.5,0.5}
\definecolor{mymauve}{rgb}{0.58,0,0.82}

\lstset{ %
  backgroundcolor=\color{white},   % choose the background color
  basicstyle=\ttfamily\footnotesize,        % size of fonts used for the code
  breaklines=true,                 % automatic line breaking only at whitespace
  captionpos=b,                    % sets the caption-position to bottom
  commentstyle=\color{mygreen},    % comment style
  escapeinside={\%*}{*)},          % if you want to add LaTeX within your code
  keywordstyle=\color{blue},       % keyword style
  stringstyle=\color{mymauve},     % string literal style
  frame=single
}

\setcounter{MaxMatrixCols}{20}

%******************************************************************************************************
%                                                                                                     *
% definition:                                                                                         *
%                                                                                                     *
%******************************************************************************************************
\newcommand{\R}{\ensuremath{\mathbb{ R }}}
\newcommand{\Q}{\ensuremath{\mathbb{ Q }}}
\newcommand{\Z}{\ensuremath{\mathbb{ Z }}}
\newcommand{\N}{\ensuremath{\mathbb{ N }}}
\newcommand{\C}{\ensuremath{\mathbb{ C }}}
\newcommand{\A}{\ensuremath{\mathbb{ A }}}
\newcommand{\F}{\ensuremath{\mathbb{ F }}}
\newcommand{\K}{\ensuremath{\mathbb{ K }}}
\newcommand{\Pb}{\ensuremath{\mathbb{ P }}}

\newcommand{\M}{\ensuremath{\mathcal{ M }}}
\newcommand{\V}{\ensuremath{\mathcal{ V }}}

\newcommand{\AAA}{\ensuremath{\mathcal{ A }}}
\newcommand{\BB}{\ensuremath{\mathcal{ B }}}
\newcommand{\CC}{\ensuremath{\mathcal{ C }}}
\newcommand{\DD}{\ensuremath{\mathcal{ D }}}
\newcommand{\EE}{\ensuremath{\mathcal{ E }}}
\newcommand{\FF}{\ensuremath{\mathcal{ F }}}
\newcommand{\KK}{\ensuremath{\mathcal{ K }}}
\newcommand{\MM}{\ensuremath{\mathcal{ M }}}
\newcommand{\PP}{\ensuremath{\mathcal{ P }}}
\newcommand{\ZZ}{\ensuremath{\mathcal{ Z }}}

\newcommand{\Set}{\text{\textbf{Set}} }


\newcommand{\imporant}[1]{ \textcolor{red}{\textbf{#1}} }

\newcommand{\bb}[1]{\mathbf{#1}}
\newcommand{\balpha}{\boldsymbol{\upalpha}}
\newcommand{\bbeta}{\boldsymbol{\upbeta}}
\newcommand{\bgamma}{\boldsymbol{\upgamma}}
\newcommand{\bdelta}{\boldsymbol{\delta}}
\newcommand{\bmu}{\boldsymbol{\upmu}}

\newcommand{\z}[1]{\Z_{#1}}
\newcommand{\e}[1]{\z{#1}^*}
\newcommand{\q}[1]{(\e{#1})^2}
\newcommand{\m}{\mathcal}

\newcommand{\zz}[1]{\ensuremath{\Z /#1\Z}}

\excludecomment{book}
\excludecomment{example}
\excludecomment{backup}

\newcommand{\zb}{z.~B. }

\begin{document}

%******************************************************************************************************
%                                                                                                     *
\begin{titlepage}
%                                                                                                     *
%******************************************************************************************************
% \vspace*{\fill}
\centering
{\huge
(Höhere Grundlagen) Homologische Algebra\\[1cm]
\textbf{v5.1.1.1.2 Ring-Darstellung = Modul}
}\\[1cm]

\textbf{Kategory GmbH \& Co. KG}\\
Präsentiert von Jörg Kunze\\
Copyright (C) 2024 Kategory GmbH \& Co. KG

\end{titlepage}

%\clearpage
%\setcounter{page}{2}
%
%\tableofcontents

\newpage

%******************************************************************************************************
%                                                                                                     *
\section*{Beschreibung}
%                                                                                                     *
%******************************************************************************************************

%******************************************************************************************************
\subsection*{Inhalt}
%******************************************************************************************************
Eine Ring-Darstellung eines Ringes $R$ ist ein Ring-Homomorphismus von $R$ in den Endomorphismus-Ring einer abelschen Gruppe. Wir finden auf diese Weise den Ring $R$ realisiert durch Gruppen-Endomorphismen. Ähnlich wie wir Gruppen als Matrixgruppen \zb über Körpern finden können.

Ring-Darstellung helfen uns Ringe besser zu verstehen. 

Umgekehrt dienen Unter-Ringe von Endomorphismus-Ringen abelscher Gruppen als Quelle für Ringe.

Ring-Darstellungen sind auch Quellen für $R$-Moduln, denn sie statten die zugrunde liegende abelsche Gruppe automatisch mit der Struktur eines $R$-Moduls aus.

Umgekehrt liefert jedes $R$-Modul eine Darstellung von $R$.

Also: Ring-Darstellung = $R$-Modul

%******************************************************************************************************
\subsection*{Präsentiert}
%******************************************************************************************************
Von Jörg Kunze

%******************************************************************************************************
\subsection*{Voraussetzungen}
%******************************************************************************************************
Kategorien-Theorie, Ringe, Gruppen, lineare Algebra

%******************************************************************************************************
\subsection*{Text}
%******************************************************************************************************
Der Begleittext als PDF und als LaTeX findet sich unter
{\tiny
   \url{https://github.com/kategory/kategoryMathematik/tree/main/v5%20H%C3%B6here%20Grundlagen/v5.1%20Homologische%20Algebra/v5.1.1.1%20Moduln}
}

%******************************************************************************************************
\subsection*{Meine Videos}
%******************************************************************************************************
Siehe auch in den folgenden Videos:\\
\\
v5.0.1 (Höher) Kategorien - Playlist\\
\url{https://www.youtube.com/playlist?list=PLqVqq9xKS5R-baIvTr9GnW0Pb8rlPig7S}

%******************************************************************************************************
\subsection*{Quellen}
%******************************************************************************************************
Siehe auch in den folgenden Seiten:\\
\url{https://math.stackexchange.com/questions/1980543/representations-of-rings}\\
\url{https://de.wikipedia.org/wiki/Ganzheitsring}\\
\url{https://de.wikipedia.org/wiki/Darstellungsring}\\
\url{}\\
\url{}\\
\url{}\\
\url{}\\
\url{}\\
\url{}\\
\url{}\\
\url{}\\
\url{}\\
\url{}\\
\url{}\\
\url{}\\
\url{}\\
\url{}\\
\url{}\\
\url{}\\

%******************************************************************************************************
\subsection*{Buch}
%******************************************************************************************************
Grundlage ist folgendes Buch:\\
"`An Introduction to Homological Algebra"'\\
Joseph J. Rotman\\
2009\\
Springer-Verlag New York Inc.\\
978-0-387-24527-0 (ISBN)\\
{\tiny
   \url{https://www.lehmanns.de/shop/mathematik-informatik/6439666-9780387245270-an-introduction-to-homological-algebra}}\\
\\
Oft zitiert:\\
"`An Introduction to Homological Algebra"'\\
Charles A. Weibel\\
1995\\
Cambridge University Press\\
978-0-521-55987-4 (ISBN)\\
{\tiny
   \url{https://www.lehmanns.de/shop/mathematik-informatik/510327-9780521559874-an-introduction-to-homological-algebra}}\\
\\
Ohne Kategorien-Theorie:\\
"`Algébre 10. Algèbre homologique"'\\
Nicolas Bourbaki\\
1980\\
Springer-Verlag \\
978-3-540-34492-6 (ISBN)\\
{\tiny
   \url{https://www.lehmanns.de/shop/mathematik-informatik/7416782-9783540344926-algebre}}

%******************************************************************************************************
\subsection*{Lizenz}
%******************************************************************************************************
Dieser Text und das Video sind freie Software. Sie können es unter den Bedingungen der
GNU General Public License, wie von der Free Software Foundation veröffentlicht, weitergeben
und/oder modifizieren, entweder gemäß Version 3 der Lizenz oder (nach Ihrer Option) jeder späteren Version.

Die Veröffentlichung von Text und Video erfolgt in der Hoffnung, dass es Ihnen von Nutzen sein wird,
aber OHNE IRGENDEINE GARANTIE, sogar ohne die implizite Garantie der MARKTREIFE oder der
VERWENDBARKEIT FÜR EINEN BESTIMMTEN ZWECK. Details finden Sie in der GNU General Public License.

Sie sollten ein Exemplar der GNU General Public License zusammen mit diesem Text erhalten haben
(zu finden im selben Git-Projekt).
Falls nicht, siehe \url{http://www.gnu.org/licenses/}.

\subsection*{Das Video}
%******************************************************************************************************
Das Video hierzu ist zu finden unter
{\tiny
   \url{https://youtu.be/JY43_07kNmA}
}

%******************************************************************************************************
%                                                                                                     *
\section{v5.1.1.1.2 Ring-Darstellung = Modul}
%                                                                                                     *
%******************************************************************************************************

%******************************************************************************************************
\subsection{Definition}
%******************************************************************************************************
Salopp: Ein $R$-Modul ist ein $R$-Vektorraum, wo $R$ ein Ring mit Eins aber nicht notwendig ein Körper ist. Es können also Inverse und die Kommutativität der Multiplikation fehlen.

%------------------------------------------------------------------------------------------------------
\begin{Definition}{R-Modul}
   Eine abelsche Gruppe $M$ (additiv geschrieben, also mit $+$ und $0$) zusammen mit einem  Ring mit Eins $R$ und einer Skalar- Multiplikation $R \times M \to M$ heißt $R$\textbf{-Modul}, wenn für alle $m,n \in M$ und $r, s \in R$ gilt
   \begin{alignat}{2}
      &(rs)m &&= r(sm)\\
      &(r+s)m &&= rm + rs\\
      &r(m+n) &&= rm +rn\\
      &1m &&= m
   \end{alignat}
\end{Definition}
Hier haben wir einen Links-Modul definiert. Da die Multiplikation nicht als kommutativ vorausgesetzt wird, müssen wir diese von Rechts-Moduln unterscheiden, wo wir die Skalare von rechts multiplizieren. 

Ein wichtiges Beispiel für nicht-kommutative Ringe sind Endomorphismen-Ringe von abelschen Gruppen.

%******************************************************************************************************
\subsection{Beispiele}
%******************************************************************************************************
\begin{enumerate}
   \item $R$ ist selbst $R$-Modul
   \item $\Z^2$ als \Z-Modul (mit Löchern)
   \item $2\Z$ als \Z-Modul (Echte Teilmenge von $R$)
   \item Ideal = Unter-$R$-Modul von $R$
   \item $\Q, \R$ sind $\Z$-Moduln (Achtung: es sind keine freien Moduln)
   \item \zz6 als \Z-Modul (stärker aufgewickelt als $R$)
   \item $(\zz6)^2$ als \zz6-Modul (Achtung Nullteiler)
   \item Freier $R$-Modul
   \item Das \Z-Modul $(\zz{12})^2$ ist nicht kürzbar
   \item $\K$-Vektorraum ist auch $\K$-Modul
\end{enumerate}

%-----------------------------------------------------------------------------------------------
\subsubsection{$R$ ist selbst $R$-Modul}
Die Ring-Axiome sind schon rein formel-technisch gleich zu den Modul-Axiomen. Nur, dass die Multiplikation eine interne ist, in dem Sinne, dass die Skalare die Elemente der additiven Gruppe selber sind.

%-----------------------------------------------------------------------------------------------
\subsubsection{$\Z^2$ als \Z-Modul (mit Löchern)}
Auf den Achsen gibt es nur die ganzzahligen Werte.
\begin{tikzpicture}
   \draw[->, thick] (0, -0.5) -- (0, 5) node(yaxis)[above]{$y$};
   \draw[->, thick] (-0.5, 0) -- (5, 0) node(xaxis)[right]{$x$};

   \draw (1, -0.05) node[below]{$1$} -- (1, 0.05);
   \draw (-0.05, 1) node[left]{$1$} -- (0.05, 1);

   \fill (1,1) circle (0.03);
   \fill (2,1) circle (0.03);
   \fill (3,1) circle (0.03);
   \fill (4,1) circle (0.03);
   \fill (5,1) circle (0.03);

   \fill (1,2) circle (0.03);
   \fill (2,2) circle (0.03);
   \fill (3,2) circle (0.03);
   \fill (4,2) circle (0.03);
   \fill (5,2) circle (0.03);

   \fill (1,3) circle (0.03);
   \fill (2,3) circle (0.03);
   \fill (3,3) circle (0.03);
   \fill (4,3) circle (0.03);
   \fill (5,3) circle (0.03);

   \fill (1,4) circle (0.03);
   \fill (2,4) circle (0.03);
   \fill (3,4) circle (0.03);
   \fill (4,4) circle (0.03);
   \fill (5,4) circle (0.03);

   \fill (1,5) circle (0.03);
   \fill (2,5) circle (0.03);
   \fill (3,5) circle (0.03);
   \fill (4,5) circle (0.03);
   \fill (5,5) circle (0.03);

\end{tikzpicture}


%-----------------------------------------------------------------------------------------------
\subsubsection{$2\Z$ als \Z-Modul (Echte Teilmenge von $R$)}
Dies ist bei Körpern nicht möglich. Da es auch keinen Null-Körper gibt, da hier immer $0 \ne 1$ gilt, ist der einzige $\K$-Vektorraum in $\K$ der ganze Körper $\K$.

%-----------------------------------------------------------------------------------------------
\subsubsection{Ideal = Unter-$R$-Modul von $R$}
Die Ideal-Axiome sind schon rein formel-technisch gleich zu den Modul-Axiomen. Nur, dass die additiven Gruppe eine Teilmenge der Skalare sind.

%-----------------------------------------------------------------------------------------------
\subsubsection{$\Q, \R$ sind $\Z$-Moduln (Achtung: es sind keine freien Moduln)}
Da es für alle $x,y \in \Q$ Werte $n_x, n_y \in \Z$ mit
\begin{equation}
   n_xx + n_yy = 0
\end{equation}
gibt, bestehen hier Relationen zwischen den Elementen.

%-----------------------------------------------------------------------------------------------
\subsubsection{\zz6 als \Z-Modul (stärker aufgewickelt als $R$)}
Obwohl es in \Z\ keine Nullteiler gibt, gibt es im \Z-Modul \zz6 ein Element welches mit einem Skalar multipliziert Null ergibt: $2 \cdot 3 = 0$. Hier ist $2 \in \Z$ aber $3 \in \zz6$.

%-----------------------------------------------------------------------------------------------
\subsubsection{$(\zz6)^2$ als \zz6-Modul (Achtung Nullteiler)}
Aufgrund der Nullteiler in \zz6 haben wir $3\vec{x} + 3\vec{x} = 0$ für alle $\vec x \in (\zz6)^2$.

%-----------------------------------------------------------------------------------------------
\subsubsection{Freier $R$-Modul}
$G$ ist die Menge der Generatoren (entspricht in etwa einer Basis)
\begin{equation}
   \left \{ \sum_{i \in I} r_i g_i \mid I \subseteq G, I \text{ endlich}, r_i \in R, g_i \in G \right \}
\end{equation}
in Worten: der $R$-Modul der endlichen Linearkombinationen von Elementen aus $G$ mit Koeffizienten aus $R$. Bei endlichem $G = \{0, \cdots, n-1 \}$ schreiben wir dafür auch
\begin{equation}
   R^n := R^{\{0, \cdots, n-1 \}},
\end{equation}
in welchem Fall wir die Elemente des Moduls als Tupel schreiben: $(r_0, r_1, r_2) \in R^3$.

Hier kann $R^2 \cong R$ gelten. D.~h. Begriffe wie Basis und Dimension stehen nicht oder zumindest zunächst nicht oder nicht so einfach zur Verfügung. So bei den Endomorphismen-Ringen. TODO: Beweis.



%-----------------------------------------------------------------------------------------------
\subsubsection{Das \Z-Modul $(\zz{12})^2$ ist nicht kürzbar}
Wir haben
\begin{equation}
   \begin{pmatrix} 1 \\ 3 \end{pmatrix} \ne
   \begin{pmatrix} 4 \\ 6 \end{pmatrix}
\end{equation}
und gleichzeitig
\begin{equation}
   4 \cdot \begin{pmatrix} 1 \\ 3 \end{pmatrix} =
   4 \cdot \begin{pmatrix} 4 \\ 6 \end{pmatrix} =
          \begin{pmatrix} 4 \\ 0 \end{pmatrix}.
\end{equation}

Das heißt, in diesem \Z-Modul gilt im allgemeinen nicht
\begin{equation}
   4 \vec x = 4 \vec y \Rightarrow \vec x = \vec y,
\end{equation}
obwohl das entsprechende innerhalb von \Z\ gilt.

%-----------------------------------------------------------------------------------------------
\subsubsection{$\K$-Vektorraum ist auch $\K$-Modul}
Da ein Körper unter anderem auch ein kommutativer Ring mit Eins ist, sind Vektorräume automatisch auch Moduln. Da abelsche Gruppen das gleiche wie \Z-Moduln sind, sind Moduln die gleichzeitige Verallgemeinerung von Vektorräumen und abelschen Gruppen.

\begin{backup}
%******************************************************************************************************
%                                                                                                     *
\section{TODO}
%                                                                                                     *
%******************************************************************************************************
\begin{itemize}
     \item Überprüfe Symbolverzeichnis
\end{itemize}


\end{backup}

\begin{backup}
    (Zur Zeit nicht benötigter Inhalt)
\end{backup}

%******************************************************************************************************
%                                                                                                     *
\begin{thebibliography}{9}
%                                                                                                     *
%******************************************************************************************************
  \bibitem[Rotman2009]{Rotman}
   	Joseph J. Rotman, \emph{An Introduction to Homological Algebra},
   	2009 Springer-Verlag New York Inc., 978-0-387-24527-0 (ISBN)

   \bibitem[Bourbaki1970]{A1-3}
      Nicolas Bourbaki, \emph{Algébre 1-3},
      2006 Springer-Verlag, 978-3-540-33849-9 (ISBN)

   \bibitem[Bourbaki1980]{A10}
      Nicolas Bourbaki, \emph{Algébre 10. Algèbre homologique},
      2006 Springer-Verlag, 978-3-540-34492-6 (ISBN)

   \bibitem[MacLane1978]{MacLane}
      Saunders Mac Lane, \emph{Categories for the Working Mathematician},
      Springer-Verlag New York Inc., 978-0-387-98403-2 (ISBN)

\end{thebibliography}

%******************************************************************************************************
%                                                                                                     *
\begin{large}
    \centerline{\textsc{Symbolverzeichnis}}
\end{large}
%                                                                                                     *
%******************************************************************************************************
\bigskip

\renewcommand*{\arraystretch}{1}

\begin{tabular}{ll}
    $R$                                 & ein kommutativer Ring mit Eins\\
    $n\Z$                               & das Ideal aller Vielfachen von $n$ in $\Z$\\
    $\zz{n}$                            & Der Restklassenring modulo $n$\\
    $\K$                                & Ein Körper\\
    $\vec x, \vec y$                    & Elemente des Moduls, wenn wir sie von den Skalaren abheben wollen. \\& Entspricht in etwa Vektoren
\end{tabular}

\end{document}
