%******************************************************** -*-LaTeX-*- ******************************
%                                                                                                  *
% v5.0.1.1.2 Kontravarianz und duale Kategorie.tex                                                 *
%                                                                                                  *
% Copyright (C) 2024 Kategory GmbH \& Co. KG (joerg.kunze@kategory.de)                             *
%                                                                                                  *
% v5.0.1.1.2 Kontravarianz und duale Kategorie is part of kategoryMathematik.                      *
%                                                                                                  *
% kategoryMathematik is free software: you can redistribute it and/or modify                       *
% it under the terms of the GNU General Public License as published by                             *
% the Free Software Foundation, either version 3 of the License, or                                *
% (at your option) any later version.                                                              *
%                                                                                                  *
% kategoryMathematik is distributed in the hope that it will be useful,                            *
% but WITHOUT ANY WARRANTY; without even the implied warranty of                                   *
% MERCHANTABILITY or FITNESS FOR A PARTICULAR PURPOSE.  See the                                    *
% GNU General Public License for more details.                                                     *
%                                                                                                  *
% You should have received a copy of the GNU General Public License                                *
% along with this program.  If not, see <http://www.gnu.org/licenses/>.                            *
%                                                                                                  *
%***************************************************************************************************

\documentclass[a4paper]{amsart}
% \documentclass[a4paper]{book}

%-----------------------------------------------------------------------------------------------------*
% package:                                                                                            *
%-----------------------------------------------------------------------------------------------------*
\usepackage{amssymb}
\usepackage{amsfonts}
\usepackage{amsmath}
\usepackage{amsthm}

\usepackage{mathabx}

\usepackage{a4wide} % a little bit smaller margins

\usepackage{graphicx}
\usepackage{hyperref}
\usepackage{algorithmic}
\usepackage{listings}
\usepackage{color}
\usepackage{colortbl}
\usepackage{sidecap}
\usepackage{comment}
\usepackage{tcolorbox}
\usepackage{collect}

\usepackage{upgreek}

% \usepackage{diagrams}

\usepackage[german]{babel}
\usepackage[none]{hyphenat}
\emergencystretch=4em

\usepackage[utf8]{inputenc} % to be able to use äöü as characters in text
\usepackage[T1]{fontenc} % to be able to use äöü in lables
\usepackage{lmodern}     % to avoid pixelation introduced by fontenc

\usepackage{hyperref}

\usepackage{tikz}
\usepackage{tikz-cd}
\usetikzlibrary{babel}

%-----------------------------------------------------------------------------------------------------*
% theorem:                                                                                            *
%-----------------------------------------------------------------------------------------------------*
\theoremstyle{definition}
\newtheorem{theorem}{Theorem}[subsection]

\newcommand{\myTheorem}[1]{%
  \newtheorem{jk#1}[theorem]{#1}
  \newenvironment{#1}[1]{%
    \expandafter\begin{jk#1} \expandafter\label{#1:##1}\textbf{(##1):}
  }{%
    \expandafter\end{jk#1}
  }
}

\myTheorem{Definition}
\myTheorem{Proposition}
\myTheorem{Satz}
\myTheorem{Theorem}
\myTheorem{Axiom}
\myTheorem{Beispiel}
\myTheorem{Anmerkung}

\definecollection{jkjkFrage}
\newtheorem{jkFrage}[theorem]{Frage}
\newenvironment{Frage}[1]{%
  \expandafter\begin{jkFrage} \expandafter\label{Frage:#1}\textbf{(#1):}
  \begin{collect}{jkjkFrage}{}{}
    \item \ref{Frage:#1} #1
  \end{collect}
}{%
  \expandafter\end{jkFrage}
}

\newcommand{\myRef}[2]{[#1 \ref{#1:#2}, ``#2'']}

\renewcommand{\proofname}{Beweis}

%-----------------------------------------------------------------------------------------------------*
% operator:                                                                                           *
%-----------------------------------------------------------------------------------------------------*
\DeclareMathOperator{\End}{End}
\DeclareMathOperator{\Ker}{Ker}
\DeclareMathOperator{\Mat}{Mat}
\DeclareMathOperator{\rank}{rank}
\DeclareMathOperator{\ggT}{ggT}
\DeclareMathOperator{\len}{len}
\DeclareMathOperator{\ord}{ord}
\DeclareMathOperator{\kgV}{kgV}
\DeclareMathOperator{\id}{id}
\DeclareMathOperator{\red}{red}
\DeclareMathOperator{\supp}{supp}
\DeclareMathOperator{\Bild}{Bild}
\DeclareMathOperator{\Rang}{Rang}
\DeclareMathOperator{\Det}{Det}
\DeclareMathOperator{\Hom}{Hom}
\DeclareMathOperator{\GL}{GL}

\DeclareMathOperator{\sub}{sub}
\DeclareMathOperator{\blk}{blk}
\DeclareMathOperator{\minimal}{minimal}
\DeclareMathOperator{\maximal}{maximal}

\DeclareMathOperator{\Dom}{Dom}
\DeclareMathOperator{\Cod}{Cod}
\DeclareMathOperator{\Obj}{Obj}

\definecolor{mygreen}{rgb}{0,0.6,0}
\definecolor{mygray}{rgb}{0.5,0.5,0.5}
\definecolor{mymauve}{rgb}{0.58,0,0.82}

\lstset{ %
  backgroundcolor=\color{white},   % choose the background color
  basicstyle=\ttfamily\footnotesize,        % size of fonts used for the code
  breaklines=true,                 % automatic line breaking only at whitespace
  captionpos=b,                    % sets the caption-position to bottom
  commentstyle=\color{mygreen},    % comment style
  escapeinside={\%*}{*)},          % if you want to add LaTeX within your code
  keywordstyle=\color{blue},       % keyword style
  stringstyle=\color{mymauve},     % string literal style
  frame=single
}

\setcounter{MaxMatrixCols}{20}

%******************************************************************************************************
%                                                                                                     *
% definition:                                                                                         *
%                                                                                                     *
%******************************************************************************************************
\newcommand{\R}{\ensuremath{\mathbb{ R }}}
\newcommand{\Q}{\ensuremath{\mathbb{ Q }}}
\newcommand{\Z}{\ensuremath{\mathbb{ Z }}}
\newcommand{\N}{\ensuremath{\mathbb{ N }}}
\newcommand{\C}{\ensuremath{\mathbb{ C }}}
\newcommand{\A}{\ensuremath{\mathbb{ A }}}
\newcommand{\F}{\ensuremath{\mathbb{ F }}}
\newcommand{\K}{\ensuremath{\mathbb{ K }}}
\newcommand{\Pb}{\ensuremath{\mathbb{ P }}}

\newcommand{\M}{\ensuremath{\mathcal{ M }}}
\newcommand{\V}{\ensuremath{\mathcal{ V }}}

\newcommand{\AAA}{\ensuremath{\mathcal{ A }}}
\newcommand{\BB}{\ensuremath{\mathcal{ B }}}
\newcommand{\CC}{\ensuremath{\mathcal{ C }}}
\newcommand{\DD}{\ensuremath{\mathcal{ D }}}
\newcommand{\EE}{\ensuremath{\mathcal{ E }}}
\newcommand{\FF}{\ensuremath{\mathcal{ F }}}
\newcommand{\KK}{\ensuremath{\mathcal{ K }}}
\newcommand{\MM}{\ensuremath{\mathcal{ M }}}
\newcommand{\PP}{\ensuremath{\mathcal{ P }}}
\newcommand{\ZZ}{\ensuremath{\mathcal{ Z }}}

\newcommand{\Set}{\text{\textbf{Set}}}
\newcommand{\Ab}{\text{\textbf{Ab}}}

\newcommand{\imporant}[1]{ \textcolor{red}{\textbf{#1}} }

\newcommand{\bb}[1]{\mathbf{#1}}
\newcommand{\balpha}{\boldsymbol{\upalpha}}
\newcommand{\bbeta}{\boldsymbol{\upbeta}}
\newcommand{\bgamma}{\boldsymbol{\upgamma}}
\newcommand{\bdelta}{\boldsymbol{\delta}}
\newcommand{\bmu}{\boldsymbol{\upmu}}

\newcommand{\z}[1]{\Z_{#1}}
\newcommand{\e}[1]{\z{#1}^*}
\newcommand{\q}[1]{(\e{#1})^2}
\newcommand{\m}{\mathcal}

\excludecomment{book}
\excludecomment{example}
\excludecomment{backup}

\newcommand{\zb}{z.~B. }

\begin{document}

%******************************************************************************************************
%                                                                                                     *
\begin{titlepage}
%                                                                                                     *
%******************************************************************************************************
% \vspace*{\fill}
\centering
{\huge
(Höhere Grundlagen) Kategorien\\[1cm]
\textbf{v5.0.1.1.2 Kontravarianz und duale Kategorie}
}\\[1cm]

\textbf{Kategory GmbH \& Co. KG}\\
Präsentiert von Jörg Kunze\\
Copyright (C) 2024 Kategory GmbH \& Co. KG

\end{titlepage}

%\clearpage
%\setcounter{page}{2}
%
%\tableofcontents

\newpage

%******************************************************************************************************
%                                                                                                     *
\section*{Beschreibung}
%                                                                                                     *
%******************************************************************************************************

%******************************************************************************************************
\subsection*{Inhalt}
%******************************************************************************************************
xxx

%******************************************************************************************************
\subsection*{Präsentiert}
%******************************************************************************************************
Von Jörg Kunze

%******************************************************************************************************
\subsection*{Voraussetzungen}
%******************************************************************************************************
Kategorie, Homomorphismus, Funktor, Hom-Mengen, Dual

%******************************************************************************************************
\subsection*{Text}
%******************************************************************************************************
Der Begleittext als PDF und als LaTeX findet sich unter
{\tiny
   \url{https://github.com/kategory/kategoryMathematik/tree/main/v5%20H%C3%B6here%20Grundlagen/v5.0.1%20Kategorien/v5.0.1.1.2%20Kontravarianz%20und%20duale%20Kategorie}
}

%******************************************************************************************************
\subsection*{Meine Videos}
%******************************************************************************************************
Siehe auch in den folgenden Videos:\\
\\
v5.0.1.0.1 (Höher) Kategorien - Axiome für Kategorien\\
\url{https://youtu.be/X8v5Kyly0KI}\\
\\
v5.0.1.0.2 (Höher) Kategorien - Kategorien\\
\url{https://youtu.be/sIaKt-Wxlog}\\
\\
v5.0.1.0.3 (Höher) Kategorien - Funktoren\\
\url{https://youtu.be/Ojf5LQGeyOU}\\
\\
v5.0.1.0.8 (Höher) Kategorien - Hom-Mengen\\
\url{https://youtu.be/bnMkDng-NnA}

%******************************************************************************************************
\subsection*{Quellen}
%******************************************************************************************************
Siehe auch in den folgenden Seiten:\\
\url{https://de.wikipedia.org/wiki/Duale_Kategorie}\\
\url{https://de.wikipedia.org/wiki/Funktor_(Mathematik)}\\
\url{https://de.wikipedia.org/wiki/Hom-Funktor}\\
\url{https://de.wikipedia.org/wiki/Potenzmengenfunktor}

%******************************************************************************************************
\subsection*{Buch}
%******************************************************************************************************
Grundlage ist folgendes Buch:\\
"`Categories for the Working Mathematician"'\\
Saunders Mac Lane\\
1998 | 2nd ed. 1978\\
Springer-Verlag New York Inc.\\
978-0-387-98403-2 (ISBN)\\
{\tiny
   \url{https://www.amazon.de/Categories-Working-Mathematician-Graduate-Mathematics/dp/0387984038}}\\

Gut für die kategorische Sichtweise ist:\\
"`Topology, A Categorical Approach"'\\
Tai-Danae Bradley\\
2020 MIT Press\\
978-0-262-53935-7 (ISBN)\\
{\tiny
\url{https://www.lehmanns.de/shop/mathematik-informatik/52489766-9780262539357-topology}}\\

Einige gut Erklärungen finden sich auch in den Einführenden Kapitel von:\\
"`An Introduction to Homological Algebra"'\\
Joseph J. Rotman\\
2009 Springer-Verlag New York Inc.\\
978-0-387-24527-0 (ISBN)\\
{\tiny \url{https://www.lehmanns.de/shop/mathematik-informatik/6439666-9780387245270-an-introduction-to-homological-algebra}}\\

Etwas weniger umfangreich und weniger tiefgehend aber gut motivierend ist:
"`Category Theory"'\\
Steve Awodey\\
2010 Oxford University Press\\
978-0-19-923718-0 (ISBN)\\
{\tiny\url{https://www.lehmanns.de/shop/mathematik-informatik/9478288-9780199237180-category-theory}}\\

Mit noch weniger Mathematik und die Konzepte motivierend ist:
"`Conceptual Mathematics: a First Introduction to Categories"'\\
F. William Lawvere, Stephen H. Schanuel\\
2009 Cambridge University Press\\
978-0-521-71916-2 (ISBN)\\
{\tiny\url{https://www.lehmanns.de/shop/mathematik-informatik/8643555-9780521719162-conceptual-mathematics}}

%******************************************************************************************************
\subsection*{Lizenz}
%******************************************************************************************************
Dieser Text und das Video sind freie Software. Sie können es unter den Bedingungen der
GNU General Public License, wie von der Free Software Foundation veröffentlicht, weitergeben
und/oder modifizieren, entweder gemäß Version 3 der Lizenz oder (nach Ihrer Option) jeder späteren Version.

Die Veröffentlichung von Text und Video erfolgt in der Hoffnung, dass es Ihnen von Nutzen sein wird,
aber OHNE IRGENDEINE GARANTIE, sogar ohne die implizite Garantie der MARKTREIFE oder der
VERWENDBARKEIT FÜR EINEN BESTIMMTEN ZWECK. Details finden Sie in der GNU General Public License.

Sie sollten ein Exemplar der GNU General Public License zusammen mit diesem Text erhalten haben
(zu finden im selben Git-Projekt).
Falls nicht, siehe \url{http://www.gnu.org/licenses/}.

\subsection*{Das Video}
%******************************************************************************************************
Das Video hierzu ist zu finden unter
{\tiny
   \url{XXX}
}

%******************************************************************************************************
%                                                                                                     *
\section{Kontravarianz und duale Kategorie}
%                                                                                                     *
%******************************************************************************************************

%******************************************************************************************************
\subsection{Duale Kategorie}
%******************************************************************************************************
\begin{Definition}{Duale Kategorie}
   Sei \CC\ eine Kategorie. Die dazu \textbf{duale Kategorie} (englisch \textbf{opposite category}, $\CC^\text{op}$) geschrieben als $\CC^*$ ist durch folgende Daten gegeben:
   \begin{itemize}
      \item Die Objekte von $\CC^*$ sind genau die Objekte von $\CC$
      \item Die Morphismen von $\CC^*$ sind genau die Morphismen von $\CC$. Ist $f \in \CC$ so schreiben wir $f^*$, wenn wir ihn als Morphismus in $\CC^*$ betrachten.
      \item $f^*$ geht in die entgegengesetzte Richtung von $f$. D.~h. wir definieren $\Cod(f^*) := \Dom(f)$ und $\Dom(f^*) := \Cod(f)$. (Mit Dom = Domain = Quelle und Cod = Codomain = Bild = Ziel). Mit anderen Worten $(f^* \colon Y \to X) \in \CC^*$, genau dann wenn $(f \colon X \to Y) \in \CC$. Mit ganz anderen Worten: Es sind die selben Morphismen, nur andersrum.
      \item Die Verknüpfung definieren wir über $f^* \circ g^* := (g \circ f)^*$.
   \end{itemize}
\end{Definition}   
Es ist leicht zu sehen, dass die Axiome einer Kategorie erfüllt sind.

\begin{Satz}{Dual der Konstruktion einer Kategorie ergibt deren duale}
   Sei $\Delta$ eine gültige Zeichenkette in ETAC, welche eine Kategorie $\CC$ definiert/konstruiert. Dann ist die duale Zeichenkette $\Delta^*$ eine Definition/Konstruktion der dualen Kategorie $\CC^*$.
\end{Satz}
\begin{proof}
   Dies ist insofern klar, als beide Definitionen von "`dual"' die Richtung der Pfeile umdrehen. Wenn wir das präzise beweisen wollen, müssen wir, glaube ich, das Dual einer Kategorie formalisieren und dann induktiv über den Aufbau gültiger Zeichenketten arbeiten.
\end{proof}

\begin{Satz}{In Dualer Kategorie gelten die dualen Sätze}
   Sei $\phi$ ein Prädikat, \CC\ eine Kategorie und gelte $\phi(\CC)$. Sei $\Phi$ die Zeichenkette aus ETAC, die $\phi$ formalisiert, $\Phi^*$ deren Dual und $\phi^*$ die Rücküberführung von $\Phi^*$ in unsere Sprache, dann gilt $\phi^*(\CC^*)$.
\end{Satz}
\begin{proof}
   Da dies ein Satz über die menschliche Sprache ist, können wir ihn nicht so ohne weiteres beweisen. Ich würde es so beweisen, dass wir, wie oben das  Dual einer Kategorie in ETAC formalisieren, dann folgenden Satz über ETAC beweisen: Falls "`$\Phi(\CC)$"' gilt/beweisbar ist dann gilt/ist beweisbar auch "`$\Phi^*(\CC^*)$"'. Danach bemühen wir wieder unser geheimes Axiom, dass Aussagen über ETAC auch in der echten Welt gelten.
\end{proof}

In diesem Sinne ist epi das selbe wie das Dual von mono, was wir schnell sehen, da die beiden Definitionen sich nur durch die Richtung der Pfeile unterscheiden. Wenn \zb in einer Kategorie \CC\ gilt, dass alle Morphismen mono sind, dann gilt in $\CC^*$, dass alle Morphismen epi sind.

%******************************************************************************************************
\subsection{Kontravarianter Funktor}
%******************************************************************************************************
\begin{Definition}{Kontravarianter Funktor via dual}
   Ein \textbf{kontravarianter Funktor} $F \colon \CC \to \DD$ ist ein Funktor $F \colon \CC^* \to \DD$. Funktoren heißen zur Abgrenzung gegenüber den kontravarianten Funktoren auch \textbf{kovariant}. 
\end{Definition}   

\begin{Definition}{Kontravarianter Funktor via Axiome}
   Ein \textbf{kontravarianter Funktor} sind zwei Abbildungen $F \colon \Obj( \CC ) \to \Obj( \DD )$ und $F \colon \Hom( \CC ) \to \Hom( \DD )$ zwischen den Klassen der Objekte und den Klassen der Morphismen, wobei dies Klassen-Abbildungen sein können, sodass für alle $C \in \Obj(\CC)$ und alle $f, g \in \Hom( \CC )$, für die wir $f \circ g$ bilden können gilt:
   \begin{alignat}{4}
      \text{ wir können } &F(g) \circ F(f) \quad &&\text{bilden}\\
      &F( f \circ g ) &&= F(g) \circ F(f)\\
      &F( \id_C) &&= \id_{F(C)}.
   \end{alignat} 
\end{Definition}
   
\begin{Satz}{Kontravariant via dual äquivalent Kontravariant via Axiome}
   Die beiden Definitionen oben führen zum selben Begriff von "`kontravarianter Funktor"'.
\end{Satz}
\begin{proof}
   Die Definition der Verknüpfung enthält genau die Vertauschung, die wir für die Verknüpfung bei den Axiomen benötigen.
\end{proof}

%******************************************************************************************************
\subsection{Kontravarianter Potenzmengen-Funktor}
%******************************************************************************************************
\begin{Definition}{Kontravarianter Potenzmengen-Funktor}
   Der \textbf{kontravarianter Potenzmengen-Funktor} $\PP \colon \Set \to \Set$ ist definiert auf den Objekten (in diesem Fall Mengen) als $\PP( X ) := $ Menge aller Teilmengen von $X$. Für ein $f \colon X \to Y$ definieren wir $\PP(f) \colon \PP(Y) \to \PP(X)$ über seine Wirkung auf Teilmengen $Y' \subseteq Y$ als
   \begin{equation}
      \PP(f)(Y') := f^{-1}(Y').
   \end{equation}
\end{Definition}   
Hier sehen wir, dass ein Funktor viel mehr ist als eine stupide Abbildung der Objekte. Er geht immer einher mit einer Abbildung der Morphismen und das ganze ist strukturverträglich. Die Potenzmengen-Bildung ist also mächtiger oder reicher als auf den ersten Blick zu erkennen ist.

Es gibt auch einen kovarianten Potenzmengen-Funktor, der sich nur in seiner Wirkung  auf die Morphismen unterscheidet.

%******************************************************************************************************
\subsection{Hom-Funktoren}
%******************************************************************************************************
\begin{Definition}{Hom-Funktor}
   Sei \CC\ eine Kategorie in der die Klassen $\Hom(X,Y)$ unabhängig von $X,Y$ immer Mengen sind (wir nennen eine solche Kategorie lokal klein). Dann definieren wir zu jedem $C \in \CC$ den \textbf{kovarianten Hom-Funktor} $\Hom(C, \_) \colon \CC \to \Set$ durch seine Wirkung auf alle $X,Y, f \colon X \to Y \in \CC$:
   \begin{alignat}{4}
      &\Hom(C, \_)( Y ) := &&\Hom(C, Y)\\
      &\Hom(C, \_)( f ) := &&\Hom(C, f ) \colon &&\Hom(C, X ) &&\to \Hom(C, Y )\\
      &                    &&                   &&h           &&\mapsto f \circ h.
   \end{alignat}
   Wir schreiben gerne prägnanter $h_C := \Hom(C, \_)$ und  $f_* := f \circ \_ := \Hom(C, f )$ schreiben.
   
   Ähnlich definieren wir den \textbf{kontravarianten Hom-Funktor} $\Hom(\_, C) \colon \CC \to \Set$ durch seine Wirkung auf alle $X,Y, f \colon X \to Y \in \CC$:
   \begin{alignat}{4}
      &\Hom(\_, C)( Y ) := &&\Hom(Y, C)\\
      &\Hom(\_, C)( f ) := &&\Hom( f, C ) \colon &&\Hom(Y, C ) &&\to \Hom(X, C )\\
      &                    &&                   &&h           &&\mapsto h \circ f.
   \end{alignat}
   Wir schreiben gerne prägnanter $h^C := \Hom(\_, C)$ und  $f^* := \_ \circ f := \Hom( f, C )$ schreiben.
\end{Definition}   
Diese Sterne haben allerdings nichts mit "`dual"' zu tun. Diese Definition für die Wirkung auf $f$ ist formeltechnisch unabhängig von $C$. Sie wirkt aber jeweils nur auf Morphismen, deren Start $C$ ist.

$f_* = f \circ \_$ drückt den Endpunkt von $h$ von $X$ in Richtung $f$ nach $Y$.
\begin{equation}
   \begin{tikzcd}
                         &C   \arrow[dl, "h"{above, name=0}] \arrow[dr, dashed, "f \circ h"{name=1}] \\
      X \arrow[rr, "f"{below}]  &        &Y
      \arrow[Rightarrow, bend right, from=0, to=1, "f \circ \_"{below}]
   \end{tikzcd}
\end{equation}


$f^* = \_ \circ f$ zieht den Startpunkt von $h$ von $Y$ gegen (kontra) die Richtung $f$ zurück nach $X$.
\begin{equation}
   \begin{tikzcd}
      X \arrow[rr, "f"{above}] \arrow[dr, dashed, "h \circ f"{left, name=1}] &   
                            &Y \arrow[dl, , "h"{right,name=0}]\\
      &C     \\
      \arrow[Rightarrow, bend right, from=0, to=1, "\_ \circ f"{below}]
   \end{tikzcd}
\end{equation}

%******************************************************************************************************
\subsection{Hom-Funktoren-Verträglichkeit}
%******************************************************************************************************
Haben wir die beiden Morphismen $f \colon X \to Y$ und $g \colon C \to D$ so gibt es zwei Wege von $\Hom(D,X)$ nach $\Hom(C,Y)$. Beide Wege liefern das selbe Ergebnis. Mit anderen Worten, $f_*$ und $g^*$ vertauschen oder, noch andere Worte, das folgende Diagramm ist kommutativ.
\begin{equation}
   \begin{tikzcd}
      \Hom(D,X)   \arrow[d, "g^*"]  \arrow[r, "f_*"] & \Hom(D,Y) \arrow[d, "g^*"]\\
      \Hom(C,X)                     \arrow[r, "f_*"] & \Hom(C,Y)
   \end{tikzcd}
\end{equation}
\begin{proof}
   Sei $h\in \Hom(D,X)$, dann gilt
   \begin{alignat}{4}
      &g^*(f_*(h))         &&= g^*(f \circ h)      &&= \\
      &(f \circ h) \circ g &&= f \circ (h \circ g) &&=\\
      &f_*(h \circ g)      &&= f_*(g^*(h)).
   \end{alignat}
\end{proof}
Hier handelt es sich also eigentlich um die Assoziativität der Verknüpfung von Funktionen. Etwas poetischer: die Vertauschung ist in Wirklichkeit eine zeitliche und keine räumliche. Das ist in dem Sinne gemeint, dass die Kommutativität die örtliche Reihenfolge der Symbole auf der Tafel irrelevant macht, während die Assoziativität die zeitliche Reihenfolge der Auswertung irrelevant macht. 

\begin{itemize}
   \item eine Zeicnung für $f_*$ und $f^*$
\end{itemize}

\begin{backup}
%******************************************************************************************************
%                                                                                                     *
\section{TODO}
%                                                                                                     *
%******************************************************************************************************
\begin{itemize}
     \item Überprüfe Symbolverzeichnis
\end{itemize}

\end{backup}

\begin{backup}
    (Zur Zeit nicht benötigter Inhalt)
\end{backup}

%******************************************************************************************************
%                                                                                                     *
\begin{thebibliography}{9}
%                                                                                                     *
%******************************************************************************************************
   \bibitem[Awodey2010]{Awodey}
      Steve Awode, \emph{Category Theory},
      2010 Oxford University Press, 978-0-19-923718-0 (ISBN)

   \bibitem[Bradley2020]{Bradley}
      Tai-Danae Bradley, \emph{Topology, A Categorical Approach},
      2020 MIT Press, 978-0-262-53935-7 (ISBN)

   \bibitem[LawvereSchanuel2009]{Lawvere}
      F. William Lawvere, Stephen H. Schanuel, \emph{Conceptual Mathematics: a First Introduction to Categories},
      2009 Cambridge University Press, 978-0-521-71916-2 (ISBN)

   \bibitem[MacLane1978]{MacLane}
      Saunders Mac Lane, \emph{Categories for the Working Mathematician},
      Springer-Verlag New York Inc., 978-0-387-98403-2 (ISBN)

   \bibitem[Rotman2009]{Rotman}
   	Joseph J. Rotman, \emph{An Introduction to Homological Algebra},
   	2009 Springer-Verlag New York Inc., 978-0-387-24527-0 (ISBN)

\end{thebibliography}

%******************************************************************************************************
%                                                                                                     *
\begin{large}
    \centerline{\textsc{Symbolverzeichnis}}
\end{large}
%                                                                                                     *
%******************************************************************************************************
\bigskip

\renewcommand*{\arraystretch}{1}

\begin{tabular}{ll}
    $P(x)$                              & ein Prädikat\\
    $A, B, C, \cdots, X, Y, Z$          & Objekte\\
    $F,G$                               & Funktoren\\
    $V, V'$                             & Vergiss-Funktoren\\
    $f, g, h, r, s, \cdots$             & Homomorphismen\\
    $\mathcal C, \mathcal D, \mathcal E, \cdots$ & Kategorien\\
    $\PP$                               & Potenzmengen-Funktor\\
    \Set                                & Die Kategorie der kleinen Mengen\\
    \Ab                                 & Kategorie der kleinen abelschen Gruppen\\
    $\Hom( X, Y)$                       & Die Klasse der Homomorphismen von $X$ nach $Y$\\
    $\alpha, \beta, \cdots$             & natürliche Transformationen oder Ordinalzahlen\\
    $\mathcal C ^{\text{op}}$           & Duale Kategorie\\
    $\DD^\CC$                           & Funktorkategorie\\
    $\textbf{Ring}, \textbf{Gruppe}$    & Kategorie der kleinen Ringe und der kleinen Gruppen\\
    $U, U', U''$                        & Universen\\
    $V_\alpha$                          & eine Menge der Von-Neumann-Hierarchie zur Ordinalzahl
                                          $\alpha$

\end{tabular}

\end{document}
