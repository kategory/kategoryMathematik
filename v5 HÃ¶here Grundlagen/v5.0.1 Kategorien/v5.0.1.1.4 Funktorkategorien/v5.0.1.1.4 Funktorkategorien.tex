%******************************************************** -*-LaTeX-*- ******************************
%                                                                                                  *
% v5.0.1.1.4 Funktorkategorien.tex                                                                 *
%                                                                                                  *
% Copyright (C) 2024 Kategory GmbH \& Co. KG (joerg.kunze@kategory.de)                             *
%                                                                                                  *
% v5.0.1.1.4 Funktorkategorien is part of kategoryMathematik.                                      *
%                                                                                                  *
% kategoryMathematik is free software: you can redistribute it and/or modify                       *
% it under the terms of the GNU General Public License as published by                             *
% the Free Software Foundation, either version 3 of the License, or                                *
% (at your option) any later version.                                                              *
%                                                                                                  *
% kategoryMathematik is distributed in the hope that it will be useful,                            *
% but WITHOUT ANY WARRANTY; without even the implied warranty of                                   *
% MERCHANTABILITY or FITNESS FOR A PARTICULAR PURPOSE.  See the                                    *
% GNU General Public License for more details.                                                     *
%                                                                                                  *
% You should have received a copy of the GNU General Public License                                *
% along with this program.  If not, see <http://www.gnu.org/licenses/>.                            *
%                                                                                                  *
%***************************************************************************************************

\documentclass[a4paper]{amsart}
% \documentclass[a4paper]{book}

%-----------------------------------------------------------------------------------------------------*
% package:                                                                                            *
%-----------------------------------------------------------------------------------------------------*
\usepackage{amssymb}
\usepackage{amsfonts}
\usepackage{amsmath}
\usepackage{amsthm}

\usepackage{mathabx}

\usepackage{a4wide} % a little bit smaller margins

\usepackage{graphicx}
\usepackage{hyperref}
\usepackage{algorithmic}
\usepackage{listings}
\usepackage{color}
\usepackage{colortbl}
\usepackage{sidecap}
\usepackage{comment}
\usepackage{tcolorbox}
\usepackage{collect}

\usepackage{upgreek}

% \usepackage{diagrams}

\usepackage[german]{babel}
\usepackage[none]{hyphenat}
\emergencystretch=4em

\usepackage[utf8]{inputenc} % to be able to use äöü as characters in text
\usepackage[T1]{fontenc} % to be able to use äöü in lables
\usepackage{lmodern}     % to avoid pixelation introduced by fontenc

\usepackage{hyperref}

\usepackage{tikz}
\usepackage{tikz-cd}
\usetikzlibrary{babel}

%-----------------------------------------------------------------------------------------------------*
% theorem:                                                                                            *
%-----------------------------------------------------------------------------------------------------*
\theoremstyle{definition}
\newtheorem{theorem}{Theorem}[subsection]

\newcommand{\myTheorem}[1]{%
  \newtheorem{jk#1}[theorem]{#1}
  \newenvironment{#1}[1]{%
    \expandafter\begin{jk#1} \expandafter\label{#1:##1}\textbf{(##1):}
  }{%
    \expandafter\end{jk#1}
  }
}

\myTheorem{Definition}
\myTheorem{Proposition}
\myTheorem{Satz}
\myTheorem{Theorem}
\myTheorem{Axiom}
\myTheorem{Beispiel}
\myTheorem{Anmerkung}

\definecollection{jkjkFrage}
\newtheorem{jkFrage}[theorem]{Frage}
\newenvironment{Frage}[1]{%
  \expandafter\begin{jkFrage} \expandafter\label{Frage:#1}\textbf{(#1):}
  \begin{collect}{jkjkFrage}{}{}
    \item \ref{Frage:#1} #1
  \end{collect}
}{%
  \expandafter\end{jkFrage}
}

\newcommand{\myRef}[2]{[#1 \ref{#1:#2}, ``#2'']}

\renewcommand{\proofname}{Beweis}

%-----------------------------------------------------------------------------------------------------*
% operator:                                                                                           *
%-----------------------------------------------------------------------------------------------------*
\DeclareMathOperator{\End}{End}
\DeclareMathOperator{\Ker}{Ker}
\DeclareMathOperator{\Mat}{Mat}
\DeclareMathOperator{\rank}{rank}
\DeclareMathOperator{\ggT}{ggT}
\DeclareMathOperator{\len}{len}
\DeclareMathOperator{\ord}{ord}
\DeclareMathOperator{\kgV}{kgV}
\DeclareMathOperator{\id}{id}
\DeclareMathOperator{\red}{red}
\DeclareMathOperator{\supp}{supp}
\DeclareMathOperator{\Bild}{Bild}
\DeclareMathOperator{\Rang}{Rang}
\DeclareMathOperator{\Det}{Det}
\DeclareMathOperator{\Hom}{Hom}
\DeclareMathOperator{\GL}{GL}

\DeclareMathOperator{\sub}{sub}
\DeclareMathOperator{\blk}{blk}
\DeclareMathOperator{\minimal}{minimal}
\DeclareMathOperator{\maximal}{maximal}

\DeclareMathOperator{\Dom}{Dom}
\DeclareMathOperator{\Cod}{Cod}
\DeclareMathOperator{\Obj}{Obj}

\definecolor{mygreen}{rgb}{0,0.6,0}
\definecolor{mygray}{rgb}{0.5,0.5,0.5}
\definecolor{mymauve}{rgb}{0.58,0,0.82}

\lstset{ %
  backgroundcolor=\color{white},   % choose the background color
  basicstyle=\ttfamily\footnotesize,        % size of fonts used for the code
  breaklines=true,                 % automatic line breaking only at whitespace
  captionpos=b,                    % sets the caption-position to bottom
  commentstyle=\color{mygreen},    % comment style
  escapeinside={\%*}{*)},          % if you want to add LaTeX within your code
  keywordstyle=\color{blue},       % keyword style
  stringstyle=\color{mymauve},     % string literal style
  frame=single
}

\setcounter{MaxMatrixCols}{20}

%******************************************************************************************************
%                                                                                                     *
% definition:                                                                                         *
%                                                                                                     *
%******************************************************************************************************
\newcommand{\R}{\ensuremath{\mathbb{ R }}}
\newcommand{\Q}{\ensuremath{\mathbb{ Q }}}
\newcommand{\Z}{\ensuremath{\mathbb{ Z }}}
\newcommand{\N}{\ensuremath{\mathbb{ N }}}
\newcommand{\C}{\ensuremath{\mathbb{ C }}}
\newcommand{\A}{\ensuremath{\mathbb{ A }}}
\newcommand{\F}{\ensuremath{\mathbb{ F }}}
\newcommand{\K}{\ensuremath{\mathbb{ K }}}
\newcommand{\Pb}{\ensuremath{\mathbb{ P }}}

\newcommand{\M}{\ensuremath{\mathcal{ M }}}
\newcommand{\V}{\ensuremath{\mathcal{ V }}}

\newcommand{\AAA}{\ensuremath{\mathcal{ A }}}
\newcommand{\BB}{\ensuremath{\mathcal{ B }}}
\newcommand{\CC}{\ensuremath{\mathcal{ C }}}
\newcommand{\DD}{\ensuremath{\mathcal{ D }}}
\newcommand{\EE}{\ensuremath{\mathcal{ E }}}
\newcommand{\FF}{\ensuremath{\mathcal{ F }}}
\newcommand{\KK}{\ensuremath{\mathcal{ K }}}
\newcommand{\MM}{\ensuremath{\mathcal{ M }}}
\newcommand{\PP}{\ensuremath{\mathcal{ P }}}
\newcommand{\ZZ}{\ensuremath{\mathcal{ Z }}}

\newcommand{\Set}{\text{\textbf{Set}}}
\newcommand{\Ab}{\text{\textbf{Ab}}}

\newcommand{\imporant}[1]{ \textcolor{red}{\textbf{#1}} }

\newcommand{\bb}[1]{\mathbf{#1}}
\newcommand{\balpha}{\boldsymbol{\upalpha}}
\newcommand{\bbeta}{\boldsymbol{\upbeta}}
\newcommand{\bgamma}{\boldsymbol{\upgamma}}
\newcommand{\bdelta}{\boldsymbol{\delta}}
\newcommand{\bmu}{\boldsymbol{\upmu}}

\newcommand{\z}[1]{\Z_{#1}}
\newcommand{\e}[1]{\z{#1}^*}
\newcommand{\q}[1]{(\e{#1})^2}
\newcommand{\m}{\mathcal}

\excludecomment{book}
\excludecomment{example}
\excludecomment{backup}

\newcommand{\zb}{z.~B. }

\begin{document}

%******************************************************************************************************
%                                                                                                     *
\begin{titlepage}
%                                                                                                     *
%******************************************************************************************************
% \vspace*{\fill}
\centering
{\huge
(Höhere Grundlagen) Kategorien\\[1cm]
\textbf{v5.0.1.1.4 Funktorkategorien}
}\\[1cm]

\textbf{Kategory GmbH \& Co. KG}\\
Präsentiert von Jörg Kunze\\
Copyright (C) 2024 Kategory GmbH \& Co. KG

\end{titlepage}

%\clearpage
%\setcounter{page}{2}
%
%\tableofcontents

\newpage

%******************************************************************************************************
%                                                                                                     *
\section*{Beschreibung}
%                                                                                                     *
%******************************************************************************************************

%******************************************************************************************************
\subsection*{Inhalt}
%******************************************************************************************************
Die Funktorkategorie aller Funktoren von einer Kategorie C in eine andere D als Objekte und der natürlichen Transformationen zwischen ihnen als Homomorphismen erfüllt alle Axiome einer Kategorie.

Das liegt zum einen daran, dass sich Eigenschaften der Morphismen, welche die Komponenten der Nat's sind, auf die Nat's übertragen, zum anderen, dass das Zusammensetzen von kommutativen Quadraten wieder kommutative Quadrate ergibt.

Funktorkategorien kommen überall und ständig vor. Viele handelsübliche Kategorien entpuppen sich als isomorph zu Funktorkategorien. So \zb die Kat. der Morphismen oder die der R-Darstellungen einer Gruppe G.

VORSICHT: es gibt Größenprobleme: Wenn die Quell-Kategorie C nicht klein ist (eine echte Klasse ist), ist ein Funktor oder eine Nat im Allgemeinen kein Element unseres Universums (eine echte Klasse). Damit kann die Funktorkategorie nicht innerhalb von U gebildet werden (existiert nicht, da echte Klassen nicht Elemente von Klassen sein können).

Die Wichtigkeit und Häufigkeit von Funktorkategorien auf der einen Seite und die Größenprobleme auf der anderen Seite sind einer DER Gründe, für die Einführung des Konzeptes "`kleine Mengen"'. 

%******************************************************************************************************
\subsection*{Präsentiert}
%******************************************************************************************************
Von Jörg Kunze

%******************************************************************************************************
\subsection*{Voraussetzungen}
%******************************************************************************************************
Kategorie, Funktor, Natürliche Transformation, kleine Kategorien

%******************************************************************************************************
\subsection*{Text}
%******************************************************************************************************
Der Begleittext als PDF und als LaTeX findet sich unter
{\tiny
   \url{https://github.com/kategory/kategoryMathematik/tree/main/v5%20H%C3%B6here%20Grundlagen/v5.0.1%20Kategorien/v5.0.1.1.4%20Funktorkategorien}
}

%******************************************************************************************************
\subsection*{Meine Videos}
%******************************************************************************************************
Siehe auch in den folgenden Videos:\\
\\
v5.0.1.0.1 (Höher) Kategorien - Axiome für Kategorien\\
\url{https://youtu.be/X8v5Kyly0KI}\\
\\
v5.0.1.0.2 (Höher) Kategorien - Kategorien\\
\url{https://youtu.be/sIaKt-Wxlog}\\
\\
v5.0.1.0.3 (Höher) Kategorien - Funktoren\\
\url{https://youtu.be/Ojf5LQGeyOU}\\
\\
v5.0.1.0.4 (Höher) Kategorien - Natürliche Transformationen\\
\url{https://youtu.be/IN7Qa-SwlD0}\\
\\
v5.0.1.0.6 (Höher) Kategorien - Mathematische Grundlagen\\
\url{https://youtu.be/ezW54mnzHMw}\\
\\
v5.0.1.0.7 (Höher) Kategorien - Große Kategorien\\
\url{https://youtu.be/Dcz6at6EXqM}


%******************************************************************************************************
\subsection*{Quellen}
%******************************************************************************************************
Siehe auch in den folgenden Seiten:\\
\url{https://de.wikipedia.org/wiki/Funktor_(Mathematik)}\\
\url{https://de.wikipedia.org/wiki/Funktorkategorie}\\
\url{https://en.wikipedia.org/wiki/Functor_category}\\
\url{https://ncatlab.org/nlab/show/functor+category}

%******************************************************************************************************
\subsection*{Buch}
%******************************************************************************************************
Grundlage ist folgendes Buch:\\
"`Categories for the Working Mathematician"'\\
Saunders Mac Lane\\
1998 | 2nd ed. 1978\\
Springer-Verlag New York Inc.\\
978-0-387-98403-2 (ISBN)\\
{\tiny
   \url{https://www.amazon.de/Categories-Working-Mathematician-Graduate-Mathematics/dp/0387984038}}\\

Gut für die kategorische Sichtweise ist:\\
"`Topology, A Categorical Approach"'\\
Tai-Danae Bradley\\
2020 MIT Press\\
978-0-262-53935-7 (ISBN)\\
{\tiny
\url{https://www.lehmanns.de/shop/mathematik-informatik/52489766-9780262539357-topology}}\\

Einige gut Erklärungen finden sich auch in den Einführenden Kapitel von:\\
"`An Introduction to Homological Algebra"'\\
Joseph J. Rotman\\
2009 Springer-Verlag New York Inc.\\
978-0-387-24527-0 (ISBN)\\
{\tiny \url{https://www.lehmanns.de/shop/mathematik-informatik/6439666-9780387245270-an-introduction-to-homological-algebra}}\\

Etwas weniger umfangreich und weniger tiefgehend aber gut motivierend ist:
"`Category Theory"'\\
Steve Awodey\\
2010 Oxford University Press\\
978-0-19-923718-0 (ISBN)\\
{\tiny\url{https://www.lehmanns.de/shop/mathematik-informatik/9478288-9780199237180-category-theory}}\\

Mit noch weniger Mathematik und die Konzepte motivierend ist:
"`Conceptual Mathematics: a First Introduction to Categories"'\\
F. William Lawvere, Stephen H. Schanuel\\
2009 Cambridge University Press\\
978-0-521-71916-2 (ISBN)\\
{\tiny\url{https://www.lehmanns.de/shop/mathematik-informatik/8643555-9780521719162-conceptual-mathematics}}

%******************************************************************************************************
\subsection*{Lizenz}
%******************************************************************************************************
Dieser Text und das Video sind freie Software. Sie können es unter den Bedingungen der
GNU General Public License, wie von der Free Software Foundation veröffentlicht, weitergeben
und/oder modifizieren, entweder gemäß Version 3 der Lizenz oder (nach Ihrer Option) jeder späteren Version.

Die Veröffentlichung von Text und Video erfolgt in der Hoffnung, dass es Ihnen von Nutzen sein wird,
aber OHNE IRGENDEINE GARANTIE, sogar ohne die implizite Garantie der MARKTREIFE oder der
VERWENDBARKEIT FÜR EINEN BESTIMMTEN ZWECK. Details finden Sie in der GNU General Public License.

Sie sollten ein Exemplar der GNU General Public License zusammen mit diesem Text erhalten haben
(zu finden im selben Git-Projekt).
Falls nicht, siehe \url{http://www.gnu.org/licenses/}.

\subsection*{Das Video}
%******************************************************************************************************
Das Video hierzu ist zu finden unter
{\tiny
   \url{XXX}
}

%******************************************************************************************************
%                                                                                                     *
\section{Funktorkategorien}
%                                                                                                     *
%******************************************************************************************************

%******************************************************************************************************
\subsection{Kategorien-Axiome}
%******************************************************************************************************
Seien $\CC, \DD$ zwei Kategorien. Die Objekte der Funktorkategorie $\DD^\CC$ sind die Funktoren $F \colon \CC \to \DD$. Die Homomorphismen zwischen zwei Funktoren $F, G \colon \CC \to \DD$ sind die natürlichen Transformationen (Nat) $F \to G$.

Seien zwei Nat's $\alpha \colon F \to G$ und $\beta \colon G \to H$ gegeben. Die Verknüpfung wird komponentenweise für alle Objekte $X \in \CC$ definiert:
\begin{equation}
   (\beta \circ \alpha)_X := \beta_X \circ \alpha_X.
\end{equation}
Dies geht, da die Komponenten $\beta_X, \alpha_X$ Morphismen in $\DD$ sind. Um nachzuweisen, dass es sich bei dieser Verknüpfung wieder um eine Nat handelt, dass also die entsprechenden Quadrate kommutativ sind, betrachten wir folgende Zeichnung:
\begin{equation}\label{kommutativ}
   \begin{tikzcd}
       F(X) \arrow[r, "\alpha_X"] \arrow[d, "F(f)"]
      &G(X) \arrow[r, "\beta_X" ] \arrow[d, "G(f)"]
      &H(X)                       \arrow[d, "H(f)"]
      \\
       F(Y) \arrow[r, "\alpha_Y"]  
      &G(Y) \arrow[r, "\beta_Y" ]  
      & H(Y)
   \end{tikzcd}
\end{equation}
Hier haben wir dann
\begin{alignat}{4}
   &H(f) \circ (\beta \circ \alpha)_X   &&= \\
   &H(f) \circ (\beta_X \circ \alpha_X) &&= (H(f) \circ \beta_X) \circ \alpha_X = \\
   &(\beta_Y \circ G(f)) \circ \alpha_X &&= \beta_Y \circ (G(f) \circ \alpha_X) = \\
   &\beta_Y \circ (\alpha_Y \circ F(f)) &&= (\beta_Y \circ \alpha_Y) \circ F(f) = \\
   &(\beta \circ \alpha)_Y \circ F(f)
\end{alignat}
Hier haben wir die Assoziativität der Verknüpfung von Morphismen in $\DD$, die Definition der Verknüpfung von Nat's und die Kommutativität der Teil-Quadrate ausgenutzt.

Wir definieren weiter die Identitäten als die komponentenweise Identitäten:
\begin{equation}
   (\id_F)_X := \id_{F(X)}.
\end{equation}
Dass diese Nat's sind und als Identitäten mit der obern definierten Verknüpfung agieren ist leicht zuu zeigen.

Das Assoziativgesetz ergibt sich ebenfalls schnell aus der komponentenweise Assoziativität. 

Damit sind alle Eigenschaften einer Kategorie erfüllt.

%******************************************************************************************************
\subsection{Schreibweise}
%******************************************************************************************************
Wir schreiben 
\begin{equation}
   \DD^\CC
\end{equation}
für die Funktor-Kategorie der Funktoren von $\CC$ nach $\DD$. Das Ziel steht also dummerweise links die Quelle oben-rechts.
Manchmal schreiben wir auch so etwas wie
\begin{equation}
   \operatorname{Fun}( \CC, \DD).
\end{equation}
Diese Hoch-Schreibweise gibt es auch bei Funktionen, wo $\{0,1,2\}^{\{0,1,2,3\}}$ eben $3^4$ Elemente hat.

%******************************************************************************************************
\subsection{Beispiele}
%******************************************************************************************************

Diskrete Kategorien sind letztlich Mengen. Ein Funktor ist dann nichts anderes als eine Funktion. In dem Fall ist $\DD^\CC$ einfach die Menge der Funktionen von $\CC$ nach $\DD$.

Wenn $\CC$ diskret/Menge ist, ist $\{0,1\}^\CC$ die Menge der Teilmengen von $\CC$.

Wenn $\boldsymbol 1 = \{ \bullet \}$ die Kategorie mit genau einem Element und nur der Identität $\id_\bullet$ ist, dann ist $\DD^{\boldsymbol{1}}$ isomorph zu $\DD$.

Wenn $\boldsymbol 2 = \{ \bullet \to \bullet \}$ die Kategorie mit genau zwei Elementen und nur den angedeuteten Morphismus neben den beiden Identitäten ist, dann ist $\DD^{\boldsymbol{2}}$ die Kategorie der Morphismen von $\DD$.

Sei nun $R$ ein kommutativer Ring mit $1$ und $G$ eine Gruppe. Wir betrachten $G$ als Kategorie mit einem Objekt und den Gruppen-Elementen als Morphismen. Wir bezeichnen mit $R$-Mod die Kategorie der $R$-Moduln. Dann ist $\text{$R$-Mod}^G$ die Kategorie der $R$-Darstellungen von $G$.

%******************************************************************************************************
\subsection{Größen-Probleme}
%******************************************************************************************************
Wenn $\CC$ nicht klein ist, ist Funktor im Allgemeinen nicht in unserem Universum. Wenn wir mit dem Universum aller Mengen rechnen, ist so ein Funktor eine echte Klasse.

Das heißt: der ganze obere Sermon gilt nur, wenn wir als Voraussetzung "`$\CC$ ist klein"' hinzufügen.

Nur wenn $\CC$ zu groß ist. $\text{\textbf{Gruppe}}^{\boldsymbol{2}}$, also die Kategorie der Morphismen der Kategorie der Gruppen ist kein Problem, auch wenn wir mit \textbf{Gruppe} die echte Klasse aller Gruppen meinen. Denn das Problem ist nicht, dass es zu viele Funktoren geben könnten, sondern dass der einzelne Funktor eine echte Klasse wird.

%******************************************************************************************************
\subsection{Deswegen benötigen wir kleine Mengen}
%******************************************************************************************************
Funktor-Kategorien kommen ständig vor, sind super nützlich, wichtig und interessant.

Funktor-Kategorien einer der Haupt-Gründe für die Einführung der Konzepte rund um kleine Mengen und kleine Kategorien. 

\begin{backup}
%******************************************************************************************************
%                                                                                                     *
\section{TODO}
%                                                                                                     *
%******************************************************************************************************
\begin{itemize}
     \item Überprüfe Symbolverzeichnis
\end{itemize}

\end{backup}

\begin{backup}
    (Zur Zeit nicht benötigter Inhalt)
\end{backup}

%******************************************************************************************************
%                                                                                                     *
\begin{thebibliography}{9}
%                                                                                                     *
%******************************************************************************************************
   \bibitem[Awodey2010]{Awodey}
      Steve Awode, \emph{Category Theory},
      2010 Oxford University Press, 978-0-19-923718-0 (ISBN)

   \bibitem[Bradley2020]{Bradley}
      Tai-Danae Bradley, \emph{Topology, A Categorical Approach},
      2020 MIT Press, 978-0-262-53935-7 (ISBN)

   \bibitem[LawvereSchanuel2009]{Lawvere}
      F. William Lawvere, Stephen H. Schanuel, \emph{Conceptual Mathematics: a First Introduction to Categories},
      2009 Cambridge University Press, 978-0-521-71916-2 (ISBN)

   \bibitem[MacLane1978]{MacLane}
      Saunders Mac Lane, \emph{Categories for the Working Mathematician},
      Springer-Verlag New York Inc., 978-0-387-98403-2 (ISBN)

   \bibitem[Rotman2009]{Rotman}
   	Joseph J. Rotman, \emph{An Introduction to Homological Algebra},
   	2009 Springer-Verlag New York Inc., 978-0-387-24527-0 (ISBN)

\end{thebibliography}

%******************************************************************************************************
%                                                                                                     *
\begin{large}
    \centerline{\textsc{Symbolverzeichnis}}
\end{large}
%                                                                                                     *
%******************************************************************************************************
\bigskip

\renewcommand*{\arraystretch}{1}

\begin{tabular}{ll}
    $P(x)$                              & ein Prädikat\\
    $A, B, C, \cdots, X, Y, Z$          & Objekte\\
    $F,G$                               & Funktoren\\
    $V, V'$                             & Vergiss-Funktoren\\
    $f, g, h, r, s, \cdots$             & Homomorphismen\\
    $\mathcal C, \mathcal D, \mathcal E, \cdots$ & Kategorien\\
    $\PP$                               & Potenzmengen-Funktor\\
    \Set                                & Die Kategorie der kleinen Mengen\\
    \Ab                                 & Kategorie der kleinen abelschen Gruppen\\
    $\Hom( X, Y)$                       & Die Klasse der Homomorphismen von $X$ nach $Y$\\
    $\alpha, \beta, \cdots$             & natürliche Transformationen oder Ordinalzahlen\\
    $\mathcal C ^{\text{op}}$           & Duale Kategorie\\
    $\DD^\CC$                           & Funktorkategorie\\
    $\textbf{Ring}, \textbf{Gruppe}$    & Kategorie der kleinen Ringe und der kleinen Gruppen\\
    $U, U', U''$                        & Universen\\
    $V_\alpha$                          & eine Menge der Von-Neumann-Hierarchie zur Ordinalzahl
                                          $\alpha$

\end{tabular}

\end{document}
